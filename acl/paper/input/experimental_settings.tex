\section{Experimental Settings}
\label{sec:experimental_settings}
  \subsection{Datasets}
  \label{subsec:datasets}
    We run experiments over two datasets: Inist (ling.) and SemEval, which
    differ in terms of language, nature and size. The following is the
    description of each dataset.

    \paragraph{Inist (ling.)} \TODO{uncertain yet}

    \paragraph{SemEval~\textnormal{\cite{kim2010semeval}}} contains 244 English
    scientific papers collected from the ACM Digital Libraries (conference and
    workshop papers). The papers represent four areas of Computer Science:
    Distributed Systems; Information Search and Retrieval; Distributed
    Artificial Intelligence -- Multiagent Systems; Social and Behavioral
    Sciences -- Economics. SemEval is divided into two disjoint sets: a training
    set containing 144 documents and a test set containing 100 documents. The
    associated keyphrases are provided by both authors and readers.

    \paragraph{}
    Table~\ref{tab:corpus_statistics} shows factual information about each
    datasets. By looking at these, one can understand the importance of methods
    like ours. Indeed, ranging from \TODO{0.0\%} to \TODO{100\%}, the percentage
    of keyphrases impossible to extract from the documents is part of the
    keyphrases our method is able to find.

  \subsection{Preprocessing}
  \label{subsec:preprocessing}
    We apply the following preprocessing steps to every document of the
    datasets: sentence segmentation, word tokenization and Part-of-Speech (POS)
    tagging. We perform sentence segmentation with the PunktSentenceTokenizer
    provided by the Python Natural Language ToolKit~\cite[NLTK]{bird2009nltk}.
    We tokenize the sentence into words using the NLTK TreebankWordTokenizer for
    English and the Bonsai word tokenizer\footnote{The Bonsai word tokenizer is
    a tool provided with the Bonsai PCFG-LA parser:
    \url{http://alpage.inria.fr/statgram/frdep/fr_stat_dep_parsing.html}.} for
    French. Finally, we use the Stanford POS
    tagger~\cite{toutanova2003stanfordpostagger} for English POS tagging and
    MElt~\cite{denis2009melt} for French POS tagging.

  \subsection{Baselines}
  \label{subsec:baselines}
    We compare TopicRank++ with TopicRank and TopicRank++ when only domain
    keyphrases are extracted (TopicRank++$_\textnormal{\textit{dom.}}$), when
    only document keyphrases are extracted
    (TopicRank++$_\textnormal{\textit{doc.}}$) and when the co-ranking is
    performed with candidate keyphrases instead of topics
    (TopicRank++$_\textnormal{\textit{cdt}}$).

  \subsection{Evaluation measures}
  \label{subsec:evaluation_measures}
    To evaluate the performance of the keyphrase extraction methods, we use the
    common measures of precision (P), recall (R) and f-score (F). We cut-off the
    extracted keyphrases at the 10 best ranking ones.

