\section{Introduction}
\label{sec: introduction}
  % - definition of keyphrases
  % - motivations
  Keyphrases are words or phrases that give a synoptic picture of what is
  important within a document. Keyphrases are useful in many tasks such as
  document indexing for digital
  libraries~\cite{gutwin1999keyphrasesfordigitallibraries} or extractive
  document summarization~\cite{litvak2008graphbased}. However, documents do not
  always contain keyphrases and the daily flow of new documents makes the manual
  keyphrase annotation impractical. As a consequence, automatic keyphrase
  annotation of single documents recently attracts a lot of attention and many
  different methods were proposed~\cite{hasan2014state_of_the_art}.

  % - definition of the automatic keyphrase annotation task
  The automatic keyphrase annotation task consists in automatically detecting
  the important concepts addressed by a document. We divide the automatic
  keyphrase annotation into two principal categories: Automatic Keyphrase
  Extraction (AKE) and Automatic Keyphrase Assignment (AKA). AKE detects
  important words or phrases occuring within the document, whereas AKA provides
  keyphrases from a domain-specific terminology, without regard to their
  occurence within the document.

  % - benefits and drawbacks of each task
  The automatic keyphrase annotation problem has mainly been tackled using AKE
  methods. However, statistics from standard evaluation datasets show a sizeable
  amount of keyphrases that do not occur within their
  documents~\cite{bougouin2013topicrank}, i.e.~keyphrases impossible to find
  using AKE methods. In opposition, AKA methods have the ability to bridge this
  ``vocabulary gap''~\cite{liu2011vocabularygap}, but they fail to discover
  novel concepts, i.e.~concepts that are not terminological entries yet (or
  variants).

  % - our propososal
  This paper presents a new method performing both AKA to bridge the
  ``vocabulary gap'' and AKE to discover potentially novel concepts. Leveraging
  training data with documents associated with reference keyphrases, we
  simultanously rank these reference keyphrases and the concepts of a document
  using a unified graph-based algorithm. An importance score is given to every
  reference keyphrase based on its relationship with the other reference
  keyphrases and its relationship with the document's concepts. Similarly, an
  importance score is given to every concept of the document based on its
  relationship with the other concepts and its relationship with the reference
  keyphrases.

  \TODO{Results insight}

  \TODO{The rest of this paper is organized as follows\dots}

