\section{Introduction}
\label{sec: introduction}
  % keyphrase?
  % => need of automatic keyphrase annotation
  Keyphrases are words or phrases that give a synoptic picture of what is
  important within a document. Keyphrases are useful in many tasks such as
  document indexing for digital
  libraries~\cite{gutwin1999keyphrasesfordigitallibraries} or extractive
  document summarization~\cite{litvak2008graphbased}. However, documents do not
  always contain keyphrases and the daily flow of new documents makes the manual
  keyphrase annotation impractical. As a consequence, automatic keyphrase
  annotation of single documents recently attracts a lot of attention and many
  different methods were proposed~\cite{hasan2014state_of_the_art}.

  % automatic keyphrase annotation?
  The automatic keyphrase annotation task consists in automatically detecting
  the important concepts, or topics, addressed by a document. We divide the
  automatic keyphrase annotation into two disjoint categories: Automatic
  Keyphrase Extraction (AKE) and Automatic Keyphrase Assignment (AKA). In the
  first hand, AKE detects important words or phrases occuring within the
  document. In the other hand, AKA provides controlled keyphrases from a
  domain-specific terminology, or controlled vocabulary, without regard to their
  occurence within the document.

  % motivations?
  The automatic keyphrase annotation problem has mainly been addressed using AKE
  methods. However, statistics from standard evaluation datasets show a sizeable
  amount of keyphrases that do not occur within the
  documents~\cite{bougouin2013topicrank}, i.e.~keyphrases impossible to find
  using AKE methods. AKA methods have the ability to bridge this ``vocabulary
  gap''~\cite{liu2011vocabularygap} but, in turn, fail to discover novel or more
  specific concepts, i.e.~concepts that are not terminological entries yet (or
  variants).

  % our proposed method?
  % => its benefits?
  This paper presents a new method performing both AKA, to bridge the
  ``vocabulary gap'', and AKE, to discover potentially novel concepts.
  Leveraging training data containing documents associated with reference
  keyphrases, we simultanously rank the training reference keyphrases and the
  concepts of a document using a unified graph-based co-ranking algorithm. An
  importance score is given to every reference keyphrase based on its
  relationship with the other reference keyphrases and its relationship with the
  document's concepts. Similarly, an importance score is given to every concept
  of the document based on its relationship with the other concepts and its
  relationship with the reference keyphrases. Despite combining both AKE and
  AKA, our approach conciders already used keyphrases (from training data) as a
  controlled vocabulary. Following this assumption reduces the cost of producing
  domain-specific data, such as thesauri, and makes our method suitable in any
  case where training data are available.

  \TODO{Results insight}

  The rest of this paper is organized as follows. Section~\ref{sec:related_work}
  presents the previous work on automatic keyphrase annotation and the previous
  employments of graph co-ranking for Natural Language Processing (NLP) tasks.
  Section~\ref{sec:topicrankpp} describes our method and
  sections~\ref{sec:experimental_settings}~and~\ref{sec:results} presents,
  respectively, our experimental settings and the comparison results of our
  method with the baselines. Finally, section~\ref{sec:conclusion} concludes our
  work and discusses future work.

