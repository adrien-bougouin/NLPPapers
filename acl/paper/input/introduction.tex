\section{Introduction}
\label{sec: introduction}
  % keyphrase?
  % => need of automatic keyphrase annotation
  %Keyphrases are words or phrases that give a synoptic picture of what is
  %important within a document. Keyphrases are useful in many tasks such as
  %document indexing for digital
  %libraries~\cite{gutwin1999keyphrasesfordigitallibraries} or extractive
  %document summarization~\cite{litvak2008graphbased}. However, documents do not
  %always contain keyphrases and the daily flow of new documents makes the manual
  %keyphrase annotation impractical. As a consequence, automatic keyphrase
  %annotation of single documents recently attracts a lot of attention and many
  %different methods were proposed~\cite{hasan2014state_of_the_art}.
  
  Keyphrases are words or phrases that give a synoptic picture of what is
  important within a document. They are useful in many tasks such as
  document indexing~\cite{gutwin1999keyphrasesfordigitallibraries}, text
  categorization~\cite{hulth-megyesi:2006:COLACL} or
  summarization~\cite{litvak2008graphbased}. However, most documents do not
  provide keyphrases, and the daily flow of new documents makes the manual
  keyphrase annotation impractical. As a consequence, automatic keyphrase
  annotation has recently received a lot of attention in the Natural Language
  Processing (NLP) community and many different methods were 
  proposed~\cite{hasan2014state_of_the_art}.


  % automatic keyphrase annotation?
  %The automatic keyphrase annotation task consists in automatically detecting
  %the important concepts, or topics, addressed by a document. We divide the
  %automatic keyphrase annotation into two disjoint categories: automatic
  %keyphrase extraction and automatic keyphrase assignment. The first detects
  %important words or phrases occuring within the document. The second provides
  %controlled keyphrases from a domain-specific terminology (controlled
  %vocabulary), even if they do not occur within the document.
  
  The task of automatic keyphrase annotation consists in identifying the main 
  concepts, or topics, addressed in a document.
  Keyphrase annotation methods fall into two broad categories: keyphrase 
  extraction and keyphrase assignment methods.
  Keyphrase extraction methods extract the most important words or phrases
  occurring in a document, while assignment methods provide controlled 
  keyphrases from a domain-specific terminology (controlled vocabulary).

  % motivations?
  %The automatic keyphrase annotation problem has mainly been addressed using
  %extraction methods, because they are more practical than keyphrase assignment
  %methods. Unlike assignment methods, extraction methods do not need a manually
  %constructed controlled vocabulary and are able to discover novel concepts,
  %i.e. concepts that do not match entries of a controlled vocabulary yet.
  %Both categories of methods are not incompatible and would benefit
  %from each other.
  
  The automatic keyphrase annotation task is often reduced to the sole 
  keyphrase extraction task. 
  Unlike assignment methods, extraction methods do not require controlled 
  vocabularies that are costly to create and to maintain. 
  %BD je vois pas le lien entre la capacite de trouver de nouveaux concepts et le fait de ne pas s'appuyer sur un vocabulaire controlé. Donc le For that reason à supprimer.
  Furthermore, they are able to identify new keyphrases, i.e., not 
  already assigned to previous documents.
  However, extraction methods often output ill-formed or inappropriate
  keyphrases~\cite{medelyan2008smalltrainingset}, and they produce only
  keyphrases that actually occur in the document.
  %BD A rajouter nos hypotheses
  Our claim is that these two kinds of methods are complementary and 
  should benefit from each other if they are performed jointly.
  
  In this work, we present a graph co-ranking approach to keyphrase 
  annotation that simultaneously performs keyphrase assignment and 
  extraction in a mutually reinforcing manner.
  We rely on documents annotated with keyphrases to build a co-occurrence 
  graph representation of the controlled vocabulary.
  A second co-occurrence graph is built from the document and connected 
  to the latter.
  Keyphrase candidates extracted from the document and entries 
  from the controlled vocabulary are then simultaneously ranked.
  Our assumption is that there is a mutually reinforcing relationship between
  the ranking of the keyphrases in the document and the ranking of
  the entries in the controlled vocabulary.
  %the information provided by the co-occurrence relations in 
  %the document and the information provided by the co-occurrence
  %relations of the entries in the controlled vocabulary.
  
  %The main intuition behind our co-ranking approach is that there is a mutually reinforcing relationship between the document and the controlled vocabulary.
%  Topic CoRank implements our hypothesis of the mutual benefit of the two kinds of keyphrases: document keyphrases and terminological keyphrases.

  We perform our experiments on three datasets of different languages and domains.
  Results show that our co-ranking approach significantly improves performance over a state-of-the-art graph-based keyphrase extraction approach.
  We further analyse how document and controlled vocabulary graphs contribute to the ranking, and show that both keyphrase extraction and assignment tasks benefit from being treated jointly.
  

  % our proposed method?
  % => its benefits?
 % This paper presents TopicCoRank, the first method that performs both keyphrase
 % extraction and keyphrase assignment, using graph co-ranking. Employed to
 % tackle various NLP
 % tasks~\cite{wan2011corankingsummarization,yan2012corankingtweetrecommendation,zhang2013wordtopicmultirank,liu2014corankingopinionmining},
 % graph co-ranking is a powerfull technique that reinforce the ranking of
 % specific entities by the ranking of other entities. We simultaneously rank the
 % keyphrase candidates of the document and the entries of a controlled
 % vocabulary. To reduce the cost of producing a controlled vocabulary, we use
 % the keyphrases annotated for training documents (reference keyphrases).
 % Comparisons with keyphrase extraction and keyphrase assignment baselines show
 % improvements while performing keyphrase extraction, keyphrase assignment and
 % both.

  The rest of this paper is organized as follows. Section~\ref{sec:related_work}
  presents the previous work on automatic keyphrase annotation.
  Section~\ref{sec:topicrankpp} describes our method,
  section~\ref{sec:experimental_settings} presents our experimental settings and
  section~\ref{sec:results} discusses the results.
  Finally, section~\ref{sec:conclusion} concludes and shows some directions 
  for future work.

