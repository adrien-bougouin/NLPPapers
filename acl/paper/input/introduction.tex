\section{Introduction}
\label{sec: introduction}
  % keyphrase?
  % => need of automatic keyphrase annotation
  Keyphrases are words or phrases that give a synoptic picture of what is
  important within a document. Keyphrases are useful in many tasks such as
  document indexing for digital
  libraries~\cite{gutwin1999keyphrasesfordigitallibraries} or extractive
  document summarization~\cite{litvak2008graphbased}. However, documents do not
  always contain keyphrases and the daily flow of new documents makes the manual
  keyphrase annotation impractical. As a consequence, automatic keyphrase
  annotation of single documents recently attracts a lot of attention and many
  different methods were proposed~\cite{hasan2014state_of_the_art}.

  % automatic keyphrase annotation?
  The automatic keyphrase annotation task consists in automatically detecting
  the important concepts, or topics, addressed by a document. We divide the
  automatic keyphrase annotation into two disjoint categories: automatic
  keyphrase extraction and automatic keyphrase assignment. The first detects
  important words or phrases occuring within the document. The second provides
  controlled keyphrases from a domain-specific terminology (controlled
  vocabulary), even if they do not occur within the document.

  % motivations?
  The automatic keyphrase annotation problem has mainly been addressed using
  extraction methods, because they are more practical than keyphrase assignment
  methods. Unlike assignment methods, extraction methods do not need a manually
  constructed controlled vocabulary and are able to discover novel concepts,
  i.e. concepts that do not match entries of a controlled vocabulary. However,
  keyphrase extraction methods often output ill-formed or inappropriate
  keyphrases. Both categories of methods are not incompatible and would benefit
  from each other.

  % our proposed method?
  % => its benefits?
  This paper presents a graph co-ranking method that performs both keyphrase
  extraction and keyphrase assignment. Graph co-ranking is a powerfull
  technique, already employed for various NLP tasks~\cite{}, that reinforce the
  ranking of specific entities by the ranking of other entities. We
  simultaneously rank the keyphrase candidates of the document and the entries
  of a controlled vocabulary. To reduce the cost of producing a controlled
  vocabulary, we use the keyphrases annotated to training documents (reference
  keyphrases).

  \TODO{Results insight}

  The rest of this paper is organized as follows. Section~\ref{sec:related_work}
  presents the previous work on automatic keyphrase annotation and the previous
  employments of graph co-ranking for Natural Language Processing (NLP) tasks.
  Section~\ref{sec:topicrankpp} describes our method,
  section~\ref{sec:experimental_settings} presents our experimental settings and
  section~\ref{sec:results} discusses the comparison results of our method with
  the baselines. Finally, section~\ref{sec:conclusion} concludes our work and
  discusses future work.

