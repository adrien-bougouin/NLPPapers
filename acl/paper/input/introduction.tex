\section{Introduction}
\label{sec: introduction}
  % - definition of keyphrases
  % - motivations
  Keyphrases are words or phrases that give a synoptic picture of what is
  important within a document. Keyphrases are useful in many tasks such as
  document clustering~\cite{han2007webdocumentclustering} and document
  summarization~\cite{litvak2008graphbased}. However, documents do not always
  contain keyphrases and the daily flow of new documents makes manual keyphrase
  assignment impractical. As a consequence, automatic keyphrase indexing of
  single documents recently attracts a lot of attention and many different
  methods were proposed~\cite{medelyan2006kea++,hasan2014state_of_the_art}.

  % - definition of the automatic keyphrase indexing task
  The automatic keyphrase indexing task consists in automatically detecting the
  important concepts addressed by a document. Previous work devides the
  automatic keyphrase indexing into two categories: Automatic Keyphrase
  Extraction (AKE) and Automatic Keyphrase Assignment (AKA). AKE is the task of
  detecting important words or phrases \emph{within} a document. AKE methods can
  identify explicit important concepts within a document, but they fail to
  discover implicit ones. In opposition, AKA methods use external information
  and detect both explicit and implicit important concepts.

  % - address the gap between previous work on AKE and AKA
  Many AKE methods have been proposed, ranging from supervised to unsupervised
  methods. Supervised AKE methods recast AKE as a binary classification task,
  e.g.~\cite{witten1999kea}, whereas unsupervised AKE methods define importance
  ranking schemes to extract the best words and phrases of a document,
  e.g.~\cite{mihalcea2004textrank}. Unlike AKE, there is a few proposals for
  AKA. AKA methods are supervised methods relying on external
  resources~\cite{medelyan2006kea++} or complex
  classifiers~\cite{liu2011vocabularygap}. Intuitively, AKA is a much difficult
  task than AKE. However, AKA is as important as there is a sizeable amount of
  keyphrases that do not appear within the documents they are assigned
  to~\cite[\TODO{find a better ref.}]{bougouin2013topicrank}.
  \TODO{Better end}

  % - our propososal
  This paper proposes a new method to tackle the AKA problem for specific
  domains. Extending the AKE method TopicRank~\cite{bougouin2013topicrank}, we
  use a graph-based ranking algorithm hover a graph with multiple edge types.
  Modeling the relations between the words, and phrases, of a document
  (keyphrase candidates), their relations with domain-specific terminological
  entries and the relations between these terminological entries, we extract
  important keyphrase candidates, as well as terminological entries highlighted
  by the keyphrase candidates.
  \TODO{Better end}
