\section{Introduction}
\label{sec: introduction}
  % - definition of keyphrases
  % - motivations
  Keyphrases are words or phrases that give a synoptic picture of what is
  important within a document. Keyphrases are useful in many tasks such as
  document clustering~\cite{han2007webdocumentclustering} and document
  summarization~\cite{litvak2008graphbased}. However, documents do not always
  contain keyphrases and the daily flow of new documents make manual keyphrase
  assignment impractical. As a consequence, automatic keyphrase indexing of
  single documents recently attracts a lot of attention and many different
  methods were proposed~\cite{hasan2014state_of_the_art}.

  % - definition of the automatic keyphrase indexing task
  The automatic keyphrase indexing task consists in providing the important
  concepts addressed by a document. Previous work devides the automatic
  keyphrase indexing into two categories: Automatic Keyphrase Extraction (AKE)
  and Automatic Keyphrase Assignment (AKA). AKE is the task of detecting
  important words or phrases \emph{within} a document. While AKE methods can
  detect explicit important concepts within a document, they fail to provide
  implicit ones. In opposition to AKE methods, AKA methods use external
  information and provide both explicit and implicit important concepts.

  % - address the gap between previous work on AKE and AKA
  Many AKE methods have been proposed, ranging from supervised to unsupervised
  methods. Supervised AKE methods recast AKE as a binary classification task,
  e.g.~\cite{witten1999kea}, whereas unsupervised AKE methods define importance
  ranking schemes to extract the best words and phrases of a document,
  e.g.~\cite{mihalcea2004textrank}. Unlike AKE, there is a few proposals for
  AKA. AKA methods are supervised methods relying on external
  resources~\cite{medelyan2006keathesaurus} or complex
  classifiers~\cite{liu2011vocabularygap}. Although AKA is a much difficult
  task, there is a sizeable amount of keyphrases that do not appears to the
  documents they must be assigned
  to~\cite[find a better ref.]{bougouin2013topicrank}. Hence, AKA is a critical
  task to solve keyphrase indexing.

  % - our propososal
  This paper proposes a new method to tackle the AKA problem for specific
  domains. Extending the graph-based AKE method
  TopicRank~\cite{bougouin2013topicrank}, we first build a bi-graph modeling
  both relations between words, and phrases, within a document (keyphrase
  candidates) and relations between terminological entries of the document's
  domain. Secondly, we rank keyphrase candidates and terminological entries
  using mutual reinforcement. At last, we extract important keyphrase candidates
  within the document, as well as terminological entries highlighted by some
  keyphrase candidates.
