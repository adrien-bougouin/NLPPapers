\section{TopicRank++}
\label{sec:topicrankpp}
  This section presents TopicRank++, a supervised extension of TopicRank that
  adds AKA. Topic\-Rank works in three steps:
  \begin{enumerate}
    \item{Clustering candidate keyphrases that belongs to the same topic.
          \newcite{bougouin2013topicrank} assumes that candidate of the same
          topic must share as many words as possible. They uses a Hierarchical
          Agglomerative Clustering (HAC) with a ``naive'' stem overlap
          similarity: at the biginning, each candidate is a single cluster and
          candidates sharing an average of $\unitfrac{1}{4}$ stemmed words with
          the candidates of a given cluster are added to this cluster.}
    \item{Building a graph of topics and ranking the topics using
          TextRank~\cite{mihalcea2004textrank}. Each topic is connected to the
          other topics by edges weighted according to the semantic strength
          between the connected topics. TextRank ranking algorithm recursively
          gives high importance to topics strongly connected to as most
          important topic as possible.}
    \item{Extracting keyphrases among the candidates of the $N$ most important
          topics. To avoid topic redundancy, TopicRank only extracts one
          keyphrase per topic. Following previous
          observations~\cite{witten1999kea}, \newcite{bougouin2013topicrank}
          extract the first occurring candidate from each topic.}
  \end{enumerate}

  To fit our need, TopicRank++ has a different graph construction. We also
  replace the classic TextRank ranking algorithm by a co-ranking variant and
  we integrate AKA during the former AKE process.

  \subsection{Graph construction}
  \label{subsec:graph_construction}
    TopicRank++ operates over a unified graph that connects two graphs
    representing the domain's reference keyphrases, the document's topics and
    the relations between them. Formally, let $G = (V, E)$ denote the unified
    graph. Reference keyphrases and topics are vertices $V$, respectively $V_k$
    and $V_t$, connected to their equals by edges
    $E_\textnormal{\textit{intra}}$ and connected to the other vertices by edges
    $E_\textnormal{\textit{outer}}$ (see Figure~\ref{fig:topicrankpp_graph}).

    \begin{figure*}
      \newcommand{\xslant}{0.25}
      \newcommand{\yslant}{0}

      \centering
      \begin{tikzpicture}[transform shape, scale=.66]
        % frame
        \node [draw,
               rectangle,
               minimum width=.7\linewidth,
               minimum height=8em,
               xslant=\xslant,
               yslant=\yslant] (domain_graph) {};
        \node [above=of domain_graph,
               xshift=.36\linewidth,
               yshift=8em,
               anchor=south east] (domain_graph_label) {domain keyphrases};

        \node [draw,
               circle,
               above=of domain_graph,
               xshift=.3\linewidth,
             yshift=5em] (domain_node1) {$V_{k_1}$};
        \node [draw,
               circle,
               above=of domain_graph,
               xshift=-.3\linewidth,
               yshift=5em] (domain_node2) {$V_{k_2}$};
        \node [draw,
               circle,
               above=of domain_graph,
               yshift=5em] (domain_node3) {$V_{k_3}$};
        \node [draw,
               circle,
               above=of domain_graph,
               xshift=.15\linewidth,
               yshift=.75em] (domain_node4) {$V_{k_4}$};
        \node [draw,
               circle,
               above=of domain_graph,
               xshift=-.15\linewidth,
               yshift=.75em] (domain_node5) {$V_{k_5}$};

        \draw [<->] (domain_node1) -- (domain_node3);
        \draw [<->] (domain_node2) -- (domain_node3);
        \draw [<->] (domain_node2) -- (domain_node4);
        \draw [<->] (domain_node4) -- (domain_node5);
        \draw [<->] (domain_node4) -- (domain_node3);

        % document
        \node [draw,
               rectangle,
               minimum width=.7\linewidth,
               minimum height=8em,
               xslant=\xslant,
               yslant=\yslant,
               above=of domain_graph,
               xshift=-2em] (document_graph) {};
        \node [below=of document_graph,
               xshift=-.36\linewidth,
               yshift=-8em,
               anchor=north west] (document_graph_label) {document topics};

        \node [draw,
               regular polygon,
               regular polygon sides=8,
               below=of document_graph,
               xshift=.3\linewidth,
               yshift=-5em] (document_node1) {$V_{t_1}$};
        \node [draw,
               regular polygon,
               regular polygon sides=8,
               below=of document_graph,
               xshift=-.3\linewidth,
               yshift=-5em] (document_node2) {$V_{t_2}$};
        \node [draw,
               regular polygon,
               regular polygon sides=8,
               below=of document_graph,
             yshift=-5em] (document_node3) {$V_{t_3}$};
        \node [draw,
               regular polygon,
               regular polygon sides=8,
               below=of document_graph,
               xshift=.15\linewidth,
               yshift=-.75em] (document_node4) {$V_{t_4}$};
        \node [draw,
               regular polygon,
               regular polygon sides=8,
               below=of document_graph,
               xshift=-.1\linewidth,
               yshift=-.75em] (document_node5) {$V_{t_5}$};
        \node [draw,
               regular polygon,
               regular polygon sides=8,
               below=of document_graph,
               yshift=-.75em] (document_node6) {$V_{t_6}$};
        \node [draw,
               regular polygon,
               regular polygon sides=8,
               below=of document_graph,
               xshift=-.175\linewidth,
               yshift=-5em] (document_node7) {$V_{t_7}$};

        \draw [<->] (document_node2) -- (document_node7);
        \draw [<->] (document_node2) -- (document_node5);
        \draw [<->] (document_node7) -- (document_node5);
        \draw [<->] (document_node7) -- (document_node3);
        \draw [<->] (document_node5) -- (document_node6);
        \draw [<->] (document_node3) -- (document_node1);
        \draw [<->] (document_node1) -- (document_node4);
        \draw [<->] (document_node3) -- (document_node4);

        % extra link
        \draw [<->, dashed] (document_node2) -- (domain_node2);
        \draw [<->, dashed] (document_node6) -- (domain_node5);
        \draw [<->, dashed] (document_node6) -- (domain_node3);
        \draw [<->, dashed] (document_node4) -- (domain_node1);
        \draw [<->, dashed] (document_node3) -- (domain_node4);

        % legend
        \node [right=of document_graph, xshift=2em, yshift=-9.25em] (legend_title) {\underline{Legend:}};
        \node [below=of legend_title, xshift=-1em, yshift=2em] (begin_inner) {};
        \node [right=of begin_inner] (end_inner) {: $E_\textnormal{\textit{inner}}$};
        \node [below=of begin_inner, yshift=1.5em] (begin_outer) {};
        \node [right=of begin_outer] (end_outer) {: $E_\textnormal{\textit{outer}}$};

        \draw (legend_title.north  -| end_outer.east) rectangle (end_outer.south -| legend_title.west);

        \draw [<->] (begin_inner) -- (end_inner);
        \draw [<->, dashed] (begin_outer) -- (end_outer);
      \end{tikzpicture}
      \caption{Example of a unified graph constructed by TopicRank++ and its two
               kinds of edges: inner- and outer-graph edges
               \label{fig:topicrankpp_graph}}
    \end{figure*}

    We create edges $E_\textnormal{\textit{intra}}$ between two reference
    keyphrases or two topics when they co-occur, respectively, as reference
    keyphrases for a document of the domain or within a sentence of the
    document. Each edge $E_{\textnormal{\textit{intra}}_{ij}}$ is weighted by
    the normalized number of co-occurrences, $w_{ij}$ between the reference
    keyphrases $V_{k_i}$ and $V_{k_j}$ or the topics $V_{t_i}$ and $V_{t_j}$.

    To unify the two graphs, we look for reference keyphrase occurrences within
    the document. The domain keyphrases can be seen as a category map of the
    domain. We try to connect the document to its potential categories. An edge
    $E_{\textnormal{\textit{outer}}_{ij}}$ is created to connect a reference
    keyphrase $V_{k_i}$ and a topic $V_{t_j}$ if the reference keyphrase is a
    member of the topic, i.e.~a keyphrase candidate that belongs to the topic.
    To accept flexions, such as plural flexions, we perform the comparison with
    stems.

  \subsection{Graph-based co-ranking}
  \label{subsec:graph_based_co_ranking}
    This ranking leverages both the reference keyphrase relations and the topic
    relations. Based on TextRank's random walk~\cite{mihalcea2004textrank}, our
    co-ranking algorithm combines the recommendations within each graph in such
    a way that the random part is replaced by a recommendation comming from the
    other graph:
    \begin{tiny}
      \begin{align}
        S(V_{t_i}) &= (1 - \lambda) \sum_{E_{\text{outer}_{ji}}}{\frac{S(V_{k_j})}{\mathlarger\sum_{E_{\text{outer}_{jm}}}{1}}} + \lambda \sum_{E_{\text{inner}_{ij}}}{\frac{w_{ij} S(V_{t_j})}{\mathlarger\sum_{E_{\text{inner}_{jm}}}{{w_{jm}}}}}\\
        S(V_{k_i}) &= (1 - \lambda) \sum_{E_{\text{outer}_{ij}}}{\frac{S(V_{t_j})}{\mathlarger\sum_{E_{\text{outer}_{mj}}}{1}}} + \lambda \sum_{E_{\text{inner}_{ij}}}{\frac{w_{ij} S(V_{k_j})}{\mathlarger\sum_{E_{\text{inner}_{jm}}}{{w_{jm}}}}}
      \end{align}
    \end{tiny}
    This formula embeds two assumptions:
    \begin{enumerate}
      \item{A reference keyphrase is important if it is strongly linked to other
            reference keyphrases and a topic is important if it strongly linked
            to other topics.}
      \item{A reference keyphrase is important in the context of the document if
            it is related to the topics within the document and a topic is
            important if it relates to domain keyphrases.}
    \end{enumerate}
    The first assumption is the former TextRank assumption, while the second is
    induced by the unification of both domain keyphrases and document topics.

    The $\lambda$ factor helps to configure the influence of the domain
    keyphrases over the whole ranking (in percentage). A lower $\lambda$ gives a
    higher influence of the domain keyphrases than a higher $\lambda$, while
    $\lambda=0.5$ balance the influence of the domain ranking and the document
    ranking.

  \subsection{Keyphrase assignment and extraction}
  \label{subsec:keyphrase_assignment_and_extraction}
    To both assign and extract keyphrases, we first sort the reference
    keyphrases and the document's topics using their importance score obtained
    with the co-ranking, then we retain the top $N$ ones.

    We assign reference keyphrases over one condition. A reference keyphrase can
    be assigned to a document if it is directly or transitively connected to a
    topic of the document. In other words, if the ranking of a reference
    keyphrase has not been influenced by the document, we do not consider this
    reference keyphrase.

    We also change the former strategy of the keyphrase extraction from the best
    ranking topics. We follow \newcite{bougouin2013topicrank} and extract the
    first occuring candidates as keyphrases, but we leverage the training data
    and prioritize the reference keyphrases within the topics. If reference
    keyphrases are included in the topic, and if they have not been assigned
    yet, we extract the first occurring one. If a top ranking topic does not
    contain one or more reference keyphrase, we extract the first occurring
    candidate.

  \subsection{Extension points}
  \label{subsec:extension_points}
    This section discusses possible customizations of TopicRank++. In
    section~\ref{subsec:graph_based_co_ranking}, we suggest to optimize the
    value of the $\lambda$ factor to set the appropriate influence of the domain
    over the document and, inversely, of the document over the domain. In
    section~\ref{subsec:keyphrase_assignment_and_extraction}, we present a pure
    mixin of AKA and AKE based on the co-ranking. This last step can be
    customized to better fit data caracteristics.

    In section~\ref{subsec:keyphrase_assignment_and_extraction}, we assume AKA
    and AKE to have the same priority. This might not be true in every case.
    Furthermore, we can assume that AKA has a higher priority, because it
    provides keyphrases that have already been judge to be keyphrases.
    Therefore, TopicRank++ could leverage a definition of the ratio of keyphrase
    to assign.

    Controlling the degree of generalization of an assigned keyphrase is also an
    interesting customization. Indeed, one may be interested in assigning very
    specific keyphrases regarding the document(\TODO{example}), whereas one may
    want to assign more general keyphrases (\TODO{example}). The minimum deph
    between a reference keyphrase and a document topic encodes this degree of
    generalization. A reference keyphrase directly connected to a document topic
    has the lowest generalization regarding the document, followed by the
    reference keyphrases they are connected to, and so on.

