\section{Candidate Selection}
\label{sec:candidate_selection}
    Previous work commonly selects candidates of either one of the following types:
    \begin{itemize}
        \item{\textbf{N-gram:} Ordered sequence of $n$ word(s), with $n$ usually set to one up to three (see Property~\ref{prop:informativity}).
              To avoid selecting an irrelevant keyphrase candidate, an n-gram containing a stopword is pruned~\cite{witten1999kea}.}
          
        \item{\textbf{POS sequence:} Word or phrase matching a given POS tag pattern. A simple, yet efficient POS pattern represents a longest sequence of nouns and adjectives~\cite{bougouin2013topicrank}.}
        \item{\textbf{NP-Chunk:} Non-recursive (minimal) noun-phrase (NP). An
              NP-chunk can be detected by pattern matching. In our experiments, we use the following POS patterns for English and French, respectively:}
            \begin{itemize}
                \item{\texttt{Np+|(A+~Nc)|Nc+}}
                \item{\texttt{Np+|(A?~Nc~A+)|(A~Nc)|Nc+}}
            \end{itemize}
    \end{itemize}
  
    These candidates are not fully consistent with properties inferred in Section~\ref{sec:keyphrase_properties}.
    Selecting n-grams respects Property~\ref{prop:informativity}, but produces candidates that contain verbs, adverbs and other words inconsistent with Property~\ref{prop:noun_phrases}.
    Conversely, selecting the longest sequences of nouns and adjectives (longest NPs) satisfies Property~\ref{prop:noun_phrases} but not Property~\ref{prop:informativity} as the length of the selected candidates is not controlled.
    Selecting NP-chunks is consistent with both properties, but it does not consider the nature of the candidate modifiers.
    We thus propose to refine this latter selection method by filtering out irrelevant modifiers.
  