\section{Conclusion}
\label{sec:conclusion}
    This work investigated the properties of human-assigned keyphrases and proposed a candidate selection method that best fit these properties.
    We performed both intrinsic and extrinsic experiments on three standard datasets varying in terms of domain, language and document length.
    %Results showed that filtering out irrelevant modifiers and prioritizing certain linguistic properties within keyphrase candidates outperforms commonly used candidate selection methods.
    Results showed that filtering out irrelevant modifiers and prioritizing certain linguistic properties within keyphrase candidates outperforms commonly used methods.
    
    This work opens new perspectives for automatic keyphrase extraction.
    Although we know that commonly used candidate selection methods produce acceptable keyphrase candidates, we demonstrated that future work would benefit from using candidates refined according to their words' nature.
    We introduced the usage of the relational property as a discriminative feature for candidate selection, but it could also be used as a feature for supervised classification of keyphrase candidates extracted by common techniques (e.g. with KEA).
