\section{Experiments}
\label{sec:experiments}
    We validate the effectiveness of our proposed candidate selection method by using two series of experiments.
    First, we provide a qualitative evaluation of the keyphrase candidates produced by our method and perform a comparison with the other methods.
    Second, we conduct an end-to-end evaluation by exploiting two keyphrase extraction systems.
    
    \subsection{Experimental settings}
    \label{subsec:experimental_settings}
        To quantify the capacity of the candidate selection methods to provide suitable candidates and avoid irrelevant ones, we compute the number of selected candidates (Cand./Doc.) and confront it with the best possible performance (maximum recall~--~R$_{\text{max}}$).
        To do so, we compute a quality ratio (QR):
        \begin{align}
            \text{QR} &= \frac{\text{R$_{\text{max}}$}}{\text{Cand./Doc.}} \times 100
        \end{align}
        The higher is QR, the better.

        \begin{table*}
            \centering
            \begin{tabular}{r|ccc|ccc|ccc}
                \toprule
                \multirow{2}{*}[-2pt]{\textbf{Method}} & \multicolumn{3}{c|}{\textbf{DUC} (\textit{English})} & \multicolumn{3}{c|}{\textbf{SemEval} (\textit{English})} & \multicolumn{3}{c}{\textbf{DEFT} (\textit{French})}\\
                \cline{2-10}
                & Cand./Doc. & R$_{\text{max}}$ & QR & Cand./Doc. & R$_{\text{max}}$ & QR & Cand./Doc. & R$_{\text{max}}$ & QR\\
                \hline
                \{1..3\}-grams & $~~~$596.2 & \textbf{90.8} & 15.2 & 2580.5 & \textbf{72.2} & $~~$2.8 & 4070.2 & \textbf{74.1} & $~~~$1.8\\
                Longest NPs & $~~~$155.6 & 88.7 & 57.0 & $~~~$646.5 & 62.4 & $~~$9.7 & $~~~$914.5 & 61.1 & $~~$6.7\\
                NP-chunks & $~~~$149.9 & 76.0 & 50.7 & $~~~$598.4 & 56.6 & $~~$9.5 & $~~~$812.3 & 63.0 & $~~$7.8\\
                LR-NPs & \textbf{$~~~$143.8} & 85.3 & \textbf{59.3} & \textbf{$~~~$538.2} & 59.4 & \textbf{11.0} & \textbf{$~~~$738.2} & 60.1 & \textbf{$~~$8.1}\\
                \bottomrule
            \end{tabular}
            \caption{Qualitative comparison of the keyphrase candidate selection methods
                     \label{tab:candidate_extraction_statistics}}
        \end{table*}

        \begin{table*}
            \centering
            \resizebox{\linewidth}{!}{
            \begin{tabular}{r@{~}|c@{~~}c@{~~}c@{~}|@{~}c@{~~}c@{~~}c@{~}|@{~}c@{~~}c@{~~}c@{~}|@{~}c@{~~}c@{~~}c@{~}|@{~}c@{~~}c@{~~}c@{~}|@{~}c@{~~}c@{~~}c}
                \toprule
                \multirow{2}{*}[-2pt]{\textbf{Method}} & \multicolumn{6}{c@{~}|@{~}}{\textbf{DUC} (\textit{English})} & \multicolumn{6}{c@{~}|@{~}}{\textbf{SemEval} (\textit{English})} & \multicolumn{6}{c}{\textbf{DEFT} (\textit{French})}\\
                \cline{2-19}
                & \multicolumn{3}{c@{~}|@{~}}{TF-IDF} & \multicolumn{3}{c@{~}|@{~}}{KEA} & \multicolumn{3}{c@{~}|@{~}}{TF-IDF} & \multicolumn{3}{c@{~}|@{~}}{KEA} & \multicolumn{3}{c@{~}|@{~}}{TF-IDF} & \multicolumn{3}{c}{KEA}\\
                \cline{2-19}
                & P & R & F & P & R & F & P & R & F & P & R & F & P & R & F & P & R & F\\
                \hline
                \{1..3\}-grams & 14.3 & 19.0 & 16.1$~~$ & 12.0 & 16.6 & 13.7$~~$ & $~~$9.0 & $~~$6.6 & $~~$7.2$~~$ & 19.4 & 13.7 & 15.9 & $~~$6.7 & 12.5 & $~~$8.6 & 13.4 & 25.3 & 17.3\\
                Longest NPs & 24.2 & 31.7 & 27.0$~~$ & \textbf{14.5} & 19.9 & 16.5$~~$ & 11.7 & $~~$7.9 & $~~$9.3$~~$ & 19.6 & 13.7 & 16.0 & $~~$9.5 & 17.6 & 12.1 & 14.1 & 26.3  &18.1\\
                NP-chunks & 21.1 & 28.1 & 23.8$~~$ & 13.5 & 18.6 & 15.4$~~$ & 11.9 & $~~$8.0 & $~~$9.5$~~$ & 19.5 & 13.7 & 16.0 & $~~$9.6 & 17.9 & 12.3 & 14.3 & 26.8 & 18.4\\
                LR-NPs & \textbf{24.3} & \textbf{32.0} & \textbf{27.2$^\dagger$} & \textbf{14.5} & \textbf{20.0} & \textbf{16.6$^\ddagger$} & \textbf{12.4} & \textbf{$~~$8.4} & \textbf{$~~$9.9$^\ddagger$} & \textbf{20.4} & \textbf{14.4} & \textbf{16.7}& \textbf{10.1} & \textbf{18.5} & \textbf{12.9} & \textbf{14.4} & \textbf{27.0} & \textbf{18.6}\\
                \bottomrule
            \end{tabular}
            }
            \caption{Comparison of TF-IDF and KEA applied on top of different candidate selection methods.
            $\ddagger$ indicates a significant improvement overall candidate sets and $\dagger$ indicates a significant improvement overall candidate sets but the longest NPs at 0.001 level using Student's t-test.
                     \label{tab:keyphrase_extraction_results}}
        \end{table*}
        
        We measure the impact of each candidate selection method on the keyphrase extraction task using the two following unsupervised and supervised keyphrase extraction methods:
        \begin{itemize}
            \item{\textbf{TF-IDF~\cite{jones1972tfidf}:} Word significancy weighting scheme.
                  Words are weighted based on their frequency in the document and the inverse number of documents in which they appear (specificity);
                  Candidates are weighted using the sum of  their words' score.}
            \item{\textbf{KEA~\cite{witten1999kea}:} Naive Bayes classifier trained on two features: the TF-IDF\footnote{KEA computes a TF-IDF based on candidate frequency, whereas our TF-IDF baseline relies on word frequency. KEA's TF-IDF is more efficient on larger documents than smaller ones.} and the first position of each candidate selected within train documents.}
        \end{itemize}
        We report the performance of TF-IDF and KEA in terms of precision~(P), 
        recall~(R) and F1-measure (F) at the top 10 keyphrases.
        Candidate and reference keyphrases are stemmed to reduce the number 
        of mismatches.
    
    \subsection{Candidate selection evaluation}
    \label{subsec:candidate_extraction_evaluation}
        Table~\ref{tab:candidate_extraction_statistics} presents the results of the intrinsic evaluation of the candidate selection methods.
        %Unsurprisingly, the best maximum recall is achieved by the $\{1..3\}$-grams selection method, but at the cost of a huge number of unrelevant candidates as indicated by the low quality ratio.
        Unsurprisingly, the best maximum recall is achieved when selecting $\{1..3\}$-grams, but at the cost of a huge number of irrelevant candidates as indicated by the low QR.
        Among the other selection methods, our method shows a competitive maximum recall while reducing the number of candidates.
        As a consequence, the LR-NPs quality outperforms other selected candidates quality, which is crucial as it directly affects the performance and time complexity of keyphrase extraction methods~\cite{wang2014keyphraseextractionpreprocessing}.
        
        \TODO{examples of true and false positives}
    
    \subsection{Keyphrase extraction evaluation}
    \label{subsec:keyphrase_extraction_evaluation}
        Table~\ref{tab:keyphrase_extraction_results} shows the results of the extrinsic evaluation of the candidate selection methods.
        Overall, we observe that the performance of TF-IDF and KEA is closely correlated with the quality of the set of selected candidates.
        Best results are then obtained when TF-IDF and KEA are applied on LR-NPs (half of them are significantly better).
        Thus, although our proposed selection method does not achieve the best maximum recall, it still outperforms the other candidate selection  methods.
        Comparing longest NPs, NP-chunks and LR-NPs demonstrates that it is efficient to use heuristic based on linguistic properties.
