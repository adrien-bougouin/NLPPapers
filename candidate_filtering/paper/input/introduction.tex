\section{Introduction}
\label{sec:section}
    Keyphrases are words or phrases that represent the main content of a document.
    Similar to an abstract, keyphrases give a synoptic picture of what is important in the document.
    Disimilar to an abstract, keyphrases are small grain units and are useful resources for many Natural Language Processing tasks: document clustering~\cite{han2007webdocumentclustering}, information retrieval~\cite{medelyan2008smalltrainingset}, document summarization~\cite{litvak2008graphbased}, etc.
    However, documents do not always contain keyphrases.
    As the daily flow of new documents grows, manually annotating documents has become impractical.
    Hence automatic keyphrase extraction recently attracts a lot of attention and many different methods are proposed~\cite{hasan2014state_of_the_art}.

    Automatic keyphrase extraction is the task of detecting important words or phrases within a document.
    Generally speaking, we divide keyphrase extraction methods into two categories: supervised and unsupervised.
    Supervised methods treat keyphrase extraction as a binary classification task, e.g.~\cite{witten1999kea}.
    Conversely, unsupervised methods usually rank keyphrase candidates by importance and select the top-ranked ones as keyphrases, e.g.~\cite{mihalcea2004textrank}.

    Although they tackle the keyphrase extraction problem differently, both supervised and unsupervised methods rely on a candidate selection step.
    Keyphrase candidate selection identifies words or phrases consistent with human-assigned keyphrase properties.
    %Although keyphrase candidate selection starts to draw attention~\cite{wang2014keyphraseextractionpreprocessing}, keyphrase extraction methods use simple heuristics: selection of n-grams, sequences of nouns and adjectives, etc.
    However, current selection methods use simple heuristics~\cite{wang2014keyphraseextractionpreprocessing}: candidates are n-grams or sequences of nouns and adjectives.
    %This work infers linguistic properties from human-assigned keyphrases and demonstrates their applicability on keyphrase candidate selection.
    This work proposes rules based on a comprehensive analysis of modifiers within human-assigned keyphrases.
    We demonstrate their applicability on keyphrase candidate selection.
    
    This paper is organized as follows.
    Section~\ref{sec:keyphrase_properties} presents an analysis of human-assigned keyphrases.
    Section~\ref{sec:candidate_selection} describes common keyphrase candidate selection methods followed by a description of our method in Section~\ref{sec:proposed_candidate_selection_method}. Finally, Section~\ref{sec:experiments} presents the expriments and Section~\ref{sec:conclusion} concludes our work.
