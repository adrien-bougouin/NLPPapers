\section{Keyphrase Properties}
\label{sec:keyphrase_properties}
    In this section, we infer keyphrase properties from human-assigned keyphrases.
    We use three standard datasets: DUC~\cite{wan2008expandrank}, SemEval~\cite{kim2010semeval} and DEFT~\cite{paroubek2012deft}.
    These datasets differ in terms of language, nature and document size (see Table~\ref{tab:datasets}), which makes our inferred properties more general.
    We avoid evaluation bias and only infer properties from train documents, while test documents are used for evaluation purpose.
    \begin{table}[!h]
        \centering
        \begin{tabular}{@{}r@{~}|@{~}c@{~~}c@{~~}c@{}}
            \toprule
            & \textbf{DUC} & \textbf{SemEval} & \textbf{DEFT}\\
            \hline
            Language & English & English & French\\
            Nature & Journalistic & Scientific & Scientific\\
            Train documents & 208 & 144 & 141\\
            Test documents & 100 & 100 & ~~93\\
            \bottomrule
        \end{tabular}
        \caption{Caracteristics of DUC, SemEval and DEFT
                 \label{tab:datasets}}
    \end{table}

    Table~\ref{tab:train_dataset_statistics} shows statistics about the train documents.
    %Reference keyphrases were automatically Part-of-Speech (POS) tagged\footnote{We use the Stanford POS tagger~\cite{toutanova2003stanfordpostagger} for English and MElt~\cite{denis2009melt} for French.} and manually reviewed for consistency.
    Reference keyphrases were automatically Part-of-Speech (POS) tagged and manually reviewed for consistency.
    The bottom part of the table presents the percentage of multi-word keyphrases that contain a certain POS.
    We do not show single-word keyphrase statistics as they are mostly nouns.
    \vspace{-.25em}
    \begin{table}[!h]
        \centering
            \begin{tabular}{@{}r@{~}|@{~}c@{~~}c@{~~}c@{}}
                \toprule
                ~ & \textbf{DUC} & \textbf{SemEval} & \textbf{DEFT}\\
                \hline
                \multicolumn{1}{@{}l@{~}|@{~}}{\textbf{Documents:}}\\
                Tokens/doc. & 912.0 & 5134.6 & 7276.7\\
                Keyphrases/doc. & 8.1 & 15.4 & 5.4\\
                Maximum recall (\%) & 96.1 & 86.5 & 81.8\\
                \hline
                \multicolumn{1}{@{}l@{~}|@{~}}{\textbf{Keyphrases:}}\\
                Unigrams (\%) & 17.1 & 20.2 & 60.2\\
                Bigrams (\%) & 60.8 & 53.4 & 24.5\\
                Trigrams (\%) & 17.8 & 21.3 & ~~8.8\\
                \hline
                \multicolumn{1}{@{}l@{~}|@{~}}{\textbf{Multi-word keyphrases}}\\
                \multicolumn{1}{@{}l@{~}|@{~}}{\textbf{with:}\hfill{}Noun (\%)} & 94.5 & 98.7 & 93.1\\
                Proper noun (\%) & 17.1 & ~~4.3 & ~~6.9\\
                Attributive adj. (\%) & 24.2 & 29.1 & ~~8.6\\
                Relational adj. (\%) & 28.9 & 24.1 & 57.6\\
                Prep. (\%) & ~~0.3 & ~~1.5 & 31.2\\
                Det. (\%) & ~~0.0 & ~~0.0 & 20.4\\
                \bottomrule
            \end{tabular}
        \caption{Statistics of the train documents.
                 The maximum recall represents the percentage of keyphrases that can be extracted from the documents.
                 \label{tab:train_dataset_statistics}}
    \end{table}
    \vspace{-1.5em}
    
    First, we observe, as noted in previous work, that most keyphrases are small-sized textual units.
    \begin{property}\label{prop:informativity}
      Keyphrases are small-sized textual units, usually containing up to three words (e.g.~``storms'', ``hurricane expert'' or ``annual hurricane forecast'').
    \end{property}

    Second, we observe that most keyphrases contain a noun and more than half of them are modified by an adjective.
    Most importantly, among these adjectives, there is a larger number of relational adjectives.
    This is also confirmed by the presence of relational adjectives in the most frequent POS tag patterns of keyphrases, as shown in Table~\ref{tab:best_patterns}.
    \begin{property}\label{prop:noun_phrases}
      Keyphrases are noun sequences (e.g.~``storms'') modified or not, most likely by a relational adjective (e.g.~``annual hurricane forecast'').
    \end{property}
    \vspace{-.5em}
    \begin{table}[!h]
        \centering
        \begin{tabular}{@{}r@{~}|@{~}l@{~}l@{~}l@{~}ll@{}}
            \toprule
            \multicolumn{1}{r}{} & \multicolumn{4}{@{}l}{\textbf{Pattern}} & \textbf{Example}\\
            \midrule
            \multirow{3}{*}{\begin{sideways}\textbf{English}\end{sideways}}
            & \texttt{Nc} & \texttt{Nc} & & & \textit{``hurricane expert''}\\ % AP880409-0015
            & \texttt{Nc} & & & & \textit{``storms''}\\ % AP880409-0015
            & \texttt{rA} & \texttt{Nc} & & & \textit{``Chinese earthquake''}\\ % AP890228-0019
            \hline
            \multirow{3}{*}{\begin{sideways}\textbf{French}\end{sideways}}
            & \texttt{Nc} & & & & \textit{``patrimoine'' (``cultural heritage'')}\\ % as_2002_007048ar
            & \texttt{Nc} & \texttt{rA} & & & \textit{``tradition orale'' (``oral tradition'')}\\ % as_2002_007048ar
            & \texttt{Np} & & & & \textit{``Indonésie'' (``Indonesia'')}\\ % as_2001_000235ar
            \bottomrule
      \end{tabular}
      \caption{Most frequent patterns -- Multex format~\cite{ide1994multext}.
               \texttt{rA} stands for \textit{relational adjective}.
               \label{tab:best_patterns}}
    \end{table}
    \vspace{-.5em}
    
    Unlike most attributive adjectives, relational adjectives express a relation with a noun (e.g.~``cultural'' is derived from ``culture'').
    Due to this denominal property, relational adjectives are highly used as taxonomic classes (e.g. the Wikipedia category \textit{cultural heritage}), which can be seen as higher-level keyphrases~\cite{fatima2011automaticdocumentannotation}.
    \TODO{They are also used in other tasks such as term extraction~\cite{daille2001relationaladjectives}.}
    
    Although Property~\ref{prop:informativity} is well-known in automatic keyphrase extraction, Property~\ref{prop:noun_phrases} has only been partially covered in previous work.
    Indeed, keyphrases are known to be mostly noun phrases.
    However, no comprehensive analysis of the keyphrase modifiers has been conducted so far, showing how relational and attributive adjectives are used in keyphrases.
    We use these findings to devise a new method that filters out irrelevant modifiers from keyphrase candidates.
