%
% File naaclhlt2015.tex
%

\documentclass[11pt,letterpaper]{article}
\usepackage{naaclhlt2015}
\usepackage{times}
\usepackage{latexsym}
%\setlength\titlebox{6.5cm} % Expanding the titlebox

\usepackage[utf8]{inputenc}
\usepackage{amsmath}
\usepackage{relsize}
\usepackage{xfrac}
\usepackage{booktabs}
\usepackage{multirow}
\usepackage{rotating}
\usepackage{lipsum}
\usepackage{color}
\usepackage{subfigure}
\usepackage{overpic}
\usepackage{fancybox}
\usepackage[ruled,vlined]{algorithm2e}
\usepackage{url}
\usepackage{graphicx}

\setlength{\textfloatsep}{0em}
\setlength{\floatsep}{0em}
\setlength{\dbltextfloatsep}{0em}
\setlength{\dblfloatsep}{0em}
\setlength{\abovecaptionskip}{.33em}
\setlength{\belowcaptionskip}{.44em}

\newtheorem{property}{Property}
\newcommand\TODO[1]{}%\textcolor{red}{[TODO #1]}}

\title{
    Linguistically Motivated Keyphrase Candidate Selection
    %Selecting Keyphrase Candidates using Relational Adjectives
    % \Thanks{
    %     The authors would like to thank the anonymous reviewers for their useful
    %     advice and comments. This work was supported by the French National Research
    %     Agency (TermITH project -- ANR-12-CORD-0029).
    % }
}

% \author{First Author \\
%   Affiliation \\
%   {\tt email@domain} \\\And
%   Second Author \\
%   Affiliation \\
%   {\tt email@domain} \\}

% \author{
%     Adrien Bougouin \and Florian Boudin \and Béatrice Daille\\
%     Université de Nantes, LINA, France\\
%     \normalsize\texttt{\{adrien.bougouin,florian.boudin,beatrice.daille\}@univ-nantes.fr
% }

\date{}

\begin{document}
    \maketitle
    
    \begin{abstract}
        % This paper analyses human-assigned keyphrases and identifies their linguistic properties.
        % We use these properties to propose a new method for selecting keyphrase candidates.
        % Through both intrinsic and extrinsic evaluations on three datasets, we show that a linguistically motivated selection of keyphrase candidates improves the keyphrase extraction performance.
        Keyphrase candidate selection is the task of producing a set of potential keyphrases.
        This work analyses human-assigned keyphrases and identifies their linguistic properties.
        We use these properties to select keyphrase candidates.
        Through both intrinsic and extrinsic comparisons with current candidate selection methods, we show that a
        linguistically motivated selection of keyphrase candidates improves the keyphrase extraction performance.
    \end{abstract}
    
    
    \section{Introduction}
\label{sec:introduction}
  % * définition de terme-clé, applications et enjeux
  Un terme-clé est un mot ou une expression polylexicale qui représente un
  concept important d'un document auquel il est associé. En pratique, plusieurs
  termes-clés représentant des concepts différents sont associés à un même
  document. Ils forment alors un ensemble de termes-clés à partir duquel il est
  possible de déduire le contenu principal du document. Du fait de leur capacité
  à synthétiser le contenu d'un document, les termes-clés sont utilisés dans
  diverses applications en Recherche d'Information (RI)~: résumé
  automatique~\cite{avanzo2005keyphrase}, classification de
  documents~\cite{han2007webdocumentclustering}, indexation
  automatique~\cite{medelyan2008smalltrainingset}, etc. Avec l'essor du
  numérique, de plus en plus de documents (articles scientifiques, articles
  journalistiques, etc.) sont accessibles depuis des médiums d'informations tels
  que Internet. Afin de permettre à un utilisateur de rapidement trouver des
  documents, ainsi que d'avoir un bref aperçu de leur contenu, les tâches
  sus-mentionnées sont nécessaires.
  Cependant, la majorité des documents ne sont pas associés avec des termes-clés
  et, compte tenu du nombre important de documents numériques, l'ajout manuel de
  ces derniers n'est pas envisageable. Pour pallier ce problème, de plus en plus
  de chercheurs s'intéressent à l'extraction automatique de termes-clés et
  certaines campagnes d'évaluations, telles que DEFT~\cite{paroubek2012deft} et
  SemEval~\cite{kim2010semeval}, proposent des tâches d'extraction automatique
  de termes-clés.

  % * qu'est-ce que l'extraction automatique de termes-clés
  % * deux écoles : indexation libre et indexation contrôlée (assignation de
  %                 termes-clés)
  %   -> nous sommes de la première école
  % * deux catégories de méthodes : supervisées et non-supervisées
  %    -> en supervisé ils utilisent la structure des documents
  %    -> très peu de travaux en non-supervisé (filtrage des candidats)
  L'extraction automatique de termes-clés, ou indexation libre, est la tâche qui
  consiste à extraire les unités textuelles les plus importantes d'un document,
  en opposition à la tâche d'assignation automatique de termes-clés, ou
  indexation contrôlée, qui consiste à assigner des termes-clés à partir d'une
  terminologie donnée~\cite{paroubek2012deft}. Parmi les méthodes d'extraction
  automatique de termes-clés existantes, nous distinguons deux catégories~: les
  méthodes supervisées et les méthodes non-supervisées. Dans le cas supervisé,
  la tâche d'extraction de termes-clés est considérée comme une tâche de
  classification binaire~\cite{witten1999kea}, où il s'agit d'attribuer la
  classe \og{}\textit{terme-clé}\fg{} ou \og{}\textit{non terme-clé}\fg{} aux
  termes-clés candidats extraits du document. Une collection de documents
  annotés en termes-clés est alors nécessaire pour l'apprentissage d'un modèle
  de classification reposant sur divers traits, allant de la simple fréquence
  aux informations structurelles du document (titre, résumé, introduction,
  conclusion, etc.). Dans le cas non-supervisé, les méthodes attribuent un
  score d'importance à chaque candidat en fonction de divers indicateurs tels
  que la fréquence et la position de la première occurrence dans le document.
  Bien que les méthodes supervisées soient en général plus performantes, la
  faible quantité de documents annotés en termes-clés disponibles, ainsi que la
  forte dépendance des modèles de classification au type des documents à partir
  desquels ils sont appris, poussent les chercheurs à s'intéresser de plus en
  plus aux méthodes non-supervisées.

  % * ici, on cherche à identifier l'échelle de difficulté d'indexation des
  %   documents en Sciences Humaines et Sociales (SHS)
  % * on dispose de 4 collections de notices de 4 disciplines différentes de
  %   SHS + 1 collection de notices de chimie (science dure)
  Dans cette article, nous nous intéressons à l'extraction non-supervisée de
  termes-clés dans les articles scientifiques, et plus particulièrement à la
  performance des méthodes d'extraction de termes-clés dans des domaines de
  spécialité. Au moyen de cinq corpus disciplinaires, notre objectif est
  d'observer et d'analyser l'échelle de difficulté pour l'extraction
  automatique de termes-clés dans des articles scientifiques appartenant à cinq
  disciplines différentes~: Archéologie, Sciences de l'Information,
  Linguistique, Psychologie et Chimie.
  \TODO{Dire pourquoi nous nous intéressons aux méthodes non-supervisées}
  \TODO{Dire pourquoi nous nous intéressons aux articles scientifiques}

  % * annonce du plan
  L'article est structuré comme suit. Un bref état de l'art est donné dans la
  section~\ref{sec:etat_de_l_art}, les données utilisées sont présentées dans la
  section~\ref{sec:presentation_des_donnees} et les expériences menées, ainsi
  que les résultats obtenus, sont décrits dans la section~\ref{sec:experiences}.
  Enfin, une analyse des résultats est donnée dans la
  section~\ref{sec:discussion}, puis une conclusion générale et des perspectives
  de travaux futurs sont présentés en
  section~\ref{sec:conclusion_et_perspectives}.


    \section{Keyphrase Properties}
\label{sec:keyphrase_properties}
    In this section, we infer keyphrase properties from human-assigned keyphrases.
    We use three standard datasets: DUC~\cite{wan2008expandrank}, SemEval~\cite{kim2010semeval} and DEFT~\cite{paroubek2012deft}.
    These datasets differ in terms of language, nature and document size (see Table~\ref{tab:datasets}), which makes our inferred properties more general.
    We avoid evaluation bias and only infer properties from train documents, while test documents are used for evaluation purpose.
    \begin{table}[!h]
        \centering
        \begin{tabular}{@{}r@{~}|@{~}c@{~~}c@{~~}c@{}}
            \toprule
            & \textbf{DUC} & \textbf{SemEval} & \textbf{DEFT}\\
            \hline
            Language & English & English & French\\
            Nature & Journalistic & Scientific & Scientific\\
            Train documents & 208 & 144 & 141\\
            Test documents & 100 & 100 & ~~93\\
            \bottomrule
        \end{tabular}
        \caption{Caracteristics of DUC, SemEval and DEFT
                 \label{tab:datasets}}
    \end{table}

    Table~\ref{tab:train_dataset_statistics} shows statistics about the train documents.
    %Reference keyphrases were automatically Part-of-Speech (POS) tagged\footnote{We use the Stanford POS tagger~\cite{toutanova2003stanfordpostagger} for English and MElt~\cite{denis2009melt} for French.} and manually reviewed for consistency.
    Reference keyphrases were automatically Part-of-Speech (POS) tagged and manually reviewed for consistency.
    The bottom part of the table presents the percentage of multi-word keyphrases that contain a certain POS.
    We do not show single-word keyphrase statistics as they are mostly nouns.
    \vspace{-.25em}
    \begin{table}[!h]
        \centering
            \begin{tabular}{@{}r@{~}|@{~}c@{~~}c@{~~}c@{}}
                \toprule
                ~ & \textbf{DUC} & \textbf{SemEval} & \textbf{DEFT}\\
                \hline
                \multicolumn{1}{@{}l@{~}|@{~}}{\textbf{Documents:}}\\
                Tokens/doc. & 912.0 & 5134.6 & 7276.7\\
                Keyphrases/doc. & 8.1 & 15.4 & 5.4\\
                Maximum recall (\%) & 96.1 & 86.5 & 81.8\\
                \hline
                \multicolumn{1}{@{}l@{~}|@{~}}{\textbf{Keyphrases:}}\\
                Unigrams (\%) & 17.1 & 20.2 & 60.2\\
                Bigrams (\%) & 60.8 & 53.4 & 24.5\\
                Trigrams (\%) & 17.8 & 21.3 & ~~8.8\\
                \hline
                \multicolumn{1}{@{}l@{~}|@{~}}{\textbf{Multi-word keyphrases}}\\
                \multicolumn{1}{@{}l@{~}|@{~}}{\textbf{with:}\hfill{}Noun (\%)} & 94.5 & 98.7 & 93.1\\
                Proper noun (\%) & 17.1 & ~~4.3 & ~~6.9\\
                Attributive adj. (\%) & 24.2 & 29.1 & ~~8.6\\
                Relational adj. (\%) & 28.9 & 24.1 & 57.6\\
                Prep. (\%) & ~~0.3 & ~~1.5 & 31.2\\
                Det. (\%) & ~~0.0 & ~~0.0 & 20.4\\
                \bottomrule
            \end{tabular}
        \caption{Statistics of the train documents.
                 The maximum recall represents the percentage of keyphrases that can be extracted from the documents.
                 \label{tab:train_dataset_statistics}}
    \end{table}
    \vspace{-1.5em}
    
    First, we observe, as noted in previous work, that most keyphrases are small-sized textual units.
    \begin{property}\label{prop:informativity}
      Keyphrases are small-sized textual units, usually containing up to three words (e.g.~``storms'', ``hurricane expert'' or ``annual hurricane forecast'').
    \end{property}

    Second, we observe that most keyphrases contain a noun and more than half of them are modified by an adjective.
    Most importantly, among these adjectives, there is a larger number of relational adjectives.
    This is also confirmed by the presence of relational adjectives in the most frequent POS tag patterns of keyphrases, as shown in Table~\ref{tab:best_patterns}.
    \begin{property}\label{prop:noun_phrases}
      Keyphrases are noun sequences (e.g.~``storms'') modified or not, most likely by a relational adjective (e.g.~``annual hurricane forecast'').
    \end{property}
    \vspace{-.5em}
    \begin{table}[!h]
        \centering
        \begin{tabular}{@{}r@{~}|@{~}l@{~}l@{~}l@{~}ll@{}}
            \toprule
            \multicolumn{1}{r}{} & \multicolumn{4}{@{}l}{\textbf{Pattern}} & \textbf{Example}\\
            \midrule
            \multirow{3}{*}{\begin{sideways}\textbf{English}\end{sideways}}
            & \texttt{Nc} & \texttt{Nc} & & & \textit{``hurricane expert''}\\ % AP880409-0015
            & \texttt{Nc} & & & & \textit{``storms''}\\ % AP880409-0015
            & \texttt{rA} & \texttt{Nc} & & & \textit{``Chinese earthquake''}\\ % AP890228-0019
            \hline
            \multirow{3}{*}{\begin{sideways}\textbf{French}\end{sideways}}
            & \texttt{Nc} & & & & \textit{``patrimoine'' (``cultural heritage'')}\\ % as_2002_007048ar
            & \texttt{Nc} & \texttt{rA} & & & \textit{``tradition orale'' (``oral tradition'')}\\ % as_2002_007048ar
            & \texttt{Np} & & & & \textit{``Indonésie'' (``Indonesia'')}\\ % as_2001_000235ar
            \bottomrule
      \end{tabular}
      \caption{Most frequent patterns -- Multex format~\cite{ide1994multext}.
               \texttt{rA} stands for \textit{relational adjective}.
               \label{tab:best_patterns}}
    \end{table}
    \vspace{-.5em}
    
    Unlike most attributive adjectives, relational adjectives express a relation with a noun (e.g.~``cultural'' is derived from ``culture'').
    Due to this denominal property, relational adjectives are highly used as taxonomic classes (e.g. the Wikipedia category \textit{cultural heritage}), which can be seen as higher-level keyphrases~\cite{fatima2011automaticdocumentannotation}.
    \TODO{They are also used in other tasks such as term extraction~\cite{daille2001relationaladjectives}.}
    
    Although Property~\ref{prop:informativity} is well-known in automatic keyphrase extraction, Property~\ref{prop:noun_phrases} has only been partially covered in previous work.
    Indeed, keyphrases are known to be mostly noun phrases.
    However, no comprehensive analysis of the keyphrase modifiers has been conducted so far, showing how relational and attributive adjectives are used in keyphrases.
    We use these findings to devise a new method that filters out irrelevant modifiers from keyphrase candidates.

    \section{Candidate Selection}
\label{sec:candidate_selection}
    Previous work commonly selects candidates of either one of the following types:
    \begin{itemize}
        \item{\textbf{N-gram:} Ordered sequence of $n$ word(s), with $n$ usually set to one up to three (see Property~\ref{prop:informativity}).
              To avoid selecting an irrelevant keyphrase candidate, an n-gram containing a stopword is pruned~\cite{witten1999kea}.}
          
        \item{\textbf{POS sequence:} Word or phrase matching a given POS tag pattern. A simple, yet efficient POS pattern represents a longest sequence of nouns and adjectives~\cite{bougouin2013topicrank}.}
        \item{\textbf{NP-Chunk:} Non-recursive (minimal) noun-phrase (NP). An
              NP-chunk can be detected by pattern matching. In our experiments, we use the following POS patterns for English and French, respectively:}
            \begin{itemize}
                \item{\texttt{Np+|(A+~Nc)|Nc+}}
                \item{\texttt{Np+|(A?~Nc~A+)|(A~Nc)|Nc+}}
            \end{itemize}
    \end{itemize}
  
    These candidates are not fully consistent with properties inferred in Section~\ref{sec:keyphrase_properties}.
    Selecting n-grams respects Property~\ref{prop:informativity}, but produces candidates that contain verbs, adverbs and other words inconsistent with Property~\ref{prop:noun_phrases}.
    Conversely, selecting the longest sequences of nouns and adjectives (longest NPs) satisfies Property~\ref{prop:noun_phrases} but not Property~\ref{prop:informativity} as the length of the selected candidates is not controlled.
    Selecting NP-chunks is consistent with both properties, but it does not consider the nature of the candidate modifiers.
    We thus propose to refine this latter selection method by filtering out irrelevant modifiers.
  
    \section{Linguistically Refined NPs (LR-NPs)}
\label{sec:proposed_candidate_selection_method}
    To get rid of unwanted modifiers within keyphrase candidates, we propose a set of heuristics drawn from the observations presented in Section~\ref{sec:keyphrase_properties}.
    
    First, we use more specific POS tag patterns to select candidates: \texttt{A?\,(Nc|Np)+} for English and \texttt{(Nc|Np)+\,A?} for French.
    These patterns ensure more consistency with Property~\ref{prop:informativity} and, in the case of French, filter out some attributive adjectives (left-side adjectives).
    
    Second, we further refine the selected keyphrase candidates by pruning modifiers according to the following rules.
    As relational adjectives often appear in reference keyphrases, we consider that they are always relevant modifiers and must be preserved in all candidates.
    We also accept every single-word compound modifier, such as an hyphenated modifier (e.g.~``graph-based'', ``data-driven'', ect.).
    Indeed, similar to relational adjectives, single-word compound modifiers introduce a relation with a noun.
    
    Other modifiers are in turn removed from the keyphrase candidates, except when they bring \textit{useful} information.
    To assess whether a modifier is useful or not, we simply compare the number of occurrences of a candidate with and without the modifier.
    If a keyphrase candidate occurs more often unmodified, we prune the candidate from the modifier (e.g.~``\textit{large storms}'' becomes ``\textit{storms}'').
    
    To detect relational adjectives, we use external resources: WordNet~\cite{miller1995wordnet} for English and WoNeF~\cite{pradet2013wonef} for French.
    %We consider an adjective to be relational if it has a pertainym in those resources, or if its suffix matches one of the known suffixes of relational adjectives\footnote{This simple detection technique requires few resources and is easy to adapt for new languages.}: ``al'', ``ant'', ``ary'', ``ic'', ``ous''  or ``ive'' for English~\cite{grabar2006terminologystructuring}; ``ain'', ``aire'', ``al'', ``el'', ``eux'', ``ien'', ``ier'', ``ique'' or ``ois'' for French~\cite{harastani2013relationaladjectivetranslation}.
    We consider an adjective to be relational if it has a pertainym in those resources, or if its suffix matches one of the known suffixes of relational adjectives: ``al'', ``ant'', ``ary'', ``ic'', ``ous''  or ``ive'' for English~\cite{grabar2006terminologystructuring}; ``ain'', ``aire'', ``al'', ``el'', ``eux'', ``ien'', ``ier'', ``ique'' or ``ois'' for French~\cite{harastani2013relationaladjectivetranslation}.

    \section{Experiments}
\label{sec:experiments}
    We validate the effectiveness of our proposed candidate selection method by using two series of experiments.
    First, we provide a qualitative evaluation of the keyphrase candidates produced by our method and perform a comparison with the other methods.
    Second, we conduct an end-to-end evaluation by exploiting two keyphrase extraction systems.
    
    \subsection{Experimental settings}
    \label{subsec:experimental_settings}
        To quantify the capacity of the candidate selection methods to provide suitable candidates and avoid irrelevant ones, we compute the number of selected candidates (Cand./Doc.) and confront it with the best possible performance (maximum recall~--~R$_{\text{max}}$).
        To do so, we compute a quality ratio (QR):
        \begin{align}
            \text{QR} &= \frac{\text{R$_{\text{max}}$}}{\text{Cand./Doc.}} \times 100
        \end{align}
        The higher is QR, the better.

        \begin{table*}
            \centering
            \begin{tabular}{r|ccc|ccc|ccc}
                \toprule
                \multirow{2}{*}[-2pt]{\textbf{Method}} & \multicolumn{3}{c|}{\textbf{DUC} (\textit{English})} & \multicolumn{3}{c|}{\textbf{SemEval} (\textit{English})} & \multicolumn{3}{c}{\textbf{DEFT} (\textit{French})}\\
                \cline{2-10}
                & Cand./Doc. & R$_{\text{max}}$ & QR & Cand./Doc. & R$_{\text{max}}$ & QR & Cand./Doc. & R$_{\text{max}}$ & QR\\
                \hline
                \{1..3\}-grams & $~~~$596.2 & \textbf{90.8} & 15.2 & 2580.5 & \textbf{72.2} & $~~$2.8 & 4070.2 & \textbf{74.1} & $~~~$1.8\\
                Longest NPs & $~~~$155.6 & 88.7 & 57.0 & $~~~$646.5 & 62.4 & $~~$9.7 & $~~~$914.5 & 61.1 & $~~$6.7\\
                NP-chunks & $~~~$149.9 & 76.0 & 50.7 & $~~~$598.4 & 56.6 & $~~$9.5 & $~~~$812.3 & 63.0 & $~~$7.8\\
                LR-NPs & \textbf{$~~~$143.8} & 85.3 & \textbf{59.3} & \textbf{$~~~$538.2} & 59.4 & \textbf{11.0} & \textbf{$~~~$738.2} & 60.1 & \textbf{$~~$8.1}\\
                \bottomrule
            \end{tabular}
            \caption{Qualitative comparison of the keyphrase candidate selection methods
                     \label{tab:candidate_extraction_statistics}}
        \end{table*}

        \begin{table*}
            \centering
            \resizebox{\linewidth}{!}{
            \begin{tabular}{r@{~}|c@{~~}c@{~~}c@{~}|@{~}c@{~~}c@{~~}c@{~}|@{~}c@{~~}c@{~~}c@{~}|@{~}c@{~~}c@{~~}c@{~}|@{~}c@{~~}c@{~~}c@{~}|@{~}c@{~~}c@{~~}c}
                \toprule
                \multirow{2}{*}[-2pt]{\textbf{Method}} & \multicolumn{6}{c@{~}|@{~}}{\textbf{DUC} (\textit{English})} & \multicolumn{6}{c@{~}|@{~}}{\textbf{SemEval} (\textit{English})} & \multicolumn{6}{c}{\textbf{DEFT} (\textit{French})}\\
                \cline{2-19}
                & \multicolumn{3}{c@{~}|@{~}}{TF-IDF} & \multicolumn{3}{c@{~}|@{~}}{KEA} & \multicolumn{3}{c@{~}|@{~}}{TF-IDF} & \multicolumn{3}{c@{~}|@{~}}{KEA} & \multicolumn{3}{c@{~}|@{~}}{TF-IDF} & \multicolumn{3}{c}{KEA}\\
                \cline{2-19}
                & P & R & F & P & R & F & P & R & F & P & R & F & P & R & F & P & R & F\\
                \hline
                \{1..3\}-grams & 14.3 & 19.0 & 16.1$~~$ & 12.0 & 16.6 & 13.7$~~$ & $~~$9.0 & $~~$6.6 & $~~$7.2$~~$ & 19.4 & 13.7 & 15.9 & $~~$6.7 & 12.5 & $~~$8.6 & 13.4 & 25.3 & 17.3\\
                Longest NPs & 24.2 & 31.7 & 27.0$~~$ & \textbf{14.5} & 19.9 & 16.5$~~$ & 11.7 & $~~$7.9 & $~~$9.3$~~$ & 19.6 & 13.7 & 16.0 & $~~$9.5 & 17.6 & 12.1 & 14.1 & 26.3  &18.1\\
                NP-chunks & 21.1 & 28.1 & 23.8$~~$ & 13.5 & 18.6 & 15.4$~~$ & 11.9 & $~~$8.0 & $~~$9.5$~~$ & 19.5 & 13.7 & 16.0 & $~~$9.6 & 17.9 & 12.3 & 14.3 & 26.8 & 18.4\\
                LR-NPs & \textbf{24.3} & \textbf{32.0} & \textbf{27.2$^\dagger$} & \textbf{14.5} & \textbf{20.0} & \textbf{16.6$^\ddagger$} & \textbf{12.4} & \textbf{$~~$8.4} & \textbf{$~~$9.9$^\ddagger$} & \textbf{20.4} & \textbf{14.4} & \textbf{16.7}& \textbf{10.1} & \textbf{18.5} & \textbf{12.9} & \textbf{14.4} & \textbf{27.0} & \textbf{18.6}\\
                \bottomrule
            \end{tabular}
            }
            \caption{Comparison of TF-IDF and KEA applied on top of different candidate selection methods.
            $\ddagger$ indicates a significant improvement overall candidate sets and $\dagger$ indicates a significant improvement overall candidate sets but the longest NPs at 0.001 level using Student's t-test.
                     \label{tab:keyphrase_extraction_results}}
        \end{table*}
        
        We measure the impact of each candidate selection method on the keyphrase extraction task using the two following unsupervised and supervised keyphrase extraction methods:
        \begin{itemize}
            \item{\textbf{TF-IDF~\cite{jones1972tfidf}:} Word significancy weighting scheme.
                  Words are weighted based on their frequency in the document and the inverse number of documents in which they appear (specificity);
                  Candidates are weighted using the sum of  their words' score.}
            \item{\textbf{KEA~\cite{witten1999kea}:} Naive Bayes classifier trained on two features: the TF-IDF\footnote{KEA computes a TF-IDF based on candidate frequency, whereas our TF-IDF baseline relies on word frequency. KEA's TF-IDF is more efficient on larger documents than smaller ones.} and the first position of each candidate selected within train documents.}
        \end{itemize}
        We report the performance of TF-IDF and KEA in terms of precision~(P), 
        recall~(R) and F1-measure (F) at the top 10 keyphrases.
        Candidate and reference keyphrases are stemmed to reduce the number 
        of mismatches.
    
    \subsection{Candidate selection evaluation}
    \label{subsec:candidate_extraction_evaluation}
        Table~\ref{tab:candidate_extraction_statistics} presents the results of the intrinsic evaluation of the candidate selection methods.
        %Unsurprisingly, the best maximum recall is achieved by the $\{1..3\}$-grams selection method, but at the cost of a huge number of unrelevant candidates as indicated by the low quality ratio.
        Unsurprisingly, the best maximum recall is achieved when selecting $\{1..3\}$-grams, but at the cost of a huge number of irrelevant candidates as indicated by the low QR.
        Among the other selection methods, our method shows a competitive maximum recall while reducing the number of candidates.
        As a consequence, the LR-NPs quality outperforms other selected candidates quality, which is crucial as it directly affects the performance and time complexity of keyphrase extraction methods~\cite{wang2014keyphraseextractionpreprocessing}.
        
        \TODO{examples of true and false positives}
    
    \subsection{Keyphrase extraction evaluation}
    \label{subsec:keyphrase_extraction_evaluation}
        Table~\ref{tab:keyphrase_extraction_results} shows the results of the extrinsic evaluation of the candidate selection methods.
        Overall, we observe that the performance of TF-IDF and KEA is closely correlated with the quality of the set of selected candidates.
        Best results are then obtained when TF-IDF and KEA are applied on LR-NPs (half of them are significantly better).
        Thus, although our proposed selection method does not achieve the best maximum recall, it still outperforms the other candidate selection  methods.
        Comparing longest NPs, NP-chunks and LR-NPs demonstrates that it is efficient to use heuristic based on linguistic properties.

    \section{Conclusion et perspectives}
\label{sec:conclusion_et_perspectives}
  Dans cet article, nous nous intéressons à la tâche d'extraction automatique de
  termes-clés dans les documents scientifiques et émettons l'hypothèse que sa
  difficulté est variable selon la discipline des documents traités. Pour
  vérifier cette hypothèse, nous disposons de notices bibliographiques réparties
  dans cinq disciplines (archéologie, linguistique, sciences de l'information,
  psychologie et chimie) auxquelles nous appliquons six systèmes d'extractions
  automatique de termes-clés différents. En comparant les termes-clés extraits
  par chaque système avec les termes-clés de référence assignés aux notices dans
  des conditions réels d'indexation, notre hypothèse se vérifie et nous
  observons l'échelle suivante (de la discipline la plus facile à la plus
  difficile)~:
  \begin{enumerate*}
    \item{Archéologie~;}
    \item{Linguistique~;}
    \item{Sciences de l'information~;}
    \item{Psychologie~;}
    \item{Chimie.}
  \end{enumerate*}

  À l'issue de nos expériences et de nos observations du contenu des notices,
  nous constatons deux facteurs ayant un impact sur la difficulté de la tâche
  d'extraction automatique de termes-clés. Tout d'abord, nous observons que
  l'organisation du résumé peut aider l'extraction de termes-clés. Un résumé
  riche en explications et en mises en relations des différents concepts est
  moins difficile à traiter qu'un résumé énumératif pauvre en explications.
  Ensuite, le vocabulaire utilisé dans une discipline peut influer sur la
  difficulté à extraire les termes-clés des documents de cette discipline. Si le
  vocabulaire spécifique contient des composés syntagmatiques dont certains
  éléments sont courants dans la discipline, alors il peut être plus difficile
  d'extraire les termes-clés des documents de cette discipline.

  Des deux facteurs identifiés émergent plusieurs perspectives de travaux
  futurs. Il peut être intéressant d'analyser le discours des documents afin de
  mesurer, en amont, le degré de difficulté de l'extraction de termes-clés. Avec
  une telle connaissance, nous pourrions proposer une méthode capable de
  s'adapter au degré de difficulté en ajustant automatiquement son paramètrage.
  Cependant, l'analyse que nous proposons dans cet article se fonde uniquement
  sur le contenu de notices appartenant à cinq disciplines. Il serait pertinent
  d'étendre cette analyse au contenu intégral des documents scientifiques, ainsi
  que d'élargir le panel de disciplines utilisées dans ce travail, afin
  d'établir des catégories de discplines plus ou moins difficiles à traiter
  (p.~ex. la chimie fait partie des disciplines expérimentales, qui sont
  difficiles à traiter). Nous oberservons aussi que le vocabulaire utilisé dans
  une discipline, en particulier celui utilisé pour les termes-clés, peut rendre
  la tâche d'extraction automatique de termes-clés plus difficile. Il est donc
  important de bénéficier de resources telles que des thésaurus pour permettre à
  une méthode d'extraction de termes-clés de s'adapter au domaine. Pour
  TopicRank, par exemple, avoir connaissance de la terminologie utilisée dans
  une discipline peut améliorer le choix du terme-clé le plus représentatif d'un
  sujet. Enfin, il serait intéressant de penser la tâche d'extraction de
  termes-clés comme une tâche d'extraction d'information pour le remplissage
  d'un formulaire. En archéologie, par exemple, il pourrait s'agir d'extraire
  les informations géographiques (pays, régions, etc.), chronologiques (période,
  culture, etc.), ou encore environnementales (animaux, végétaux, etc.).


    
    % \section*{Acknowledgements}
    %     The authors would like to thank the anonymous reviewers for their useful
    %     advice and comments. This work was supported by the French National Research
    %     Agency (TermITH project -- ANR-12-CORD-0029).
    
    \bibliographystyle{naaclhlt2015}
    \bibliography{../../biblio}
\end{document}
