\section{Experimental Settings}
\label{sec:experimental_settings}
  \subsection{Datasets}
  \label{subsec:datasets}
    We conduct our experiments on four datasets: Linguistics, Archaeology, DUC
    and SemEval. They differ in terms of language, nature and size. The following
    is the description of each dataset.

    \paragraph{Linguistics} is a collection of 715 bibliographic records that we
    collected from the French digital library Inist\footnote{The dataset will
    soon be available.}. The bibliographic records
    contain abstracts of scientific papers in the Linguistics area. The
    keyphrases are annotated by professional indexers. Indexers were
    given the instruction to assign keyphrases from a controlled vocabulary and to extract new
    concepts or very specific keyphrases from the bibliographic records. We divided the
    corpus into two disjoint sets: a training set containing 515 bibliographic
    records and a test set containing 200 bibliographic records.
    
    \paragraph{Archaeology} is a collection of 718 bibliographic records that we
    also collected from the French digital library Inist\footnote{The dataset will
    soon be available.}. The bibliographic records
    contain abstracts of scientific papers in the Archaeology area. Similar to
    the Linguistics dataset, the keyphrases are annotated by professional
    indexers. We divided the corpus into two disjoint sets: a training set
    containing 515 bibliographic records and a test set containing 200
    bibliographic records.

    \paragraph{DUC~\textnormal{\cite{wan2008expandrank}}} is a collection of 308
    news articles categorized within 30 topics. The documents have been manually
    annotated by students with 0.70 of inter-agreement (Kappa). DUC has no
    training set. For each document, we use the other
    documents of its topic to build the controlled graph.

    \paragraph{SemEval~\textnormal{\cite{kim2010semeval}}} contains 244 English
    scientific papers collected from the ACM Digital Libraries (conference and
    workshop papers). The papers represent four areas of Computer Science:
    Distributed Systems; Information Search and Retrieval; Distributed
    Artificial Intelligence -- Multiagent Systems; Social and Behavioral
    Sciences -- Economics. SemEval is divided into two disjoint sets: a training
    set containing 144 documents and a test set containing 100 documents. The
    associated keyphrases are provided by both authors and readers. As we do for
    DUC, we use keyphrases annotated on training documents of the same Computer Science
    area to build the controlled graph.
    
    \begin{table*}
        \centering
        \resizebox{\linewidth}{!}{
          \begin{tabular}{l|ccc|ccc|ccc|ccc}
            \toprule
            \multirow{2}{*}{\textbf{Method}} & \multicolumn{3}{c|}{\textbf{Linguistics}} & \multicolumn{3}{c|}{\textbf{Archaeology}} & \multicolumn{3}{c|}{\textbf{DUC}} & \multicolumn{3}{c}{\textbf{SemEval}}\\
            \cline{2-13}
            & P & R & F$^{~~}$ & P & R & F$^{~~}$ & P & R & F$^{~~}$ & P & R & F\\
            \hline
            TopicRank & 11.3 & 13.1 & 12.0$^{~~}$ & 28.1 & 19.1 & 22.2$^{~~}$ & 17.8 & 22.7 & 19.7$^{~~}$ & 14.6 & 10.1 & 11.8\\
            KEA++ & 11.6 & 13.0 & 12.1$^{~~}$ & 23.5 & 16.2 & 18.8$^{~~}$ & --- & --- & ---$^{~~}$ & --- & --- & ---\\
            \hline
            TopicCoRank$_\textnormal{\textit{extr}}$ & 14.6 & 16.8 & 15.4$^{\dagger}$ & 36.4 & 24.6 & 28.7$^{\dagger}$ & 25.5 & 32.4 & 28.1$^{\dagger}$ & 15.2 & 10.6 & 12.4\\
            TopicCoRank$_\textnormal{\textit{assign}}$ & \textbf{24.9} & \textbf{28.9} & \textbf{26.4$^{\dagger}$} & \textbf{49.1} & \textbf{33.3} & \textbf{38.8}$^{\dagger}$ & 25.9 & 33.3 & 28.8$^{\dagger}$ & 11.6 & ~~8.3 & ~~9.5\\
            \hline
            TopicCoRank & 19.0 & 22.0 & 20.1$^{\dagger}$ & 39.5 & 26.6 & 31.1$^{\dagger}$ & \textbf{28.4} & \textbf{36.6} & \textbf{31.5$^{\dagger}$} & \textbf{16.4} & \textbf{11.6} & \textbf{13.4}\\
            \bottomrule
          \end{tabular}
        }
        \caption{Results of TopicCoRank and the baselines at 10 keyphrases for each datasets.
                 Precision (P), Recall (R) and F-score (F) are reported in percentages. $\dagger$ indicates
                 a significant F-score improvement over TopicRank and KEA++ at 0.001
                 level using Student’s t-test.
                 \label{tab:comparison_results}}
    \end{table*}
    
    \paragraph{}
    Table~\ref{tab:corpus_statistics} shows factual information about the
    datasets. These datasets cover different types of document. Linguistics and Archeology
    contain the shortest documents within which a sizable amount of reference
    keyphrases cannot be extracted from the content of the documents (missing keyphrases), DUC contains larger documents where
    most keyphrases can be found within the text and SemEval is composed of the largest
    documents with fewer missing keyphrases than Linguistics and Archaeology, but still more
    than DUC.

  \subsection{Preprocessing}
  \label{subsec:preprocessing}
    We apply the following preprocessing steps to each document: sentence segmentation, word tokenization and Part-of-Speech (POS)
    tagging. We perform sentence segmentation with the PunktSentenceTokenizer
    provided by the Python Natural Language ToolKit~\cite[NLTK]{bird2009nltk}.
    We tokenize the sentences into words using the NLTK TreebankWordTokenizer for
    English and the Bonsai word tokenizer\footnote{The Bonsai word tokenizer is
    a tool provided with the Bonsai PCFG-LA parser:
    \url{http://alpage.inria.fr/statgram/frdep/fr_stat_dep_parsing.html}.} for
    French. Finally, we use the Stanford POS
    tagger~\cite{toutanova2003stanfordpostagger} for English POS tagging and
    MElt~\cite{denis2009melt} for French POS tagging.

  \subsection{Baselines}
  \label{subsec:baselines}
    To show the effectiveness of our approach, we
    compare TopicCoRank with TopicRank and KEA++. We use the official KEA++
    release\footnote{\url{http://www.nzdl.org/Kea/download.html}} and apply it to
    the Linguistics and Archaeology datasets with the Inist's thesauri\footnote{The thesauri will be soon available}. For DUC and
    SemEval, thesauri are not available. We cannot apply KEA++ to DUC and SemEval.

    We also show the performance of TopicCoRank when it only extracts keyphrases
    (TopicCoRank$_\textnormal{\textit{extr}}$) and when it only assigns keyphrases
    (TopicCoRank$_\textnormal{\textit{assign}}$).
    
  \subsection{Evaluation measures}
  \label{subsec:evaluation_measures}
    To evaluate the performance of the keyphrase extraction methods, we use the
    common measures of precision (P), recall (R) and f-score (F):
    \begin{align}
        \text{P} = \frac{\text{\textit{TP}}}{\text{\textit{TP}}+\text{\textit{FP}}},\ \text{R} = \frac{\text{\textit{TP}}}{\text{\textit{TP}} + \text{\textit{FN}}},\ \text{F} = \frac{2 \text{PR}}{\text{P} + \text{R}}
    \end{align}
    where $\textnormal{\textit{TP}}$ is the total number of correctly extracted keyphrases (true positives),
    $\textnormal{\textit{FP}}$ is the total number of erroneously extracted keyphrases (false positives) and
    $\textnormal{\textit{FN}}$ is the total number of positive keyphrases that have not been extracted (false negatives).
    We cut off the extracted keyphrases at the 10 best ranking ones and perform the
    comparisons with stemmed keyphrases.
    
    Consistent with \newcite{hassan2010conundrums}, we also report the performance of
    TopicCoRank and the baselines in terms of precision-recall curves. To generate the
    curves, we vary the evaluation cut-off from 1 to the total number of
    extracted/assigned keyphrases.

