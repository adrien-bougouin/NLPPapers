\section{Conclusion}
\label{sec:conclusion}
     In this paper, we have proposed a co-ranking approach to performing keyphrase
  extraction and keyphrase assignment jointly. Our method,
  TopicCoRank, builds two graphs: one with the
  document topics and one with controlled keyphrases (training keyphrases). We
  design a strategy to unify the two graphs and rank by importance topics and
  controlled keyphrases using a co-ranking vote.
  We performed experiments on three French datasets of  humanities:  linguistics, information science and archaeology. Results showed that our approach benefits from both controlled
  keyphrases and document topics, improving both keyphrase extraction and keyphrase assignment
  baselines. TopicCoRank can be used to annotate keyphrases in scientific domains in a close way of profession l indexers do it.
    
    %
    Of course,   TopicCoRank still suffers from limitations that are shared by many keyphrase extraction and assignment methods.  Our keyphrase extraction method adopts a ``naive'' topical clustering that may be replaced by a more sophisticated one in order to build more reliable concept clusters gathering all keyphrase variants.  The keyphrase assignment method uses as training data the pre-assigned keywords of scientific documents belonging to the same domain. Other  unsupervised constructions of the domain graph (using Wikipedia data, bootstrapping techniques, etc.) could be explored as well as  more complex unification strategies in order to more accurately connect the document and the domain graph.


