\section{Introduction}
\label{sec:section}
  Keyphrases are single or multi-word expressions that represent the main topics
  of a document. Keyphrases are useful in many tasks such as information
  retrieval~\cite{medelyan2008smalltrainingset}, document
  summarization~\cite{litvak2008graphbased} or document
  clustering~\cite{han2007webdocumentclustering}. Since the last decade,
  Internet has became the main source of information, but the available
  documents are not always associated with keyphrases and the enrichment
  provided by the mentioned tasks cannot be achieved. To allow these
  enrichments, researchers tend to automatically extract keyphrases from
  documents.

  Automatic keyphrase extraction methods are divided into two categories:
  supervised and unsupervisedmethods. Supervised methods typically recast the
  keyphrase extraction as a binary classification task~\cite{witten1999kea}.
  For unsupervised methods, keyphrase extraction is often considered as a
  ranking task and many approaches are
  used~\cite{barker2000nounphrasehead,mihalcea2004textrank}. Although they work
  differently, both supervised and unsupervised methods rely on a preliminary
  candidate extraction step. The candidate extraction identifies single and
  multi-word expressions that have the same syntactic properties than a
  keyphrase. These expressions are the only textual units that can be extracted
  as keyphrases. Therefore, keyphrase candidates play an important role for
  automatic keyphrase extraction.
  
  In this paper, we focus on the candidate extraction step. Various methods are
  commonly employed to extract keyphrase candidates. Usually, filtered n-grams,
  NP-chunks or word sequences matching given patterns are
  extracted~\cite{hulth2003keywordextraction}. Despite the fact that some
  researchers try to find more suitable keyphrase
  candidates~\cite{hulth2003keywordextraction,kim2009reexaminingautomatickeyphraseextraction},
  there is, in our knowledge, no work to compare the various methods which are
  used. We seek to do this by comparing the performances of existing keyphrase
  extraction methods depending on the employed candidate extraction method and
  by looking at the extracted candidates and the maximum recall that can be
  achieved for each candidate extraction method. Also, we investigate the
  feasibility of using term as candidate keyphrases. Terms are grammatically
  arranged sequences of words designating a concept and treated as single units,
  terminological phrases. For this reason, terms seem to be suitable for
  keyphrase candidate extraction.

  This paper is organized as follows.
  Section~\ref{sec:definition_of_candidate_keyphrases} presents our datasets and
  study their reference keyphrases, Section~\ref{sec:candidate_extraction}
  presents the candidate extraction and term extraction methods and
  Section~\ref{sec:keyphrase_extraction} describes the three keyphrase
  extraction methods that we use in our intrinsic and extrinsic evaluations
  presented in Section~\ref{sec:evaluation}. Finally,
  Section~\ref{sec:conclusion} conclude this work.

\section{Definition of Candidate Keyphrases}
\label{sec:definition_of_candidate_keyphrases}
  Candidate keyphrases are textual units which can be selected as keyphrases
  of a document. Therefore, they must have the same syntactic properties than
  ground truth keyphrases. This section aims to determine those properties by
  analyzing three standard evaluation datasets.

  \subsection{Keyphrase Extraction Datasets}
  \label{subsec:keyphrase_extraction_datasets}
    Keyphrase extraction datasets are used to train or evaluate keyphrase
    extraction methods. They are collections of documents paired with reference
    keyphrases given by authors, readers or both. In this work, we use three
    standard datasets which differ in terms of document size,  type and
    language.

    The \textbf{DUC} dataset \cite{over2001duc} is a collection of 308 English
    news articles covering about 30 topics (e.g. tornadoes, gun control, etc.).
    This collection is the test dataset of the DUC-2001 summarization evaluation
    campaign. This part of DUC-2001 is the only one that contains keyphrases,
    annotated by \newcite{wan2008expandrank}. We split the collection into two
    sets: a training set containing 208 documents and a test set containing 100
    documents.

    The \textbf{SemEval} dataset \cite{kim2010semeval} contains 284 English
    scientific papers collected from the ACM Digital Libraries (conference and
    workshop papers). The papers are divided into three sets: a trial set
    containing 40 documents (unused in this work), a training set containing 144
    documents and a test set containing 100 documents. The associated keyphrases
    are provided by both authors and readers.

    The \textbf{DEFT} dataset \cite{Paroubek2012deft} is a collection of 234
    French scientific papers belonging to the Humanities and Social Sciences
    domain. DEFT is divided into two sets: a training set containing 141
    documents and a test set containing 93 documents. References are author
    keyphrases.

  \subsection{Reference Keyphrases Analysis}
  \label{subsec:keyphrase_analysis}
    Despite the fact that the data are not homogeneous, this section aims to
    find the syntactic properties of most keyphrases for English and French. To
    avoid from biasing our work we only exploit information from the training
    data.

    Table~\ref{tab:train_dataset_statistics} shows statistics extracted from our
    datasets. It represents information about the documents (language, nature,
    size, etc.) and their associated keyphrases. First, keyphrases are studied
    regarding their number of words, which is a clue about their degree of
    informativity. Secondly, the amount of multi-word keyphrases containing a
    potentially relevant part-of-speech (POS) is given to allow us to infer
    syntactic properties of keyphrases\footnote{We know that word keyphrases are
    mostly nouns or proper nouns, so we focus on multi-word expressions.}. In
    this purpose, reference keyphrases have been automatically POS tagged and
    manually checked.
    \begin{table}[h]
      \centering
      \begin{tabular}{@{~}r@{~}r@{~}c@{~}c@{~}c@{~}}
        \toprule
        & \multirow{2}{*}[-2pt]{\textbf{Statistic}} & \multicolumn{3}{c}{\textbf{Corpus}}\\
        \cmidrule{3-5}
        & & DUC & SemEval & DEFT\\
        \midrule
        \multirow{6}{*}[-2pt]{\begin{sideways}\textbf{Documents}\end{sideways}} & Language & English & English & French\\
        & Type & News & Papers & Papers\\
        & Number & 208 & 144 & 141\\
        & Token average & 912.0 & 5134.6 & 7276.7\\
        & Keyphrase average & 8.1 & 15.4 & 5.4\\
        & Missing keyphrases & 3.9\% & 13.5\% & 18.2\%\\
        \addlinespace[1.5\defaultaddspace]
        \multirow{5}{*}[-2pt]{\begin{sideways}\textbf{Keyphrases}\end{sideways}} & Unigrams & 17.1\% & 20.2\% & 60.2\%\\
        & Bigrams & 60.8\% & 53.4\% & 24.5\%\\
        & Trigrams & 17.8\% & 21.3\% & $~~$8.8\%\\
        & Quadrigrams & $~~$3.0\% & $~~$3.9\% & $~~$4.2\%\\
        & N-grams (N $\geq$ 5) & $~~$1.3\% & $~~$1.3\% & $~~$2.4\%\\
        \addlinespace[1.5\defaultaddspace]
        \multirow{5}{*}[-2pt]{\begin{sideways}\textbf{Multi-word keyphrases}\end{sideways}} & Cont. nouns & 94.5\% & 98.7\% & 93.3\%\\
        & Cont. proper nouns & 17.1\% & $~~$4.3\% & $~~$6.9\%\\
        & Cont. adjectives & 50.0\% & 50.2\% & 65.5\%\\
        & Cont. verbs & $~~$1.0\% & $~~$4.0\% & $~~$1.0\%\\
        & Cont. adverbs & $~~$1.6\% & $~~$0.7\% & $~~$1.3\%\\
        & Cont. prepositions & $~~$0.3\% & $~~$1.5\% & 31.2\%\\
        & Cont. determiners & $~~$0.0\% & $~~$0.0\% & 20.4\%\\
        & Cont. others & $~~$1.5\% & $~~$2.5\% & 11.8\%\\
        \addlinespace[.5\defaultaddspace]
        \bottomrule
      \end{tabular}
      \caption{Training dataset statistics. As a matter of consistency regarding
               the evaluation of keyphrase extraction methods, the missing
               keyphrases are determined based on the stemmed forms.
               \label{tab:train_dataset_statistics}}
    \end{table}

    \begin{table*}
      \centering
      \begin{tabular}{rlllll}
        \toprule
        & \multicolumn{4}{l}{\textbf{Pattern}} & \textbf{Example}\\
        \midrule
        \multirow{5}{*}[-2pt]{\begin{sideways}\textbf{English}\end{sideways}} & Nc & Nc & & & hurricane expert\\ % AP880409-0015
        & A & Nc & & & turbulent summer\\ % AP88049-0015
        & Nc & & & & storms\\ % AP880409-0015
        & A & Nc & Nc & & annual hurricane forecast\\ % AP880409-0015
        & Nc & Nc & Nc & & hurricane reconnaissance flights\\ % AP890529-0030
        \addlinespace[1.5\defaultaddspace]
        \multirow{5}{*}[-2pt]{\begin{sideways}\textbf{French}\end{sideways}} & Nc & & & & patrimoine -- cultural heritage\\ % as_2002_007048ar
        & Nc & A & & & tradition orale -- oral tradition\\ % as_2002_007048ar
        & Np & & & & Indonésie -- Indonesia\\ % as_2001_000235ar
        & Nc & Sp & D & Nc & conservation de la nature -- nature conservation\\ % as_2005_011742ar
        & Nc & Sp & Nc & & traduction en anglais -- English translation\\ % meta_2003_006958ar
        \bottomrule
      \end{tabular}
      \caption{Frequent part-of-speech patterns (Multex format) for English and
               French keyphrases. \label{tab:best_patterns}}
    \end{table*}

    Keyphrase statistics of Table~\ref{tab:train_dataset_statistics} show that
    most of the keyphrases are unigrams or bigrams ($\simeq$~$80\%$). Also, the
    number of n-gram keyphrases decreases when $n$ increases ($n\geq3$). We
    conclude that keyphrases are more often less informative.
    
    \begin{property}\label{prop:informativity}
      Keyphrases bear the minimum information representing an important concept
      (e.g. ``T-2 Buckeye'' instead of ``two-seat T-2 Buckeye'').
    \end{property}

    Table~\ref{tab:train_dataset_statistics} shows the ratio of multi-word
    keyphrases containing a part-of-speech that are intuitively relevant for
    keyphrases. In both English and French, almost every keyphrases contain
    nouns and half of the keyphrases are modified by one or more adjectives. In
    French, we also observe that some keyphrases tend to contain prepositions
    and determiners. Additionally, unexpected part-of-speech, such as foreign
    words and coordinating conjunctions, are not rare.

    \begin{property}\label{prop:noun_phrases}
      Keyphrases are mostly nouns (e.g. ``storms'') that can be modified by one
      or more adjectives (e.g. ``\underline{annual} hurricane forecast'').
      French keyphrases containing multiple nouns can contain prepositions and
      determiners (e.g. ``conservation \underline{de la} nature'' -- ``nature
      conservation'').
    \end{property}

    To give an idea of the observed syntactic patterns,
    Table~\ref{tab:best_patterns} shows the five most frequent ones for the
    English and French dataset.

\section{Candidate Extraction}
\label{sec:candidate_extraction}
  The goal of the candidate extraction is to reduce the potential keyphrases to
  a set of relevant ones. Two benefits in reducing the possibilities are that it
  diminishes the execution time of the keyphrase extraction methods and removes
  noises that may deteriorate their results. We distinguish two categories of
  candidates: the positive candidates, which actually match with reference
  keyphrases, and the off-target candidates, which do not match with reference
  keyphrases. Among the off-target candidates, we also distinguish the
  information bearing candidates, which can help to improve the keyphrase
  extraction, and the unrelevant candidates, which are seen as candidate
  extraction errors.

  In this section, we present the common textual units that are used or could be
  used as candidates for keyphrase extraction.

  \paragraph{N-grams} are ordered sequences of $n$ words. From a text, every
  sequences of a given size $n$ are extracted. Extracting n-grams is the best
  way to obtain as much positive and information bearing candidates as possible,
  but it also gives many unrelevant candidates. Therefore, the extracted
  candidates are filtered with a list of stop words: conjunctions, prepositions,
  determiners  and common words. Stop words must not be found at the beginning
  or at the end of a
  candidate~\cite{witten1999kea,hulth2003keywordextraction}.

  \paragraph{POS tag patterns} are used to specify the syntactic form of the
  textual units to extract as keyphrases. \newcite{hulth2003keywordextraction}
  experimented with the frequent POS tag patterns of her training
  data\footnote{Frequent patterns are the ones that appear ten or more times.},
  whereas other researchers only use the longest sequences of nouns (including
  proper nouns) and adjectives, considering them as noun phrases.

  \paragraph{NP-chunks} are non-recursive noun phrases, i.e. noun phrases that
  do not contain other noun phrases. \newcite{hulth2003keywordextraction} used
  NP-chunks in her experiments, arguing that it makes the candidate extraction
  process less arbitrary -- which is the case when extracting n-grams -- and,
  most of all, more linguistic. Also, extracting such minimal noun phrases is
  consistent with both Property~\ref{prop:informativity} and
  Property~\ref{prop:noun_phrases}.

  \paragraph{Terminological phrases (terms)} are word sequences designating a
  concept and treated as single units for a specific domain. Considering that a
  keyphrase is the vector of an idea, or a concept, it seems relevant to extract
  keyphrases from  a set of terminological phrases. Besides, as well as
  extracting NP-chunks, extracting terminological phrases as keyphrase
  candidates seems less arbitrary and more linguistic.

  Acabit~\cite{daille2003acabit} and
  TermSuite\footnote{\url{http://www.ttc-project.eu}} are two term extraction
  systems that we propose to use for keyphrase candidate extraction. Acabit uses
  linguistic knowledge to extract the multi-word terms and their variants
  (morphological and syntagmatic). Based on syntactic and morphological clues,
  only the variants that preserve the base-term semantics are extracted.
  TermSuite is a tool for monolingual and bilingual term extraction that
  implements the state-of-the-art method for term extraction and variant
  detections.

  In their review of automatic term extraction methods,
  \newcite{castellvi2001automatictermdetection} stated that alongside term
  extraction there is the document indexing task. Indeed, the textual units that
  index a given document, i.e. best describe its content, often are
  terminological phrases. Hence, we believe that the complex noun phrase
  sub-compounding method of \newcite{evans1996nounphraseanalysis} could be
  suitable for keyphrase candidate extraction. The aim of their method is to
  extract both highly informative and less informative terms. Highly informative
  terms are complex noun phrases (e.g. ``the quality of surface of treated
  stainless steel strip'') and less  informative terms are their meaningful
  sub-compounds. Four types of sub-compounds are defined:
  \begin{enumerate}
    \item{Lexical atoms, i.e. semantically coherent phrases (e.g. ``stainless
          steel'')
          \label{item:lexical_atom}}
    \item{Head modifier pairs (e.g. ``treated steel)
          \label{item:head_modifier}}
    \item{Cross preposition modification pairs (e.g. ``surface quality'')
          \label{item:cross_preposition_modifier}}
    \item{Sub-compounds (e.g. ``stainless steel strip'')
          \label{item:subcompound}}
  \end{enumerate}
  The \ref{item:lexical_atom}$^\text{st}$,
  \ref{item:cross_preposition_modifier}$^\text{rd}$ and
  \ref{item:subcompound}$^\text{th}$ types represent continuous textual units,
  whereas the \ref{item:head_modifier}$^\text{nd}$ type represents both
  continuous and discontinuous textual units. In our work, we do not consider
  the head modifiers (\ref{item:head_modifier}), because we aim to only extract
  keyphrases contained in the analyzed document. For the same reason, we do not
  generate cross preposition modification pairs
  (\ref{item:cross_preposition_modifier}), which are not always containde within
  the document.

\section{Keyphrase Extraction}
\label{sec:keyphrase_extraction}
  The keyphrase extraction is the task of identifying single or multi-word
  expressions that best represent a document. After the extraction of the
  keyphrase candidates, an unsupervised or supervised method is applied to,
  respectively, rank or classify the candidates. Unsupervised methods do not
  need a human intervention, they may need a collection of non-annotated
  documents (TF-IDF), a set of similar
  documents~\cite[ExpandRank]{wan2008expandrank} or only the analyzed
  document~\cite[TopicRank]{bougouin2013topicrank}. As for supervised methods,
  such as KEA~\cite{witten1999kea}, they need manually annotated data to learn
  what are keyphrases. Supervised methods usually achieve better results than
  unsupervised methods.

  In this section, we present the three keyphrase extraction methods that we
  retain for our work. The first method uses a collection of non-annotated
  documents (TF-IDF), the second method only needs the analyzed document
  (TopicRank) and the last method requires annotated documents for a prior
  training (KEA).

  \paragraph{TF-IDF} is a weighting scheme that represents the significance of a
  word in a given document. A significant word must be both frequent in the
  document (TF) and specific to it (IDF). The specificity is determined using a
  collection of documents. The intuition is that the lower is the amount of
  documents containing a word, the higher is the specificity of the word.

  The keyphrase candidates are scored according to the sum of the TF-IDF weight
  of their words and the $k$ keyphrases with the highest scores are extracted as
  keyphrases.

  \paragraph{TopicRank} aims to extract keyphrases that best represent the main
  topics of a document. One keyphrase is extracted from each of the $k$ most
  significant topics, leading to a set of topically unredundant keyphrases.

  First, the keyphrase candidates are clustered by topics, assuming that
  candidates sharing enough words belong to the same topic. Secondly, the
  topics are used as graph nodes linked altogether. The TextRank graph-based
  ranking is then applied to rank the topics according to the strength of their
  links. The smaller are the offset distances between the candidates of two
  topics, the higher is the strength between the two topics. A topic strongly
  connected to many topics is highly significant and gives more significancy to
  every topic it is connected to. Finally, the most significant topics are
  selected and one candidate per each is chosen as a keyphrase. Assuming that a
  topic is first introduced by its generic form, the best keyphrase candidate
  for a topic is the one that appears first into the document.

  \paragraph{KEA} is a supervised keyphrase extraction method that learns to
  identify keyphrases based on two features. Indeed, a training collection is
  used to compute probabilities to have keyphrases given specific TF-IDF
  weights\footnote{The TF-IDF weights computed for KEA are based on the
  candidate frequency, not word frequency.} and specific first positions for
  candidates. A Naive Bayes classifier make usage of the two learned
  distributions (TF-IDF and first position) to classify the keyphrase candidates
  extracted from an analyzed document.

\section{Evaluation}
\label{sec:evaluation}
  To better understand the candidate extraction methods, we perform two
  evaluations. The first evaluation is intrinsic, it shows the quantity of
  extracted candidates compared to the quantity of positive candidates. The
  second evaluation is extrinsic, it compares the candidate extraction method
  when they are used with either TF-IDF, TopicRank or KEA.

  \subsection{Datasets}
  \label{subsec:datasets}
    The datasets used to evaluate the methods are the test sets of DUC,
    SemEval and DEFT (see Section~\ref{subsec:keyphrase_extraction_datasets}).
    Table~\ref{tab:test_dataset_statistics} reports the information about these
    test sets. As they are coherent with the training sets, we can make usage of
    the properties inferred in
    Section~\ref{sec:definition_of_candidate_keyphrases}.
    \begin{table}
      \centering
      \begin{tabular}{@{~}r@{~}c@{~}c@{~}c@{~}}
        \toprule
        \multirow{2}{*}[-2pt]{\textbf{Statistic}} & \multicolumn{3}{c}{\textbf{Corpus}}\\
        \cmidrule{2-4}
        & DUC & SemEval & DEFT\\
        \midrule
        Language & English & English & French\\
        Type & News & Papers & Papers\\
        Documents & 100 & 100 & 93\\
        Token average & 877.3 & 5177.7 & 6839.4\\
        Keyphrase average & 7.9 & 14.7 & 5.2\\
        Tokens/keyphrase & 2.1 & 2.1 & 1.6\\
        Missing keyphrases & 2.8\% & 22.1\% & 21.1\% \\
        \bottomrule
      \end{tabular}
      \caption{Test dataset statistics. As a matter of consistency regarding
               the evaluation of keyphrase extraction methods, the missing
               keyphrases are determined based on the stemmed forms.
               \label{tab:test_dataset_statistics}}
    \end{table}

  \subsection{Candidate Extraction}
  \label{subsec:candidate_extraction}
%    \begin{table*}
%      \centering
%      \begin{tabular}{rcccccc}
%        \toprule
%        \multirow{2}{*}[-2pt]{\textbf{Method}} & \multicolumn{2}{c}{\textbf{DUC}} & \multicolumn{2}{c}{\textbf{SemEval}} & \multicolumn{2}{c}{\textbf{DEFT}}\\
%        \cmidrule(r){2-3}\cmidrule(lr){4-5}\cmidrule(l){6-7}
%        & Candidates & Rmax & Candidates & Rmax & Candidates & Rmax\\
%        \midrule
%        \{1..2\}-grams & $~~$49098 & 76.6 & 163358 & 61.0 & 238678 & 67.3\\
%        \{1..3\}-grams & $~~$59623 & \textbf{90.8} & 258054 & \textbf{72.2} & 378526 & 74.1\\
%        %\{1..4\}-grams & $~~$78024 & 92.6 & 365151 & 74.1 & 533753 & 78.2\\
%        Learned patterns & $~~$31764 & 90.6 & 122741 & 69.8 & 199789 & \textbf{76.5}\\
%        Longest NPs & $~~$15559 & 88.7 & $~~$64649 & 62.4 & $~~$85047 & 61.1\\
%        NP-chunks & $~~$14994 & 76.0 & $~~$59839 & 56.6 & $~~$75548 & 63.0\\
%        Sub-compounds & $~~$17181 & 90.6 & $~~$71224 & 64.4 & $~~$86866 & 61.1\\
%        Acabit & $~~~~$2377 & 26.7 & $~~$13214 & 17.6 & $~~$11106 & 13.4\\
%        TermSuite & $~~$16173 & 46.1 & $~~$49449 & 32.3 & $~~$60162 & 52.8\\
%        \bottomrule
%      \end{tabular}
%      \caption{Candidate extraction statistics. Rmax stands for maximum recall,
%               i.e. the percentage of candidates that match with reference
%               keyphrases. \label{tab:candidate_extraction_statistics}}
%    \end{table*}
    \begin{table*}
      \centering
      \begin{tabular}{rcccccc}
        \toprule
        \multirow{2}{*}[-2pt]{\textbf{Method}} & \multicolumn{2}{c}{\textbf{DUC}} & \multicolumn{2}{c}{\textbf{SemEval}} & \multicolumn{2}{c}{\textbf{DEFT}}\\
        \cmidrule(r){2-3}\cmidrule(lr){4-5}\cmidrule(l){6-7}
        & Cand./Doc. & Rmax & Cand./Doc. & Rmax & Cand./Doc. & Rmax\\
        \midrule
        \{1..2\}-grams & $~~$491.0 & 76.6 & 1633.6 & 61.0 & 2566.4 & 67.3\\
        \{1..3\}-grams & $~~$596.2 & \textbf{90.8} & 2580.5 & \textbf{72.2} & 4070.2 & 74.1\\
        Learned patterns & $~~$317.6 & 90.6 & 1227.4 & 69.8 & 2148.3 & \textbf{76.5}\\
        Longest NPs & $~~$155.6 & 88.7 & $~~$646.5 & 62.4 & $~~$914.5 & 61.1\\
        NP-chunks & $~~$149.9 & 76.0 & $~~$598.4 & 56.6 & $~~$812.3 & 63.0\\
        Sub-compounds & $~~$171.8 & 90.6 & $~~$712.2 & 64.4 & $~~$934.0 & 61.1\\
        Acabit & $~~~~$23.8 & 26.7 & $~~$132.1 & 17.6 & $~~$119.4 & 13.4\\
        TermSuite & $~~$161.7 & 46.1 & $~~$494.5 & 32.3 & $~~$646.9 & 52.8\\
        \bottomrule
      \end{tabular}
      \caption{Candidate extraction statistics. Rmax stands for maximum recall,
               i.e. the percentage of candidates that match with reference
               keyphrases. \label{tab:candidate_extraction_statistics}}
    \end{table*}

    This section presents the intrinsic evaluation of the candidate extraction
    methods presented in Section~\ref{sec:candidate_extraction}. The aim is to
    compare the methods in terms of quantity of extracted candidates and
    percentage of keyphrases that can be found.

    \subsubsection{Method Settings}
    \label{subsubsec:method_settings}
      The parameters of the keyphrase candidate extraction methods are chosen to
      fit the keyphrase properties inferred from the training sets (see
      Section~\ref{sec:definition_of_candidate_keyphrases}).

      According to Property~\ref{prop:informativity}, we try two \textbf{n-gram
      extraction} methods with a low $n$ value: the first method extracts
      filtered n-grams where $n \in \{1..2\}$ and the second method extracts
      filtered n-grams where $n \in \{1..3\}$. The stop words used for the
      filtering are part of the IR Multilingual Resources provided by the
      University of Neuchâtel (UniNE).

      We define two \textbf{pattern matching} methods. The first one learns the
      keyphrases POS tag patterns from the training documents and extracts
      textual units, filtered by stop words, that match with one of the
      patterns. The second method follows both Property~\ref{prop:noun_phrases}
      and previous works on keyphrase extraction by looking for the longest
      sequences of nouns and adjectives, supposedly the longest noun phrases.

      The \textbf{NP-chunk extraction} is performed using pattern matching. Only
      basic regular expressions are used:
      \begin{itemize}
        \item{(Np+) $|$~(A+~Nc) $|$~(N+), for English}
        \item{(Np+) $|$~(A?~Nc~A+) $|$~(A~Nc) $|$~(Nc+),
              for French}
      \end{itemize}

      The \textbf{sub-compounding} method of
      \newcite{evans1996nounphraseanalysis} requires complex noun phrases. For
      comparison purpose, we input the longest NPs extracted by pattern
      matching, assuming that they are complex NPs.

      \textbf{Acabit} and \textbf{TermSuite} are both used to build a
      terminology from each training dataset. Acabit and TermSuite respectively
      detect 16929 and 30779 terms in DUC, 49537 and 74610 terms in SemEval and
      78438 and 125841 terms in DEFT. The extracted keyphrase candidates of a
      document are its textual units that belong to the terminology.

    \subsubsection{Result Analysis}
    \label{subsubsec:candidate_extraction_result_analysis}
      Table~\ref{tab:candidate_extraction_statistics} shows the number of
      textual units extracted by each candidate extraction method and indicates
      the maximum recall that could be achieved, i.e. the percentage of
      reference keyphrases that could eventually be found.

      Globally, the extraction  of n-grams gives many candidates and allows a
      high maximum recall, whereas the extraction of more specific textual units
      provides few candidates and allows a lower maximum recall. However, for
      the extraction of the longest noun phrases, NP-chunks and the longest noun
      phrases plus their sub-compounds, the drop in the maximum recall is not
      significant compared to the drop in the number of candidates. At this
      point, we can say that it benefits the processing time by avoiding
      off-target candidates, but it is not excluded that the avoided off-target
      candidates are unrelevant candidates which may introduce noise that
      dampens the keyphrase extraction performances. Also, the extraction of
      textual units matching learned patterns seems to stand as a compromise
      between the extraction of n-grams and longest NPs, NP-chunks or longest
      NPs plus their sub-compounds.

      The addition of sub-compoubds to the longest NPs does not seem to improve.
      A few candidates are added and the maximum recall barely increases. Our
      conclusion is that reference keyphrases can be found without modifiers
      within the document.

      Acabit and TermSuite provide a few candidates and do not allow a good
      maximum recall. Acabit only extracts complex terms, which explains the gap
      between its candidates and the one obtained using TermSuite.
      Figure~\ref{fig:candidate_intersections} compares TermSuite candidates
      with two candidate sets that allow a decent maximum recall while providing
      either many or few candidates: respectively $\{1..3\}$-grams and longest
      NPs. This figure shows that almost every positive candidates extracted by
      TermSuite are also extracted by the other methods, which extract even
      more positive candidates. Therefore, the only reason why TermSuite
      candidates could help to better extract keyphrases is the quality of the
      candidates. In fact, the $\{1..3\}$-grams may cover almost all the
      TermSuite candidates, but it extract so much arbitrary candidates that the
      risk of introducing noise dampening performances is increased. In the
      contrary TermSuite candidates do not significantly overlap the longest
      NPs. At this point, we can not decide whether one candidate set is better
      than the other. It is also not excluded that the two sets are
      complementary.
      \begin{figure*}
        \begin{minipage}{.32\textwidth}
          \centering
          \subfigure[TermSuite candidates compared to $\{1..3\}$-grams and \textbf{DUC} references.\label{subfig1:candidate_intersections}]{
            \begin{overpic}[height=.9\linewidth]{include/duc_refs_term_suite_1_2_3_grams.eps}
              \put(5,45){$\{1..3\}$-grams}
              \put(60,42){TermSuite}
              \put(80,72){Refs}
            \end{overpic}
          }
        \end{minipage}
        \hfill
        \begin{minipage}{.32\textwidth}
          \centering
          \subfigure[TermSuite candidates compared to $\{1..3\}$-grams and \textbf{SemEval} references.\label{subfig2:candidate_intersections}]{
            \begin{overpic}[height=.9\linewidth]{include/semeval_refs_term_suite_1_2_3_grams.eps}
              \put(7,45){$\{1..3\}$-grams}
              \put(62,40){TermSuite}
              \put(83,64){Refs}
            \end{overpic}
          }
        \end{minipage}
        \hfill
        \begin{minipage}{.32\textwidth}
          \centering
          \subfigure[TermSuite candidates compared to $\{1..3\}$-grams and \textbf{DEFT} references.\label{subfig3:candidate_intersections}]{
            \begin{overpic}[height=.9\linewidth]{include/deft_refs_term_suite_1_2_3_grams.eps}
              \put(7,46){$\{1..3\}$-grams}
              \put(64,45){TermSuite}
              \put(88,61){Refs}
            \end{overpic}
          }
        \end{minipage}
        \hfill
        \begin{minipage}{.32\textwidth}
          \centering
          \vspace{1em}
          \subfigure[TermSuite candidates compared to longest noun phrases and \textbf{DUC} references.\label{subfig4:candidate_intersections}]{
            \hspace{1.6em}
            \begin{overpic}[height=.48\linewidth]{include/duc_refs_term_suite_longest_nps.eps}
              \put(10,18){Longest}\put(19,6){NPs}
              \put(43,45){TermSuite}
              \put(38,74){Refs}
            \end{overpic}
            \hspace{1.6em}
          }
        \end{minipage}
        \hfill
        \begin{minipage}{.32\textwidth}
          \centering
          \vspace{.9em}
          \subfigure[TermSuite candidates compared to longest noun phrases and \textbf{SemEval} references.\label{subfig5:candidate_intersections}]{
            \hspace{1.8em}
            \begin{overpic}[height=.48\linewidth]{include/semeval_refs_term_suite_longest_nps.eps}
              \put(15,18){Longest}\put(24,6){NPs}
              \put(45,45){TermSuite}
              \put(48,75){Refs}
            \end{overpic}
            \hspace{1.8em}
          }
        \end{minipage}
        \hfill
        \begin{minipage}{.32\textwidth}
          \centering
          \vspace{1.6em}
          \subfigure[TermSuite candidates compared to longest noun phrases and \textbf{DEFT} references.\label{subfig6:candidate_intersections}]{
            \hspace{2.4em}
            \begin{overpic}[height=.42\linewidth]{include/deft_refs_term_suite_longest_nps.eps}
              \put(12,19){Longest}\put(23,5){NPs}
              \put(40,45){TermSuite}
              \put(48,75){Refs}
            \end{overpic}
            \hspace{2.4em}
          }
        \end{minipage}
        \caption{Intersection of TermSuite candidates with $\{1..3\}$-grams,
                 which allows a near perfect maximum recall, and the longest
                 NPs, which extracts a few candidates and allows a lower, but
                 still decent, maximum recall.
                 \label{fig:candidate_intersections}}
      \end{figure*}

  \subsection{Keyphrase Extraction}
  \label{subsec:keyphrase_extraction}
    This section presents the extrinsic evaluation of the keyphrase candidate
    extraction methods. The performances of TF-IDF, TopicRank and KEA are
    compared when they extract keyphrases from each candidate set. By doing so,
    we seek to determine whether some candidate extraction methods introduce
    noises that some particular methods are not able to efficiently deal with.

    \subsubsection{Evaluation Measures}
    \label{subsubsec:keyphrase_extraction_evaluation_measures}
      The performances of TF-IDF, TopicRank and KEA are evaluated in terms of
      precision, recall and f-score (f1-measure) when a maximum of 10 keyphrases
      are extracted. To allow morphological variations and reduce the missing
      keyphrases problem, extracted keyphrases and reference keyphrases are
      compared using stemmed forms.

    \subsubsection{Result Analysis}
    \label{subsubsec:candidate_extraction_result_analysis}
      Tables~\ref{tab:tfidf_results},~\ref{tab:topicrank_results}~and~\ref{tab:kea_results}
      show the performances of respectively TF-IDF, TopicRank and KEA when
      they are applied with either one of the studied candidate extraction
      methods.

      Globally, for every methods the best performances are achieved when using
      matching patterns (learned patterns, longest NPs, NP-chunks, etc.). As
      suggested in Section~\ref{subsubsec:candidate_extraction_result_analysis},
      despite the fact they provide a low number of candidates and allow a lower
      maximum recall, these methods provide better quality candidate sets by
      avoiding unrelevant candidates extracted with n-gram methods.

      Most of the time, the longest NPs plus their sub-compounds do not improve
      the performances achieved with the longest NPs themselves. Adding
      sub-compounds to the longest NPs does not add much positive candidates and
      does not add candidates bearing enough information to improve the
      extraction.

      The keyphrase extraction methods themselves behave differently. KEA have
      competitive performances with every candidate extraction methods while
      TF-IDF and TopicRank see their performances improving when the quality of
      the candidate set improves. Also, performances of TF-IDF seem more stable
      than the ones of TopicRank. This is mainly due to the TF-IDF weighting
      (also used in KEA) which can better manage non-specific candidates
      (unrelevant candidates). TopicRank only uses the analysed document, so it
      cannot make usage of such specificity measure and its performances are
      more dependant to the quality of the candidate set that is used.

      When applied to domain specific datasets, such as SemEval and DEFT,
      TermSuite candidates  lead to good performances.
      Section~\ref{subsubsec:candidate_extraction_result_analysis} stated that
      longest NPs, which seem to be the best candidates, and TermSuite
      candidates are not comparable, but they seem competitive, so a combination
      of these could contribute to the creation of an even better candidate set.

      \begin{table*}
        \centering
        \begin{tabular}{rccccccccc}
          \toprule
          \multirow{2}{*}[-2pt]{\textbf{Method}} & \multicolumn{3}{c}{\textbf{DUC}} & \multicolumn{3}{c}{\textbf{SemEval}} & \multicolumn{3}{c}{\textbf{DEFT}}\\
          \cmidrule(r){2-4}\cmidrule(lr){5-7}\cmidrule(l){8-10}
          & P & R & F & P & R & F & P & R & F\\
          \midrule
          \{1..2\}-grams & 14.7 & 19.5 & 16.5 & 10.3 & $~~$7.0 & $~~$8.3 & $~~$8.1 & 15.1 & 10.4\\
          \{1..3\}-grams & 14.3 & 19.0 & 16.1 & $~~$9.0 & $~~$6.0 & $~~$7.2 & $~~$6.7 & 12.5 & $~~$8.6\\
          %\{1..4\}-grams & 13.7 & 18.2 & 15.4 & $~~$8.4 & $~~$5.6 & $~~$6.7 & $~~$6.7 & 12.5 & $~~$8.6\\
          Learned patterns & 19.1 & 25.4 & 21.5 & 10.7 & $~~$7.3 & $~~$8.6 & $~~$7.0 & 13.1 & $~~$9.0\\
          Longest NPs & 24.2 & 31.7 & \textbf{27.0} & 11.7 & $~~$7.9 & $~~$9.3 & $~~$9.5 & 17.6 & 12.1\\
          NP-chunks & 21.1 & 28.1 & 23.8 & 11.9 & $~~$8.0 & \textbf{$~~$9.5} & $~~$9.6 & 17.9 & 12.3\\
          Sub-compounds & 22.8 & 29.9 & 25.5 & 10.8 & $~~$7.2 & $~~$8.6 & $~~$9.2 & 17.2 & 11.9\\
          Acabit & 15.3 & 19.6 & 17.0 & $~~$8.6 & $~~$6.1 & $~~$7.1 & $~~$2.4 & $~~$5.6 & $~~$3.3\\
          TermSuite & 17.4 & 23.2 & 19.6 & 11.5 & $~~$8.3 & \textbf{$~~$9.5} & 11.3 & 21.1 & \textbf{14.5}\\
          \bottomrule
        \end{tabular}
        \caption{Comparison of candidate extraction methods, when extracting 10
                 keyphrases with \textbf{TF-IDF}. Results are expressed as a
                 percentage of precision (P), recall (R) and f-score (F).
                 \label{tab:tfidf_results}}
      \end{table*}
      \begin{table*}
        \centering
        \begin{tabular}{rccccccccc}
          \toprule
          \multirow{2}{*}[-2pt]{\textbf{Method}} & \multicolumn{3}{c}{\textbf{DUC}} & \multicolumn{3}{c}{\textbf{SemEval}} & \multicolumn{3}{c}{\textbf{DEFT}}\\
          \cmidrule(r){2-4}\cmidrule(lr){5-7}\cmidrule(l){8-10}
          & P & R & F & P & R & F & P & R & F\\
          \midrule
          \{1..2\}-grams & 10.2 & 14.1 & 11.7 & 11.9 & $~~$8.2 & $~~$9.6 & $~~$5.8 & 11.0 & $~~$7.5\\
          \{1..3\}-grams & $~~$7.8 & 10.7 & $~~$8.9 & $~~$9.5 & $~~$6.7 & $~~$7.7 & $~~$6.2 & 11.4 & $~~$8.0\\
          %\{1..4\}-grams & $~~$7.1 & $~~$9.7 & $~~$8.1 & & & & & & \\
          Learned patterns & 14.9 & 19.8 & 16.7 & 12.2 & $~~$4.6 & \textbf{$~~$9.9} & $~~$8.8 & 16.1 & 11.3\\
          Longest NPs & 17.7 & 23.2 & 19.8 & 11.6 & $~~$7.9 & $~~$9.3 & 11.6 & 21.5 & \textbf{14.9}\\
          NP-chunks & 13.3 & 21.5 & 18.3 & 11.7 & $~~$8.0 & $~~$9.4 & 11.1 & 20.7 & 14.4\\
          Sub-compounds & 18.3 & 24.0 & \textbf{20.5} & 11.3 & $~~$7.7 & $~~$9.0 & 11.6 & 21.5 & \textbf{14.9}\\
          Acabit & 11.2 & 14.2 & 12.3 & 10.2 & $~~$7.1 & $~~$8.3 & $~~$3.4 & $~~$7.9 & $~~$4.7\\
          TermSuite & 10.4 & 13.9 & 11.7 & $~~$8.8 & $~~$6.4 & $~~$7.4 & $~~$9.6 & 18.5 & 12.4\\
          \bottomrule
        \end{tabular}
        \caption{Comparison of candidate extraction methods, when extracting 10
                 keyphrases with \textbf{TopicRank}. Results are expressed as a
                 percentage of precision (P), recall (R) and f-score (F).
                 \label{tab:topicrank_results}}
      \end{table*}
      \begin{table*}
        \centering
        \begin{tabular}{rccccccccc}
          \toprule
          \multirow{2}{*}[-2pt]{\textbf{Method}} & \multicolumn{3}{c}{\textbf{DUC}} & \multicolumn{3}{c}{\textbf{SemEval}} & \multicolumn{3}{c}{\textbf{DEFT}}\\
          \cmidrule(r){2-4}\cmidrule(lr){5-7}\cmidrule(l){8-10}
          & P & R & F & P & R & F & P & R & F\\
          \midrule
          \{1..2\}-grams & 12.3 & 17.1 & 14.1 & 19.2 & 13.6 & 15.8 & 13.1 & 24.5 & 16.9\\
          \{1..3\}-grams & 12.0 & 16.6 & 13.7 & 19.4 & 13.7 & 15.9 & 13.4 & 25.3 & 17.3\\
          %\{1..4\}-grams & 11.7 & 16.1 & 13.4 & 19.5 & 13.8 & 16.0 & 13.7 & 25.7 & 17.6\\
          Learned patterns & 12.9 & 17.8 & 14.8 & 19.6 & 13.8 & \textbf{16.1} & 14.7 & 27.6 & 19.0\\
          Longest NPs & 14.5 & 19.9 & 16.5 & 19.6 & 13.7 & 16.0 & 14.1 & 26.3 & 18.1\\
          NP-chunks & 13.5 & 18.6 & 15.4 & 19.5 & 13.7 & 16.0 & 14.3 & 26.8 & 18.4\\
          Sub-compounds & 14.6 & 20.0 & \textbf{16.7} & 19.3 & 13.5 & 15.8 & 14.1 & 26.3 & 18.1\\
          Acabit & 14.6 & 19.1 & 16.3 & 15.6 & 10.8 & 12.6 & $~~$4.7 & 10.5 & $~~$6.4\\
          TermSuite & 12.5 & 17.2 & 14.3 & 13.8 & 10.0 & 11.5 & 14.7 & 28.0 & \textbf{19.1}\\
          \bottomrule
        \end{tabular}
        \caption{Comparison of candidate extraction methods, when extracting 10
                 keyphrases with \textbf{KEA}. Results are expressed as a
                 percentage of precision (P), recall (R) and f-score (F).
                 \label{tab:kea_results}}
      \end{table*}
%      \begin{table*}
%        \centering
%        \begin{tabular}{rccccccccc}
%          \toprule
%          \multirow{2}{*}[-2pt]{\textbf{Method}} & \multicolumn{3}{c}{\textbf{DUC}} & \multicolumn{3}{c}{\textbf{SemEval}} & \multicolumn{3}{c}{\textbf{DEFT}}\\
%          \cmidrule(r){2-4}\cmidrule(lr){5-7}\cmidrule(l){8-10}
%          & P & R & F & P & R & F & P & R & F\\
%          \midrule
%          \{1..2\}-grams & 15.5 & 20.5 & 17.4 & 10.4 & $~~$7.0 & \textbf{$~~$8.3} & $~~$3.0 & $~~$6.2 & $~~$4.0\\
%          \{1..3\}-grams & 13.7 & 18.0 & 15.3 & $~~$3.4 & $~~$2.3 & $~~$2.7 & $~~$1.9 & $~~$4.2 & $~~$2.6\\
%          %\{1..4\}-grams & $~~$7.7 & 10.1 & $~~$8.6 & $~~$1.4 & $~~$1.0 & $~~$1.1 & $~~$1.1 & $~~$2.4 & $~~$1.5\\
%          Learned patterns & 18.7 & 24.3 & 20.8 & $~~$4.6 & $~~$3.1 & $~~$1.2 & $~~$1.9 & $~~$4.2 & $~~$2.6\\
%          Longest NPs & 22.8 & 29.5 & \textbf{25.3} & $~~$3.7 & $~~$2.5 & $~~$3.0 & $~~$4.6 & $~~$9.2 & $~~$6.1\\
%          NP-chunks & 20.6 & 27.3 & 23.1 & $~~$8.4 & $~~$5.7 & $~~$6.7 & $~~$4.9 & $~~$9.7 & $~~$6.4\\
%          Sub-compounds & 21.2 & 27.3 & 23.5 & $~~$3.5 & $~~$2.4 & $~~$2.8 & $~~$4.6 & $~~$9.2 & $~~$6.1\\
%          Acabit & 15.3 & 19.8 & 17.0 & $~~$7.6 & $~~$5.2 & $~~$6.1 & $~~$2.9 & $~~$6.6 & $~~$4.0\\
%          TermSuite & & & & & & & & & \\
%          \bottomrule
%        \end{tabular}
%        \caption{Comparison of candidate extraction methods, when extracting 10
%                 keyphrases with \textbf{SingleRank}. Results are expressed as a
%                 percentage of precision (P), recall (R) and f-score (F).
%                 \label{tab:singlerank_results}}
%      \end{table*}

\section{Conclusion}
\label{sec:conclusion}
  \textcolor{red}{\lipsum[1]}

