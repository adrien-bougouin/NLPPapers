%%%%%%%%%%%%%%%%%%%%%%%%%%%%%%%%%%%%%%%%%%%%%%%%%%%%%%%%%%%%%%%%%%%%%%%%%%%%%%%%
% File ijcnlp2013.tex
%
% Contact  sudeshna@cse.iitkgp.ernet.in
%
% Based on the style files for IJCNLP-2011...
%
% Based on the style files for EACL 2006 by
% e.agirre@ehu.es or Sergi.Balari@uab.es
%
% Based on the style files for ACL 08 by Joakim Nivre and Noah Smith

\documentclass[11pt]{article}
\usepackage{ijcnlp2013}
\usepackage{times}
\usepackage{url}
\usepackage{latexsym}

\usepackage[utf8]{inputenc}
\usepackage{amsmath}
\usepackage{relsize}
\usepackage{xfrac}
\usepackage{booktabs}
\usepackage{multirow}
\usepackage{todonotes}

\setlength{\textfloatsep}{\baselineskip}

\title{TopicRank: Graph-Based Topic Ranking for Keyphrase Extraction}
\author{
  Adrien Bougouin \and Florian Boudin \and Béatrice Daille\\
  Université de Nantes, LINA, France\\
  {\tt \{adrien.bougouin,florian.boudin,beatrice.daille\}@univ-nantes.fr}
}
\date{}

\begin{document}
  \maketitle

  \begin{abstract}
    Keyphrase extraction is the task of identifying single or multi-word
    expressions that represent the main topics of a document. In this paper we
    present TopicRank, a graph-based keyphrase extraction method that relies on
    a topical representation of the document. Candidate keyphrases are clustered
    into topics and used as vertices in a complete graph. A graph-based ranking
    model is applied to assign a significance score to each topic. Keyphrases
    are then generated by selecting a candidate from each of the top-ranked
    topics. We conducted experiments on four evaluation datasets of different
    languages and domains. Results show that TopicRank significantly outperforms
    state-of-the-art methods on three datasets.
  \end{abstract}

  \section{Extraction de termes-clés avec TopicRank}
\label{sec:extraction_de_termes_cles_avec_topicrank}
  \begin{itemize}
    \item{principe général}
    \item{identification des sujets}
    \item{ordonancement des sujets}
    \begin{itemize}
      \item{construction du graphe de sujets}
      \item{ordonnancement dans le graphe de sujets}
    \end{itemize}
    \item{selection des termes-clés}
  \end{itemize}



  \section*{Acknowledgments}
    The authors would like to thank the anonymous reviewers for their useful
    advice and comments. This work was supported by the French National
    Research Agency (TermITH project -- ANR-12-CORD-0029).

  \bibliographystyle{acl}
  \bibliography{biblio}
\end{document}

