%\section{Introduction}
  \begin{frame}{Introduction}
    \begin{block}<1->{Keyphrases}
      \begin{itemize}
        \item{Word or multi-word expressions}
        \item{Overview of a document's content
              $\Rightarrow$ \textbf{Main topics}}
      \end{itemize}
    \end{block}

    \uncover<3->{
      \vfill

      \begin{block}{Applications}
        \begin{multicols}{2}
          \begin{itemize}
            \item{Document indexing}
            \item{Document clustering}
            \item{Text summarization}
            \item{Query expansion}
            \item{Targeted advertising}
            \item{etc.}
          \end{itemize}
        \end{multicols}
      \end{block}
    }

    \visible<2>{
      \begin{textblock*}{.95\textwidth}(.025\textwidth, -.6\textheight)
        \setbeamertemplate{blocks}[rounded][shadow=true]

        \begin{exampleblock}{\footnotesize \textcolor{red}{Project Euclid} and
                             the role of \textcolor{red}{research libraries} in
                             \textcolor{red}{scholarly publishing}}
          \footnotesize Project Euclid, a \textcolor{red}{joint electronic
          journal publishing initiative} of \textcolor{red}{Cornell University
          Library} and \textcolor{red}{Duke University Press} is discussed in
          the broader contexts of the changing patterns of
          \textcolor{red}{scholarly communication} and the publishing scene of
          \textcolor{red}{mathematics}. Specific aspects of the project such as
          \textcolor{red}{partnerships} and the creation of an
          \textcolor{red}{economic model} are presented as well as what it takes
          to be a publisher. Libraries have gained important and relevant
          experience through the creation and management of digital libraries,
          but they need to develop further skills if they want to adopt a new
          role in the life cycle of scholarly communication.
        \end{exampleblock}
      \end{textblock*}
    }
  \end{frame}

