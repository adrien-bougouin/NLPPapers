%\section{Introduction}
\begin{frame}{Introduction}
    \begin{block}<1->{Keyphrases}
      \begin{itemize}
        \item{Word or multi-word expressions}
        \item{Overview of a document's content
              $\Rightarrow$ \textbf{Main topics}}
      \end{itemize}
    \end{block}

    \visible<2>{
      \begin{textblock*}{.95\textwidth}(.025\textwidth, -.1\textheight)
        \setbeamertemplate{blocks}[rounded][shadow=true]

        \begin{exampleblock}{\footnotesize \textcolor{BrickRed}{Project
                             Euclid} and the role of
                             \textcolor{BrickRed}{research libraries} in
                             \textcolor{BrickRed}{scholarly publishing}}
          \footnotesize Project Euclid, a \textcolor{BrickRed}{joint
          electronic journal publishing initiative} of
          \textcolor{BrickRed}{Cornell University Library} and
          \textcolor{BrickRed}{Duke University Press} is discussed in the
          broader contexts of the changing patterns of
          \textcolor{BrickRed}{scholarly communication} and the publishing
          scene of \textcolor{BrickRed}{mathematics}. Specific aspects of the
          project such as \textcolor{BrickRed}{partnerships} and the creation
          of an \textcolor{BrickRed}{economic model} are presented as well as
          what it takes to be a publisher. Libraries have gained important and
          relevant experience through the creation and management of digital
          libraries, but they need to develop further skills if they want to
          adopt a new role in the life cycle of scholarly communication.
        \end{exampleblock}
      \end{textblock*}
    }

    \visible<3->{
      \vfill

      \begin{block}{Applications}
        \begin{multicols}{2}
          \begin{itemize}
            \item{Document indexing}
            \item{Document clustering}
            \item{Text summarization}
            \item{Query expansion}
            \item{Targeted advertising}
            \item{etc.}
          \end{itemize}
        \end{multicols}
      \end{block}
    }
  \end{frame}

