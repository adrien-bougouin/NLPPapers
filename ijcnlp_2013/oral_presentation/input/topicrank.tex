\section{TopicRank}
\begin{frame}{TopicRank}
    \begin{columns}
      \begin{column}{.45\textwidth}
        \begin{enumerate}
          \item<2->{Candidate extraction}
          \item<3->{Candidate clustering}
          \item<4->{Graph creation}
          \item<5->{Topic ranking}
          \item<6->{Keyphrase selection}
        \end{enumerate}

        \visible<7>{
          \begin{textblock*}{.95\textwidth}(.025\textwidth, -.6\textheight)
            \setbeamertemplate{blocks}[rounded][shadow=true]

            \begin{exampleblock}{\footnotesize Project Euclid and the role of
                                 \textcolor{BurntOrange}{research libraries} in
                                 scholarly publishing}
              \footnotesize Project Euclid, a joint electronic journal
              publishing initiative of \textcolor{BrickRed}{Cornell University
              Library} and Duke University Press is discussed in the broader
              contexts of the changing patterns of scholarly communication and
              the publishing scene of mathematics. [\dots]
              \textcolor{BrickRed}{Libraries} have gained important and relevant
              experience through the creation and management of
              \textcolor{BrickRed}{digital libraries}, but they need to develop
              further skills if they want to adopt a new role in the life cycle
              of scholarly communication.
            \end{exampleblock}
          \end{textblock*}
        }

        \visible<9>{
          \begin{textblock*}{.95\textwidth}(.025\textwidth, -.6\textheight)
            \setbeamertemplate{blocks}[rounded][shadow=true]

            \begin{exampleblock}{\footnotesize Project Euclid and the role of
                                 research libraries in
                                 \textcolor{BurntOrange}{scholarly publishing}}
              \footnotesize Project Euclid, a joint electronic journal
              publishing initiative of Cornell University Library and Duke
              University Press is discussed in the broader contexts of the
              changing patterns of scholarly communication and the
              \textcolor{BrickRed}{publishing scene} of mathematics. Specific
              aspects of the project such as partnerships and the creation of
              an economic model are presented as well as what it takes to be a
              \textcolor{BrickRed}{publisher}. [\dots]
            \end{exampleblock}
          \end{textblock*}
        }

        \visible<11>{
          \begin{textblock*}{.95\textwidth}(.025\textwidth, -.425\textheight)
            \setbeamertemplate{blocks}[rounded][shadow=true]

            \begin{exampleblock}{\footnotesize \textcolor{BurntOrange}{Project
                                 Euclid} and the role of research libraries in
                                 scholarly publishing}
              \footnotesize [\dots] Specific aspects of the
              \textcolor{BrickRed}{project such} as partnerships and the
              creation of an economic model are presented as well as what it
              takes to be a publisher. [\dots]
            \end{exampleblock}
          \end{textblock*}
        }

        \visible<13>{
          \begin{textblock*}{.95\textwidth}(.025\textwidth, -.48\textheight)
            \setbeamertemplate{blocks}[rounded][shadow=true]

            \begin{exampleblock}{\footnotesize Project Euclid and the
                                 \textcolor{BurntOrange}{role} of research
                                 libraries in scholarly publishing}
              \footnotesize [\dots] Libraries have gained important and relevant
              experience through the creation and management of digital
              libraries, but they need to develop further skills if they want to
              adopt a \textcolor{BrickRed}{new role} in the life cycle of
              scholarly communication.
            \end{exampleblock}
          \end{textblock*}
        }
      \end{column}

      \begin{column}{.55\textwidth}
        \alt<2->{ % next slides
          \alt<3->{ % next slides
            \alt<4->{ %next slides
              \alt<6->{ % next slides
                \alt<16->{ % next slides
                  \alt<17->{ % next slides
                    \scriptsize
                    \begin{tabular}{l}
                      \toprule
                      Keyphrases\\
                      \midrule
                      \cellcolor{pink}{research libraries}\\
                      \cellcolor{pink}{scholarly publishing}\\
                      \cellcolor{pink}{project euclid}\\
                      role\\
                      creation\\
                      \cellcolor{pink}{scholarly communication}\\
                      \cellcolor{pink}{mathematics}\\
                      specific aspects\\
                      \cellcolor{pink}{joint electronic journal publishing initiative}\\
                      \cellcolor{pink}{partnerships}\\
                      \dots\\
                      \bottomrule
                    \end{tabular}
                  }{ % current slide
                    \scriptsize
                    \begin{tabular}{l}
                      \toprule
                      Keyphrases\\
                      \midrule
                      research libraries\\
                      scholarly publishing\\
                      project euclid\\
                      role\\
                      creation\\
                      scholarly communication\\
                      mathematics\\
                      specific aspects\\
                      joint electronic journal publishing initiative\\
                      partnerships\\
                      \dots\\
                      \bottomrule
                    \end{tabular}
                  }
                }{ % current slide
                  \scriptsize
                  \begin{tabular}{cl}
                    \toprule
                    ID & Cluster\\
                    \midrule
                    \multirow{1}{*}{C01} & cornell university library; digital libraries;\\
                    & \alt<8->{\textcolor{BrickRed}{research libraries}}{research libraries}; libraries\\
                    \multirow{1}{*}{C03} & publishing scene; \alt<10->{\textcolor{BrickRed}{scholarly publishing}}{scholarly publishing};\\
                    & publisher\\
                    C02 & \alt<12->{\textcolor{BrickRed}{project euclid}}{project euclid}; project such\\
                    C04 & \alt<14->{\textcolor{BrickRed}{role}}{role}; new role\\
                    C16 & \alt<15->{\textcolor{BrickRed}{creation}}{creation}\\
                    C06 & \alt<15->{\textcolor{BrickRed}{scholarly communication}}{scholarly communication}\\
                    C09 & \alt<15->{\textcolor{BrickRed}{mathematics}}{mathematics}\\
                    C12 & \alt<15->{\textcolor{BrickRed}{specific aspects}}{specific aspects}\\
                    C10 & \alt<15->{\textcolor{BrickRed}{joint electronic journal publishing initiative}}{joint electronic journal publishing initiative}\\
                    C08 & \alt<15->{\textcolor{BrickRed}{partnerships}}{partnerships}\\
                    \dots & \dots\\
                    \bottomrule
                  \end{tabular}
                }
              }{ % current slide
                \begin{tikzpicture}[scale=.185,
                                    align=center,
                                    every node/.style={transform shape},
                                    main node/.style={text centered,
                                                      circle,
                                                      draw=JungleGreen,
                                                      fill=JungleGreen!20,
                                                      inner sep=1.5pt,
                                                      font=\Large\bfseries}]
                  \foreach \number/\pos in {1/above,2/above,3/above,4/above,5/above,6/above,7/above,8/above,9/above,10/above,11/below,12/below,13/below,14/below,15/below,16/below,17/below,18/below,19/below}{
                    \mycount=\number
                    \advance\mycount by -1
                    \multiply\mycount by 19
                    \advance\mycount by 0
                    \ifthenelse{\number > 9}{
                      \node[main node, label={\pos:\Large\textbf{\visible<5->{\pagerank{\number}}}}] (N-\number) at (\the\mycount:15cm) {C\number};
                    }{
                      \node[main node, label={\pos:\Large\textbf{\visible<5->{\pagerank{\number}}}}] (N-\number) at (\the\mycount:15cm) {C0\number};
                    }
                  }
                  \foreach \number in {1,...,18}{
                    \mycount=\number
                    \advance\mycount by 1
                    \foreach \numbera in {\the\mycount,...,19}{
                      \path[JungleGreen!50] (N-\number) edge (N-\numbera);
                    }
                  }
                  \foreach \number/\pos in {1/above,2/above,3/above,4/above,5/above,6/above,7/above,8/above,9/above,10/above,11/below,12/below,13/below,14/below,15/below,16/below,17/below,18/below,19/below}{
                    \mycount=\number
                    \advance\mycount by -1
                    \multiply\mycount by 19
                    \advance\mycount by 0
                    \ifthenelse{\number > 9}{
                      \node[main node, label={\pos:\Large\textbf{\visible<5->{\pagerank{\number}}}}] (N-\number) at (\the\mycount:15cm) {C\number};
                    }{
                      \node[main node, label={\pos:\Large\textbf{\visible<5->{\pagerank{\number}}}}] (N-\number) at (\the\mycount:15cm) {C0\number};
                    }
                  }
                \end{tikzpicture}
              }
            }{ % current slide
              \scriptsize
              \begin{tabular}{cl}
                \toprule
                ID & Cluster\\
                \midrule
                \multirow{1}{*}{C01} & cornell university library; digital libraries;\\
                & research libraries; libraries\\
                C02 & project euclid; project such\\
                \multirow{1}{*}{C03} & publishing scene; scholarly publishing;\\
                & publisher\\
                C04 & role; new role\\
                C05 & important\\
                C06 & scholarly communication\\
                C07 & further skills\\
                C08 & partnerships\\
                C09 & mathematics\\
                C10 & joint electronic journal publishing initiative\\
                C11 & contexts\\
                C12 & specific aspects\\
                C13 & economic model\\
                C14 & duke university press\\
                C15 & relevant experience\\
                C16 & creation\\
                C17 & life cycle\\
                C18 & patterns\\
                C19 & management\\
                \bottomrule
              \end{tabular}
            }
          }{ % current slide
            \begin{exampleblock}{\footnotesize \textcolor{BrickRed}{Project
                                 Euclid} and the \textcolor{BrickRed}{role} of
                                \textcolor{BrickRed}{research libraries} in
                                \textcolor{BrickRed}{scholarly publishing}}
              \footnotesize \textcolor{BrickRed}{Project Euclid}, a
              \textcolor{BrickRed}{joint electronic journal publishing
              initiative} of \textcolor{BrickRed}{Cornell University Library}
              and \textcolor{BrickRed}{Duke University Press} is discussed in
              the broader \textcolor{BrickRed}{contexts} of the changing
              \textcolor{BrickRed}{patterns} of \textcolor{BrickRed}{scholarly
              communication} and the \textcolor{BrickRed}{publishing scene} of
              \textcolor{BrickRed}{mathematics}. \textcolor{BrickRed}{Specific
              aspects} of the \textcolor{BrickRed}{project such} as
              \textcolor{BrickRed}{partnerships} and the
              \textcolor{BrickRed}{creation} of an \textcolor{BrickRed}{economic
              model} are presented as well as what it takes to be a
              \textcolor{BrickRed}{publisher}. \textcolor{BrickRed}{Libraries}
              have gained \textcolor{BrickRed}{important} and
              \textcolor{BrickRed}{relevant experience} through the
              \textcolor{BrickRed}{creation} and
              \textcolor{BrickRed}{management} of \textcolor{BrickRed}{digital
              libraries}, but they need to develop \textcolor{BrickRed}{further
              skills} if they want to adopt a \textcolor{BrickRed}{new role} in
              the \textcolor{BrickRed}{life cycle} of
              \textcolor{BrickRed}{scholarly communication}.
            \end{exampleblock}
          }
        }{ % current slide
          \begin{exampleblock}{\footnotesize Project Euclid and the role of
                               research libraries in scholarly publishing}
            \footnotesize Project Euclid, a joint electronic journal publishing
            initiative of Cornell University Library and Duke University Press
            is discussed in the broader contexts of the changing patterns of
            scholarly communication and the publishing scene of mathematics.
            Specific aspects of the project such as partnerships and the
            creation of an economic model are presented as well as what it takes
            to be a publisher. Libraries have gained important and relevant
            experience through the creation and management of digital libraries,
            but they need to develop further skills if they want to adopt a new
            role in the life cycle of scholarly communication.
          \end{exampleblock}
        }
      \end{column}
    \end{columns}
  \end{frame}

