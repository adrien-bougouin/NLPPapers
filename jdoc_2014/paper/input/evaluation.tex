\section{Évaluation}
\label{sec:evaluation}
  Pour valider notre approche, nous réalisons une expérience d'extraction
  automatique de termes-clés à partir d'une collection de documents de test dont
  nous connaissons déjà les termes-clés. L'objectif est de proposer la méthode
  capable d'extraire le plus de termes-clés en commun avec ces termes-clés de
  référence, associés pour chaque document.

  \subsection{Collection de test}
  \label{subsec:donnees_de_test}
    Pour notre expérience, nous utilisons la collection de documents
    Wikinews\footnote{\url{https://github.com/adrien-bougouin/WikinewsKeyphraseCorpus}}.
    Wikinews est une collection de 100 articles journalistiques français
    extraits du site Web WikiNews\footnote{\url{http://fr.wikinews.org/}} entre
    les mois de mai et décembre 2012, puis annotés en termes-clés avec l'aide de
    trois étudiants pour chaque article.

  \subsection{Mesures d'évaluation}
  \label{subsec:mesures_d_evaluation}
    Nous mesurons la performance de chaque méthode d'extraction automatique de
    termes-clés en termes de précision (P), rappel (R) et F-mesure (F) lorsque
    10 termes-clés sont extraits, c'est-à-dire les 10 termes-clés candidats
    ayant le meilleur score d'importance. La précision mesure la capacité d'une
    méthode à extraire un minimum de termes-clés erronés, le rappel mesure sa
    capacité à fournir le plus de termes-clés corrects et la F-mesure est le
    compromis entre la précision et le rappel.

  \subsection{Méthodes de référence pour l'extraction de termes-clés}
  \label{subsec:systemes_de_reference_pour_l_extraction_de_termes_cles}
    % Comment les baselines sont-elles choisies ?
    Dans nos expériences, nous comparons TopicRank avec trois autres
    méthodes non-supervisées d'extraction automatique de termes-clés. Nous
    choisissons TextRank et SingleRank, les deux méthodes qui sont la
    fondation des méthodes à base de graphe, ainsi que la pondération TF-IDF,
    qui consiste à donner un score d'importance élevé aux termes-clés dont les
    mots sont fréquents et spécifiques au document analysé\footnote{Il est
    important de noter que la notion de spécificité calculée par TF-IDF requière
    une collection de documents (non annotée). Ceci, rend donc TF-IDF difficile
    à battre avec une méthode tirant uniquement profit des informations
    contenues dans le document.}.

  \subsection{Comparaison de TopicRank avec l'existant}
  \label{subsec:comparaison_de_topicrank_avec_l_existant}
    % Que représente le tableau ?
    Le tableau~\ref{tab:resultats_globaux} montre la performance de TopicRank
    comparée à celle des trois méthodes de référence.
    % Que peut-on dire globalement ?
    %Globalement, TopicRank donne de meilleurs résultats que les méthodes de
    %référence utilisées.
    % Que peut-on dire de plus ? (analyse plus approfondie)
    Comparée aux méthodes à base de graphe existantes, TopicRank donne des
    résultats significativement meilleurs, et confirme ainsi l'importance de
    grouper les termes-clés candidats représentant le même sujet afin de
    rassembler des informations (complémentaires) utiles à l'algorithme
    d'ordonnancement par importance. De plus, il est important de noter le gain
    de TopicRank vis-à-vis de la méthode TF-IDF, car cette dernière fait l'usage
    de statistiques extraites de documents supplémentaires, alors que TopicRank
    utilise uniquement le document à analyser. 
    \begin{table}
      \centering
      \begin{tabular}{rccc}
        \toprule
        \multirow{2}{*}[-2pt]{\textbf{Méthode}} & \multicolumn{3}{c}{\textbf{WikiNews}}\\
        \cmidrule(r){2-4}
        & P & R & F\\
        \midrule
        TF-IDF & 33,9 & 35,9 & 34,3$^{~}$\\
        TextRank & $~~$9,3 & $~~$8,3 & $~~$8,6$^{~}$\\
        SingleRank & 19,4 & 20,7 & 19,7$^{~}$\\
        TopicRank & \textbf{35,0} & \textbf{37,5} & \textbf{35,6}$^\dagger$\\
        \bottomrule
      \end{tabular}
      \caption{Résultats de l'extraction de 10 termes-clés avec TF-IDF,
               TextRank, SingleRank et TopicRank. $\dagger$ indique une
               amélioration significative de TopicRank vis-à-vis de TextRank et
               SingleRank, à 0,001 pour le t-test de Student.
               \label{tab:resultats_globaux}}
    \end{table}

