\section{Introduction}
\label{sec:introduction}
  Un ensemble de termes-clés est un ensemble de mots ou d'expressions
  polylexicales, c'est-à-dire des séquences de mots grammaticalement correctes,
  permettant de caractériser le contenu principal d'un document. Du fait de leur
  propriété synthétique, les ensembles de termes-clés sont utilisés dans de
  nombreuses applications telles que l'indexation automatique de
  documents~\cite{medelyan2008smalltrainingset}. Pourtant, de nombreux
  documents, tels que ceux accessibles par Internet, ou collections de
  documents, telles que les actes de conférences, n'en sont pas accompagnées. La
  quantité de document à traiter est aujourd'hui trop importante pour que
  l'extraction de leurs termes-clés soit effectuée manuellement. C'est pourquoi
  de nombreux chercheurs se penchent sur la problématique de l'extraction
  automatique de termes-clés.

  L'extraction automatique de termes-clés consiste à sélectionner dans un
  document les unités textuelles (mots et séquences de mots) les plus
  importantes. Parmi les différentes méthodes d'extraction automatique de
  termes-clés proposées dans la littérature, deux grandes catégories émergent~:
  les méthodes supervisées et les méthodes non-supervisées. Les premières
  réduisent la tâche d'extraction de termes-clés en une tâche de classification
  binaire~\cite{witten1999kea}, où il s'agit d'attribuer la classe
  \og{}\textit{terme-clé}\fg{} ou \og{}\textit{non terme-clé}\fg{} aux
  différents candidats extraits à partir du document. Dans ce cas, une
  collection de documents dont nous connaissons les termes-clés est nécessaire
  pour apprendre à faire la distinction entre \og{}\textit{terme-clé}\fg{} et
  \og{}\textit{non terme-clé}\fg{}. Au contraire, les méthodes non-supervisées
  n'utilisent pas de données ayant nécessitées un travail manuel tel que
  l'annotation en termes-clés. Ces méthodes se contentent d'ordonner les
  termes-clés candidats selon un score d'importance attribué en fonction de
  divers indicateurs tels que leur fréquence ou leur position dans le document
  analysé. Les méthodes supervisées sont plus performantes que les méthodes
  non-supervisées, mais leur besoin en données annotées en termes-clés (pour
  l'apprentissage) pousse les chercheurs à proposer des méthodes non-supervisées
  compétitives avec les méthodes supervisées.

  Les méthodes d'extraction de termes-clés non-supervisées les plus étudiées
  sont sans conteste celles fondées sur TextRank~\cite{mihalcea2004textrank},
  qui est une méthode d'ordonnancement d'unités textuelles à partir de graphe.
  Les graphes sont un moyen naturel de représenter les unités textuelles et
  leurs relations. Pour l'extraction de termes-clés, l'idée est de représenter
  le document sous la forme d'un graphe dans lequel les n\oe{}uds correspondent
  aux mots et où chaque mot est relié aux mots dont il est proche dans le
  document. Ensuite, un algorithme ordonne les mots par importance, selon le
  principe de recommandation~: un mot est d'autant plus important s'il est
  proche d'un grand nombre de mots et si les mots dont il est proche sont eux
  aussi importants. Les mots les plus importants (mots clés) sont finalement
  assemblés pour générer les termes-clés du document analysé.

  Dans cet article, nous présentons TopicRank, une méthode non-supervisée
  d'extraction de termes-clés basée sur TextRank. TopicRank groupe les
  termes-clés candidats selon leur appartenance à un sujet, représente le
  document sous la forme d'un graphe complet de sujets, ordonne les sujets selon
  leur importance, puis sélectionne pour chacun des meilleurs sujets son
  candidat le plus représentatif. La notion de sujet est vague, tant elle peut
  exprimer un thème ou un domaine général (par exemple, \og le traitement
  automatique des langues~\fg) ou plus spécifique (par exemple, \og l'extraction
  automatique de termes-clés~\fg). Ici, nous nous intéressons aux sujets les
  plus spécifiques, car ils caractérisent avec plus de précision le contenu d'un
  document.

  L'article est structuré comme suit. Tout d'abord, nous décrivons le
  fonctionnement de TopicRank
  (section~\ref{sec:extraction_de_termes_cles_avec_topicrank}), puis nous
  présentons notre évaluation (section~\ref{sec:evaluation}) et enfin, nous
  concluons et présentons les perspectives de ce travail
  (section~\ref{sec:conclusion_et_perspectives}).
