\section{Extraction de termes-clés avec TopicRank}
\label{sec:extraction_de_termes_cles_avec_topicrank}
  % Qu'est-ce que TopicRank ?
  TopicRank modélise un document sous la forme d'un graphe de sujets.
  % Quels problèmes résout-il ?
  Cette méthode se différencie des autres méthodes à base de graphe, car, plutôt
  que de chercher les mots importants du document, elle cherche ses sujets
  importants.
  % Quel en est le fonctionnement général ?
  Dans un premier temps, les sujets sont identifiés, dans un second temps ils
  sont ordonnés, puis les candidats les plus représentatifs des $k$ meilleurs
  sujets sont extraits comme termes-clés.

  \subsection{Identification des sujets}
  \label{subsec:identification_des_sujets}
    % Que nous faut-il pour identifier les sujets ?
    La première étape de l'identification des sujets consiste à extraire les
    termes-clés candidats.
    % Quels candidats composent les sujets ?
    Pour ce faire, nous suivons \newcite{wan2008expandrank} et
    \newcite{hassan2010conundrums} en extrayant les plus longues séquences de
    noms, de noms propres et d'adjectifs.

    % Comment détectons nous deux candidats appartenant au même sujet ?
    La seconde étape de l'identification des sujets consiste à grouper les
    termes-clés candidats lorsqu'ils appartiennent au même sujet.
    Dans le soucis de proposer une méthode ne faisant pas l'usage de données
    supplémentaires, nous optons pour un groupement lexical des candidats.
    Ceux-ci sont groupés lorsqu'ils partagent un nombre suffisant (définition
    d'un seuil) de mots.

  \subsection{Ordonnancement des sujets}
  \label{subsec:ordonnancement_des_sujets}
    % Quel est le but de l'ordonnancement ?
    % Comment est-il effectué ?
    L'ordonnancement des sujets a pour objectif de trouver quels sont ceux qui
    ont le plus d'importance dans le document analysé. À l'instar de
    TextRank~\cite{mihalcea2004textrank}, l'importance des sujets est déterminée
    à partir d'un graphe.

    % Comment le graphe est-il construit ?
    Les sujets du document analysé composent les n\oe{}uds d'un graphe complet.
    Chaque n\oe{}ud est lié aux autres par une relation qui représente la force
    du lien sémantique entre eux. Plus les candidats de deux sujets sont proches
    dans le document analysé, plus le lien sémantique entre ces deux sujets est
    fort.

    % Comment le graphe est-il utilisé pour ordonner les sujets ?
    % Quelle est l'intuition de PageRank/TextRank ?
    Une fois le graphe construit, l'algorithme d'ordonnancement de TextRank est
    utilisé pour identifier quels sont les sujets les plus importants du
    document. Cet ordonnancement se fonde sur le principe de recommandation,
    ou de vote, c'est-à-dire un sujet est d'autant plus important qu'il est
    fortement lié à un grand nombre de sujets et que les sujets avec lesquels il
    est fortement lié sont importants.

  \subsection{Sélection des termes-clés}
  \label{subsec:selection_des_termes_cles}
    % De quoi s'agit-il ?
    La sélection des termes-clés est la dernière étape de TopicRank. Elle
    consiste à chercher les termes-clés candidats qui représentent le mieux les
    sujets importants qui sont abordés dans le document. Dans le but de ne pas
    extraire de termes-clés redondants, un seul candidat est sélectionné par
    sujet.
    % Quel en est le but ?
    Ainsi, lorsque $k$ termes-clés doivent être extraits, nous nous assurons
    de couvrir exactement $k$ sujets disctincts.

    % Quelles sont les différentes stratégies envisageable ?
    La difficulté de ce principe de sélection réside dans la capacité à trouver
    parmi plusieurs termes-clés candidats d'un même sujet celui qui le
    représente le mieux. Dans ce travail, nous faisons l'hypothèse qu'un sujet
    est tout d'abord introduit par sa forme la plus appropriée. Nous
    sélectionnons donc, pour chaque sujet, le candidat qui apparaît en premier
    dans le document analysé.

