\section{TermITH-Eval Format}
\label{sec:termith_eval_format}
    In the purpose of the TermITH-Eval dataset, we had to tackle the challenge that complex annotations combining automatic extractions, manual annotations, as well as scoring information, would occur within our document.
    Our choice for dealing with such a complex document structure was to use the TEI guidelines, which particularly offer customization facilities for the identification of an optimal trade-off between full compliance to the TEI architecture and integration of project specific constraints.
    More precisely we integrated two extensions to the TEI standard representation:
    \begin{itemize}
        \item{
            We used the work done in~\cite{romary2010tbxgoestei} to complement the TEI guidelines with terminological entries compliant to ISO standard 30042 (TBX, TermBase eXchange).
            This in turn has now become a proposal to the TEI consortium;
        }
        \item{
            We heavily experimented with the new proposal (\url{https://github.com/TEIC/TEI/issues/374}) for an in-document stand-off annotation element, which would allow to class together groups of annotations (e.g. from the same term extraction process).
            We also added TBX entries as possible body objects (in the sense of the Open Annotation framework) to the stand-off proposal.
        }
    \end{itemize}
    
    All in all, this work of compiling the best of existing but also on-going standardisation efforts, has proved to be highly effective for our project, especially when keyphrase extraction outputs had to be sent for evaluation, and we see this as a possible reference framework for similar projects.
