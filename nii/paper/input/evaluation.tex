\section{Experimental Settings}
\label{sec:experimental_settings}
  \subsection{Dataset}
  \label{subsec:dataset}
    In this work, we use the SemEval corpus. Built for the task 5 of
    SemEval-2010~\cite{kim2010semeval}, Sem\-Eval contains 244 English
    scientific papers collected from the ACM Digital Libraries. We use SemEval's
    training set (144 documents) and test set (100 documents) with their sets of
    combined author- and reader-assigned keyphrases.

  \subsection{Baselines}
  \label{subsec:baselines}
    \TODO{Baselines derived from our method, then common baselines}

  \subsection{Preprocessing}
  \label{subsec:preprocessing}
%    For each document, we apply the following preprocessing steps: sentence
%    segmentation, word tokenization and Part-of-Speech tagging. For sentence
%    segmentation, we use the PunktSentenceTokenizer provided by the Python
%    Natural Language ToolKit~\cite[NLTK]{bird2009nltk}. For word tokenization,
%    we use the NLTK TreebankWordTokenizer for English and the Bonsai word
%    tokenizer\footnote{The Bonsai word tokenizer is a tool provided with the
%    Bonsai PCFG-LA parser:
%    \url{http://alpage.inria.fr/statgram/frdep/fr_stat_dep_parsing.html}.} for
%    French. As for Part-of-Speech tagging, we use the Stanford POS
%    tagger~\cite{toutanova2003stanfordpostagger} for English and
%    MElt~\cite{denis2009melt} for French.
    %%%%%%%%%%%%%%%%%%%%%%%%%%%%%%%%%%%%%%%%%%%%%%%%%%%%%%%%%%%%%%%%%%%%%%%%%%%%
    For our method, as well as all baselines, we use Topic\-Rank's outputs.
    Therefore, our results can directly be compared to results
    in~\cite{bougouin2013topicrank}.

  \subsection{Evaluation Measures}
  \label{subsec:evaluation_measures}
    We evaluate the performances of our method and the baselines in terms of
    precision (P), recall (R) and f-score (f1-measure, F) when at most 10
    keyphrases are extracted. In order to reduce mismatches due to flexions such
    as plural, we also stem candidate and reference keyphrases during the
    evaluation.

\section{Results}
\label{sec:results}
  \TODO{Explain better results of SVM than SVM+Clusters (more than one keyphrase
  per cluster)}
  \TODO{Explain poor results of SVM+Clusters on dependent features}
  \TODO{Most of all the different sampling is responsible for it}

%  \begin{table*}
%    \centering
%    \begin{tabular}{rrrrrrrrrr}
%      \toprule
%      & \multicolumn{9}{c}{Feature sets}\\
%      \cmidrule{2-10}
%      Method & \multicolumn{3}{c}{Undependent} & \multicolumn{3}{c}{Dependent} & \multicolumn{3}{c}{Combination}\\
%      \cmidrule(r){2-4}
%      \cmidrule{5-7}
%      \cmidrule(l){8-10}
%      & \multicolumn{1}{c}{P} & \multicolumn{1}{c}{R} & \multicolumn{1}{c}{F} & \multicolumn{1}{c}{P} & \multicolumn{1}{c}{R} & \multicolumn{1}{c}{F} & \multicolumn{1}{c}{P} & \multicolumn{1}{c}{R} & \multicolumn{1}{c}{F}\\
%      \midrule
%      %TopicRank+SVM   & 17.2 & 12.0 & 14.0
%      %                & 14.3 & 10.3 & 11.9
%      %                & 19.4 & 13.4 & 15.7\\
%      TopicRank+SVM    & 21.5 & 15.1 & 17.6
%                       & 9.0 & 6.5 & 7.5
%                       & 24.2 & 16.7 & 19.6\\
%      Clustering+ SVM  & 13.3 & 9.3 & 10.8
%                       & 0.2 & 0.1 & 0.2
%                       & 11.9 & 8.4 & 9.7\\
%      SVM              & 15.0 & 10.5 & 12.2
%                       &  &  & 
%                       &  &  & \\
%      \bottomrule
%    \end{tabular}
%    \caption{
%             \label{tab:baseline_comparison}}
%  \end{table*}
%  \begin{figure}
%    \begin{tikzpicture}%[scale=.75]
%      \pgfkeys{/pgf/number format/.cd, fixed, fixed zerofill, precision=1}
%      \begin{axis}[axis lines=left,
%                   symbolic x coords={Undependent, Dependent, Combination},
%                   xtick=data,
%                   enlarge x limits=0.25,
%                   %x=.25\linewidth,
%                   xticklabel style={anchor=north east, xshift=.75em,
%                   yshift=.5em, rotate=12.5},
%                   nodes near coords,
%                   nodes near coords align={vertical},
%                   every node near coord/.append style={font=\scriptsize},
%                   ytick={0.0, 5.0, 10.0, 15.0, 20.0, 25.0},
%                   y=0.02\linewidth,
%                   ymin=0.0,
%                   ymax=22.0,
%                   ybar=10pt,
%                   ylabel=F,
%                   ylabel style={at={(ticklabel* cs:1)},
%                                 anchor=south,
%                                 rotate=270},
%                   legend style={at={(0.6, 1.0)},
%                                 anchor=north}]
%        \addplot[black!66,
%                 pattern=north west lines,
%                 pattern color=black!40] coordinates{
%          (Undependent, 12.2)
%          (Dependent, 0.0)
%          (Combination, 0.0)
%        };
%        \addplot[black!66,
%                 pattern=north east lines,
%                 pattern color=black!40] coordinates{
%          (Undependent, 10.8)
%          (Dependent, 0.2)
%          (Combination, 9.7)
%        };
%        \addplot[black!66,
%                 pattern=horizontal lines,
%                 pattern color=black!40] coordinates{
%          (Undependent, 17.6)
%          (Dependent, 7.5)
%          (Combination, 19.6)
%        };
%        \legend{SVM, Clustering+SVM, TopicRank+SVM}
%      \end{axis}
%    \end{tikzpicture}
%    \caption{
%             \label{fig:baseline_comparison}}
%  \end{figure}
  \begin{figure}[h]
    \begin{tikzpicture}%[scale=.75]
      \pgfkeys{/pgf/number format/.cd, fixed, fixed zerofill, precision=1}
      \begin{axis}[axis lines=left,
                   symbolic x coords={TopicRank+SVM, Clustering+SVM, SVM},
                   xtick=data,
                   enlarge x limits=0.25,
                   %x=.25\linewidth,
                   xticklabel style={anchor=west, rotate=-22.25},
                   nodes near coords,
                   nodes near coords align={vertical},
                   every node near coord/.append style={font=\scriptsize},
                   ytick={0.0, 5.0, 10.0, 15.0, 20.0, 25.0},
                   y=0.025\linewidth,
                   ymin=0.0,
                   ymax=22.0,
                   ybar=5pt,
                   ylabel=F,
                   ylabel style={at={(ticklabel* cs:1)},
                                 anchor=south,
                                 rotate=270},
                   legend style={at={(1.0, 1.0)},
                                 anchor=north east}]
        \addplot[Cerulean,
                 pattern=north east lines,
                 pattern color=Cerulean] coordinates{
          (SVM, 12.2)
          (Clustering+SVM, 10.8)
          (TopicRank+SVM, 17.6)
        };
        \addplot[YellowGreen,
                 pattern=north west lines,
                 pattern color=YellowGreen] coordinates{
          (SVM, 0.0)
          (Clustering+SVM, 0.2)
          (TopicRank+SVM, 7.5)
        };
        \addplot[RedOrange,
                 pattern=horizontal lines,
                 pattern color=RedOrange] coordinates{
          (SVM, 0.0)
          (Clustering+SVM, 9.7)
          (TopicRank+SVM, 19.6)
        };
        \legend{Undependent features, Dependent features, All features}
      \end{axis}
    \end{tikzpicture}
    \caption{
             \label{fig:baseline_comparison}}
  \end{figure}

  \TODO{t-test}

  \begin{table}[h]
    \centering
    \begin{tabular}{|r|rrr|}
      \hline
      Method & \multicolumn{1}{c}{P} & \multicolumn{1}{c}{R} & \multicolumn{1}{c|}{F}\\
      \hline
      KEA           & 18.8 & 13.3 & 15.4\\
      TF-IDF        & 13.2 & 8.9 & 10.5\\
      TopicRank     & 14.9 & 10.3 & 12.1\\
      TopicRank+SVM & 24.2 & 16.7 & 19.6\\
      \hline
      Upper bound   & 37.6 & 25.8 & 30.3\\
      \hline
    \end{tabular}
    \caption{
             \label{tab:state_of_the_art_comparison}}
  \end{table}

%\section{Error Analysis}
%\label{sec:error_analysis}

