\section{Introduction}
\label{sec:introduction}
  Since the last decade, the amount of information available on the web is
  constantly increasing. While the number of documents continues to grow, the
  need for efficient information retrieval methods becomes increasingly
  important. One way to improve retrieval effectiveness is to use
  keyphrases~\cite{jones1999phrasier}. Keyphrases are single or multi-word
  expressions that represent the main content of a document. There is, however,
  only a small number of documents that have keyphrases associated with them.
  Keyphrase extraction has then attracted a lot of attention recently and many
  different approaches were proposed~\cite{hasan2014state_of_the_art}.

  Generally speaking, keyphrase extraction methods can be categorized into two
  main categories: supervised and unsupervised approaches. Supervised approaches
  treat keyphrase extraction as a binary classification task, where each phrase
  is labeled either as ``keyphrase'' or ``non-keyphrase''. Many unsupervised
  methods have been proposed, applying classifiers such as Naive Bayes
  classifiers~\cite{witten1999kea}, multilayer
  perceptrons~\cite{sarkar2010neuralnetwork} and SVMs~\cite{zhang2006svm} with
  various features such as the first position~\cite{witten1999kea}, document
  sections~\cite{nguyen2007keadocumentstructure} and known keyphrase
  distributions~\cite{frank1999keafrequency}. Conversely, unsupervised
  approaches usually rank phrases by importance and select the top-ranked ones
  as keyphrases. Unsupervised approaches proposed so far have involved a number
  of techniques including clustering~\cite{liu2009keycluster}, graph-based
  ranking~\cite{mihalcea2004textrank} and even
  both~\cite{bougouin2013topicrank}.

  In the current state of the keyphrase extraction task, supervised methods
  outperform unsupervised methods. Taking advantage of statistical features
  and/or linguistic features extracted from training documents paired with
  reference keyphrases, supervised methods are able to identify among candidate
  keyphrases the ones most likely to be keyphrases. In opposition, unsupervised
  methods only rely on the analysed document (sometimes using extra information)
  to determine the importance of each candidate keyphrase in regard of the
  document.

  In order to extract the most likely keyphrases without neglecting their actual
  importance within the analysed document, we present a work that combines both
  supervised and unsupervised visions of the keyphrase extraction task. Using
  the unsupervised keyphrase extraction method
  Topic\-Rank~\cite{bougouin2013topicrank}, we group candidate keyphrases that
  represent the same topic, determine the most important topics and apply
  machine learning to extract the most likely keyphrase of each important topic.
  \TODO{Results show that\dots}

%  The rest of this paper is organized as follows. Section~\ref{sec:topicrank}
%  presents TopicRank,
%  Section~\ref{sec:supervised_keyphrase_selection_from_topical_clusters}
%  describes our new approach, Section~\ref{sec:experimental_settings} details
%  our experimentations and Section~\ref{sec:results} presents the evaluation
%  results. Finally, we conclude and discuss about future work in
%  Section~\ref{sec:conclusion_and_future_work}.

