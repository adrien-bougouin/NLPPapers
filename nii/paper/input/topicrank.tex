\section{TopicRank}
\label{sec:topicrank}
  TopicRank aims to extract keyphrases that best represent the main topics
  (concepts or ideas) of a document using both candidate clustering and
  graph-based ranking.

  At first, TopicRank clusters candidate keyphrases that belongs to the same
  topic, considering that candidate of the same topic must share as many words
  as possible. This topical clustering uses a Hierarchical Agglomerative
  Clustering (HAC) with a ``naive'' stem overlap similarity: at the biginning,
  each candidate is a single cluster and candidates sharing an average of
  $\unitfrac{1}{4}$ stemmed words with the candidates of a given cluster are
  added to this cluster.

  Secondly, TopicRank builds a graph of topics and ranks the topics using
  TextRank~\cite{mihalcea2004textrank}. Every topic is connected to other topics
  by edges weighted according to the semantic strength between the connected
  topics. Then, TextRank ranking algorithm gives high importance to topics
  strongly connected to as most topic as possible. Additionally, important
  topics contribute more to the importance of the topics they are strongly
  connected to.

  At last, TopicRank extract keyphrases among the candidates of the $N$ most
  important topics. To avoid topic redundancy, only one keyphrase per topic is
  extracted. Following previous observations, the strategy to extract the best
  keyphrase of a topic takes its candidate that appears first in the document.
  This strategy is currently the weakest link of TopicRank.
  \newcite{bougouin2013topicrank} showed that the best possible performance of
  TopicRank (with a ``perfect'' strategy) can be much higher than its current
  performance. There is still room for improvement.

