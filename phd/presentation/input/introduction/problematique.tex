%\begin{frame}{Introduction}\framesubtitle{Problématique}
%  \begin{itemize}
%    \item{Les termes-clés ne sont pas toujours fournis}
%    \item{Les termes-clés ne sont pas toujours adaptés}
%  \end{itemize}
%\end{frame}
%
%\begin{frame}{Introduction}\framesubtitle{Problématique}
%  \begin{exampleblock}{\small Termes-clés de deux organismes documentaires}\footnotesize
%    \vspace{-.5em}
%    \begin{itemize}
%      \item{Inist~:}
%      \vspace{-1.25em}
%      \begin{multicols}{3}
%        \begin{itemize}
%          \item{Mailhac}
%          \item{Aude}
%          \item{Mourrel-Ferrat}
%          \item{Olonza}
%          \item{Hérault}
%          \item{céramique}
%          \item{\underline{typologie}}
%          \item{\underline{décor}}
%          \item{chronologie}
%          \item{diffusion}
%          \item{production}
%          \item{commerce}
%          \item{répartition}
%          \item{oppidum}
%          \item{analyse}
%          \item{fouille ancienne}
%          \item{le Cayla}
%          \item{micassé}
%          \item{\underline{céramique non-} \underline{tournée}}
%          \item{echange}
%          \item{\underline{age du Fer}}
%          \item{La Tène}
%          \item{Europe}
%          \item{France}
%          \item{celtes}
%          \item{distribution}
%          \item{cartographie}
%          \item{habitat}
%          \item{site fortifié}
%          \item{identification}
%          \item{étude du matériel}
%        \end{itemize}
%      \end{multicols}
%      \vspace{-1em}
%      \item{Revues.org~:}
%      \vspace{-1.25em}
%      \begin{multicols}{3}
%        \begin{itemize}
%          \item{\underline{typologie}}
%          \item{\underline{décors}}
%          \item{\underline{céramique non} \underline{tournée}}
%          \item{\underline{âge du Fer}}
%          \item{Protohistoire}
%          \item{deuxième âge du Fer}
%          \item{commerce et échanges}
%          \item{Languedoc occidental}
%        \end{itemize}
%      \end{multicols}
%      \vspace{-1em}
%    \end{itemize}
%  \end{exampleblock}
%\end{frame}

\begin{frame}{Introduction}\framesubtitle{Problématique}
  Comment identifier automatiquement les termes-clés d'un document~?
  \begin{enumerate}
    \item{Dans le contexte général}
    \item{En domaines de spécialité}
  \end{enumerate}

  \vspace{1em}

  \large\textbf{$\Rightarrow$ Indexation (automatique) par termes-clés}
\end{frame}

