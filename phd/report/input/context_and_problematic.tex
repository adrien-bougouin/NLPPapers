  \chapter{Contexte et problématique}
    \section{Contexte}
      %% Expression du besoin
      Avec l'essor du numérique, le Web occupe aujourd'hui une place importante
      dans notre société. Celui-ci contient tous types d'informations
      (culturelles, historiques, scientifiques, etc.) qu'il rend disponibles
      pour tous. Cependant, le Web est en constante expansion et le nombre
      croissant d'informations disponibles compliquent leur accès (leur
      recherche). Pour résoudre ce problème, il nous faut nous donner les moyens
      de représenter et d'organiser efficacement les documents numériques. Dans
      le contexte de la recherche scientifique, résoudre ce problème est
      d'autant plus important que la recherche a un impact sur le développement
      des pays. Favoriser l'accès aux productions scientifiques, que ce soit au
      niveau mondial ou national, favorise les avancées scientifiques et
      contribue donc au développement des pays. C'est dans ce but que sont
      créées les bibliothèques numériques, telles que la Bibliothèque
      Scientifique Numérique (BSN) fondée en 2009 par le ministère de
      l'enseignement supérieur et de la recherche français.

      %% Réponse à ce besoin (introduction de la notion de termes-clés)
      Pour comprendre comment l'accès aux informations est facilité, alors même
      que leur quantité augmente, nous pouvons prendre l'exemple de l'Institut
      de l'Information Scientifique et Technique (InIST) dont les activités
      s'organisent dans le cadre de la BSN. \TODO{L'InIST construit de notices
      bibliographiques.}\TODO{À quoi servent les notices ?}\TODO{Introduire les
      descripteur $\rightarrow$ termes-clés.}\TODO{Revenir sur la quantité de
      données et le besoin d'automatiser le travail de l'InIST.}

      %% Parallèle entre assignation et extraction de termes-clés
      % TODO dire que l'assignation doit être fondée sur les candidats ?

      %%

      %%

    \section{Problématique}

