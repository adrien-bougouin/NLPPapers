  \chapter{Contexte et problématique}
    \section{Contexte}
      %% Expression du besoin
      Avec l'essor du numérique, le Web occupe aujourd'hui une place importante
      dans notre société. Celui-ci contient tous types d'informations
      (culturelles, historiques, scientifiques, etc.) qu'il rend disponibles
      pour tous. Cependant, le Web est en constante expansion et le nombre
      croissant d'informations disponibles compliquent leur accès, leur
      recherche. Pour résoudre ce problème, il nous faut nous donner les moyens
      de représenter et d'organiser efficacement les documents numériques. Dans
      le contexte de la recherche scientifique, résoudre ce problème est
      d'autant plus important que la recherche a un impact sur le développement
      des pays. Favoriser l'accès aux productions scientifiques, que ce soit au
      niveau mondial ou national, favorise les avancées scientifiques et
      contribue donc au développement des pays. C'est dans ce but que sont
      créées les bibliothèques numériques telles que la Bibliothèque
      Scientifique Numérique (\textsc{Bsn}) fondée en 2009 par le ministère de
      l'enseignement supérieur et de la recherche français.

      %% Réponse à ce besoin (introduction de la notion de termes-clés)
      Afin de mieux comprendre comment l'accès aux informations peut être
      facilité, prenons l'exemple de l'Institut de l'Information Scientifique et
      Technique (Inist) dont les activités s'organisent aujourd'hui dans le
      cadre de la \textsc{Bsn}. Créé en 1988, l'Inist possède l'une des plus
      importantes collections de publications scientifiques d'Europe et fournit
      plusieurs services pour la recherche d'information, dont le maintient de
      bases de données bibliographiques. Ces bases, organisées par thématiques
      (p.~ex.~la base Francis pour les domaines des lettres et des sciences
      humaines et sociales~---~\textsc{Shs}), contiennent des notices
      bibliographiques qui décrivent le contenu de chaque production
      scientifique. Ce sont les descriptions de ces notices bibliographiques qui
      facilitent la recherche d'information, en permettant d'établir une
      correspondance entre un besoin d'un utilisateur (p. ex. un chercheur) et
      certains documents. Elles incluent des connaissances explicitées dans les
      documents, telles que le titre et les auteurs, mais aussi des
      connaissances qui ne le sont pas toujours~: le résumé et les mots-clés,
      que nous décidons d'appeler termes-clés afin d'éviter toute ambiguïté sur
      leur nature~---~monolexicale ou polylexicale. Lorsque les résumés ne sont
      pas fournis par les auteurs des documents, ce sont des ingénieurs
      documentalistes (indexeurs professionnels) qui les rédigent. Quand aux
      termes-clés, qu'ils soient fournis ou non par les auteurs, les indexeurs
      professionnels en fournissent toujours de nouveaux, cela afin que les
      notices soient homogènes (le même terme-clé pour représenter le même
      concept) et aussi afin d'assurer que le vocabulaire utilisé soit, autant
      que possible, issu du vocabulaire de la discipline des documents.

      %% Besoin d'aller vers une solution automatique (surcharge => travail
      %% baclé)

      %% 2 types d'indexations

      %% 2 catégories de méthodes

    \section{Problématique}
      %% Projet ANR

