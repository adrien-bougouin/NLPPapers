\chapter{Soumissions et publications}
  Dans ce chapitre, nous énumérons et décrivons brièvement les travaux
  publiés, ou en cours de relecture, dans des conférences ou des revues.

  \section{\textit{État de l'art des méthodes d'extraction automatique de
           termes-clés} (cf. Annexe~\ref{sec:recital2013})}
    Cet article a été publié en 2013 dans la Rencontre des Étudiants Chercheurs
    en Informatique pour le Traitement Automatique des Langues (RECITAL). Il
    s'agit d'une version courte de l'état de l'art présenté dans le
    chapitre~\ref{chap:etat_de_l_art}.

  \section{\textit{TopicRank: Graph-Based Topic Ranking for Keyphrase
           Extraction} (cf. Annexe~\ref{sec:ijcnlp2013})}
  \label{sec:resume_ijcnlp2013}
    Cet article a été publié en 2013 dans la conférence internationale IJCNLP.
    En nous appuyant sur les faiblesses des méthodes à base de graphe pour
    l'extraction automatique de termes-clés (ordonnancement fondé sur les mots,
    redondance des termes-clés extraits, etc), nous avons proposé TopicRank.
    Dans un premier temps, TopicRank regroupe les termes-clés candidats en
    sujets. Dans un second temps, il ordonne ces sujets par importance selon la
    méthode TextRank et sélectionne dans chacun des meilleurs sujets le
    terme-clé candidat qui le représente le mieux.

  \section{\textit{Influence des domaines de spécialité dans l'extraction de
           termes-clés} (cf. Annexe~\ref{sec:taln2014})}
    Cet article à été accepté pour une présentation dans la conférence TALN qui
    aura lieu en Juillet 2014. À partir des cinq corpus disciplinaires
    constitués dans le cadre du projet Termith et de deux méthodes d'extraction
    automatique de termes-clés, nous montrons que certaines disciplines sont
    plus difficiles à traiter que d'autres et proposons une analyse des facteurs
    influant sur cette difficulté.

  \section{\textit{TopicRank : ordonnancement de sujets pour l'extraction
           automatique de termes-clés} (cf. Annexe~\ref{sec:tal2014})}
    Cet article, soumis (et accepté en seconde relecture) à la revue TAL, est
    une version étendue de l'article présenté à IJCNLP (cf.
    Section~\ref{sec:resume_ijcnlp2013}). Il présente des expériences
    supplémentaires justifiant nos choix pour le groupement en sujets et la
    sélection du candidat le plus représentatif par sujet, ainsi qu'une analyse
    des erreurs lors du groupement en sujets.

  \section{\textit{Selecting Candidates for Automatic Keyphrase Extraction} (cf.
           Annexe~\ref{sec:coling2014})}
    Dans cet article, soumis à la conférence COLING, nous proposons une nouvelle
    méthode de sélection de termes-clés candidats. Fondée sur les hypothèses que
    (1) tous les adjectifs ne contribuent pas au sens d'un terme-clé candidat
    et (2) les adjectifs relationnels sont, par leur nature taxonomique, plus
    pertinents en tant que modifieurs dans un terme-clé, notre méthode n'accepte
    dans les candidats de la forme Nom(s)-Adjectif que les adjectifs dénominaux
    (l'une des propriétés des adjectifs relationnels) et les adjectifs qui
    co-occurrent plus d'une fois avec le(s) nom(s) du candidat (ils contribuent
    au sens du candidat).

