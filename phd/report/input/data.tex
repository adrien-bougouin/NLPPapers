\chapter{Présentation des données}
  \section{Données du projet Termith}
    %% Présentation des collections 
    Dans le cadre du projet Termith, nous disposons de cinq collections de
    notices bibliographiques fournies par l'Inist. Ces cinq collections
    représentent cinq disciplines~: l'archéologie, les sciences de
    l'information, la linguistique, la psychologie et la chimie. Le corpus
    d'archéologie est composé de 718 notices représentant des articles français
    parus entre 2001 et 2012 dans 22 revues (\textit{Paléo}, \textit{Le bulletin
    de la Société préhistorique française}, etc.)~; Le corpus des sciences de
    l'information contient 706 notices d'articles français publiés entre 2001 et
    2012 dans six revues (\textit{Documentaliste -- Sciences de l'information},
    \textit{Document numérique}, etc.)~; Le corpus de linguistique est constitué
    de 715 notices d'articles français parus entre 2000 à 2012 dans 12 revues
    (\textit{Linx~---~Revue des linguistes de l'Université Paris Ouest Nanterre
    La Défense}, \textit{Travaux de linguistique}, etc.)~; Le corpus de
    psychologie contient 720 notices d'articles français publiés entre 2001 et
    2012 dans sept revues (\textit{Enfance}, \textit{Revue internationale de
    psychologie et de gestion des comportements organisationnels}, etc.)~; Le
    corpus de chimie est composé de 782 notices d'articles français publiés
    entre 1983 et 2012 dans quatre revues (\textit{Comptes Rendus de l'Académie
    des Sciences}, \textit{Comptes Rendus Chimie}, etc.).
    
    %% Explication du procesus decréation des notices
    Chaque notice contient le titre, le résumé et les termes-clés associés au
    document qu'elle représente. Au total, il peut y avoir quatre ensembles de
    termes-clés différents~: les termes-clés des auteurs et les termes-clés des
    indexeurs de l'Inist en français, en anglais et en espagnole. Parmi ces
    quatre ensembles, nous prenons pour référence les termes-clés français
    fournis par les indexeurs de l'Inist, car notre objectif est d'automatiser
    l'indexation telle qu'elle est effectuée à l'Inist. Le processus
    d'indexation de l'Inist se déroule en deux étapes~: la reconnaissance des
    concepts contenant l'information dans les documents à indexer, puis la
    représentation de ces concepts dans le language documentaire (le vocabulaire
    contrôlé doit être utilisé en priorité). Ce processus fourni donc aussi bien
    des termes-clés présents dans les documents que des termes-clés qui n'y sont
    pas.

    Le tableau~\ref{tab:statistiques_des_corpus} présente les caractéristiques
    principales des collections de notices présentées ci-dessus. Elles sont de
    petite taille et sont rédigées différemment selon les disciplines
    (cf.~figure~\ref{fig:exemple_notice_inist}). Les notices d'archéologie, par
    exemple, font l'objet d'un effort de présentation du contexte historique lié
    aux travaux présentés, tandis que les notices de chimie, principalement des
    comptes rendus d'expériences, décrivent sommairement les expériences
    réalisées (noms des expériences, éléments chimiques impliqués, etc.). Les
    termes-clés associés aux documents varient en nombre et en complexité. En
    archéologie, par exemple, nous observons qu'un grand nombre de termes-clés
    sont des entités nommées principalement composées d'un seul mot
    (p. ex.~\og{}Paléolithique\fg{}, \og{}Europe\fg{}, etc.), tandis qu'en
    chimie, nous observons un usage fréquent de notions générales (dans la
    discipline) nécessitant une spécialisation presque systématique
    (p. ex..~\og{}\underline{réaction} topotactique\fg{},
    \og{}\underline{réaction} sonochimique\fg{}, \og{}\underline{réaction}
    électrochimique\fg{}, etc.). Enfin, il est important de noter la faible
    proportion de termes-clés apparaissant dans les notices -- rappel maximum
    pouvant être obtenu. Par exemple, dans le corpus des sciences de
    l'information, uniquement 2,8 termes-clés peuvent être extraits des notices
    parmi les 8,7 associés aux notices, en moyenne. Il est donc important de ne
    pas utiliser que le contenu du résumé des notices pour extraire les
    termes-clés.

    \begin{table}
      \centering
      \begin{tabular}{@{~}r|c@{~~}c@{~~}c@{~~}c@{~~}c@{~}}
        \toprule
        & & \textbf{Sciences} & & &\\ \textbf{Statistique} & \textbf{Archéologie} & \textbf{de} & \textbf{Linguistique} & \textbf{Psychologie} & \textbf{Chimie}\\ & & \textbf{l'Information} & & &\\
        \hline
        Documents & 718 & 706 & 715 & 720 & 782\\
        Mots/doc. & 219,1 & 119,7 & 156,7 & 185,7 & 105,2\\
        Termes-clés/doc. & 16,6 & 8,5 & 8,7 & 11,6 & 12,8\\
        Mots/terme-clé & 1,3 & 1,7 & 1,8 & 1,6 & 2,2\\
        Rappel max. & 62,9~\% & 32,4~\% & 38,8~\% & 27,1~\% & 23,7~\%\\
        \bottomrule
      \end{tabular}
      \caption{Caractéristiques des cinq corpus disciplinaires. La ligne
               \textit{Rappel max.} indique le pourcentage de termes-clés
               pouvant être extraits à partir du titre ou du résumé des notices.
               Cela donne un aperçu des performances maximales que peuvent
               atteindre des méthodes tenant uniquement compte du contenu des
               documents traités.
               \label{tab:statistiques_des_corpus}}
    \end{table}

    \begin{figure}
      \framebox[\linewidth]{ % archeologie_09-0054907
        \parbox{.99\linewidth}{\small
          \textbf{Variabilité du Gravettien de Kostienki (bassin moyen du Don)
          et des ter-}
          ~\hfill\underline{\textit{Archéologie}}\\
          \textbf{ritoires associés}\\

          Dans la région de Kostienki-Borschevo, on observe l'expression, à ce
          jour, la plus orientale du modèle européen de l'évolution du
          Paléolithique supérieur. Elle est différente à la fois du modèle
          Sibérien et du modèle de l'Asie centrale. Comme ailleurs en Europe, le
          Gravettien apparaît à Kostienki vers 28 ka (Kostienki 8 /II/). Par la
          suite, entre 24-20 ka, les techno-complexes gravettiens sont
          représentés au moins par quatre faciès dont deux, ceux de Kostienki
          21/III/ et Kostienki 4 /II/, ressemblent au Gravettien occidental et
          deux autres, Kostienki-Avdeevo et Kostienki 11/II/, sont des faciès
          propres à l'Europe de l'Est, sans analogie à l'Ouest.\\

          \textbf{Termes-clés~:} \underline{Europe}, Kostienko,
          \underline{Borschevo}, variation, typologie, industrie osseuse,
          industrie lithique, Europe centrale, \underline{Avdeevo},
          \underline{Paléolithique supérieur}, \underline{Gravettien}.
        }
      }
      ~\\~\\
      \framebox[\linewidth]{ % linguistique_08-0265302
        \parbox{.99\linewidth}{\small
          \textbf{Termes techniques et marqueurs d'argumentation : pour
          débusquer}
          \hfill\underline{\textit{Linguistique}}\\
          \textbf{l'argumentation cachée dans les articles de recherche}\\

          Les articles de recherche présentent les résultats d'une expérience
          qui modifie l'état de la connaissance dans le domaine concerné. Le
          lecteur néophyte a tendance à considérer qu'il s'agit d'une simple
          description et à passer à côté de l'argumentation au cours de laquelle
          le scientifique cherche à convaincre ses pairs de l'innovation et de
          l'originalité présentées dans l'article et du bien-fondé de sa
          démarche tout en respectant la tradition scientifique dans laquelle il
          s'insère. Ces propriétés spécifiques du discours scientifique peuvent
          s'avérer un obstacle supplémentaire à la compréhension, surtout
          lorsqu'il s'agit d'un article en langue étrangère. C'est pourquoi il
          peut être utile d'incorporer dans l'enseignement   des langues de
          spécialité une sensibilisation aux marqueurs linguistiques
          (terminologiques et argumentatifs), qui permettent de dépister le
          développement de cette rhétorique. Les auteurs s'appuient sur deux
          articles dans le domaine de la microbiologie.\\

          \textbf{Termes-clés~:} Langue scientifique,
          \underline{argumentation},  \underline{rhétorique},  \underline{langue
          de spécialité}, \underline{enseignement des langues}, linguistique
          appliquée,  \underline{discours scientifique},  \underline{article de
          recherche}. 
        }
      }
      ~\\~\\
      \framebox[\linewidth]{ % chimie_90-0137940
        \parbox{.99\linewidth}{\small
          \textbf{Etude d'un condensat acide
          isocyanurique-urée-formaldéhyde}
          \hfill\underline{\textit{Chimie}}\\

          La synthèse d'un condensat acide isocyanurique-urée-formaldéhyde
          utilisant la pyridine en tant que solvant a été effectuée par réaction
          sonochimique.\\

          \textbf{Termes-clés~:} \underline{Réaction sonochimique}, hétérocycle
          azote, cycle 6 chaînons, ether.
        }
      }
      \caption{Exemple de notices Inist. Les termes-clés soulignés sont ceux qui
               peuvent être extraits depuis le titre et le résumé des notices.
               \label{fig:exemple_notice_inist}}
    \end{figure}

  \section{Autres données}
    Dans le cadre de publications internationales, nous utilisons aussi d'autres
    collections de données~: Inspec, \textsc{Duc}, SemEval, WikiNews et
    \textsc{Deft}. Parmi ces cinq collections de données, Inspec et
    \textsc{Duc} sont issues de travaux
    ultérieurs~\citep{hulth2003keywordextraction,wan2008expandrank} et SemEval
    et \textsc{Deft} sont issues de campagnes
    d'évaluations~\citep{kim2010semeval,paroubek2012deft}. Voici une brève
    description de ces collections~:

    \textbf{Inspec}~\citep{hulth2003keywordextraction} est une collection en
    anglais de résumés d'articles scientifiques provenant de la base de données
    Inspec~\footnote{\url{http://inspecdirect.theiet.org/}}. Cette collections
    contient 2~000 résumés répartis en trois ensemble~: un ensemble
    d'entraînement de 1~000 résumés, un ensemble de développement de 500 résumés
    et un ensemble de test de 500 résumés. Les termes-clés de référence de cette
    collection sont assignés par des indexeurs professionnels. Ils sont
    disponibles en deux versions, l'une pour laquelle les termes-clés sont
    définis librement par les indexeurs, l'autre pour laquelle les termes-clés
    sont contrôlés avec un vocabulaire spécifique.

    \textbf{DUC}~\citep{over2001duc} est une collection en anglais issue des
    données de la campagne d'évaluation DUC-2001. Cette campagne d'évaluation
    concerne les méthodes de résumé automatique, elle ne contient donc
    originellement pas d'annotations en termes-clés. Cependant, les 308 articles
    journalistiques de la partie test de DUC-2001 ont été annotés par
    \cite{wan2008expandrank}.

    \textbf{SemEval}~\citep{kim2010semeval} est la collection en anglais fournie
    lors de la campagne d'évaluation SemEval-2010 pour la tâche d'extraction
    automatique de termes-clés. Cette collection contient 244 articles
    scientifiques (conférences et ateliers) issus de la librairie numérique
    \textsc{Acm}. La collection est répartie en deux sous-ensembles, un ensemble
    de 144 documents d'entraînement et un ensemble de 100 documents de test. Les
    termes-clés assignés par les auteurs et des lecteurs sont disponibles pour
    cette collection.

    \textbf{WikiNews}\footnote{\url{https://github.com/adrien-bougouin/WikinewsKeyphraseCorpus}}
    est une collection de 100 articles journalistiques en français que nous
    avons extraits du site Web WikiNews\footnote{\url{http://fr.wikinews.org/}}
    entre les mois de mai et décembre 2012. Chaque document a été annoté par au
    moins trois étudiants. Les termes-clés des différents étudiants ont été
    groupés et les redondances lexicales ont été automatiquement supprimées.

    \textbf{DEFT}~\citep{paroubek2012deft} est la collection fournie lors de la
    campagne d'évaluation DEFT-2012 pour la tâche d'extraction automatique de
    termes-clés. Celle-ci contient 234 documents en français issus de quatre
    revues de Sciences Humaines et Sociales. La collection est divisée en deux
    sous-ensembles, un ensemble d'entraînement contenant 141 documents et un
    ensemble de test contenant 93 documents.  Seuls les termes-clés des auteurs
    sont disponibles pour cette collection.

