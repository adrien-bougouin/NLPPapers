\chapter{Éléments d'appréciation factuels et personnels}
  \section{Projet professionnel}
    \begin{modules}{professionnels}
      \module{STIMP05}{Méthodologie de la recherche}
      \module{PRESENS14}{Stage DCACE 1ère année - \og{}Pédagogie en enseignement
                         supérieur\fg{}}
    \end{modules}

    \begin{enseignements}
      % 2012-2013
      \enseignement{Technologies Web 1 A}
                   {TP}
                   {11} % 10,4
                   {L3}
                   {LP SIL}
      \enseignement{Projet Tuteuré --- S2}
                   {PRJ}
                   {7} % 7,0
                   {L1}
                   {DUT Informatique}
      \enseignement{Projet Tuteuré --- S2}
                   {PRJ}
                   {7} % 7,0
                   {L1}
                   {DUT Informatique}
      % 2013-2014
      \enseignement{Conc. Doc. Num.}
                   {TP}
                   {14} % 13,2
                   {L1}
                   {DUT Informatique}
      \enseignement{Concept. Objet}
                   {TD}
                   {16} % 16,0
                   {L2}
                   {DUT Informatique}
      \enseignement{Technologies Web 1 A}
                   {TP}
                   {11} % 10,4
                   {L3}
                   {LP SIL}
      \enseignement{CO-concept.}
                   {TP}
                   {11} % 10,4
                   {L1}
                   {DUT Informatique}
      \enseignement{Projet Tuteuré --- S1}
                   {PRJ}
                   {4} % 4,0
                   {L1}
                   {DUT Informatique}
      \enseignement{Projet Tuteuré --- S2}
                   {PRJ}
                   {5} % 5,0
                   {L1}
                   {DUT Informatique}
    \end{enseignements}

  \section{Aspects scientifiques}
    \begin{modules}{scientifiques}
      \module{STIM05}{Estimation des incertitudes}
      \module{STIM02}{Journée des doctorants - Obligatoire pour les doctorants
                      de 2ème année}
    \end{modules}

    \clearpage
    \subsection{Publications}
      \noindent{Titre~: État de l'art des méthodes d'extraction automatique de
                termes-clés}

      \noindent{Auteur~: Adrien Bougouin}

      \noindent{Conférence~: RECITAL 2013}

      \noindent{Présentation~: Poster}\\

      %%%%%%%%%%%%%%%%%%%%

      \noindent{Titre~: TopicRank: Graph-Based Topic Ranking for Keyphrase
                Extraction}

      \noindent{Auteurs : Adrien Bougouin, Florian Boudin et Béatrice Daille}

      \noindent{Conférence : IJCNLP 2013}

      \noindent{Présentation~: Orale}\\

      %%%%%%%%%%%%%%%%%%%%

      \noindent{Titre~: Influence des domaines de spécialité dans l'extraction
                de termes-clés}

      \noindent{Auteurs : Adrien Bougouin, Florian Boudin et Béatrice Daille}

      \noindent{Conférence : TALN 2014}

      \noindent{Présentation~: Orale (à venir)}

    \subsection{Participations à la vie de la recherche}
      \begin{itemize}
        \item{Participation à l'école d'automne EARIA 2012 (École d'Automne en
              Recherche d'Information et Applications)~;}
        \item{Participation à la réunion de démarrage du projet ANR TermITH
              (ANR-12-CORD-0029) du 13 décembre 2012~;}
        \item{Participation à l'organisation de la conférence TALN 2013~;}
        \item{Participation à la réunion du projet ANR TermITH (ANR-12-CORD-0029) du
              15 juillet 2013~;}
        \item{Participation à la réunion du projet ANR TermITH (ANR-12-CORD-0029) du
              10 janvier 2014~;}
        \item{Participation à la réunion du projet ANR TermITH (ANR-12-CORD-0029) du
              24 avril 2014.}
      \end{itemize}


    \subsection{État qualitatif de l'avancement du travail de thèse}
      Niveau d'avancement :
      \begin{itemize}
        \item{Comme prévu.}
      \end{itemize}

      Après la réalisation d'un état de l'art et la proposition d'une nouvelle
      méthode d'extraction automatique de termes-clés, le travail de cette
      seconde année s'est centré sur la sélection des termes-clés candidats,
      c'est-à-dire les unités textuelles traitées par une méthode d'extraction
      automatique de termes-clés. Cette étape est cruciale car elle fixe la
      performance maximale que peut atteindre une méthode d'extraction
      automatique de termes-clés.

