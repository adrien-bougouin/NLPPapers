\chapter{Conclusion et perspectives}
\label{chap:main-conclusion}
  \smallchaptercite{
    [\dots] the task is far from being solved [\dots]
  }{
    \newcite{hasan2014state_of_the_art}
  }

  Dans cette thèse, nous nous sommes intéressé au problème de l'indexation par
  termes-clés en domaines de spécialités. Parce que l'identification des
  termes-clés dépend de critères nombreux et variés (fréquenciels, structurels,
  sémantique, etc.) et parce qu'ils sont parfois implicites aux documents,
  l'indexation automatique par termes-clés est une tâche difficile et les
  performances qu'atteignent les méthodes actuelles sont très en dessous des
  performances d'un grand nombre d'autres applications du
  \textsc{Taln}~\cite{hasan2014state_of_the_art}. Tout d'abord, nous avons fait
  le choix de traiter ce problème dans sa généralité, puis nous nous sommes
  ensuite concentré sur les documents en domaines de spécialités et sur leur
  indexation particulière effectuée par des indexeurs professionnels. Dans le
  chapitre~\ref{chap:main-domain_independent_keyphrase_extraction}, nous avons
  présenté deux contributions à l'extraction automatique de termes-clés~; dans
  le chapitre~\ref{chap:main-domain_specific_keyphrase_annotation}, nous avons
  présenté l'indexation manuelle réalisée en domaine de spécialité et notre
  contribution à son automatisation. Nous avons aussi présenté le protocole et
  les résultats d'une campagne d'évaluation manuelle que nous avons réalisé.

  Notre première contribution à l'extraction de termes-clés concerne l'étape
  préliminaire de sélection des termes-clés candidats. En nous fondant sur une
  analyse des propriétés linguistiques des termes-clés référence et de leurs
  adjectifs, nous avons proposé une méthode qui sélectionne les séquences de
  noms modifiés par un adjectif s'il est jugé utile, non superflu. Un adjectif
  utile se distingue par sa catégorie (relationnel ou composé complexe) et son
  usage fréquent dans le document~; un adjectif superflu est un adjectif
  qualificatif qui modifie un nom utilisé plus fréquemment en autonomie. Pour
  valider ce travail, nous avons réalisé deux évaluations. La première
  évaluation, intrinsèque, compare l'ensemble des termes-clés candidats
  sélectionnés par notre méthode à ceux sélectionnés par les méthodes les plus
  utilisées. Si l'on fait l'hypothèse qu'un ensemble de candidats a une qualité
  d'autant meilleure que son nombre de candidats est réduit et que la
  performance maximale qu'il permet d'atteindre est élevée, alors notre méthode
  fournit des candidats de meilleure qualité que les autres méthodes. La seconde
  évaluation, extrinsèque, compare les performances des deux méthodes
  d'extraction de termes-clés \textsc{Tf-Idf} et \textsc{Kea} lorsqu'elles
  utilisent les candidats de notre méthodes ou d'une des autres méthodes. Les
  meilleures performances qu'obtiennent ces méthodes sont presque toujours
  obtenues avec les candidats que sélectionne notre méthode. Nous avons donc
  présenté une méthode simple qui filtre efficacement les adjectifs superflus
  pour améliorer la sélection des termes-clés candidats en amont de l'extraction
  de termes-clés.

  Notre seconde contribution à l'extraction de termes-clés concerne
  l'ordonnancement par importance des termes-clés candidats à partir du graphe
  de cooccurrences de mots du document. Dans ce travail, nous avons remis en
  cause l'ordonnancement des mots dans le graphe, puisque c'est l'importance des
  termes-clés candidats qui nous intéresse. Pour y remédier, nous avons proposé
  d'utiliser un graphe de sujets représentés par des groupes de termes-clés
  candidats, d'ordonner ces sujets par importance et d'extraire comme terme-clé
  le candidat le plus représentatif de chacun des sujets les plus importants.
  Pour valider ce travail, nous avons comparé la performance de notre méthode,
  TopicRank, à celle d'autres méthodes à base de graphe. Les résultats montrent
  que TopicRank améliore significativement les méthodes à base de graphe
  précédentes. Ses performances restent tout de même très faibles. Pour
  améliorer TopicRank, il reste des pistes que nous n'avons pas explorées. Tout
  d'abord, l'identification des sujets que nous effectuons en groupant les
  termes-clés candidats qui partagent le plus possible de mots ne tient pas
  compte de la sémantique des mots. Nous pourions utiliser des algorithmes pour
  la détection de sujets, tels que \textsc{Lda}~\cite{blei2003lda}. Ensuite, le
  choix du terme-clé candidat le plus représentatif de chaque sujet n'est pas
  optimal. Utiliser de l'apprentissage supervisé pour apprendre à reconnaître le
  représentant d'un sujet est une solution~; utiliser des méthodes d'étiquetage
  de sujets~\cite{lau2011topiclabeling} en est une autre. Cette dernière
  perspective est intéressante, car une méthode générative permettra de proposer
  des termes-clés sous une forme qui n'occurre pas nécessairement dans le
  document.

  Notre contribution à l'indexation par termes-clés en domaines de spécialités
  étend notre seconde contribution pour y ajouter la capacité à assigner des
  termes-clés, à la manière d'un indexeur professionnel. Pour cela, notre
  nouvelle méthode, TopicCoRank, représente le domaine de spécialité du document
  avec un graphe de termes-clés de référence attribués à des documents du même
  domaine, le connecte au graphe de sujets et ordonne conjointement sujets et
  termes-clés de référence. Les termes-clés obtenus à partir du graphe de sujets
  sont extraits et les termes-clés obtenus à partir du graphe du domaine sont
  assignés. À notre connaissance, TopicCoRank est la première méthode qui
  réalise conjointement extraction et assignement de termes-clés. Comparé à
  l'état de l'art, elle donne de meilleures performances en domaines de
  spécialités. Néanmoins, TopicCoRank est encore à un stade préliminaire et nous
  avons plusieurs perspectives d'amélioration. Parmi elles, nous souhaiterions
  étudier différents schémas de connexion des deux graphes, étudier différentes
  pondératons des arêtes et améliorer la généralisation de TopicCoRank à des
  documents autres qu'en domaines de spécialités. Nous avons vu que
  l'unification des deux graphes que nous avons proposé favorise l'assignement
  de termes-clés présents dans le document. Afin de mieux ordonner les autres
  termes-clés du domaine, nous envisageons d'analyser plus en profondeur le
  contenu des documents utilisés pour construire le graphe du domaine. En lien
  avec cette perspective, nous pourrions aussi pondérer les arêtes (internes et
  externes) du graphe unifié avec les probabilités d'occurrences dans le
  document et dans le domaine. Enfin, pour améliorer la généralisation de
  TopicCoRank à tout type de documents, il nous faudra nous intéresser au manque
  d'homogénéité de l'indexation par termes-clés de référence. Modifier notre
  construction du graphe du domaine est une solution, mais nous pensons que ce
  problème est plus lié à l'évaluation automatique.

  Pour conclure, nous avons proposé des méthodes pour améliorer l'extraction de
  termes-clés et l'indexation par termes-clés dans son ensemble (extraction et
  assignement) en nous fondant sur les pratiques documentaires pour le maintient
  de bases de données bibliographiques. Bien que nos travaux donnent des
  résultats intéressants, ceux-ci restent encore relativement faibleset
  \og{}la tâche est loin d'être résolue\fg{}~\cite{hasan2014state_of_the_art}.

