\chapter{Introduction}
\label{chap:main-introduction}
  \chaptercite{
    Seeking information is an important activity for everyone who uses computers
    in daily life. While the Internet is a bountiful source of all types of
    knowledge, locating relevant documents is still a great challenge [\dots]
    Keyphrases help organise documents and retrieve them based on content.
  }{
    \newcite{medelyan2008smalltrainingset}
  }

  %-----------------------------------------------------------------------------

  \section{Contexte}
  \label{sec:main-introduction-context}
    Avec l'essor du numérique, le Web occupe aujourd'hui une place importante
    dans notre société. Celui-ci contient tout type d'information (culturelle,
    historique, scientifique, etc.) qu'il rend disponible pour tous. Cependant,
    le Web est en constante expansion et le nombre croissant d'informations
    disponibles complique leur accès, leur recherche. Pour résoudre ce problème,
    il faut représenter et organiser efficacement les documents numériques. Dans
    le contexte de la recherche scientifique (en domaines de spécialités) il est
    important de résoudre ce problème, car favoriser l'accès aux productions
    scientifiques favorise les avancées scientifiques. C'est pourquoi sont
    créées des bibliothèques numériques telles que la Bibliothèque scientifique
    numérique (\textsc{Bsn}) fondée en 2009 par le ministère de l'enseignement
    supérieur et de la recherche français.

    Afin de mieux comprendre comment l'accès aux informations peut être
    facilité, prenons l'exemple de l'Institut de l'information scientifique et
    technique (Inist), avec lequel nous collaborons et dont les activités
    s'organisent autour de la \textsc{Bsn}. Créé en 1988, l'Inist possède l'une
    des plus importantes collections de publications scientifiques d'Europe et
    fournit plusieurs services pour la recherche d'information (\textsc{Ri}),
    dont le maintient de bases de données bibliographiques. Ces dernières sont
    composées de notices bibliographiques décrivant les documents
    scientifiques~: titre, auteur(s), résumé et termes-clés. Parmi ces
    métadonnées, les termes-clés font partie des plus importantes pour la
    \textsc{Ri}. Ce sont des mots ou des expressions qui représentent le contenu
    principal du document. Ils permettent de résumer le document, de le
    catégoriser et de l'indexer. C'est cette indexation des documents par leur
    termes-clés qui permet de faciliter la recherche d'information~: lorsqu'une
    personne recherche une information particulière, elle formule une requête qui
    est comparée à l'indexation par termes-clés du document afin de déterminer
    s'il est pertinent pour la personne. Au sein de l'Inist, l'indexation par
    termes-clés est réalisée dans le respect des bonnes pratiques
    documentaires~\cite{guinchat1996techniquesdocumentaires}~: les indexeurs
    professionnels ne donnent pas leur opinion au travers des termes-clés
    (impartialité), donnent tous les termes-clés nécessaire à la caractérisation
    du contenu du document (exhaustivité et spécificité) et utilisent un
    vocabulaire contrôlé et identique pour tous les documents du même domaine
    (conformité et homogénéité). 

    L'indexation manuelle des documents par leurs termes-clés est une tâche
    coûteuse et chronophage. Soucieux de faciliter le travail d'indexation par
    termes-clés, que ce soit pour des productions scientifiques ou des documents
    d'autres natures, de nombreux chercheurs s'intéressent à son automatisation,
    en témoignent le nombre grandissant d'articles scientifiques à ce
    sujet~\cite{hasan2014state_of_the_art} ainsi que l'émergence de campagnes
    d'évaluation~\cite{kim2010semeval,paroubek2012deft}. Dans cette thèse, nous
    proposons des méthodes pour l'indexation automatique par termes-clés.
    Travaillant d'abord d'un point de vue généraliste, nous nous focalisons
    ensuite sur l'indexation par termes-clés en domaines de spécialités.

  %-----------------------------------------------------------------------------

  \section{Problématique}
  \label{sec:main-introduction-problem_statement}
    Étant donné un document, l'indexation automatique par termes-clés consiste à
    trouver les unités textuelles qui décrivent son contenu principal. La
    difficulté de cette tâche réside dans l'identification des éléments
    importants vis-à-vis de son contenu, ainsi que leur représentation avec les
    unités textuelles appropriées. La première difficulté est d'ordre
    sémantique~: il faut réussir à comprendre le document pour en extraire
    l'essence~; la seconde est d'ordre linguistique et terminologique~: il faut
    déterminer les propriétés linguistiques des termes-clés et connaître le
    vocabulaire du domaine auquel appartient le document. Par ailleurs, la forme
    la plus appropriée pour un terme-clé n'est pas nécessairement présente
    dans le contenu du document (le terme-clé est implicite).

    Plutôt que de comprendre le document, les méthodes d'indexation par
    termes-clés de la littérature se fondent sur des statistiques et
    des modélisations particulières du document. Pour ce qui est de l'usage
    d'unités textuelles appropriées, elles se contentent le plus souvent des
    unités textuelles qui occurrent dans le document. De manière général, des
    termes-clés candidats sont sélectionnés dans le document d'après des
    critères prédéfinis (ce doit être des groupes nominaux, par exemple), ces
    candidats sont analysés et les termes-clés sont extraits d'entre eux en
    fonction du résultat de l'analyse.

    L'analyse des termes-clés candidats du document peut être réalisée avec deux
    approches~: supervisée ou non supervisée. En général, l'approche supervisée
    consiste à analyser les caractéristiques des termes-clés de données
    manuellement indexées pour apprendre à reconnaître les termes-clés. Elle
    consiste donc à chercher les candidats qui sont le plus vraissemblablement
    les termes-clés, tandis que l'approche non supervisée consiste à chercher
    les candidats les plus importants dans le contenu du document.

    Cette thèse s'inscrit dans le cadre du projet \textsc{Anr} Termith
    (\textsc{Anr-12-Cord-0029}), qui s'intéresse à l'accès à l'information
    numérique en domaines de spécialités. Notre objectif est de proposer une
    méthode d'indexation automatique par termes-clés en domaines de spécialités.
    Dans un premier temps, nous proposons une méthode non supervisée,
    généraliste et applicable dans tous les scénarii d'utilisation. Dans un
    second temps, nous proposons une méthode supervisée adaptée aux domaines de
    spécialités.

  %-----------------------------------------------------------------------------

  \section{Hypothèses}
  \label{sec:main-introduction-hypothesis}
    Notre première hypothèse concerne la sélection des candidats et leur impact
    sur la suite du processus d'indexation par termes-clés. Selon nous,
    l'indexation gagne en efficacité lorsque la qualité de l'ensemble de
    candidats sélectionnés augmente. Cette qualité peut être quantifiée à partir
    de deux critères~: le nombre de candidats sélectionnés et le nombre de
    termes-clés qui s'y trouvent. Paradoxalement, le premier doit être minimisé,
    car un espace de recherche trop grand augmente la difficulté de
    l'indexation~\cite{hasan2014state_of_the_art}, et le second maximisé. Afin
    de trouver le meilleur compromis entre ces deux conditions, nous proposons
    d'analyser plus finement les propriétés linguistiques des termes-clés.
    Identifier ces propriétés doit permettre d'affiner les critères de
    sélection, donc de réduire le nombre de candidats sélectionnés, sans
    éliminer les termes-clés du document.
    
    Notre seconde hypothèse concerne la représentation du document utilisée par
    une catégorie particulière de méthodes non supervisées. Cette catégorie de
    méthodes modélise le document par un graphe de mots connectés entre eux
    lorsqu'ils sont sémantiquement liés, analyse ce graphe afin d'en déterminer
    les mots les plus importants (les mots-clés), puis les utilisent pour
    générer les termes-clés. Selon nous, ce n'est pas l'importance des mots
    qu'il faut déterminer, mais l'importance de ce qu'ils représentent. De plus,
    si plusieurs mots ou expression véhiculent le même sujet, la même idée,
    alors leurs relations sémantiques doivent être mutualisée afin d'améliorer
    la précision de l'analyse.
    
    Enfin, notre troisième hypothèse concerne l'usage des données indexées
    manuellement. Alors que dans la littérature, ces données servent à apprendre
    à identifier les termes-clés parmi les candidats selon diverses
    caractéristiques discriminantes, nous pensons qu'elles peuvent aussi servir
    à contextualiser le document à indexer dans son domaine. Selon nous, cette
    contextualisation permet d'identifier plus précisément les éléments
    importants du document s'ils sont importants dans son contexte global
    (domaine), et même de faire émerger des termes-clés implicites. Si cette
    hypothèse est juste, alors la contextualisation est d'autant plus importante
    en domaines de spécialités, car elle favorise un indexation par termes-clés
    respectueuse des pratiques
    documentaires~\cite{guinchat1996techniquesdocumentaires}.

  %-----------------------------------------------------------------------------

  \section{Plan de thèse}
  \label{sec:main-introduction-outline}
    Cette thèse est organisée de la manière suivante. Tout d'abord, le
    chapitre~\ref{chap:main-state_of_the_art} présente l'état de l'art en
    indexation automatique par termes-clés, puis le
    chapitre~\ref{chap:main-data_description} introduit les données avec
    lesquelles nous travaillons. Nos contributions sont détaillées dans les
    chapitres~\ref{chap:main-domain_independent_keyphrase_extraction}
    et~\ref{chap:main-domain_specific_keyphrase_annotation}. Le
    chapitre~\ref{chap:main-domain_independent_keyphrase_extraction} présente
    nos travaux fondés sur les deux premières hypothèses et le
    chapitre~\ref{chap:main-domain_specific_keyphrase_annotation} sur la
    troisième hypothèse. Ce dernier ce concentre sur l'indexation en domaine de
    spécialité. Il présente tout d'abord l'indexation manuelle réalisée par les
    indexeurs professionnel, puis notre contribution et enfin, une campagne
    d'évaluation manuelle de nos travaux en domaines de spécialités. Pour
    terminer, le chapitre~\ref{chap:main-conclusion} dresse le bilan de notre
    travail et présente quelques perspectives.

