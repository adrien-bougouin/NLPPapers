\chapter{Introduction}
\label{chap:main-introduction}
  \chaptercite{
    Rechercher des informations est une activité fréquente pour quiconque
    utilise quotidiennement un ordinateur. Alors qu'Internet est une source
    abondante d'information de tout genre, trouver les documents pertinents est
    encore difficile. [...] Les termes-clés aident à organiser et retrouver ces
    documents d'après leur contenu.
%    Seeking information is an important activity for everyone who uses computers
%    in daily life. While the Internet is a bountiful source of all types of
%    knowledge, locating relevant documents is still a great challenge [\dots]
%    Keyphrases help organise documents and retrieve them based on content.
  }{
    \newcite{medelyan2008smalltrainingset}
  }{.75\linewidth}{\justify}

  %-----------------------------------------------------------------------------

  \section{Contexte}
  \label{sec:main-introduction-context}
    La société contemporaine dans laquelle nous vivons se situe en pleine ère de
    l'information. Cette ère succède l'ère moderne, durant laquelle de
    nombreuses découvertes et avancées scientifiques ont été faites~; durant
    laquelle des connaissances considérables ont été acquises. Elle  est aussi
    marquée par le début de la mondialisation, qui, sur le plan scientifique,
    favorise la dissémination et la production de nouvelles connaissances.
    Jusqu'alors rangées sous la forme de documents papiers dans des
    bibliothèques, où des documentalistes les indexent et aident ensuite
    scientifiques et particuliers à y accèder le plus efficacement possible, les
    connaissances sont devenues trop nombreuses et leur stockage physique
    inadapté~\cite{rider1946thegreatdilemmaofworldorganization}. L'ère de
    l'information débute vers la fin des années 1940 et apporte une solution à
    ce problème~: l'informatisation des données. Cette informatisation présente
    tout d'abord l'avantage de pouvoir stocker les connaissances sur des
    supports pérennes, de capacité de plus en plus grande (de quelques
    mégaoctets à plusieurs gigaoctets) et de taille de plus en plus réduite (du
    disque dur au \textsc{Dvd}). Très vite, la communauté scientifique y voit
    aussi un moyen pour améliorer la recherche d'information, en indexant les
    documents qui contiennent les connaissances, en proposant des interfaces
    pour permettre à un utilisateur de formuler une requète et en cherchant les
    documents pertinents vis-à-vis de cette requète~\cite{salton1975tfidf}.

    Produit de cette ère de l'information, le réseau informatique mondial
    Internet en est aussi devenu l'un des acteurs principaux. En effet, si
    l'informatisation et l'indexation des données facilite leur recheche,
    Internet facilite leur accès depuis les bases de données informatisées qui y
    sont connectées. Médium d'information mondial et accessible de
    tous\footnote{En 2012, l'Institut national de la statistique et des études
    économiques (Insee) estimait qu'environ 80~\% des français sont connectés à
    Internet.}, il favorise donc la transition depuis les bibliothèques
    traditionnelles vers des bibliothèques numériques. Ces dernières combinent
    le savoir faire des documentalistes avec les techniques du Traitement
    automatique des langues (\textsc{Tal}) et de la Recherche d'information
    (\textsc{Ri}) pour informatiser les données et faciliter leur accès et leur
    recherche.

    Cette thèse s'inscrit dans le cadre du projet \textsc{Anr} Termith
    (\textsc{Anr-12-Cord-0029}), qui s'intéresse à l'accès à l'information
    numérique en domaines de spécialité et qui s'articule lui même autour du
    travail de l'Institut de l'information scientifique et technique (Inist). Né
    en 1988 de la fusion du Centre de documentation scientifique et technique
    (\textsc{Cdst}) et du Centre de documentation sciences humaines
    (\textsc{Cdsh}), tout deux fondés en 1970 pendant les débuts de
    l'informatisation des données, l'Inist possède deux des plus importantes
    bases de données informatisées d'Europe~: \textsc{Pascal} en sciences
    exactes et \textsc{Francis} en sciences humaines. Aujourd'hui acteur de la
    Bibliothèque scientifique numérique (\textsc{Bsn}) fondée en 2009 par le
    ministère de l'enseignement supérieur et de la recherche français, l'une de
    ses missions est de faciliter l'accès à la recherche mondiale au travers de
    la production de notices bibliographiques associées à des mots-clés, que
    nous appelons ici termes-clés. \TODO{c'est devenu difficile}

    Soucieux de faciliter le travail d'indexation par termes-clés de toute sorte
    de document (résumé d'une notice bibliographiqe, article scientifique,
    article journalistique, nouvelle, etc.) et pour toute sorte d'application
    (indexation, résumé, publicité ciblée, etc.), de nombreux chercheurs
    s'intéressent à son automatisation. En témoignent le nombre grandissant
    d'articles scientifiques à ce sujet~\cite{hasan2014state_of_the_art} ainsi
    que l'émergence de campagnes
    d'évaluation~\cite{kim2010semeval,paroubek2012deft}.

  %-----------------------------------------------------------------------------

  \section{Problématique}
  \label{sec:main-introduction-problem_statement}
    Étant donné un document, l'indexation automatique par termes-clés consiste à
    trouver les unités textuelles qui décrivent son contenu principal. La
    difficulté de cette tâche réside dans l'identification des éléments
    importants vis-à-vis de son contenu, ainsi que leur représentation avec les
    unités textuelles appropriées. La première difficulté est d'ordre
    sémantique~: il faut réussir à comprendre le document pour en extraire
    l'essence~; la seconde est d'ordre linguistique et terminologique~: il faut
    déterminer les propriétés linguistiques des termes-clés et connaître le
    vocabulaire du domaine auquel appartient le document. Par ailleurs, la forme
    la plus appropriée pour un terme-clé n'est pas nécessairement présente dans
    le contenu du document, elle peut être implicite.

    Plutôt que de comprendre le document, les méthodes d'indexation par
    termes-clés de la littérature se fondent sur des statistiques et
    des modélisations particulières du document. Pour ce qui est de l'usage
    d'unités textuelles appropriées, elles se contentent le plus souvent de
    celles qui occurrent dans le document. De manière général, des
    termes-clés candidats sont sélectionnés dans le document d'après des
    critères prédéfinis (par exemple, ce doit être des groupes nominaux), ces
    candidats sont analysés et les termes-clés sont extraits d'entre eux en
    fonction du résultat de l'analyse.

    L'analyse des termes-clés candidats du document peut être réalisée avec deux
    approches~: supervisée ou non supervisée. En général, l'approche supervisée
    consiste à analyser les caractéristiques des termes-clés de données
    manuellement indexées pour apprendre à reconnaître les termes-clés. Elle
    consiste donc à chercher les candidats qui sont le plus vraissemblablement
    les termes-clés, tandis que l'approche non supervisée consiste à chercher
    les candidats les plus importants dans le contenu du document.
    
    \TODO{Notre objectif est de}
    Dans cette thèse, nous proposons des méthodes pour l'indexation automatique
    par termes-clés. Travaillant d'abord d'un point de vue généraliste, nous
    nous focalisons ensuite sur l'indexation par termes-clés de notices
    bibliographiqe en domaines de spécialité.
    \TODO{nos travaux sur le fr et en}

  %-----------------------------------------------------------------------------

  \section{Hypothèses}
  \label{sec:main-introduction-hypothesis}
    Notre première hypothèse concerne la sélection des termes-clés candidats et
    leur impact sur la suite du processus d'indexation par termes-clés. Selon
    nous, l'indexation gagne en efficacité lorsque la qualité de l'ensemble des
    candidats sélectionnés augmente.
    %
    Il s'agit là d'une hypothèse triviale~: si l'un des composants d'une chaîne
    de traitement fait des erreurs, alors celles-ci peuvent se répercuter sur
    les autres composants et dégrader leur performance. Cependant, de nombreux
    travaux utilisent encore des méthodes de sélection de candidats grossières,
    ou des filtres linguistiques suffisant pour sélectionner des candidats
    de la même forme que les termes-clés, mais produisant un nombre important 
    d'erreurs. Ces méthodes présentent l'avantage d'être facile à mettre en
    \oe{}uvre pour des performances d'indexation par termes-clés satisfaisantes.
    \TODO{mais maintenant quand même}
    %
    Pour améliorer la qualité de la sélection, nous pensons qu'il faut
    s'intéresser à deux propriétés de l'ensemble de candidats sélectionnés~: le
    nombre de termes-clés qui se trouvent parmi les candidats et le nombre total
    de candidats sélectionnés. Paradoxalement, le premier doit être maximisé,
    tandis que le second doit être minimisé, car un espace de recherche trop
    grand augmente la difficulté de
    l'indexation~\cite{hasan2014state_of_the_art}.
    %
    Notre objectif est donc de trouver des propriétés linguistiques plus fines
    afin d'obtenir le meilleur compromis entre ces deux conditions.
    
    Notre seconde hypothèse concerne la détection des mots et expressions
    importants vis-à-vis d'un document. Selon nous ce n'est pas l'importance de
    ces mots et expressions qui doit être déterminée, mais l'importance de ce
    qu'ils représentent. Nous parlons de sujet.
    %
    Les sujets abordés dans un document sont effectivement véhiculés par des
    unités textuelles. Il faut donc en déterminer l'importance en analysant
    l'usage de ces unités textuelles. Cependant, un sujet n'est pas toujours
    représenté par une unique unité textuelle. Si plusieurs unités textuelles
    sont utilisées pour représenter le même sujet dans un document, alors
    déterminer indépendamment l'importance de chaque unité textuelle engendre au
    moins 2 types d'erreurs~:
    \begin{enumerate}
      \item{Redondance~: plusieurs termes-clés proposés peuvent représenter le
            même sujet~;}
      \item{Imprécision~: pour chaque unité textuelle, l'importance du sujet est
            différente car elle ne tient pas compte de toute les références à ce
            sujet dans le document.}
    \end{enumerate}
    %
    Notre objectif est donc de mutualiser l'analyse des unités textuelles qui
    véhiculent les mêmes sujets afin d'éviter la redondance et de mieux capturer
    l'importance de ses sujets.
    
    Enfin, notre troisième hypothèse concerne l'usage des données indexées
    manuellement pour l'indexation automatique d'un document du même domaine que
    celles-ci. Nous pensons qu'il est possible de tirer profit de ces données
    pour (1) améliorer la précision de l'identification des unités textuelles
    importantes en situant le document dans son contexte global et (2)
    assigner des termes-clés du domaine importants vis-à-vis du document.
    %
    La première perspective d'amélioration est générique à tout document, tandis
    que la seconde se fonde plus sur l'indexation par termes-clés pratiquée en
    domaines de spécialité par les documentalistes (indexeurs professionnels). Utiliser des
    données déjà indexées doit permettre de proposer des termes-clés conformes
    au langage documentaire employé pour le domaine auquel appartient le
    document. De plus, cela peut résoudre le problème des termes-clés implicite
    au document.
    %
    Notre objectif est donc de trouver une représentation unifiant celle du
    document à celle de son domaine, puis de proposer une méthode d'analyse
    capable d'identifier les unités textuelles importante vis-à-vis du document
    et du domaine.

  %-----------------------------------------------------------------------------

  \section{Mise en \oe{}uvre}
  \label{sec:main-introduction-realisation}
  Les contributions que nous présentons dans cette thèse sont fondées sur les
  hypothèses que nous venons d'exprimer. Nous proposons trois
  contributions\TODO{...}

  %-----------------------------------------------------------------------------

  \section{Plan de thèse}
  \label{sec:main-introduction-outline}
    Cette thèse est organisée de la manière suivante. Tout d'abord, le
    chapitre~\ref{chap:main-state_of_the_art} présente l'état de l'art en
    indexation automatique par termes-clés, puis le
    chapitre~\ref{chap:main-data_description} introduit les données avec
    lesquelles nous travaillons. Nos contributions sont détaillées dans les
    chapitres~\ref{chap:main-domain_independent_keyphrase_extraction}
    et~\ref{chap:main-domain_specific_keyphrase_annotation}. Le
    chapitre~\ref{chap:main-domain_independent_keyphrase_extraction} présente
    nos travaux fondés sur les deux premières hypothèses et le
    chapitre~\ref{chap:main-domain_specific_keyphrase_annotation} sur la
    troisième hypothèse. Ce dernier ce concentre sur l'indexation en domaine de
    spécialité. Il présente tout d'abord l'indexation manuelle réalisée par les
    indexeurs professionnel, puis notre contribution et enfin, une campagne
    d'évaluation manuelle de nos travaux en domaines de spécialité. Pour
    terminer, le chapitre~\ref{chap:main-conclusion} dresse le bilan de notre
    travail et présente quelques perspectives.

