\chapter{Introduction}
\label{chap:main-introduction}
  \chaptercite{
    Seeking information is an important activity for everyone who uses computers
    in daily life. While the Internet is a bountiful source of all types of
    knowledge, locating relevant documents is still a great challenge [\dots]
    Keyphrases help organise documents and retrieve them based on content.
  }{
    \newcite{medelyan2008smalltrainingset}
  }

  %-----------------------------------------------------------------------------

  \section{Contexte}
  \label{sec:main-introduction-context}
    Avec l'essor du numérique, le Web occupe aujourd'hui une place importante
    dans notre société. Celui-ci contient tout type d'information (culturelle,
    historique, scientifique, etc.) qu'il rend disponible pour tous. Cependant,
    le Web est en constante expansion et le nombre croissant d'informations
    disponibles complique leur accès, leur recherche. Pour résoudre ce problème,
    il faut représenter et organiser efficacement les documents numériques. Dans
    le contexte de la recherche scientifique (en domaines de spécialités) il est
    important de résoudre ce problème, car favoriser l'accès aux productions
    scientifiques favorise les avancées scientifiques. C'est pourquoi sont
    créées des bibliothèques numériques telles que la Bibliothèque scientifique
    numérique (\textsc{Bsn}) fondée en 2009 par le ministère de l'enseignement
    supérieur et de la recherche français.

    Afin de mieux comprendre comment l'accès aux informations peut être
    facilité, prenons l'exemple de l'institut de l'information scientifique et
    technique (Inist), avec lequel nous collaborons et dont les activités
    s'organisent autour de la \textsc{Bsn}. Créé en 1988, l'Inist possède l'une
    des plus importantes collections de publications scientifiques d'Europe et
    fournit plusieurs services pour la recherche d'information (\textsc{Ri}),
    dont le maintient de bases de données bibliographiques. Ces dernières sont
    composées de notices bibliographiques décrivant les documents
    scientifiques~: titre, auteur(s), résumé et termes-clés. Parmi ces
    métadonnées, les termes-clés font partie des plus importantes pour la
    \textsc{Ri}. Ce sont des mots ou des expressions qui représentent le contenu
    principal d'un document. Ils permettent de résumer le document, de le
    catégoriser et de l'indexer. C'est cette indexation des documents par leur
    termes-clés qui permet de faciliter la recherche d'information~: lorsqu'une
    personne recherche un information particulière, elle formule une requête qui
    est comparée à l'indexation par termes-clés du document afin de déterminer
    s'il est pertinent pour la personne. Au sein de l'Inist, l'indexation par
    termes-clés est réalisée dans le respect des bonnes pratiques
    documentaires~\cite{guinchat1996techniquesdocumentaires}~: les indexeurs
    professionnels ne donnent pas leur opinion au travers des termes-clés
    (impartialité), donnent tous les termes-clés nécessaire à la caractérisation
    du contenu du document (exhaustivité et spécificité) et utilisent un
    vocabulaire contrôlé et identique pour tous les documents du même domaine
    (conformité et homogénéité). 

    L'indexation manuelle des documents par leurs termes-clés est une tâche
    coûteuse et chronophage. Soucieux de faciliter le travail d'indexation par
    termes-clés, que ce soit pour des productions scientifiques ou des documents
    d'autres natures, de nombreux chercheurs s'intéressent à son automatisation,
    en témoignent le nombre grandissant d'articles scientifiques à ce
    sujet~\cite{hasan2014state_of_the_art} ainsi que l'émergence de campagnes
    d'évaluation~\cite{kim2010semeval,paroubek2012deft}. Dans cette thèse, nous
    proposons des méthodes pour l'indexation automatique par termes-clés.
    Travaillant d'abord d'un point de vue généraliste, nous nous focalisons
    ensuite sur l'indexation par termes-clés en domaines de spécialités.

  %-----------------------------------------------------------------------------

  \section{Problématique}
  \label{sec:main-introduction-problem_statement}
    Étant donné un document textuel de nature quelconque, l'indexation
    automatique par termes-clés consiste à lui attribuer les unités textuelles
    qui décrivent son contenu. La difficulté de cette tâche réside dans
    l'identification des éléments importants dans son contenu et de leur
    représentation avec les unités textuelles les plus appropriées, explicites
    ou implicites dans le document. La première difficulté est d'ordre
    sémantique~: il faut réussir à comprendre le document~; la seconde est
    d'ordre linguistique et terminologique~: il faut déterminer les propriétés
    linguistiques des termes-clés et connaître le vocabulaire du domaine auquel
    appartient le document.

    Dans la littérature, les méthodes d'indexation par termes-clés ne sont pas
    capables de comprendre le document. Elles se fondent sur des statistiques et
    des modélisations particulières du document. Pour ce qui est de l'usage
    d'unités textuelles appropriées, elles se contentent le plus souvent des
    unités textuelles qui occurrent dans le document. De ce dernier sont donc
    sélectionnés des termes-clés candidats en fonction de critères préétablis
    (ce doit être un groue nominal, par exemple), les candidats sont analysés et
    les termes-clés sont extraits d'entre eux selon le résultat de l'analyse.

    L'analyse des termes-clés candidats du document peut être réalisée avec deux
    approches~: supervisée et non supervisée. En général, l'approche supervisée
    apprends les caractéristiques discriminantes des termes-clés à partir de
    documents indexés manuellement. Elle cherche les candidats qui sont le plus
    probablement des termes-clés, tandis que l'approche non supervisée cherche
    les candidats les plus importants dans le document.

    Notre objectif principal est de proposer une méthode supervisée d'indexation
    automatique par termes-clés en domaines de spécialités. À l'instar d'un
    indexeur professionnel, cette méthode doit être exhaustive et fournir des
    termes-clés spécifiques au document et conforme au vocabulaire de son
    domaine. Convaincus par le fait de chercher les termes-clés les plus
    importants du document, notre objectif secondaire est de proposer une
    première méthode non supervisée, puis de l'étendre avec une approche
    supervisée de sorte à trouver les unités textuelles importantes vis-à-vis du
    document et de son domaine (représenté par des données indexées
    manuellement).

  %-----------------------------------------------------------------------------

  \section{Hypothèses}
  \label{sec:main-introduction-hypothesis}
    \TODO{nos travaux se déroulent autour de trois hypothèse principales}

    \TODO{pour être de bonne qualité, les candidats sélectionnés ne doivent pas
    être nombreux, tout en permettant une performance maximal élevée}

    \TODO{il ne faut pas ordonner les unités textuelles, mais ce qu'elles
    représentes}

    \TODO{pour permettre un indexation de meilleure qualité, il ne faut pas
    uniquement modélisé le document, mais il faut aussi modélisé son domaine et
    le situer dans celui-ci}

  %-----------------------------------------------------------------------------

  \section{Plan de thèse}
  \label{sec:main-introduction-outline}
    Cette thèse est organisée de la manière suivante. Tout d'abord, le
    chapitre~\ref{chap:main-state_of_the_art} présente l'état de l'art en
    indexation automatique par termes-clés, puis le
    chapitre~\ref{chap:main-data_description} introduit les données avec
    lesquelles nous travaillons. Nos contributions sont détaillées dans les
    chapitres~\ref{chap:main-domain_independent_keyphrase_extraction}
    et~\ref{chap:main-domain_specific_keyphrase_annotation}. Le
    chapitre~\ref{chap:main-domain_independent_keyphrase_extraction} s'intéresse
    au deux contributions à l'extraction de termes-clés et le
    chapitre~\ref{chap:main-domain_specific_keyphrase_annotation} s'intéresse à
    notre contribution à l'indexation par termes-clés en domaines de
    spécialités. Ce dernier présente tout d'abord la méthodologie d'indexation
    manuelle en domaine de spécialité, il présente ensuite la méthode que nous
    proposons pour s'en approcher et il termine par la description et l'analyse
    des résultats d'une campagne d'évaluation manuelle que nous avons réalisé
    pour évaluer notre travail. Finalement, le
    chapitre~\ref{chap:main-conclusion} dresse le bilan de notre travail et
    présente quelques perspectives.

