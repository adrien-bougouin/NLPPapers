\chapter{Introduction}
\label{chap:main-introduction}

  %-----------------------------------------------------------------------------

  \section{Contexte}
  \label{sec:main-introduction-context}
    %% Expression du besoin
    Avec l'essor du numérique, le Web occupe aujourd'hui une place importante
    dans notre société. Celui-ci contient tous types d'informations
    (culturelles, historiques, scientifiques, etc.) qu'il rend disponibles
    pour tous. Cependant, le Web est en constante expansion et le nombre
    croissant d'informations disponibles compliquent leur accès, leur
    recherche. Pour résoudre ce problème, il faut représenter et organiser
    efficacement les documents numériques. Dans le contexte de la recherche
    scientifique il est important de résoudre ce problème, car favoriser
    l'accès aux productions scientifiques, que ce soit au niveau national ou
    international, favorise les avancées scientifiques. C'est pourquoi
    certains gouvernements créent des bibliothèques numériques telles que la
    Bibliothèque Scientifique Numérique (\textsc{Bsn}) fondée en 2009 par le
    ministère de l'enseignement supérieur et de la recherche français.

    %% Réponse à ce besoin (introduction de la notion de termes-clés)
    Afin de mieux comprendre comment l'accès aux informations peut être
    facilité, prenons l'exemple de l'Institut de l'Information Scientifique et
    Technique (Inist), avec lequel nous collaborons et dont les activités
    s'organisent aujourd'hui dans le cadre de la \textsc{Bsn}. Créé en 1988,
    l'Inist possède l'une des plus importantes collections de publications
    scientifiques d'Europe et fournit plusieurs services pour la Recherche
    d'Information (\textsc{Ri}), dont le maintient de bases de données
    bibliographiques. Ces dernières sont composées de notices bibliographiques
    dont les éléments décrivent chaque document~: titre, auteur(s), résumé et
    termes-clés. Parmi ces éléments, l'ensemble des termes-clés est l'un des
    plus importants pour la recherche d'information. En effet, les termes-clés
    sont les mots ou les expressions qui représentent les concepts importants
    d'un document\footnote{Un terme-clé est plus communément appelé mot-clé.
    Cependant, un mot-clé n'étant pas uniquement monolexical, nous utilisons
    la notion de \textit{terme-clé} pour lever toute ambiguïté. Lorsque dans
    la suite nous parlons de \textit{mots-clés}, cela ne concerne donc que les
    monolexicaux.} et c'est cette conceptualisation qui, au delà de résumer et
    de catégoriser les documents, permet d'établir une correspondance entre le
    besoin d'un utilisateur et les documents qui y répondent. Les termes-clés
    sont parfois fournis par les auteurs, mais leur assignation est subjective
    et peut varier selon les individus. Dans un soucis d'homogénéité, les
    organismes tels que l'Inist font appel à des ingénieurs documentalistes
    (indexeurs professionnels) qui assignent des termes-clés en s'assurant,
    entre autres, qu'ils respectent le vocabulaire spécifique à la discipline
    du document.

    %% Besoin d'aller vers une solution automatique (surcharge => travail
    %% bâclé)
    L'extraction manuelle des termes-clés des documents est une tâche coûteuse
    et chronophage. Humainement, assigner le mieux possible les termes-clés
    d'un document nécessite de le lire dans son intégralité. Cependant,
    lorsque la quantité de documents à traiter est trop importante et que le
    temps imparti pour traiter un document ne peut excéder une certaine
    limite, un indexeur aura tendance à ne considérer qu'un sous-ensemble du
    document (p.~ex. le titre et le résumé uniquement), au risque de dégrader
    la qualité de l'indexation.
    Conscients de ce problème, que ce soit pour des articles
    scientifiques ou des documents d'autres natures, de nombreux chercheurs
    s'intéressent à la tâche d'extraction automatique de termes-clés, en
    témoignent le nombre grandissant d'articles scientifiques à ce
    sujet~\cite{hasan2014state_of_the_art} ainsi que l'émergence de campagnes
    d'évaluation des méthodes d'extraction automatique de
    termes-clés~\cite{kim2010semeval,paroubek2012deft}.

    %% Définition et types d'indexations
    L'extraction automatique de termes-clés consiste à extraire du contenu
    d'un document les concepts qui y sont importants, c'est-à-dire qui le
    caractérisent le mieux. Dans un article d'archéologie, par exemple, des
    termes-clés valides peuvent concerner les types de travaux (fouilles,
    mises en valeur d'artefacts, etc.), des données géographiques (pays,
    régions, etc.), des données chronologiques (années, périodes,
    sous-périodes, etc.) ou encore des données religieuses (dieux, cultes,
    etc.). Il existe deux types d'indexation pouvant être réalisées par la
    tâche d'extraction de termes-clés~: l'indexation libre et l'indexation
    contrôlée~\cite{paroubek2012deft}. La première consiste à assigner des
    termes-clés sans aucune contrainte, alors que la seconde consiste à
    assigner des termes-clés contraints par un vocabulaire (une terminologie)
    spécifique au domaine des documents qui sont traités. Pour ces deux
    indexations, nous observons aussi le phénomène d'indexation silencieuse.
    Cette indexation, difficile à reproduire automatiquement, fait apparaître
    des termes-clés qui ne sont pas présents dans le document auquel ils sont
    assignés~\cite{liu2011vocabularygap}. Il peut s'agir de reformulations
    d'expressions utilisées dans le document (p.~ex. \og{}acquisition des
    langues secondes\fg{} devient \og{}acquisition d'une langue seconde\fg{})
    ou de concepts plus généraux décrivant une catégorie à laquelle appartient
    le document (p.~ex. \og{}psycholinguistique\fg{}).

    %% Fonctionnement des méthodes d'extraction automatique de termes-clés
    Dans la littérature, nous observons un comportement commun à toutes les
    méthodes~: les documents sont tout d'abord enrichis linguistiquement
    (segmentés en phrases, segmentés en mots, étiquetés grammaticalement,
    etc.), des termes-clés candidats en sont extraits, puis analysés (ordonnés
    ou classifiés) afin de sélectionner ceux qu'il faut extraire comme
    termes-clés. Nous observons aussi deux catégories de méthodes d'extraction
    automatique de termes-clés~: les méthodes supervisées et les méthodes
    non-supervisées. Les premières réduisent la tâche d'extraction de
    termes-clés à une tâche de classification~\cite{witten1999kea}.
    Entraînées à partir de collections de documents annotés en termes-clés,
    celles-ci classent les termes-clés candidats en tant que
    \textit{terme-clé} ou \textit{non terme-clé}. Quant aux secondes, elles
    ordonnent généralement les termes-clés candidats selon leur importance
    dans le document~\cite{wan2008expandrank}. En règle générale, ce sont les
    méthodes supervisées qui sont les plus performantes. Cependant, la forte
    dépendance des méthodes supervisées vis-à-vis du domaine des documents
    d'apprentissages utilisés poussent les chercheurs à s'intéresser de plus
    en plus aux méthodes non-supervisées~\cite{hassan2010conundrums}. De
    plus, selon la nature des documents à traiter, des données d'apprentissage
    peuvent ne pas être disponible.

  %-----------------------------------------------------------------------------

  \section{Problématique}
  \label{sec:main-introduction-problem_statement}
    %% Cadre de la thèse
    Dans le cadre du projet \textsc{Anr} Termith\footnote{Terminologie et
    Indexation de Textes en sciences Humaines~:
    \url{http://www.atilf.fr/ressources/termith/}.}, en partenariat avec
    l'Atilf\footnote{Analyse et Traitement Informatique de la Langue
    Française.}, l'Inist, le Lidilem\footnote{Linguistique et Didactique des
    Langues Étrangères et Maternelles.} et l'Inria\footnote{Institut National
    de Recherche en Informatique et en Automatique.} (Nancy et Saclay), notre
    objectif est d'automatiser le processus d'indexation des notices
    bibliographiques réalisé par les ingénieurs documentalistes de l'Inist~:
    indexation à la fois libre, contrôlée et silencieuse. En comparaison avec
    les performances des méthodes réalisant d'autres tâches du Traitement
    Automatique des Langues (\textsc{Tal}), telles que l'étiquetage
    grammatical (plus de 95~\% de précision), les performances des
    méthodes d'extraction automatique de termes-clés ne sont pas
    satisfaisantes (moins de 50~\% de précision). Il reste encore des
    questions auxquelles il est difficile de répondre, parmi elles~:
    \begin{itemize}
      \item{Sous quelles formes les termes-clés apparaissent-ils dans les
            documents~? Quelle est la nature des termes-clés candidats~?
            \cite{wang2014keyphraseextractionpreprocessing}}
      \item{Comment identifier les termes-clés qui n'apparaissent pas dans les
            documents~? \cite{liu2011vocabularygap}}
      \item{Quelles relations les termes-clés entretiennent-ils entre eux dans
            les documents~? \cite{boudin2013centrality_measures,
            lahiri2014boudinlike}}
    \end{itemize}
    mais aussi~:
    \begin{itemize}
      \item{Comment tenir compte de la subjectivité de la tâche lors du
            processus d'évaluation automatique~? Comment détecter
            automatiquement l'extraction d'un termes-clés équivalant à l'un
            des termes-clés de référence assignés aux
            documents~? \cite{zesch2009rprecision}}
    \end{itemize}

    %% Définir les difficultés

  %-----------------------------------------------------------------------------

  \section{Hypothèses}
  \label{sec:main-introduction-hypothesis}

  %-----------------------------------------------------------------------------

  \section{Objectifs}
  \label{sec:main-introduction-goals}

