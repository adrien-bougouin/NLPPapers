\chapter{Introduction}
\label{chap:main-introduction}
  \chaptercite{
    Rechercher des informations est une activité fréquente pour quiconque
    utilise quotidiennement un ordinateur. Alors qu'Internet est une source
    abondante d'information de tout genre, trouver les documents\index{document@Document} pertinents est
    encore difficile. [...] Les termes-clés\index{terme-cle@Terme-clé} aident à organiser et retrouver ces
    documents\index{document@Document} d'après leur contenu.
%    Seeking information is an important activity for everyone who uses computers
%    in daily life. While the Internet is a bountiful source of all types of
%    knowledge, locating relevant documents\index{document@Document} is still a great challenge [\dots]
%    Keyphrases help organise documents\index{document@Document} and retrieve them based on content.
  }{
    \newcite{medelyan2008smalltrainingset}
  }{.75\linewidth}{\justify}

  %-----------------------------------------------------------------------------

  \section{Contexte}
  \label{sec:main-introduction-context}
    La société contemporaine dans laquelle nous vivons se situe en pleine ère de
    l'information. Cette ère succède l'ère moderne, durant laquelle de
    nombreuses découvertes et avancées scientifiques ont été faites~; durant
    laquelle des connaissances considérables ont été acquises. Elle  est aussi
    marquée par le début de la mondialisation, qui, sur le plan scientifique,
    favorise la dissémination et la production de nouvelles connaissances.
    Jusqu'alors rangées sous la forme de documents\index{document@Document} papiers dans des
    bibliothèques, où des documentalistes les indexent et aident ensuite
    scientifiques et particuliers à y accéder le plus efficacement possible, les
    connaissances sont devenues trop nombreuses et leur stockage physique
    inadapté~\cite{rider1946thegreatdilemmaofworldorganization}. L'ère de
    l'information débute vers la fin des années 1940 et apporte une solution à
    ce problème~: l'informatisation des données. Cette informatisation présente
    tout d'abord l'avantage de pouvoir stocker les connaissances sur des
    supports pérennes, de capacité de plus en plus grande (de quelques
    mégaoctets à plusieurs gigaoctets) et de taille de plus en plus réduite (du
    disque dur au \textsc{Dvd}). Très vite, la communauté scientifique y voit
    aussi un moyen pour améliorer la recherche d'information, en indexant les
    documents\index{document@Document} qui contiennent les connaissances, en proposant des interfaces
    pour permettre à un utilisateur de formuler une requête et en cherchant les
    documents\index{document@Document} pertinents vis-à-vis de cette requête~\cite{salton1975tfidf}.

    Produit de cette ère de l'information, le réseau informatique mondial
    Internet en est aussi devenu l'un des acteurs principaux. En effet, si
    l'informatisation et l'indexation des données facilite leur recherche,
    Internet facilite leur accès depuis les bases de données informatisées qui y
    sont connectées. Médium d'information mondial et accessible de
    tous\footnote{En 2012, l'Institut national de la statistique et des études
    économiques (Insee) estimait qu'environ 80~\% des français sont connectés à
    Internet.}, il favorise donc la transition depuis les bibliothèques
    traditionnelles vers des bibliothèques numériques. Ces dernières combinent
    le savoir faire des documentalistes avec les techniques du Traitement\index{traitement@Traitement}
    automatique des langues (\textsc{Tal}) et de la Recherche d'information
    (\textsc{Ri}) pour informatiser les données et faciliter leur accès et leur
    recherche.

    Cette thèse s'inscrit dans le cadre du projet \textsc{Anr} Termith
    (\textsc{Anr-12-Cord-0029}), qui s'intéresse à l'accès à l'information
    numérique en domaines\index{domaine@Domaine} de spécialité\index{specialite@Spécialité} et qui s'articule lui-même autour du
    travail de l'Institut de l'information scientifique et technique (Inist). Né
    en 1988 de la fusion du Centre de documentation scientifique et technique
    (\textsc{Cdst}) et du Centre de documentation sciences humaines
    (\textsc{Cdsh}), tout deux fondés en 1970 pendant les débuts de
    l'informatisation des données, l'Inist possède deux des plus importantes
    bases de données informatisées d'Europe~: \textsc{Pascal} en sciences
    exactes et \textsc{Francis} en sciences humaines. Aujourd'hui acteur de la
    Bibliothèque scientifique numérique (\textsc{Bsn}) fondée en 2009 par le
    ministère de l'enseignement supérieur et de la recherche français, l'une de
    ses missions est de faciliter l'accès à la recherche mondiale au travers de
    la production de notices bibliographiques\footnote{Une notice
    bibliographique contient les informations factuelles d'un document\index{document@Document} (titre,
    auteurs, affiliation des auteurs, etc.), ainsi qu'un résumé.} associées à
    des mots\index{mot@Mot}-clés d'indexation, que nous appelons ici termes-clés\index{terme-cle@Terme-clé}. Avec la
    croissance du nombre\index{nombre@Nombre} de productions scientifiques, l'indexation manuelle par
    termes-clés\index{terme-cle@Terme-clé} est de plus en plus difficile. Cette tâche nécessite un travail
    de maintenance des ressources terminologiques utilisées pour
    l'indexation~\cite{guinchat1996techniquesdocumentaires} et
    des effectifs humains conséquents afin de tenir la charge journalière de
    données à indexer.

    Soucieux de faciliter le travail d'indexation par termes-clés\index{terme-cle@Terme-clé} de toute sorte
    de document\index{document@Document} (résumé d'une notice bibliographique, article scientifique,
    article journalistique, nouvelle, etc.) et pour toute sorte d'application
    (indexation, résumé, publicité ciblée, etc.), de nombreux chercheurs
    s'intéressent à son automatisation. En témoignent le nombre\index{nombre@Nombre} grandissant
    d'articles scientifiques à ce sujet\index{sujet@Sujet}~\cite{hasan2014state_of_the_art} ainsi
    que l'émergence de campagnes
    d'évaluation~\cite{kim2010semeval,paroubek2012deft}.

  %-----------------------------------------------------------------------------

  \section{Problématique}
  \label{sec:main-introduction-problem_statement}
    Étant donné un document\index{document@Document}, l'indexation automatique\index{indexation automatique@Indexation automatique} par termes-clés\index{terme-cle@Terme-clé} consiste à
    trouver les unités textuelles\index{unite textuelle@Unité textuelle} qui décrivent son contenu principal. La
    difficulté de cette tâche réside dans l'identification des éléments
    importants vis-à-vis de son contenu, ainsi que leur représentation avec les
    unités textuelles\index{unite textuelle@Unité textuelle} appropriées. La première difficulté est d'ordre
    sémantique~: il faut réussir à comprendre le document\index{document@Document} pour en extraire
    l'essence~; la seconde est d'ordre linguistique et terminologique~: il faut
    déterminer les propriétés linguistiques des termes-clés\index{terme-cle@Terme-clé} et connaître le
    vocabulaire\index{vocabulaire@Vocabulaire} du domaine\index{domaine@Domaine} auquel appartient le document\index{document@Document}. Par ailleurs, la forme
    la plus appropriée pour un terme-clé\index{terme-cle@Terme-clé} n'est pas nécessairement présente dans
    le contenu du document\index{document@Document}, elle peut être implicite.

    Plutôt que de comprendre le document\index{document@Document}, les méthodes\index{methode@Méthode} d'indexation automatique\index{indexation automatique@Indexation automatique}
    par termes-clés\index{terme-cle@Terme-clé} de la littérature se fondent sur des statistiques et
    des modélisations particulières de celui-ci. Pour ce qui est de l'usage
    d'unités textuelles\index{unite textuelle@Unité textuelle} appropriées, elles se contentent le plus souvent de
    celles qui occurrent dans le document\index{document@Document}. De manière générale, des
    termes-clés\index{terme-cle@Terme-clé} candidats sont sélectionnés dans le document\index{document@Document} d'après des
    critères prédéfinis (par exemple\index{exemple@Exemple}, ce doit être des groupes nominaux\index{groupe nominal@Groupe nominal}), ces
    candidats sont analysés et les termes-clés\index{terme-cle@Terme-clé} sont ensuite extraits d'entre eux
    en fonction du résultat\index{resultat@Résultat} de l'analyse.

    L'analyse des termes-clés\index{terme-cle@Terme-clé} candidats du document\index{document@Document} peut être réalisée avec deux
    approches~: supervisée ou non supervisée. En général, l'approche supervisée
    consiste à analyser les caractéristiques des termes-clés\index{terme-cle@Terme-clé} de données
    manuellement indexées pour apprendre à reconnaître les termes-clés\index{terme-cle@Terme-clé}. Elle
    consiste donc à chercher les candidats qui sont le plus vraisemblablement
    les termes-clés\index{terme-cle@Terme-clé}, tandis que l'approche non supervisée consiste à chercher
    les candidats les plus importants dans le contenu du document\index{document@Document}.
    
    Notre objectif est d'améliorer l'indexation par termes-clés\index{terme-cle@Terme-clé} en domaines\index{domaine@Domaine} de
    spécialité\index{specialite@Spécialité}. Cette indexation doit être de qualité documentaire, c'est-à-dire
    qu'elle doit respecter les principes que suivent les documentalistes, ou
    indexeurs professionnels~\cite{guinchat1996techniquesdocumentaires}. Nous
    travaillons d'abord d'un point de vue généraliste, puis nous nous focalisons
    sur l'indexation par termes-clés\index{terme-cle@Terme-clé} en domaines\index{domaine@Domaine} de spécialité\index{specialite@Spécialité}.

  %-----------------------------------------------------------------------------

  \section{Hypothèses}
  \label{sec:main-introduction-hypothesis}
    Dans cette thèse, nous formulons trois hypothèses.

    Notre première hypothèse concerne la sélection\index{selection@Sélection} des termes-clés\index{terme-cle@Terme-clé} candidats et
    leur impact sur la suite du processus d'indexation par termes-clés\index{terme-cle@Terme-clé}. Selon
    nous, l'indexation gagne en efficacité lorsque la qualité de l'ensemble\index{ensemble@Ensemble} des
    candidats sélectionnés augmente.
    %
    Il s'agit là d'une hypothèse triviale~: si l'un des composants d'une chaîne
    de traitement\index{traitement@Traitement} fait des erreurs, alors celles-ci peuvent se répercuter sur
    les autres composants et dégrader leur performance\index{performance@Performance}. Cependant, de nombreux
    travaux utilisent encore des méthodes\index{methode@Méthode} de sélection\index{selection@Sélection} de candidats grossières,
    ou des filtres linguistiques suffisant pour sélectionner des candidats
    de la même forme que les termes-clés\index{terme-cle@Terme-clé}, mais produisant un nombre\index{nombre@Nombre} important 
    d'erreurs. Ces méthodes\index{methode@Méthode} présentent l'avantage d'être facile à mettre en
    \oe{}uvre pour des performances\index{performance@Performance} d'indexation par termes-clés\index{terme-cle@Terme-clé} satisfaisantes.
    Nous observons toutefois une prise de conscience de l'importance\index{importance@Importance} de la
    sélection\index{selection@Sélection} des candidats, et des travaux récents montrent que nous gagnerions
    à affiner sa réalisation~\cite{wang2014keyphraseextractionpreprocessing}.
    %
    Pour améliorer la qualité de la sélection\index{selection@Sélection}, nous pensons qu'il faut
    s'intéresser à deux propriétés de l'ensemble\index{ensemble@Ensemble} de candidats sélectionnés~: le
    nombre\index{nombre@Nombre} de termes-clés\index{terme-cle@Terme-clé} qui se trouvent parmi les candidats et le nombre\index{nombre@Nombre} total
    de candidats sélectionnés. Paradoxalement, le premier doit être maximisé,
    tandis que le second doit être minimisé, car un espace de recherche trop
    grand augmente la difficulté de
    l'indexation~\cite{hasan2014state_of_the_art}.
    %
    Notre objectif est de trouver des propriétés linguistiques plus fines
    afin d'obtenir le meilleur\index{meilleur@Meilleur} compromis entre ces deux conditions.
    
    Notre seconde hypothèse concerne la détection des mots\index{mot@Mot} et expressions
    importants vis-à-vis d'un document\index{document@Document}. Selon nous ce n'est pas l'importance\index{importance@Importance} de
    ces mots\index{mot@Mot} et expressions qui doit être déterminée, mais l'importance\index{importance@Importance} de ce
    qu'ils représentent. Nous parlons de sujet\index{sujet@Sujet}.
    %
    Les sujets\index{sujet@Sujet} abordés dans un document\index{document@Document} sont véhiculés par une ou
    plusieurs unités textuelles\index{unite textuelle@Unité textuelle}. Il faut donc en déterminer l'importance\index{importance@Importance} en analysant
    l'usage de ces unités textuelles\index{unite textuelle@Unité textuelle}. Par ailleurs, si plusieurs unités
    textuelles sont utilisées pour représenter le même sujet\index{sujet@Sujet} dans un document\index{document@Document} il
    ne faut pas déterminer l'importance\index{importance@Importance} de ce sujet\index{sujet@Sujet} indépendamment de chaque
    unité textuelle\index{unite textuelle@Unité textuelle}, car cela peut engendrer au moins deux types d'erreurs~:
    \begin{itemize}
      \item{Redondance~: plusieurs termes-clés\index{terme-cle@Terme-clé} proposés représentent le
            même sujet\index{sujet@Sujet}~;}
      \item{Imprécision~: pour chaque unité textuelle\index{unite textuelle@Unité textuelle}, l'importance\index{importance@Importance} du sujet\index{sujet@Sujet} est
            différente car elle ne tient pas compte de toutes les références\index{reference@Référence} à ce
            sujet\index{sujet@Sujet} dans le document\index{document@Document}.}
    \end{itemize}
    %
    Notre objectif est de mutualiser l'analyse des unités textuelles\index{unite textuelle@Unité textuelle} qui
    véhiculent les mêmes sujets\index{sujet@Sujet} afin d'éviter la redondance et mieux capturer
    l'importance\index{importance@Importance} de ces sujets\index{sujet@Sujet}.
    
    Enfin, notre troisième hypothèse concerne l'usage de données indexées
    manuellement pour l'indexation automatique\index{indexation automatique@Indexation automatique} d'un document\index{document@Document} du même domaine\index{domaine@Domaine}. Nous pensons qu'il est possible de tirer profit de ces données
    pour (1) améliorer la précision de l'identification des unités textuelles\index{unite textuelle@Unité textuelle}
    importantes en situant le document\index{document@Document} dans son contexte global et (2)
    assigner des termes-clés\index{terme-cle@Terme-clé} du domaine\index{domaine@Domaine} qui sont importants vis-à-vis du document\index{document@Document}.
    %
    La première perspective d'amélioration est générique à tout document\index{document@Document}, tandis
    que la seconde se fonde plus sur l'indexation par termes-clés\index{terme-cle@Terme-clé} telle qu'elle
    est pratiquée en
    domaines\index{domaine@Domaine} de spécialité\index{specialite@Spécialité} (par les indexeurs professionnels). Utiliser des
    données déjà indexées doit permettre de proposer des termes-clés\index{terme-cle@Terme-clé} conformes
    au langage documentaire du domaine\index{domaine@Domaine} auquel appartient le
    document\index{document@Document}. De plus, cela peut résoudre le problème des termes-clés\index{terme-cle@Terme-clé} implicites
    au document\index{document@Document}.
    %
    Notre objectif est de trouver une représentation unifiant celle du
    document\index{document@Document} à celle de son domaine\index{domaine@Domaine}, puis de proposer une méthode\index{methode@Méthode} d'analyse
    capable d'identifier les unités textuelles\index{unite textuelle@Unité textuelle} importantes vis-à-vis du document\index{document@Document}
    et du domaine\index{domaine@Domaine}.

  %-----------------------------------------------------------------------------

  \section{Mise en \oe{}uvre}
  \label{sec:main-introduction-realisation}
  Les contributions que nous présentons dans cette thèse sont fondées sur les
  hypothèses que nous venons d'exprimer. Nous proposons trois contributions,
  appliquées à deux langues~: français et anglais. Nous disposons de quatre
  collections de données pour travailler dans le contexte général~: deux en
  français et deux en anglais. Dans le cadre du projet Termith, l'Inist met à
  notre disposition quatre autres collections de données en domaines\index{domaine@Domaine} de
  spécialité\index{specialite@Spécialité}. Ces collections sont en français et couvrent la linguistique, les
  sciences de l'information, l'archéologie et la chimie. En complément, l'Inist
  met à notre disposition des indexeurs professionnels. Nous tirons profit de
  leur expertise pour la réalisation de nos travaux, ainsi que pour la mise en
  place d'une campagne d'évaluation manuelle\index{evaluation manuelle@Évaluation manuelle} de nos travaux.

  %-----------------------------------------------------------------------------

  \section{Plan de thèse}
  \label{sec:main-introduction-outline}
    Cette thèse est organisée de la manière suivante. Tout d'abord, le
    chapitre~\ref{chap:main-state_of_the_art} présente l'état de l'art en
    indexation automatique\index{indexation automatique@Indexation automatique} par termes-clés\index{terme-cle@Terme-clé}, puis le
    chapitre~\ref{chap:main-data_description} introduit les données avec
    lesquelles nous travaillons. Nos contributions sont détaillées dans les
    chapitres~\ref{chap:main-domain_independent_keyphrase_extraction}
    et~\ref{chap:main-domain_specific_keyphrase_annotation}. Le
    chapitre~\ref{chap:main-domain_independent_keyphrase_extraction} présente
    nos travaux fondés sur les deux premières hypothèses et le
    chapitre~\ref{chap:main-domain_specific_keyphrase_annotation} sur la
    troisième hypothèse. Ce dernier ce concentre sur l'indexation en domaine\index{domaine@Domaine} de
    spécialité\index{specialite@Spécialité}. Il présente tout d'abord l'indexation manuelle réalisée par les
    indexeurs professionnel, puis notre contribution et enfin, une campagne
    d'évaluation manuelle\index{evaluation manuelle@Évaluation manuelle} de nos travaux en domaines\index{domaine@Domaine} de spécialité\index{specialite@Spécialité}. Pour
    terminer, le chapitre~\ref{chap:main-conclusion} dresse le bilan de notre
    travail et présente quelques perspectives.

