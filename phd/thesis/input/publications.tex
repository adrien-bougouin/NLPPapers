\chapter*{Publications}
\addcontentsline{toc}{chapter}{Publications}
  \section*{Publication en revue}
  \addcontentsline{toc}{section}{Publication en revue}

  TopicRank : ordonnancement de sujets pour l'extraction automatique de
  termes-clés\\\textbf{Adrien Bougouin et Florian
  Boudin}\\\cite{bougouin2014topicrank}

  ~\\\textbf{Résumé~:}
  Les termes-clés sont les mots ou les expressions polylexicales qui
  représentent le contenu principal d'un document. Ils sont utiles pour diverses
  applications telles que l'indexation automatique ou le résumé automatique,
  mais ne sont cependant pas disponibles pour la plupart des documents. La
  quantité de ces documents étant de plus en plus importante, l'extraction
  manuelle des termes-clés n'est pas envisageable et la tâche d'extraction
  automatique de termes-clés suscite alors l'intérêt des chercheurs. Dans cet
  article nous présentons \textit{Topic\-Rank}, une méthode non supervisée à
  base de graphe pour l'extraction de termes-clés. Cette méthode groupe les
  termes-clés candidats en sujets, ordonne les sujets et extrait de chacun des
  meilleurs sujets le terme-clé candidat qui le représente le mieux. Les
  expériences réalisées montrent une amélioration significative vis-à-vis de
  l'état de l'art des méthodes à base de graphe pour l'extraction non supervisée
  de termes-clés.

  ~\\Publié dans la revue Traitement Automatique des Langues (\textsc{Tal}
  55-1).

  \section*{Publication en conférence internationale avec acte}
  \addcontentsline{toc}{section}{Publication en conférence internationale avec
  acte}

  \textit{TopicRank: Graph-Based Topic Ranking for Keyphrase
  Extraction}\\\textbf{Adrien Bougouin, Florian Boudin et Béatrice
  Daille}\\\cite{bougouin2013topicrank}

  ~\\\textit{
    \textbf{Abstract:}
    Keyphrase extraction is the task of identifying single or multi-word
    expressions that represent the main topics of a document. In this paper we
    present TopicRank, a graph-based keyphrase extraction method that relies on
    a topical representation of the document. Candidate keyphrases are clustered
    into topics and used as vertices in a complete graph. A graph-based ranking
    model is applied to assign a significance score to each topic. Keyphrases
    are then generated by selecting a candidate from each of the top-ranked
    topics. We conducted experiments on four evaluation datasets of different
    languages and domains. Results show that TopicRank significantly outperforms
    state-of-the-art methods on three datasets.
  }

  ~\\Publié dans les actes de la conférence \textit{International Joint
  Conference on Natural Language Processing (\textsc{Ijcnlp})}.

  \section*{Publications en conférence national avec acte}
  \addcontentsline{toc}{section}{Publications en conférence national avec acte}

  Influence des domaines de spécialité dans l'extraction de
  termes-clés\\\textbf{Adrien Bougouin, Florian Boudin et Béatrice
  Daille}\\\cite{bougouin2014difficulty}

  ~\\\textbf{Résumé~:}
  Les termes-clés sont les mots ou les expressions polylexicales qui
  représentent le contenu principal d'un document. Ils sont utiles pour
  diverses applications, telles que l'indexation automatique ou le résumé
  automatique, mais ne sont pas toujours disponibles. De ce fait, nous nous
  intéressons à l'extraction automatique de termes-clés et, plus
  particulièrement, à la difficulté de cette tâche lors du traitement de
  documents appartenant à certaines disciplines scientifiques. Au moyen de
  cinq corpus représentant cinq disciplines différentes (archéologie,
  linguistique, sciences de l'information, psychologie et chimie), nous
  déduisons une échelle de difficulté disciplinaire et analysons les facteurs
  qui influent sur cette difficulté.

  ~\\Publié dans les actes de la conférence Traitement Automatique du Langage
  Naturel (\textsc{Taln}).

  \vspace{2.5em}\hrule\vspace{2.5em}

  ~\\État de l'art des méthodes d'extraction automatique de
  termes-clés\\\textbf{Adrien Bougouin, Florian Boudin et Béatrice
  Daille}\\\cite{bougouin2013stateoftheart}

  ~\\\textbf{Résumé~:}
  Cet article présente les principales méthodes d'extraction automatique de
  termes-clés. La tâche d'extraction automatique de termes-clés consiste à
  analyser un document pour en extraire les expressions (phrasèmes) les plus
  représentatives de celui-ci. Les méthodes d'extraction automatique de
  termes-clés sont réparties en deux catégories : les méthodes supervisées et
  les méthodes non supervisées. Les méthodes supervisées réduisent la tâche
  d'extraction de termes-clés à une tâche de classification binaire (tous les
  phrasèmes sont classés parmi les termes-clés ou les non termes-clés). Cette
  classification est possible grâce à une phase préliminaire d'apprentissage,
  phase qui n'est pas requise par les méthodes non-supervisées. Ces dernières
  utilisent des caractéristiques (traits) extraites du document analysé (et
  parfois d'une collection de documents de références) pour vérifier des
  propriétés permettant d'identifier ses termes-clés.

  ~\\Publié dans les actes de la conférence Rencontre de Étudiants Chercheurs en
  Informatique pour le Traitement Automatique des Langues (\textsc{Recital}).
