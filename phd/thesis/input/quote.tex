\chapter*{}

\chaptercite{
  Si à la place de l'éducation massive que nous avons (devons avoir) de nos
  jours avec un curriculum, dès lors que nous avons tous des ordinateurs,
  tous connectés à d'importantes bibliothèques où chacun peut poser n'importe
  quelle question et recevoir une réponse, des références, n'importe quelle
  chose pouvant l'intéresser (peu importe à quel point cela puisse
  paraître étrange pour un autre), alors chacun demanderait, et trouverait, et
  avancerait à son propre rythme, dans la direction qu'il souhaite suivre, quand
  il le souhaite, alors tout le monde aimerait apprendre. De nos jours, ce que
  les gens appellent apprendre est imposé et chacun est forcé d'apprendre en
  classe la même chose que les autres, le même jour et au même rythme. Mais tout
  le monde est différent. Pour certains c'est trop rapide, pour d'autres trop
  lent, ou encore inadapté. Donnons à chacun la chance, en plus de l'école (je
  ne dis pas qu'il faut abolir l'école, mais en supplément de l'école), de
  suivre leur propre curiosité.
%  [\dots] if instead of having mass education as we now have --- must have ---
%  with a curriculum, once we have outlets, computer outlets in every home, each
%  of them hooked up to enormous libraries where anyone can ask any question and
%  be given answers, be given reference material, be something you're interested
%  in knowing --- from an early age, however silly it might seem to someone else,
%  it's what you're interested in --- then you ask, and you can find out, and you
%  can follow it up, and you can do it in your own home, at your own speed, in
%  your own direction, in your own time, then everyone will enjoy learning.
%  Nowadays, what people call learning is forced on you and everyone is forced to
%  learn the same thing on the same day at the same speed in class. And everyone
%  is different. For some it goes too fast, for some too slow, for some in the
%  wrong direction. But give them a chance in addition to school --- I don't say
%  we abolish school, but in addition to school --- to follow up their own bent
%  from the start.
}{
  Isaac Asimov (1988)
}{.5\linewidth}{\justify}

