\begin{itemize}
  \item{contient un candidat porteur d'informations utiles}
  \item{permet d'observer un grand nombre de n-grammes erronnés}
  \item{contient plusieurs example de NP-chunks}
  \item{contient de bonnes séquences de noms et d'adjectifs}
  \item{permet la création d'un arbre peigne pour la sélection de You et al. (2009)}
  \item{montre des termes-clés de référence}
  \item{présence de candidats redondants}
\end{itemize}
