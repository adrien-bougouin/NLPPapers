\chapter{Ressources}
\label{chap:main-data_description}
  \chaptercite{
    [\dots] to fully understand the strengths and weaknesses of a keyphrase
    extractor, it is essential to evaluate it on multiple datasets.
  }{
    \newcite{hassan2010conundrums}
  }

  \section{Introduction}
  \label{sec:main-data_description-introduction}
    Pour les travaux de recherche en indexation automatique par termes-clés, des
    collections de données sont nécessaires à l'évaluation et à la comparaison
    des nouveaux travaux aux précédents. De nombreuses collections sont
    accessibles publiquement\footnote{Un grand nombre de collections de données
    est accessible depuis le dépôt GitHub de Su Nam Kim (\textit{snkim})~:
    \url{https://github.com/snkim/AutomaticKeyphraseExtraction}.}, elles
    couvrent différentes langues (français, anglais, etc.), différents domaines
    (météorologie, sciences humaines et sociales, informatique, etc.) et des
    documents de différentes natures (résumés, articles scientifiques, articles
    journalistiques, etc.). Cette diversité est essentielle à
    la compréhension des points forts et des points faibles d'une
    méthode~\cite{hassan2010conundrums}, car différents facteurs peuvent
    influencer les performances des méthodes d'indexation par termes-clés.
    \newcite{hasan2014state_of_the_art} en énoncent quatre~:
    \begin{itemize}
      \item{si un document est long, alors le nombre de termes-clés candidats
            pour celui-ci est élevé et l'indexation par termes-clés est plus
            difficile que pour un document court~;}
      \item{si le contenu d'un document est structuré (par exemple un article
            scientifique réparti en sections), alors une méthode tenant compte
            de cette structure est avantagée~;}
      \item{si des changements thématiques surviennent dans un document, une
            méthode qui utilise la position de la première occurrence des
            candidats peut être désavantagée~;}
      \item{si des sujets sans relation sont abordés dans un même document,
            alors une méthode qui tisse des liens sémantiques entre les
            termes-clés candidats est pénalisée.}
    \end{itemize}
    Il peut aussi y avoir des variations de performances dues à la mauvaise
    qualité de certains documents. Dans le cas d'articles scientifiques, par
    exemple, le format original est très souvent le \textsc{Pdf}, qui est
    ensuite converti en texte depuis les flux de données du \textsc{Pdf} ou avec
    des outils d'\textsc{Ocr} (\textit{Optical Character Recognition}).
    Cependant, ces procédés ne sont pas parfaits, ils ajoutent parfois des fautes
    d'orthographe (mauvais caractères et problèmes d'accents) et ils ne traitent
    pas toujours correctement les environnements tels que les notes de pieds de
    page, les tableaux, les figures et les équations. Deux autres
    facteurs, que \newcite{hasan2014state_of_the_art} n'évoquent pas, sont la
    proportions de termes-clés pouvant uniquement être assignés et la catégorie
    des annotateurs des termes-clés de référence. Nous identifions trois
    catégories d'annotateurs~: les auteurs des articles, des lecteurs et des
    indexeurs professionnels. Parmis ces trois catégories, les lecteurs sont les
    annotateurs les plus libres et impartiaux. Ils n'ont aucun biais vis-à-vis
    de l'article ou de leur profession. Les auteurs cherchent à attirer les
    lecteurs, et les indexeurs professionnels aspirent à une annotation homogène
    entre tous les documents d'un même domaine grâce à des vocabulaires
    contrôlés. Comparer plusieurs méthodes d'indexation par termes-clés à l'un
    ou l'autre de ces trois annotateurs peux ne pas conduire aux mêmes
    conclusions. 

    Dans nos travaux de recherche, nous utilisons cinq collections de données~:
    Termith, \textsc{Deft}, Wikinews, SemEval et \textsc{Duc}. En accord avec la
    vision de \newcite{hassan2010conundrums}, celles-ci couvrent deux langues
    (français et anglais), un large éventail de domaines (sciences humaines et
    sociales, informatique, météorologie, catastrophes naturelles, etc.) et
    trois natures de documents (résumé, articles scientifiques et articles
    journalistiques). Aussi, les annotateurs de chaque collection
    n'appartiennent pas à la même catégorie.

  %-----------------------------------------------------------------------------

  \section{Termith}
  \label{sec:main-data_description-termith_data}
    Termith est le projet \textsc{Anr} qui finance nos travaux. Dans le cadre de
    ce projet, nous disposons de cinq collections de notices bibliographiques de
    l'Inist (Institut de l'information scientifique et de la recherche).
    Une notice bibliographique est une entrée d'un catalogue bibliographique.
    Chaque entrée décrit un document avec différentes méta-données. Il s'agit
    d'informations factuelles (titre du document, nom des auteurs, affiliation
    des auteurs, éditeur, résumé du document, termes-clés des auteurs, etc.) et
    d'informations (parfois) générées par un indexeur professionnel (résumé,
    termes-clés, etc.). Pour les termes-clés, l'indexeur est aidé d'une
    terminologie du domaine (vocabulaire contrôlé) et d'un système à base de
    règles définies manuellement qui lui propose un terme-clé de la
    terminologie selon les occurrences d'unités textuelles prédéfinies dans le
    résumé de la notice ou selon les termes-clés déjà proposés.
    
    Le projet Termith s'intéresse principalement aux sciences humaines et
    sociales (\textsc{Shs}). Quatre des collections de notices représentent
    chacune une discipline de \textsc{Shs}~: linguistique, sciences de
    l'information, psychologie et archéologie. La cinquième collection
    représente la chimie. Le corpus de linguistique est constitué de 715 notices
    d'articles français parus entre 2000 et 2012 dans 12 revues
    (\textit{Linx~---~Revue des linguistes de l'Université Paris Ouest Nanterre
    La Défense}, \textit{Travaux de linguistique}, etc.)~; le corpus des
    sciences de l'information contient 706 notices d'articles français publiés
    entre 2001 et 2012 dans six revues (\textit{Documentaliste -- Sciences de
    l'information}, \textit{Document numérique}, etc.)~; le corpus de
    psychologie contient 720 notices d'articles français publiés entre 2001 et
    2012 dans sept revues (\textit{Enfance}, \textit{Revue internationale de
    psychologie et de gestion des comportements organisationnels}, etc.)~; le
    corpus d'archéologie est composé de 718 notices représentant des articles
    français parus entre 2001 et 2012 dans 22 revues (\textit{Paléo}, \textit{Le
    bulletin de la Société préhistorique française}, etc.)~; le corpus de chimie
    est composé de 782 notices d'articles français publiés entre 1983 et 2012
    dans quatre revues (\textit{Comptes Rendus de l'Académie des Sciences},
    \textit{Comptes Rendus Chimie}, etc.).

    Le tableau~\ref{tab:termith} présente les caractéristiques du corpus
    Termith. Chaque discipline est divisée en deux sous-ensembles\footnote{Les
    revues sont réparties équitablement entre chaque ensemble.}~: un ensemble
    d'apprentissage (appr.) composé de 506 à 582 notices selon la discipline et
    un ensemble de test contenant 200 notices. S'agissant de résumés, les
    documents sont courts (cf. figure~\ref{fig:example_inist}), ils ont en
    moyenne 156,7 mots. Quant aux termes-clés, pour des raisons inhérentes au
    projet Termith, nous utilisons les termes-clés des indexeurs professionnels.
    Ils sont concis (un à deux mots principalement) et, du fait de la
    méthodologie d'annotation, ils ne sont majoritairement pas présents dans les
    résumés (ils doivent être assignés).
    \TODO{si toutes les disciplines sont prêtes, alors expliquer les différences entre elles}

    \begin{table}[!h]
      \centering
      \resizebox{\linewidth}{!}{
        \begin{tabular}{l|c@{~~}c@{~~}c@{~~}c|c@{~~}c@{~~}c@{~~}c}
          \toprule
          \multirow{2}{*}{\textbf{Corpus}} & \multicolumn{4}{c|}{\textbf{Documents}} & \multicolumn{4}{c}{\textbf{Termes-clés}}\\
          \cline{2-9}
          & Langue & Nature & Quantité & Mots moy. & Annotateur & Quantité moy. & \og{}À assigner\fg{} & Mots moy.\\
          \hline
          Linguistique & & & & & & &\\
          \hfill{}Appr. & Français & Résumé & 515 & 160,5 & Professionnel & 8,6 & 60,6~\% & 1,7\\
           \hfill{}Test & \ditto & \ditto & 200 & 147,0 & \ditto & 8,9 & 62,8~\% & 1,8\\
          \hline
          Sciences de l'infor- & & & & & & &\\
          mation\hfill{}Appr. & Français & Résumé & 506 & & & &\\
          \hfill{}Test & \ditto & \ditto & 200 & & & &\\
          \hline
          Psychologie & & & & & & &\\
          \hfill{}Appr. & Français & Résumé & 520 & & & &\\
          \hfill{}Test & \ditto & \ditto & 200 & & & &\\
          \hline
          Archéologie & & & & & & &\\
          \hfill{}Appr. & Français & Résumé & 518 & & & &\\
          \hfill{}Test & \ditto & \ditto & 200 & & & &\\
          \hline
          Chimie & & & & & & &\\
          \hfill{}Appr. & Français & Résumé & 582 & & & &\\
          \hfill{}Test & \ditto & \ditto & 200 & & & &\\
          \bottomrule
        \end{tabular}
      }

      \caption{Corpus Termith
               \label{tab:termith}}
    \end{table}

    \begin{figure}[!h]
%      \framebox[\linewidth]{ % archeologie_09-0054907
%        \parbox{.99\linewidth}{\small
%          \textbf{Variabilité du Gravettien de Kostienki (bassin moyen du Don)
%          et des ter-}
%          ~\hfill\underline{\textit{Archéologie}}\\
%          \textbf{ritoires associés}\\
%
%          Dans la région de Kostienki-Borschevo, on observe l'expression, à ce
%          jour, la plus orientale du modèle européen de l'évolution du
%          Paléolithique supérieur. Elle est différente à la fois du modèle
%          Sibérien et du modèle de l'Asie centrale. Comme ailleurs en Europe, le
%          Gravettien apparaît à Kostienki vers 28 ka (Kostienki 8 /II/). Par la
%          suite, entre 24-20 ka, les techno-complexes gravettiens sont
%          représentés au moins par quatre faciès dont deux, ceux de Kostienki
%          21/III/ et Kostienki 4 /II/, ressemblent au Gravettien occidental et
%          deux autres, Kostienki-Avdeevo et Kostienki 11/II/, sont des faciès
%          propres à l'Europe de l'Est, sans analogie à l'Ouest.\\
%
%          \textbf{Termes-clés~:} \underline{Europe}, Kostienko,
%          \underline{Borschevo}, variation, typologie, industrie osseuse,
%          industrie lithique, Europe centrale, \underline{Avdeevo},
%          \underline{Paléolithique supérieur}, \underline{Gravettien}.
%        }
%      }
%      ~\\~\\
      \framebox[\linewidth]{ % linguistique_08-0265302
        \parbox{.99\linewidth}{\small
          \textbf{Termes techniques et marqueurs d'argumentation : pour
          débusquer}
          \hfill\underline{\textit{Linguistique}}\\
          \textbf{l'argumentation cachée dans les articles de recherche}\\

          Les articles de recherche présentent les résultats d'une expérience
          qui modifie l'état de la connaissance dans le domaine concerné. Le
          lecteur néophyte a tendance à considérer qu'il s'agit d'une simple
          description et à passer à côté de l'argumentation au cours de laquelle
          le scientifique cherche à convaincre ses pairs de l'innovation et de
          l'originalité présentées dans l'article et du bien-fondé de sa
          démarche tout en respectant la tradition scientifique dans laquelle il
          s'insère. Ces propriétés spécifiques du discours scientifique peuvent
          s'avérer un obstacle supplémentaire à la compréhension, surtout
          lorsqu'il s'agit d'un article en langue étrangère. C'est pourquoi il
          peut être utile d'incorporer dans l'enseignement   des langues de
          spécialité une sensibilisation aux marqueurs linguistiques
          (terminologiques et argumentatifs), qui permettent de dépister le
          développement de cette rhétorique. Les auteurs s'appuient sur deux
          articles dans le domaine de la microbiologie.\\

          \textbf{Termes-clés~:} Langue scientifique, \underline{argumentation},
          \underline{rhétorique}, \underline{langue de spécialité},
          \underline{enseignement des langues}, linguistique appliquée,
          \underline{discours scientifique}, \underline{article de recherche}. 
        }
      }
%      ~\\~\\
%      \framebox[\linewidth]{ % chimie_90-0137940
%        \parbox{.99\linewidth}{\small
%          \textbf{Etude d'un condensat acide
%          isocyanurique-urée-formaldéhyde}
%          \hfill\underline{\textit{Chimie}}\\
%
%          La synthèse d'un condensat acide isocyanurique-urée-formaldéhyde
%          utilisant la pyridine en tant que solvant a été effectuée par réaction
%          sonochimique.\\
%
%          \textbf{Termes-clés~:} \underline{Réaction sonochimique}, hétérocycle
%          azote, cycle 6 chaînons, ether.
%        }
%      }
      \caption[Exemple de notice Inist]{
        Exemple de notice Inist. Les termes-clés soulignés sont ceux qui peuvent
        être extraits.
        \label{fig:example_inist}
      }
    \end{figure}

  %-----------------------------------------------------------------------------

  \section[\textsc{Deft}]{\textsc{Deft}~\textnormal{\large\cite{paroubek2012deft}}}
  \label{sec:main-data_description-deft_data}
    \textsc{Deft} est une campagne d'évaluation francophone qui s'intéresse
    chaque année à un domaine particulier du \textsc{Tal}. Le corpus éponyme que
    nous utilisons dans nos travaux est la collection de documents construite
    dans le cadre de l'édition 2012 de \textsc{Deft}, édition portée sur
    l'extraction de termes-clés, d'une part, et sur l'assignement de
    termes-clés, d'une autre part. Le corpus \textsc{Deft} est composé de 234
    articles français publiés entre 2003 et 2008 dans quatre revues des Sciences
    Humaines et Sociales~: \textit{Anthropologie et Sociétés},
    \textit{\textsc{Ttr}~: traduction, terminologie, rédaction},
    \textit{\textsc{Meta}: Research in Hermeneutics, Phenomenology, and
    Practical Philosophy} et \textit{Revue des Sciences de l'Éducation}. Il
    s'agit du corpus utilisé pour la tâche d'extraction de termes-clés de
    \textsc{Deft}-2012.
    
    Le tableau~\ref{tab:deft} présente les différentes caractéristiques du
    corpus \textsc{Deft}. Celui-ci est divisé en deux
    sous-ensembles\footnote{Les revues sont réparties équitablement entre chaque
    ensemble.}~: un ensemble d'apprentissage composé de 141 articles et un
    ensemble de test contenant 93 articles. Contrairement aux documents de
    Termith, ceux de \textsc{Deft} sont des articles complets et leur taille est
    plus importante (en moyenne 7~102,9 mots). Les termes-clés dont nous
    disposons sont ceux des auteurs. Ils sont moins nombreux comparés à ceux de
    Termith et ils peuvent majoritairement être extraits depuis le contenu des
    documents.
    \begin{table}[!h]
      \centering
      \resizebox{\linewidth}{!}{
        \begin{tabular}{l|c@{~~}c@{~~}c@{~~}c|c@{~~}c@{~~}c@{~~}c}
          \toprule
          \multirow{2}{*}{\textbf{Corpus}} & \multicolumn{4}{c|}{\textbf{Documents}} & \multicolumn{4}{c}{\textbf{Termes-clés}}\\
          \cline{2-9}
          & Langue & Nature & Quantité & Mots moy. & Annotateur & Quantité moy. & \og{}À assigner\fg{} & Mots moy.\\
          \hline
          \hfill{}Appr. & Français & Scientifique & 141 & 7~276,7 & Auteur & 5,4 & 18,2~\% & 1,7\\
          \hfill{}Test & \ditto & \ditto & ~~93 & 6~839,4 & \ditto & 5,2 & 21,1~\% & 1,6\\
          \bottomrule
        \end{tabular}
      }
      \caption{Corpus \textsc{Deft}
               \label{tab:deft}}
    \end{table}

    Le corpus \textsc{Deft} est aussi un corpus au contenu bruité, imparfait.
    Du fait de la conversion en texte, les caractères spéciaux ne sont pas
    toujours reconnus et la segmentation en paragraphes a parfois lieu en milieu
    de phrase. Les
    figures~\ref{fig:example_deft_ko}~et~\ref{fig:example_deft_ok} montrent deux
    exemples d'un document bruité et d'un document non bruité, respectivement.
    \begin{figure}[!h]
      \framebox[\linewidth]{ % meta_2005_019828ar
        \parbox{.99\linewidth}{\small

          ~~~~Considérée comme une \og{}problem solving activity\fg{} (Guilford
          1975), la créativité, démystifiée, fait partie du quotidien du
          traducteur. Victimes d'idées préconçues et erronées sur la notion de
          \og{}fidélité\fg{}, beaucoup de traducteurs sont insécurisés face à
          leur créativité. Ils peuvent alors, comme en témoigne un de nos
          exemples, manquer de courage et jouer la carte de la stratégie du
          \og{}playing it safe\fg{}, ou bien, lorsque, comme dans un autre cas,
          leur statut social et professionnel leur donne une certaine assurance,
          garder leurs solutions créatives et revendiquer leur
          \og{}trahison\fg{}, toutefois sans pour autant essayer de trouver des
          légitimations à leurs solutions. Légitimations qui restent la plupart
          du temps au stade de \og{}mécanismes de justification\fg{} ponctuels.
          Une analyse des besoins nous permet de montrer comment ces
          justifications hétéroclites et éparses peuvent venir s'intégrer dans
          un édifice théorique cohérent, s'appuyant notamment sur des fondements
          cognitivistes, susceptible de donner au traducteur le courage de sa
          créativité.

          ~~~~Pour pouvoir déterminer l.utilité d.un quelconque apport théorique
          à la pratique du traducteur, il

          ~~~~faut commencer par examiner s.il existe un besoin en la matière et
          quelle en est la nature. Nous le

          ~~~~ferons à l.aide de deux corpus qui se complètent. Le premier est
          la transcription du débat mené par

          ~~~~un groupe de quatre
          \og{}\&amp;\#x00A0;semi-professionnels\&amp;\#x00A0;\fg{} de
          l.Institut de traducteurs et interprètes de
          
          ~~~~[\dots]\\

          \textbf{Termes-clés~:} \underline{créativité}, didactique de la
          traduction, \underline{cognitivisme}, analyse conversationnelle,
          théorie de la traduction. 
        }
      }
      \caption[Exemple de document de \textsc{Deft}]{
        Exemple de document de \textsc{Deft}. Les termes-clés soulignés sont
        ceux qui peuvent être extraits.
        \label{fig:example_deft_ko}
      }
    \end{figure}

    \begin{figure}[!h]
      \framebox[\linewidth]{ % as_2004_008571ar
        \parbox{.99\linewidth}{\small
          ~~~~Bien qu'un grand nombre de travaux ethnographiques novateurs aient
          été suscités par l'\og{}espace interculturel\fg{} que se partagent
          Australiens autochtones et non autochtones, notamment dans le domaine
          des arts visuels, les chercheurs ont accordé moins d'attention aux
          représentations rituelles publiques auxquelles les Aborigènes ont
          donné un nouvel essor en tant qu'instruments politiques. On a encore
          moins écrit sur la (re)construction interne de l'identité sociale
          autochtone et sa projection dans la production de rituels publics sur
          la scène néocoloniale australienne contemporaine. Tout en effectuant
          une remise à jour des recherches précédentes sur la question, le
          présent article montre comment, au cours des dix dernières années, les
          leaders rituelles aînées d'une petite localité d'Australie centrale
          ont inauguré une phase entièrement nouvelle de représentations
          rituelles - une phase qui diffère substantiellement des formes
          antérieures d'expérience cérémonielle, qui étaient étroitement liées à
          la négociation et à l'échange des matériaux rituels.

          ~~~~Pour M. Nampijinpa L.

          ~~~~Depuis que l'anthropologie \og{}a découvert\fg{} la religion
          australienne – à partir du milieu du XIXe siècle avec les ouvrages de
          Spencer et Gillen dont les travaux de terrain ont alimenté les
          recherches de Durkheim, ethnologue en chambre, et de ses héritiers -
          on s'est beaucoup intéressé aux manifestations rituelles de la
          cosmologie aborigène connue sous le nom de \og{}Dreaming\fg{},
          c'est-à-dire \og{}Rêve\fg{} ou \og{}Récit du Rêve\fg{}. Et bien que la
          fréquence de ce type de représentations cérémonielles ait diminué chez
          les Aborigènes, cette diminution quantitative n’affecte en rien les
          résultats analytiques issus de l’étude des usages contemporains du
          champ rituel. [\dots]\\

          \textbf{Termes-clés~:} \underline{dussart}, \underline{aborigènes},
          \underline{femmes}, \underline{identité}, \underline{rituel},
          \underline{warlpiri}, \underline{australie}.
        }
      }
      \caption[Autre exemple de document de \textsc{Deft}]{
        Autre exemple de document de \textsc{Deft}. Les termes-clés soulignés
        sont ceux qui peuvent être extraits.
        \label{fig:example_deft_ok}
      }
    \end{figure}

    \newcite{paroubek2012deft} établissent un point de comparaison à l'aide
    d'étudiants de master en ingénierie multilingue. Le
    tableau~\ref{tab:deft_human_tests} montre les résultats de l'indexation par
    termes-clés obtenus pour chacun de ces étudiants. Ceux-ci ont jugé la tâche
    difficile, comme en témoigne les faibles résultats obtenus (f-mesure moyenne
    de 21,6~\%). L'indexation par terme-clé est subjective et ils éprouvent des
    difficultés dans le cas ou une expression dans le texte est reformulée. Ils
    donnent l'exemple de \og{}traduction française et allemande\fg{} qui est
    représentée par le terme-clé \og{}traduction allemande et traduction
    française\fg{}. Ils notent aussi la présence de termes-clés d'un même champ
    sémantique et soulignent la contre-intuitivité de cette redondance. Ils
    donnent l'exemple des termes-clés \og{}interprète\fg{} et
    \og{}interprétation\fg{}. Les conclusions ne sont pas encourageantes pour
    l'indexation automatique par termes-clés. Il est possible que les problèmes
    rencontrés lors des tests humains soient dus à la nature des annotations. La
    redondance qui semble contre-intuitive en est un exemple. Nous pouvons en
    effet supposer qu'un auteur à recours à ce genre de procédé pour être
    certains d'attirer tout lecteur potentiel, par exemple l'un effectuant une
    recherche par mot-clé avec \og{}interprète\fg{} et l'autre avec
    \og{}interprétation\fg{}.
    \begin{table}[!h]
      \centering
      \begin{tabular}{l|ccccccc}
        \toprule
          \textbf{Mesure} & \textbf{P1} & \textbf{P2} & \textbf{P3} & \textbf{P4} & \textbf{P5} & \textbf{P6} & \textbf{P7}\\
        \hline
        Précision~\hfill(\%) & 25,0 & 20,0 & 16,7 & 11,8 & 29,2 & 29,2 & 20,8\\
        Rappel~\hfill(\%) & 25,0 & 20,8 & 16,7 & ~~8,3 & 29,2 & 29,2 & 20,8\\
        F-mesure~\hfill(\%) & 25,0 & 20,4 & 16,7 & ~~9,8 & 29,2 & 29,2 & 20,8\\
        \bottomrule
      \end{tabular}
      \caption[Résultats de tests humains sur le corpus \textsc{Deft}]{
        Résultats de tests humains (sept personnes --- P1$..$P7) sur le corpus
        \textsc{Deft}
        \label{tab:deft_human_tests}
      }
    \end{table}

  %-----------------------------------------------------------------------------

  \section[Wikinews]{Wikinews~\textnormal{\large\cite{bougouin2013topicrank}}}
  \label{sec:main-data_description-wikinews_data}
    Wikinews est une collection de 100 articles journalistiques en français
    que nous avons collecté sur le site web d'information collabotatif
    WikiNews\footnote{\url{http://fr.wikinews.org/}} entre les mois de mai et
    décembre 2012\footnote{Les documents et les annotations du corpus Wikinews
    sont disponibles sur le dépôt GitHub suivant~:
    \url{https://github.com/adrien-bougouin/WikinewsKeyphraseCorpus}}.
    
    Le tableau~\ref{tab:wikinews} donne les détails de ce corpus. S'agissant
    d'articles journalistiques, les documents de Wikinews sont de petite taille
    (cf. figure~\ref{fig:example_wikinews}), mais ils sont plus longs que ceux
    de Termith (en moyenne 308,5 mots). Les termes-clés sont annotés par des
    lecteurs. Le nombre de mots qui les composent est du même ordre de grandeur
    que celui des corpus Termith et \textsc{Deft} et le pourcentage de
    termes-clés ne pouvant pas être extraits (\og{}à assigner\fg{}) est très
    faible (7,6~\%).

    \begin{table}[!h]
      \centering
      \resizebox{\linewidth}{!}{
        \begin{tabular}{l|c@{~~}c@{~~}c@{~~}c|c@{~~}c@{~~}c@{~~}c}
          \toprule
          \multirow{2}{*}{\textbf{Corpus}} & \multicolumn{4}{c|}{\textbf{Documents}} & \multicolumn{4}{c}{\textbf{Termes-clés}}\\
          \cline{2-9}
          & Langue & Nature & Quantité & Mots moy. & Annotateur & Quantité moy. & \og{}À assigner\fg{} & Mots moy.\\
          \hline
          \hfill{}Test & Français & Journalistique & 100 & 308,5 & Lecteur & 9,6 & 7,6~\% & 1,7\\
          \bottomrule
        \end{tabular}
      }

      \caption{Corpus Wikinews
               \label{tab:wikinews}}
    \end{table}

    \begin{figure}[!h]
      \framebox[\linewidth]{ % 44960
        \parbox{.99\linewidth}{\small
          \textbf{Météo du 19 août 2012~: alerte à la canicule sur la Belgique
          et le Luxembourg}\\

          ~~~~A l'exception de la province de Luxembourg, en alerte jaune,
          l'ensemble de la Belgique est en vigilance orange à la canicule. Le
          Luxembourg n'est pas épargné par la vague du chaleur : le nord du pays
          est en alerte orange, tandis que le sud a était placé en alerte rouge.

          ~~~~En Belgique, la température n'est pas descendue en dessous des
          23°C cette nuit, ce qui constitue la deuxième nuit la plus chaude
          jamais enregistrée dans le royaume. Il se pourrait que ce dimanche
          soit la journée la plus chaude de l'année. Les températures seront
          comprises entre 33 et 38°C. Une légère brise de côte pourra faiblement
          rafraichir l'atmosphère. Des orages de chaleur sont a prévoir dans la
          soirée et en début de nuit.

          ~~~~Au Luxembourg, le mercure devrait atteindre 32°C ce dimanche sur
          l'Oesling et jusqu'à 36°C sur le sud du pays, et 31 à 32°C lundi. Une
          baisse devrait intervenir pour le reste de la semaine. Néanmoins, le
          record d'août 2003 (37,9°C) ne devrait pas être atteint.\\

          \textbf{Termes-clés~:} \underline{luxembourg}, \underline{alerte},
          \underline{météo}, \underline{belgique}, \underline{août 2012},
          \underline{chaleur}, \underline{température}, \underline{chaude},
          \underline{canicule}, \underline{orange}, \underline{la plus chaude}.
        }
      }
      \caption[Exemple de document de Wikinews]{
        Exemple de document de Wikinews. Les termes-clés soulignés sont ceux qui
        peuvent être extraits.
        \label{fig:example_wikinews}
      }
    \end{figure}

    Le tableau~\ref{tab:wikinews_kappa} montre l'accord inter-annotateur
    $\kappa$~\cite{fleiss1971kappa} pour l'annotation manuelle des termes-clés
    de Wikinews. Les termes-clés sont annotés librement, c'est-à-dire sans
    règles ou méthodologie définie, par au moins trois étudiants de master en
    \textsc{Tal}. L'accord entre les annotateurs est très faible. Il montre la
    subjectivité de la tâche d'indexation par termes-clés et le besoin de
    définir une méthodologie et des règles précises pour associer des
    termes-clés à un document.

    \begin{table}[!h]
      \centering
      \begin{tabular}{r|c}
        \toprule
        \textbf{Corpus} & $\boldsymbol{\kappa}$\\
        \hline
        Test & -0,1\\
        \bottomrule
      \end{tabular}

      \caption{Accord inter-annotateur $\kappa$~\cite{fleiss1971kappa} sur le
               corpus Wikinews
               \label{tab:wikinews_kappa}}
    \end{table}

  %-----------------------------------------------------------------------------

    \section[SemEval]{SemEval~\textnormal{\large\cite{kim2010semeval}}}
  \label{sec:main-data_description-semeval_data}
    À l'instar de \textsc{Deft}, SemEval est une campagne d'évaluation
    internationale. Le corpus éponyme dont nous disposons est la collection de
    documents construite pour la tâche 5 de l'édition 2010 de SemEval, tâche
    consacrée à l'extraction de termes-clés à partir d'articles scientifiques.
    Le corpus SemEval est constitué de 244 articles scientifiques en anglais
    issus de la bibliothèque numérique de l'\textsc{Acm} (\textit{Association
    for Computing Machinery}). Cette bibliothèque regroupe les articles de
    plusieurs disciplines et répartie dans différentes catégorie. Les
    documents du corpus SemEval concernent les catégories C2.4
    (\textit{Distributed Systems} --- Systèmes distribués), H3.3
    (\textit{Information Search and Retrieval} --- Recherche d'information),
    I2.11 (\textit{Distributed Artificial Intelligence -- Multiagent Systems}
    --- Intelligence artificielle distribuée -- Systèmes multi-agents) et J4
    (\textit{Social and Behavioral Sciences -- Economics} --- Sciences sociales
    et comportementales -- Économie) de la classification \textsc{Acm} de 1998.
    
    Le tableau~\ref{tab:semeval} présente les caractéristiques de SemEval. La
    collection est répartie en deux sous-ensembles\footnote{Les quatres
    catégories \textsc{Acm} sont réparties équitablement entre chaque
    ensemble.}~: un ensemble de 144 documents d'apprentissage et un ensemble de
    100 documents de test. À l'instar des documents de \textsc{Deft}, ceux de
    SemEval sont des articles scientifiques de taille importante (en moyenne
    5~152,3 mots). Trois jeux d'annotations sont fournis avec SemEval~: les
    termes-clés des auteurs, les termes-clés de lecteurs et les termes-clés des
    deux premiers jeux d'annotations. Les travaux précédents utilisent le
    troisième. Afin de pouvoir se comparer à ces travaux, nous utilisons nous
    aussi la combinaison des termes-clés des auteurs et des lecteurs.
    Du fait de cette combinaison, le nombre de termes-clés par document est plus
    élevé que celui des autres corpus dont nous disposons. Ils contiennent en
    moyenne 2,1 mots et sont majoritairement extractibles depuis le contenu des
    articles.

    \begin{table}[!h]
      \centering
      \resizebox{\linewidth}{!}{
        \begin{tabular}{l|c@{~~}c@{~~}c@{~~}c|c@{~~}c@{~~}c@{~~}c}
          \toprule
          \multirow{2}{*}{\textbf{Corpus}} & \multicolumn{4}{c|}{\textbf{Documents}} & \multicolumn{4}{c}{\textbf{Termes-clés}}\\
          \cline{2-9}
          & Langue & Nature & Quantité & Mots moy. & Annotateur & Quantité moy. & \og{}À assigner\fg{} & Mots moy.\\
          \hline
          \hfill{}Appr. & Anglais & Scientifique & 144 & 5~134,6 & Auteur~/~Lecteur & 15,4 & 13,5~\% & 2,1\\
          \hfill{}Test & \ditto & \ditto & 100 & 5~177,7 & \ditto & 14,7 & 22,1~\% & 2,1\\
          \bottomrule
        \end{tabular}
      }

      \caption{Corpus SemEval
               \label{tab:semeval}}
    \end{table}

    La figure~\ref{fig:example_semeval} donne un exemple de document de SemEval.
    La structure des articles est préservée. Celle-ci permet d'améliorer les
    performances de l'extraction de termes-clés dans certains
    travaux~\cite{nguyen2007keadocumentstructure,lopez2010humb}.

    \begin{figure}[!h]
      \framebox[\linewidth]{ % C-17
        \parbox{.99\linewidth}{\small
          \textbf{Deployment Issues of a VoIP Conferencing System in a Virtual
          Conferencing Environment}\\

          \textbf{ABSTRACT}

          ~~~~Real-time services have been supported by and large on
          circuitswitched networks. Recent trends favour services ported on
          packet-switched networks. For audio conferencing, we need to consider
          many issues - scalability, quality of the conference application,
          floor control and load on the clients/servers - to name a few. In this
          paper, we describe an audio service framework designed to provide a
          Virtual Conferencing Environment (VCE). The system is designed to
          accommodate a large number of end users speaking at the same time and
          spread across the Internet. The framework is based on Conference
          Servers [14], which facilitate the audio handling, while we exploit
          the SIP capabilities for signaling purposes. Client selection is based
          on a recent quantifier called "Loudness Number" that helps mimic a
          physical face-to-face conference. We deal with deployment issues of
          the proposed solution both in terms of scalability and interactivity,
          while explaining the techniques we use to reduce the traffic. We have
          implemented a Conference Server (CS) application on a campus-wide
          network at our Institute.\\

          \textbf{1. INTRODUCTION}

          ~~~~Today's Internet uses the IP protocol suite that was primarily
          designed for the transport of data and provides best effort data
          delivery. Delay-constraints and characteristics separate traditional
          data on the one hand from voice \& video applications on the other.
          Hence, as progressively time-sensitive voice and video applications
          are deployed on the Internet, the inadequacy of the Internet is
          exposed. Further, we seek to port telephone services on the Internet.
          [\dots]\\

          \textbf{Termes-clés~:} \underline{voip conferencing system},
          \underline{packet-switched network}, \underline{audio service
          framework}, \underline{virtual conferencing environment},
          \underline{conference server}, \underline{loudness number},
          \underline{partial mixing}, \underline{voice activity detectection},
          three simultaneous speakers sufficiency, \underline{vad technique},
          \underline{vce}, \underline{voip}, real-time audio,
          \underline{simultaneous speakers}, \underline{sip}.
        }
      }
      \caption[Exemple de document de SemEval]{
        Exemple de document de SemEval. Les termes-clés soulignés sont ceux qui
        peuvent être extraits.
        \label{fig:example_semeval}
      }
    \end{figure}

    Le tableau~\ref{tab:semeval_annotators} montre la quantité de termes-clés
    annotés par les auteurs, des lecteurs ou les deux. Ces chiffres montrent la
    différence de stratégie entre les auteurs et des lecteurs. Les auteurs
    donnent peu de termes-clés (nous le voyons aussi avec \textsc{Deft})
    comparés aux lecteurs et l'intersection des deux annotations ne couvre qu'un
    tiers des termes-clés des auteurs.

    \begin{table}[!h]
      \centering
      \begin{tabular}{l|ccc}
        \toprule
        \multirow{2}{*}{\textbf{Corpus}} & \multicolumn{3}{c}{\textbf{Annotateur}}\\
        \cline{2-4}
        & Auteur & Lecteur & Combinaison\\
        \hline
        \hfill{}Appr. & 559 & 1824 & 2223\\
        \hfill{}Test & 387 & 1217 & 1482\\
        \bottomrule
      \end{tabular}

      \caption{Nombre de termes-clés annotés dans SemEval, en fonction des
               annotateurs
               \label{tab:semeval_annotators}}
    \end{table}

  %-----------------------------------------------------------------------------

  \section[\textsc{Duc}]{\textsc{Duc}~\textnormal{\large\cite{wan2008expandrank}}}
  \label{sec:main-data_description-duc_data}
    \textsc{Duc} est une campagne d'évaluation internationale portée sur le
    résumé automatique. Notre collection de documents \textsc{Duc} est issue du
    corpus construit dans le cadre de l'édition 2001 de la
    campagne~\cite{over2001duc}. Dans leurs travaux en extraction automatique de
    termes-clés, \newcite{wan2008expandrank} utilisent les 308 documents de test
    du corpus de la campagne et demandent à deux étudiants d'en annoter les
    termes-clés. Il s'agit de 308 articles journalistiques publiés par six média
    d'information différents~: \textit{Associated Press Newswire},
    \textit{Foreign Broadcast Information Service}, \textit{Financial Times},
    \textit{Los Angeles Times}, \textit{San Jose Mercury News} et \textit{Wall
    Street Journal}. Ceux-ci couvrent 30 sujets d'actualités (tornades, contrôle
    des armes à feu, etc.).
    
    Le tableau~\ref{tab:duc} donne les détails du corpus \textsc{Duc}. Les
    documents sont des articles journalistiques (cf.
    figure~\ref{fig:example_duc}). Comme ceux de Wikinews, ces documents sont
    courts. Ils sont cependant plus élaborés et sont constitués de 900,7 mots en
    moyenne. Cette différence mise à part, \textsc{Duc} et Wikinews sont très
    similaires. Leur proportion de termes-clés à assigner est très faible. Ils
    laissent penser que l'extraction de termes-clés est plus aisée sur les
    articles journalistiques. L'annotation par des lecteurs joue peut-être aussi
    un rôle important.

    \begin{table}[!h]
      \centering
      \resizebox{\linewidth}{!}{
        \begin{tabular}{l|c@{~~}c@{~~}c@{~~}c|c@{~~}c@{~~}c@{~~}c}
          \toprule
          \multirow{2}{*}{\textbf{Corpus}} & \multicolumn{4}{c|}{\textbf{Documents}} & \multicolumn{4}{c}{\textbf{Termes-clés}}\\
          \cline{2-9}
          & Langue & Nature & Quantité & Mots moy. & Annotateur & Quantité moy. & \og{}À assigner\fg{} & Mots moy.\\
          \hline
          \hfill{}Test & Anlgais & Journalistique & 308 & 900,7 & Lecteur & 8,1 & 3,5~\% & 2,1\\
          \bottomrule
        \end{tabular}
      }

      \caption{Corpus \textsc{Duc}
               \label{tab:duc}}
    \end{table}

    \begin{figure}[!h]
      \framebox[\linewidth]{ % AP890708-0135
        \parbox{.99\linewidth}{\small
          \textbf{Forests, brush, grass burn in the hot, dry west}\\

          ~~~~Thousands more acres of brush and timber went up in smoke Saturday
          in seven states in the West, threatening homes in some places, and
          firefighters contended with wind and high temperatures.\\

          ~~~~``As the day heats up, you'll get these reburns going out and the
          trees dry out and they'll torch,'' said Forest Service spokesman Ed
          Christian in Wyoming. ``We hope Mother Nature cooperates with us,''
          said Mary Plumb of the federal Bureau of Land Management in Utah.
          Record highs included 97 at Cheyenne, Wyo., and 100 at Denver, while
          Casper, Wyo., tied its record of 100. That was Denver's fifth
          consecutive day at 100 degrees or higher
          
          ~~~~[\dots]\\

          \textbf{Termes-clés~:} \underline{forest fires},
          \underline{firefighters}, \underline{fire crews}, \underline{fire
          lines}, \underline{fire season}, \underline{federal firefighting
          efforts}.
        }
      }
      \caption[Exemple de document de \textsc{Duc}]{
        Exemple de document de \textsc{Duc}. Les termes-clés soulignés sont ceux
        qui peuvent être extraits.
        \label{fig:example_duc}
      }
    \end{figure}

  %-----------------------------------------------------------------------------

  \section{Conclusion}
  \label{sec:main-data_description-conclusion}
    Dans ce chapitre, nous présentons les ressources que nous utilisons dans nos
    travaux de recherche. Pour analyser nos travaux, les évaluer et les comparer
    aux autres, nous disposons de cinq collections de données~: Termith,
    \textsc{Deft}, Wikinews, SemEval et \textsc{Duc}. Celles-ci ne partagent pas
    toutes la même langue, le même domaine, des documents de même nature et la
    même catégorie d'annotateurs. Cette diversité est importante, car elle nous
    permet de mieux comprendre les points forts et les points faibles de nos
    travaux.

    Nous donnons aussi les caractéristiques de chaque corpus. Certaines d'entre
    elles sont stables~: les termes-clés sont des mots simples ou des
    expressions dépassant rarement deux mots et le nombre de termes-clés par
    documents gravite autour de dix. Nous retrouvons ces deux observations dans
    de nombreux travaux de recherche en indexation par termes-clés. Dans les
    travaux précédents, la longueur des candidats est souvent limitée à trois
    mots et les évaluations sont souvent effectuées lorsque les méthodes
    extraient cinq, dix et quinze termes-clés.

