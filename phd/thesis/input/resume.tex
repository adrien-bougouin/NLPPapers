Un terme-clé, couramment appelé mot-clé, est un mot ou une expression qui
représente un des aspects les plus importants d'un document. À l'instar d'un
résumé, les termes-clés attribués à un document donnent une représentation
synthétique de ce dernier et permettent à un lecteur de se faire une idée rapide
et précise de son contenu sans en lire l'intégralité. En plus de cela, ils
permettent d'indexer des documents pour la recherche d'information. Bien
qu'avantageux, les termes-clés ne sont pas toujours disponibles et il faut donc
indexer automatiquement les documents par leur termes-clés. Pour réaliser cette
tâche, la communauté scientifique peine toujours à atteindre le seuil
psychologique de 50~\% de précision. Dans cette thèse, nous nous intéressons à
la tâche d'indexation automatique par termes-clés et proposons trois nouvelles
méthodes. Notre démarche s'organise en deux temps.

Dans un premier temps, nous nous intéressons à l'indexation par termes-clés dans
un contexte généraliste et proposons une méthode pour sélectionner des
termes-clés candidats dans un document et une méthode pour ordonner par
importance les termes-clés candidats. Avant de proposer une méthode de sélection
des candidats, nous étudions les propriétés linguistiques des termes-clés
français et anglais et montrons que la catégorie des adjectifs, en particulier
s'il est relationnel, permet de décider si un adjectif doit faire partie d'un
terme-clé candidat. Fondée sur cette analyse, notre méthode sélectionne les
séquences de noms et d'adjectifs comme candidats, puis supprime de ces derniers
les adjectifs superflus. La seconde méthode que nous proposons, TopicRank, se
situe en aval de la sélection des candidats. Il s'agit du c\oe{}ur de
l'indexation par termes-clés, qui consiste à déterminer quels sont les candidats
les plus important dans le document. TopicRank, est une méthode dite \og{}à base
de graphe\fg{}, c'est-à-dire qu'elle projette des entités du document dans un
graphe et utilise un algorithme qui simule le concept du vote pour déterminer
celles les plus importantes. TopicRank groupe les termes-clés candidats qui
véhiculent le même sujet, projettent les sujets dans le graphe et extrait un
seul terme-clé par sujet important. Nos expériences réalisées sur des ressources
de langue et de nature différentes, montrent des performances significativement
supérieures aux précédentes méthodes \og{}à base de graphe\fg{}.

Dans un second temps, nous adaptons notre travail au contexte de l'indexation
par termes-clés en domaines de spécialités. Après une étude de la méthodologie
d'indexation par termes-clés réalisée manuellement par des indexeurs
professionels, nous étendons TopicRank pour simuler leur comportement. Notre
méthode, TopicCoRank, ajoute à TopicRank un graphe qui représente le domaine de
spécialité du document, modélisé par les termes-clés de référence dans le
domaine. Grâce à ce second graphe unifié au graphe de sujets initial,
TopicCoRank possède la rare capacité à fournir des termes-clés pertinents, même
lorsqu'ils n'aparaissent pas dans le document analysé. Appliqué à cinq domaines
de spécialités, TopicCoRank améliore significativement TopicRank.
