Un terme-clé, couramment appelé mot-clé, est un mot ou une expression qui
représente un des aspects les plus importants d'un document. À l'instar d'un
résumé, les termes-clés attribués à un document donnent une représentation
synthétique de ce dernier et permettent à un lecteur de se faire une idée rapide
et précise de son contenu sans en lire l'intégralité. En plus de cela, ils
permettent d'indexer des documents de sorte qu'une personne puisse aisément les
trouver. Bien qu'avantageux dans un monde où la majorité des connaissances
passées et futures s'amassent sur le Web et où l'indexation des documents
numériques joue donc un rôle primordial pour la recherche d'information, les
termes-clés ne sont pas toujours disponibles et il est donc nécessaire de les
associer automatiquement aux documents. Étudiée depuis quelques décennies, la
tâche d'indexation automatique par termes-clés est encore aujourd'hui une tâche
pour laquelle nous peinons à atteindre le \og{}seuil psychologique\fg{} de 50~\%
de précision. Dans cette thèse, nous nous intéressons à l'indexation automatique
par termes-clés et proposons trois nouvelles méthodes.

Afin d'améliorer l'indexation automatique par termes-clés, nous proposons dans
un premier temps une méthode de sélection des termes-clés candidats d'un
document. Les termes-clés candidats sont les unités textuelles du document qui
sont succeptiblent d'être des termes-clés. Après leur sélection, une méthode
d'indexation par termes-clés doit encore déterminer leur importance. Notre
méthode de sélection des candidats s'appuie sur une étude des propriétés
linguistiques des termes-clés que nous avons réalisée sur différentes
collections de données. Avec cette méthode, nous montrons que tirer partie de la
catégorie des adjectifs présents dans les termes-clés permet d'améliorer la
qualité de la sélection des termes-clés candidats. Dans un second temps, nous
proposons une méthode non supervisée d'indexation par termes-clés. Celle-ci
utilise un graphe pour représenter les sujets abordés dans le document, puis
elle ordonne les sujets par importance en utilisant une adaptation de
l'algorithme historique du célèbre moteur de recherche Google. Après avoir
montré les bonnes performances de cette méthode, nous l'étendons afin de tirer
profit des connaissances spécifiques à un domaine donné. Cette dernière méthode,
plus performante que la précédente, possède la rare capacité à fournir des
termes-clés qui ne sont pas nécessairement présents dans le contenu du document.

