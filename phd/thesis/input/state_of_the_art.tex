\chapter[Indexation automatique par termes-clés]{Indexation automatique\\par termes-clés}
\label{part:main-state_of_the_art}
  \epigraph{
    {\Large``}There is a need for tools that can automatically create
    keyphrases. Although keyphrases are very useful, only a small minority of
    the many documents that are available on-line today have
    keyphrases.{\Large''}
  }{
    \newcite{turney1999learningalgorithms}
  }

  \section{Introduction}
  \label{sec:main-state_of_the_art-introduction}
    Les termes-clés\index{Terme-cle@Terme-clé \textit{(keyphrase)}|textbf},
    souvent appelés mots-clés\index{Mot-cle@Mot-clé
    \textit{(keyword)}|textbf}\footnote{Lorsque nous utilisons le terme
    \og{}mot-clé\fg{}, il s'agit d'un terme-clé composé d'un et un seul mot.},
    sont les unités textuelles (mots ou expressions) qui caractérisent le mieux
    le contenu principal d'un document~: les sujets qu'il aborde, ses idées,
    etc. Ceux-ci donnent une description précise de ce que contient un document
    et peuvent servir à la Recherche d'Information (\textsc{Ri}). Nous parlons
    donc d'indexation par termes-clés\index{Indexation par
    termes-cles@Indexation par termes-clés \textit{(automatic keyphrase
    annotation)}|textbf}. Cette indexation ne doit toutefois pas être confondue
    avec l'indexation dite \og{}plein texte\fg{} au c\oe{}ur de nombreux
    systèmes de \textsc{Ri}. L'indexation plein texte pondère tous les mots d'un
    document selon leur importance relative à celui-ci, tandis que l'indexation
    par termes-clés fournit un ensemble restreint de mots ou expressions qui
    représentent ses sujets importants, explicites ou non.

    Dans la littérature, nous distinguons deux catégories d'indexations
    automatiques par termes-clés~: l'une libre, l'autre contrôlée. L'indexation
    libre\index{Indexation libre@Indexation libre \textit{(free indexing)}|see
    {Extraction automatique de termes-clés \textit{(automatic keyphrase
    extraction)}}} consiste à extraire d'un document les unités textuelles
    jugées les plus importantes dans son contexte. Nous parlons d'extraction
    automatique de termes-clés\index{Extraction automatique de
    termes-cles@Extraction automatique de termes-clés \textit{(automatic
    keyphrase extraction)}|textbf}. L'indexation contrôlée\index{Indexation
    controlee@Indexation contrôlée \textit{(controlled indexing)}|see
    {Assignement automatique de termes-clés \textit{(automatic keyphrase
    assignment)}}} fournit les termes-clés en se fondant sur un vocabulaire
    contrôlé et sans se restreindre aux unités textuelles présentes dans le
    document. Nous parlons d'assignement automatique de
    termes-clés\index{Assignement automatique de termes-cles@Assignement
    automatique de termes-clés \textit{(automatic keyphrase
    assignment)}|textbf}.

    Dans la suite, nous commençons par introduire une étape préliminaire à la
    plupart des méthodes d'extraction et d'assignement automatique de
    termes-clés~: la sélection des termes-clés candidats, qui devient un objet
    d'étude à part entière~\cite{wang2014keyphraseextractionpreprocessing}.
    Ensuite, nous présentons les tâches d'extraction automatique de termes-clés
    et d'assignement automatique de termes-clés. Nous terminons par une
    description du processus d'évaluation des méthodes d'indexation par
    termes-clés.

  %-----------------------------------------------------------------------------

  \section{Sélection des termes-clés candidats}
  \label{sec:main-state_of_the_art-keyphrase_candidate_selection}
    % Quel est l'objectif ?
    La sélection des termes-clés candidats\index{Selection des termes-cles
    candidats@Sélection des termes-clés candidats \textit{Keyphrase candidate
    selection}|textbf}\index{Terme-cle candidat@Terme-clé candidat
    \textit{(keyphrase candidate)}|textbf} consiste à déterminer quelles sont
    les unités textuelles qui sont potentiellement des termes-clés, c'est-à-dire
    les unités textuelles qui ont des particularités similaires à celles des
    termes-clés définis par des humains. Nous savons par exemple que les
    termes-clés sont majoritairement constitués de noms et d'adjectifs
    (\TODO{exemple}). Cette étape présente deux avantages. Le premier est la
    réduction du temps de calcul nécessaire à l'extraction ou à l'assignement
    des termes-clés. Le second est la suppression d'unités textuelles non
    pertinentes pouvant affecter négativement les performances de
    l'ordonnancement. Pour distinguer les différents candidats sélectionnés,
    nous définissons deux catégories~: les candidats positifs\index{Candidat
    positif@Candidat positif \textit{(positive candidate)}|textbf}, qui
    correspondent aux termes-clés assignés par des humains (termes-clés de
    référence), et les candidats négatifs\index{Candidat negatif@Candidat
    négatif \textit{(negative candidate)}|textbf}. Parmi les candidats négatifs,
    nous distinguons deux sous-catégories~: les candidats porteurs
    d'indices\index{Candidat porteur d'indices@Candidat porteur d'indices
    \textit{(clue candidate)}|textbf} de différentes natures pouvant influencer
    l'extraction ou l'assignement de candidats positifs (\TODO{exemple}) et les
    candidats non pertinents\index{Candidat non pertinent@Candidat non pertinent
    \textit{(irrelevant candidate)}|textbf}, que nous considérons comme des
    erreurs de sélection.

    Plusieurs méthodes de sélection de candidats sont utilisées, de la simple
    sélection de n-grammes, de chunks nominaux ou d'unités textuelles
    grammaticalement définies, jusqu'à la sélection de candidats à partir de
    méthodes plus complexes.

    ~\\Les n-grammes\index{N-gramme@N-gramme \textit{(n-gram)}|textbf} sont
    toutes les séquences ordonnées de $n$ mots adjacents. La sélection des
    n-grammes est très exhaustive, elle fournit un grand nombre de termes-clés
    candidats, ce qui maximise la quantité de candidats positifs, la quantité de
    candidats porteurs d'indices, mais aussi la quantité de candidats non
    pertinents. Pour réduire cette dernière, il est courant de filtrer les
    n-grammes avec un
    antidictionnaire\index{Antidictionnaire@Antidictionnaire
    \textit{(stopwords)}|textbf}, selon le principe suivant~: un n-gramme
    contenant un mot de l'antidictionnaire en début ou en fin n'est pas
    considéré comme un terme-clé candidat. L'antidictionnaire regroupe les mots
    fonctionnels de la langue (conjonctions, prépositions,~etc.) et les mots
    d'usage courant (\og{}particulier\fg{}, \og{}près\fg{}, \og{}beaucoup\fg{},
    etc.).
    
    Malgré son aspect bruité, la sélection des n-grammes est largement utilisée
    pour l'indexation par
    termes-clés~\cite{witten1999kea,hulth2003keywordextraction,medelyan2009humancompetitivetagging}.

    \begin{example}
      \TODO{$\{1..3\}$-grammes à partir d'une phrase de l'exemple fil rouge}
    \end{example}

    ~\\Les \textit{chunks} nominaux\index{NP-chunk@NP-\textit{chunk}|textbf}
    sont des syntagmes\index{Syntagme@Syntagme
    \textit{(syntagm)}|textbf}\footnote{Syntagme~: unité syntaxique
    intermédiaire entre le mot et la phrase. Aussi appelé groupe, le syntagme
    constitue une unité de sens dont chaque constituant conserve sa
    signification et sa syntaxe propre.} non récursifs (ou minimaux) dont la
    tête est un nom, accompagné de ses éventuels déterminants et modifieurs
    usuels. Ils sont linguistiquement définis et leur sélection est donc plus
    fiable que celle des n-grammes. \newcite{hulth2003keywordextraction} le
    montre dans ses expériences consacrées à l'apport de connaissances
    linguistiques pour l'extraction automatique de termes-clés. Cependant, ses
    propos sont nuancés par un autre de ses constats~: l'usage de la catégorie
    grammaticale des mots des n-grammes dans une méthode supervisée d'extraction
    automatique de termes-clés permet d'éliminer les n-grammes grammaticalement
    incorrects et induit de meilleures performances qu'avec les \textit{chunks}
    nominaux.

    \begin{example}
      \TODO{NP-\textit{chunks} à partir d'une phrase de l'exemple fil rouge}
    \end{example}

    ~\\La sélection d'unités textuelles qui forment des séquences
    grammaticalement définies\index{Sequence grammaticalement definie@Séquence
    grammaticalement définie \textit{(POS sequence)}|textbf} permet de contrôler
    avec précision la nature et la grammaticalité des candidats sélectionnés.
    Pour cela, il faut définir des patrons grammaticaux tels que \texttt{(<NOM>
    | <ADJ>)*}, qui représente les plus longues séquences de noms et
    d'adjectifs, d'après la syntaxe des expressions rationnelles.

    À l'instar des \textit{chunks} nominaux, la sélection des séquences
    grammaticalement définies est plus fondée linguistiquement que celle des
    n-grammes. Dans ses travaux, \newcite{hulth2003keywordextraction}
    sélectionne les candidats à partir des patrons des termes-clés de référence
    les plus fréquents dans ses données. D'autres chercheurs,
    tels que \newcite{wan2008expandrank}, se contentent des plus longues
    séquences de noms (noms propres inclus) et d'adjectifs.

    \begin{example}
      \TODO{Plus longues séquences de noms et d'adjectifs à partir d'une phrase de l'exemple fil rouge}
    \end{example}

    ~\\En plus des trois précédentes méthodes de sélection de termes-clés
    candidats, d'autres chercheurs comme
    \newcite{huang2006semanticnetworkstructureanalysis} proposent un filtrage
    des candidats sélectionnés.
    Ils filtrent tout d'abord les candidats peu fréquents dans le document, puis
    mettent ensuite en compétition ceux ayant un mot en commun. Pour chaque
    groupe de candidats mis en compétitions, un seul candidat peut être retenu
    comme terme-clé candidat, celui le plus fréquent. Un candidat peut être en
    compétition dans différents groupes. Dans ce cas, il doit être le
    \og{}vainqueur\fg{} de chaque groupe pour être retenu.

    Le travail de \newcite{huang2006semanticnetworkstructureanalysis}, sur la
    sélection des termes-clés candidats, fait partie d'un système d'extraction
    de termes-clés qu'ils ont développés. Ce dernier a fait l'objet d'une
    évaluation, mais aucune étude n'a été conduite pour montrer l'efficacité de
    leur filtrage hors des frontières de leur système.

    ~\\Contrairement à \newcite{huang2006semanticnetworkstructureanalysis},
    \newcite{you2009refinedcandidateset} ont pour objectif précis d'améliorer
    la qualité de l'extraction automatique de termes-clés en réduisant le nombre
    de candidats sélectionnés sans perdre de candidats positifs ni même
    augmenter la complexité algorithmique de la sélection. Leur méthode se
    divise en deux étapes: sélection de candidats préliminaires grâce à une
    liste de mots-clés, puis réduction de cet ensemble de candidats
    préliminaires à un ensemble non redondant de candidats porteurs de sens.

    Dans un premier temps, \newcite{you2009refinedcandidateset} extraient les
    $k$ mots les plus fréquents du document, excluant les mots d'un
    antidictionnaire, et les définissent comme mots-clés du document. Ensuite,
    chaque occurrence de chaque mot-clé sert à définir au plus sept candidats
    préliminaires~:
    \begin{enumerate*}
      \item{le mot-clé lui-même,}
      \item{un bi-gramme commençant par le mot-clé,}
      \item{un bi-gramme se terminant par le mot-clé,}
      \item{un tri-gramme commençant par le mot-clé,}
      \item{un tri-gramme se terminant par le mot-clé,}
      \item{un quadri-gramme commençant par le mot-clé et}
      \item{un quadri-gramme se terminant par le mot-clé\footnote{La taille
            maximale de 4 pour les n-grammes est fixée par les auteurs après
            une analyse statistique de leurs données. Cette taille peut diverger
            selon les données, auquel cas les candidats préliminaires
            sélectionnés pour chaque occurrence d'un mots-clés peut excéder
            sept.}.}
    \end{enumerate*}

    \begin{example}
      \TODO{Un 1-gramme, deux 2-grammes, deux 3-grammes et deux 4-grammes à partir d'une phrase de l'exemple fil rouge}
    \end{example}

    Dans un second temps, \newcite{you2009refinedcandidateset} analysent chaque
    7-uplets de $\{1..4\}$-grammes et retirent les candidats préliminaires jugés
    incomplets, non porteurs de sens. À la fin de cette étape, deux candidats
    préliminaires peuvent être sélectionnés à partir de chaque 7-uplet~: l'un
    commençant par le mot-clé et l'autre se terminant par le mot-clé. Pour cela,
    les auteurs utilisent des arbres. Un mot-clé est la racine de
    chaque arbre et les fils de chaque noeuds sont leurs mots qui les précèdent
    ou les succèdent dans les candidats préliminaires. Un mot apparaissant moins
    de deux fois est un
    n\oe{}ud feuille. Pour sélectionner les termes-clés candidats, les auteurs
    parcourent ensuite l'arbre en profondeur et ne retiennent que les
    séquences constituées uniquement de n\oe{}uds non feuilles et de mots ne
    faisant pas partie d'un antidictionnaire.

    \TODO{Exemple de graphe obtenu à partir de l'exemple précédent (faire
    référence à ce graphe dans le paragraphe précédent)}

    Comparée à la sélection des $\{1..3\}$-grammes, utilisée dans plusieurs
    travaux, la méthode de \newcite{you2009refinedcandidateset} réduit
    significativement le nombre de candidats (environ 75~\%). Leurs expériences
    montrent cependant que leur méthode de sélection n'améliore pas
    significativement les performances des méthodes d'indexation par
    termes-clés.

    ~\\Pour contribuer à la sélection des termes-clés candidats,
    \newcite{newman2012bayesiantextsegmentation} s'intéressent aux travaux
    effectués dans le domaine de la segmentation des phrases en mots et à
    l'adaptation de l'un d'eux~\cite{goldwater2009bayesianwordsegmentation} pour
    la détection des frontières séparant les mots et les expressions dans la
    phrase. Leur méthode parcours le document pour apprendre un lexique en
    suivant un processus de Dirichlet. Ce lexique est ensuite considéré comme
    l'ensemble des termes-clés candidats pour le document.

    En utilisant un simple ordonnancement par fréquence pour la tâche
    d'extraction de termes-clés, ce qui n'est pas une heuristique suffisante
    pour déterminer l'importance d'un terme-clé candidat, leur approche obtient
    de meilleurs résultats que la majorité des méthodes états de l'art. Ceci
    tant à montrer la capacité de leur méthode à fournir des candidats de très
    bonne qualité.

  %-----------------------------------------------------------------------------

  \section{Extraction automatique de termes-clés}
  \label{sec:main-state_of_the_art-automatic_keyphrase_extraction}
    L'extraction automatique de termes-clés est la tâche la plus utilisée pour
    faire de l'indexation par termes-clés. Les méthodes d'extraction automatique
    de termes-clés effectuent principalement un ordonnancement ou une
    classification des termes-clés candidats, puis choisissent les $k$
    meilleurs. L'ordonnancement est principalement réalisé selon une approche
    non supervisée, tandis que la classification est réalisée selon une approche
    supervisée. Contrairement à l'approche non supervisée, l'approche supervisée
    requiert des données d'entraînement afin d'apprendre à partir des
    caractéristiques des termes-clés qui y sont manuellement annotés.

    \subsection{Approche non supervisée}
    \label{subsec:main-state_of_the_art-automatic_keyphrase_extraction-unsupervised_keyphrase_extraction}
      Les méthodes non supervisées d'extraction de termes-clés ordonnent
      principalement les termes-clés candidats par importance. Du fait qu'elles
      ne requièrent pas de données d'entraînement, elles ont la particularité de
      s'abstraire du domaine et de la langue des documents qu'elles traitent.
      Les termes-clés candidats sont analysés avec des règles simples fondées
      sur des traits statistiques extraits du document ou d'un corpus de
      référence non annoté.

      De nombreuses méthodes sont proposées. Certaines se fondent uniquement
      sur des statistiques et d'autres les combinent avec des représentations
      plus complexes du document, principalement des groupes sémantiques et des
      graphes de cooccurrences de mots. Dans la suite, nous présentons ces
      différentes méthodes, des plus simples aux plus sophistiquées. Notez qu'il
      existe aussi une étude comparative des performances de certaines d'entre
      elles~\cite{hassan2010conundrums}.

      \subsubsection{Méthodes statistiques}
      \label{subsubsec:main-state_of_the_art-automatic_keyphrase_extraction-unsupervised_keyphrase_extraction-statistical_approaches}
        Les méthodes statistiques se fondent majoritairement sur le nombre de
        cooccurrences des termes-clés candidats (souvent assimilé à leur
        fréquence) ou sur le nombre de mots qui les composent, soit dans le
        document, soit dans un corpus de référence, ou bien les deux.

        ~\\\textsc{Tf-Idf}~\cite{jones1972tfidf} et Likey~\cite{paukkeri2010likey}
        sont deux méthodes similaires qui comparent le comportement d'une unité
        textuelle dans le document avec son comportement dans un corpus de
        référence. L'objectif est de trouver celles dont le
        comportement dans le document varie positivement comparé à leur
        comportement global dans le corpus. Dans \textsc{Tf-Idf} et Likey ceci
        s'exprime par le fait qu'une unité textuelle a une forte importance
        vis-à-vis du document si elle y est très fréquente et si elle l'est
        peu dans le corpus de référence:
        \begin{align}
          \text{\textsc{Tf-Idf}}(\text{\textit{ut}}) &= \textsc{Tf}(\text{\textit{ut}}) \times \log\left(\frac{N}{\textsc{Df}(\text{\textit{ut}})}\right) \label{math:tfidf}\\
          \notag\\
          \text{Likey}(\text{\textit{ut}}) &= \frac{\text{rang}_{\text{document}}(\text{\textit{ut}})}{\text{rang}_{\text{corpus}}(\text{\textit{ut}})} \label{math:likey}
        \end{align}\\
        Dans \textsc{Tf-Idf}, $\textsc{Tf}$ (\textit{Term Frequency}) représente
        le nombre d'occurrences d'une unité textuelle \textit{ut} dans le
        document, $\textsc{Df}$ (\textit{Document Frequency}) représente le
        nombre de documents du corpus de référence dans lequel elle apparaît et
        $N$ est le nombre total de documents du corpus de référence. Plus le
        score \textsc{Tf-Idf} d'une unité textuelle est élevé, plus celle-ci est
        importante dans le document. Dans Likey, le rang d'une unité textuelle
        dans le document et dans le corpus est obtenu à partir de son nombre
        d'occurrences, respectivement dans le document et dans le corpus. Plus
        le rapport entre ces deux rangs est faible, plus l'unité textuelle
        évaluée est importante dans le document.

        La granularité de l'unité textuelle peut être définie au mot ou au
        terme-clé candidat. Si la granularité choisie est le mot, il est courant
        de déterminer le score d'importance des termes-clés candidat en faisant
        la somme du score (\textsc{Tf-Idf} ou Likey) des mots qui les composent.
        Cependant, il est important de noter que faire cette somme favorise les
        plus longues séquences de mots et fait monter des candidats redondants
        dans le classement par importance (\TODO{exemple de redondance}).

        ~\\Okapi (ou \textsc{Bm}25) \cite{robertson1999okapi} est une mesure
        alternative à \textsc{Tf-Idf}. En Recherche d'Information (\textsc{Ri}),
        celle-ci est plus utilisée que le \textsc{Tf-Idf}. Bien que l'extraction
        automatique de termes-clés soit une discipline à la frontière entre le
        \textsc{Tal} et la \textsc{Ri}, la méthode de pondération Okapi n'a, à
        notre connaissance, pas été appliquée pour l'extraction de termes-clés.
        Dans l'article de \newcite{claveau2012vectorisation}, Okapi est décrit
        comme un \textsc{Tf-Idf} prenant mieux en compte la longueur des
        documents. Cette dernière est utilisée pour normaliser le $\textsc{Tf}$
        ($\textsc{Tf}_{\textsc{Bm}25}$)~:
        \begin{align}
          \text{Okapi}(\text{\textit{ut}}) &= \textsc{Tf}_{\textsc{Bm}25}(\text{\textit{ut}}) \times \log\left(\frac{N - \textsc{Df}(\text{\textit{ut}}) + 0,5}{\textsc{Df}(\text{\textit{ut}}) + 0,5}\right) \label{math:okapi}\\
          \notag\\
          \textsc{Tf}_{\textsc{Bm}25}(\text{\textit{ut}}) &= \frac{\textsc{Tf}(\text{\textit{ut}}) \times (k_1 + 1)}{\textsc{Tf}(\text{\textit{ut}}) + k_1 \times \left(1 - b + b \times \frac{\textsc{Dl}}{\textsc{Dl}_{\text{moyenne}}}\right)} \label{math:tf_bm25}
        \end{align}\\
        où $k_1$ est une constante fixée à 2, où $b$ est une constante fixée à
        $0,75$, où $\textsc{Dl}$ (\textit{Document Length}) représente la
        longueur du document et où $\textsc{Dl}_{moyenne}$ représente la
        longueur moyenne des documents du corpus de référence.

        ~\\Le travail de \newcite{barker2000nounphrasehead} est un autre exemple
        d'utilisation de la fréquence pour extraire les termes-clés. Se reposant
        sur des fondements plus linguistiques, ils utilisent des groupes
        nominaux comme candidats et tiennent compte à la fois de leur fréquence
        et de celle de leur tête nominale pour déterminer leur importance. Ils
        définissent un candidat important comme étant un candidat informatif et
        fréquent. L'informativité est ici assimilée à sa taille, en nombre de
        mots~: plus il contient de mots, plus il est informatif. Pour éviter les
        répétitions, jugées inesthétiques, de tels candidats informatifs sont
        parfois abrégés et leur fréquence réel ne reflète pas leur usage. C'est
        pourquoi \newcite{barker2000nounphrasehead} proposent d'utiliser la
        fréquence de la tête des candidats pour décider s'ils doivent être
        extraits ou non. Leur méthode fonctionne en quatre étapes. Ils extraient
        tout d'abord les $n$ noms les plus fréquents, ils gardent uniquement
        les groupes nominaux contenant un de ces noms, puis les ordonnent selon
        le produit de leur taille et de leur fréquence réel. Enfin, ils
        extraient les $k$ groupes nominaux de meilleur rang.

        ~\\\newcite{tomokiyo2003languagemodel} tentent aussi de vérifier
        statistiquement deux propriétés que doit respecter un terme-clé candidat
        pour être extrait comme terme-clé~:
        \begin{itemize}
          \item{informativité : un terme-clé doit capturer au moins une des
                idées essentielles exprimées dans le document analysé;}
          \item{grammaticalité : un terme-clé doit être bien formé
                syntaxiquement.}
        \end{itemize}
        Pour vérifier ces deux propriétés, trois modèles de langue
        ($\textsc{Ml}$) sont utilisés (cf. figure~\ref{fig:klml}). Les deux
        premiers modèles, l'un uni-gramme, $\textsc{Ml}_{\text{document}}^1$,
        l'un n-gramme, $\textsc{Ml}_{\text{document}}^N$, sont construits à
        partir du document. Le dernier, un modèle n-gramme,
        $\textsc{Ml}_{\text{référence}}^N$, est construit à partir d'un corpus
        de référence, c'est le modèle de référence. Il fournit une vision
        globale de la distribution des n-grammes dans la langue (français,
        anglais, etc.). De ce fait, plus la probabilité d'un terme-clé candidat
        selon le modèle n-gramme du document diverge positivement par rapport à
        sa probabilité selon le modèle de référence, plus il respecte la
        propriété d'informativité (cf. équation~\ref{math:informativeness}), et
        plus la probabilité d'un terme-clé candidat selon le modèle n-gramme du
        document diverge positivement par rapport à sa probabilité selon le
        modèle uni-gramme du document, plus il respecte la propriété de
        grammaticalité (cf. équation~\ref{math:phraseness}). La divergence est
        exprimée en terme de coût avec la divergence Kullback-Leibler (cf.
        équation \ref{math:kullbackleibler}). Les termes-clés candidats sont
        ordonnés dans l'ordre décroissant de la somme des scores d'informativité
        et de grammaticalité, puis les $k$ termes-clés candidats de meilleur
        rang sont extraits comme termes-clés.
        \begin{align}
          \text{informativité}(\text{\textit{candidat}}) &= \textsc{Kl}_{\text{\textit{candidat}}}(\textsc{Ml}_{\text{document}}^{N} \| \textsc{Ml}_{\text{référence}}^{N}) \label{math:informativeness}\\
          \notag\\
          \text{grammaticalité}(\text{\textit{candidat}}) &= \textsc{Kl}_{\text{\textit{candidat}}}(\textsc{Ml}_{\text{document}}^{N} \| \textsc{Ml}_{\text{document}}^{1}) \label{math:phraseness}\\
          \notag\\
          \textsc{Kl}_{\text{\textit{candidat}}}(\textsc{Ml} \| \textsc{Ml}') &= \textsc{Ml}(\text{\textit{candidat}}) \log \frac{\textsc{Ml}(\text{\textit{candidat}})}{\textsc{Ml}'(\text{\textit{candidat}})} \label{math:kullbackleibler}\\
          \notag\\
          \textsc{Ml}(\text{\textit{candidat}} = m_1\ m_2\ \dots\ m_k) &= \prod_{i = 1}^k P(m_i | m_{i - (N - 1)} m_{i - ((N - 1) - 1)} \dots m_{i - 1}) \notag
        \end{align}
        \begin{figure}
          \centering

          \begin{tikzpicture}
            \node [fill=@verticalgreen] (ml_n_ref) {$\textsc{Ml}_{\textnormal{référence}}^{N}$};
            \node [fill=@verticalgreen, right=of ml_n_ref, xshift=1em] (ml_n_doc) {$\textsc{Ml}_{\textnormal{document}}^{N}$};
            \node [fill=@verticalgreen, below=of ml_n_doc] (ml_1_doc) {$\textsc{Ml}_{\textnormal{document}}^{1}$};

            \path [<->] (ml_n_ref) edge node [above, yshift=.75em] {informativité} (ml_n_doc);
            \path [<->] (ml_n_doc) edge node [right, xshift=.75em] {grammaticalité} (ml_1_doc);
          \end{tikzpicture}

          \caption{Illustration des deux propriétés d'informativité et de
                   grammaticalité pouvant être induites entre trois modèles de
                   langues~\cite{tomokiyo2003languagemodel}
                   \label{fig:klml}}
        \end{figure}

        \label{subsubsec:main-state_of_the_art-automatic_keyphrase_extraction-unsupervised_keyphrase_extraction-statistical_approaches:ilp}
        ~\\Tout comme \newcite{tomokiyo2003languagemodel},
        \newcite{ding2011binaryintegerprogramming} tentent de définir des
        propriétés visant à affiner l'extraction de termes-clés. Les termes-clés
        sont ici restreints aux mots-clés. Ils expriment leurs propriétés sous
        la forme de contraintes dans un système d'optimisation (programmation
        par les entiers) qui explore l'espace des solutions possibles (toutes
        les combinaisons de mots à extraire). Les contraintes sont les
        suivantes~:
        \begin{itemize}
          \item{taille: les mots-clés extraits ne doivent pas être en nombre
                supérieur à $k$~;}
          \item{couverture : les mots-clés doivent couvrir le plus possible de
                sujets abordés dans le document~;}
          \item{cohérence : les mots-clés doivent être cohérents entre eux.}
        \end{itemize}
        La couverture de chaque sujet d'une solution est calculée avec le modèle
        \textit{Latent Dirichlet Allocation} (\textsc{Lda}) \cite{blei2003lda}
        qui donne la probabilité d'occurrence d'un mot sachant un sujet.
        Celle-ci ne doit pas être inférieur à un certain seuil pour qu'une
        solution soit retenue. La contrainte de cohérence est calculée entre
        chaque paire de mots d'une solution. Si deux mots cooccurrent plus que
        selon un seuil donné, alors ceux-ci peuvent être présents dans la même
        solution, sinon la solution n'est pas satisfaisante. Ces deux
        contraintes réduisent le champ des possibilités. Il faut ensuite
        trouver quel ensemble de mots-clés parmi les solutions satisfaisantes
        est le meilleur. Pour cela, un score d'importance des mots est calculé
        et l'ensemble de mots-clés pour lequel la somme du score d'importance
        des mots est la plus élevée est extrait. Ce score est une somme pondérée
        du score \textsc{Tf-Idf} du mot, d'un bonus s'il est présent dans le
        titre du document et d'un autre bonus s'il est présent dans la première
        phrase du document.

      \subsubsection{Méthodes par regroupement}
      \label{subsubsec:main-state_of_the_art-automatic_keyphrase_extraction-unsupervised_keyphrase_extraction-clustering_approaches}
        Les méthodes par regroupement utilisent des groupes d'unités textuelles
        partageant une ou plusieurs caractéristiques (similarité lexicale,
        similarité sémantique, etc.).

        ~\\Dans leur méthode, \newcite{matsuo2004wordcooccurrence} groupent les
        candidats les plus fréquents selon leur relation de cooccurrences (nous
        parlons abusivement de relation sémantique), puis comparent les
        termes-clés candidats du document avec les groupes de candidats les plus
        fréquents. Faisant l'hypothèse qu'un terme-clé candidat qui cooccurre
        plus que selon toute probabilité avec les candidats fréquents d'un ou
        plusieurs groupes est plus vraisemblablement un terme-clé, ils utilisent
        la mesure $\chi^2$ pour calculer le biais entre la fréquence de
        cooccurrence attendue et la fréquence de cooccurrence réelle d'un
        terme-clé candidat avec les candidats de chaque groupe~:
        \begin{align}
          \chi^2(\text{\textit{candidat}}) = \sum_{g} \frac{(\text{fréquence}(\text{\textit{candidat}}, g) - n_tp_g)^2}{n_tp_g}
        \end{align}
        où $n_tp_g$ représente la fréquence de cooccurrence attendue entre le
        terme-clé candidat et le groupe $g$, $n_t$ étant le nombre de candidats
        avec lesquels le terme-clé candidat analysé cooccurre et $p_g$ étant la
        probabilité d'occurrence du groupe $g$ avec d'autres candidats.
        
        Lors de leurs expériences, les auteurs se sont aperçus que certains
        candidats peuvent être sémantiquement liées à des candidats fréquents
        dans un domaine plus général que celui du document. En supposant que ces
        cas spéciaux soient ceux ayant le plus fort biais, ils suppriment du
        $\chi^2$ l'argument maximum de la sommation~:
        \begin{align}
          \chi^2{'}(\text{\textit{candidat}}) = \chi^2 - \max_{g}\left\{\frac{(\text{fréquence}(\text{\textit{candidat}}, g) - n_tp_g)^2}{n_tp_g}\right\}
        \end{align}
        Les termes-clés extraits sont les $k$ termes-clés candidats ayant le
        plus fort biais mesuré avec la mesure $\chi^2{'}$.

        ~\\Dans l'algorithme KeyCluster, \newcite{liu2009keycluster} utilisent
        aussi un regroupement sémantique, mais dans leur cas ils ne considèrent
        que les mots du document, mots d'un antidictionnaire exclus. le mot le
        plus central de chaque groupe est sélectionné comme mot de référence et
        sert à l'extraction des termes-clés: chaque terme-clé candidat contenant
        au moins un mot de référence est extrait comme terme-clé. Cette méthode
        présente l'avantage d'offrir une bonne couverture des sujets abordés
        dans un document, car tous les groupes sémantiques sont représentés par
        au moins un terme-clé. Cependant, aucune pondération n'est proposée pour
        ordonner les termes-clés.

      \subsubsection{Méthodes à base de graphe}
      \label{subsubsec:main-state_of_the_art-automatic_keyphrase_extraction-unsupervised_keyphrase_extraction-graph_based_approaches}
        Les approches à base de graphe sont actuellement les plus populaires.
        Utilisés dans de nombreuses applications du
        \textsc{Tal}~\cite{kozareva2013textgraphs}, les graphes ont l'avantage
        de présenter de manière simple et efficace les unités textuelles d'un
        document ainsi que les relations qu'elles entretiennent.

        ~\\\newcite{mihalcea2004textrank} proposent TextRank, une méthode
        d'ordonnancement d'unités textuelles à partir d'un graphe pour le résumé
        automatique et l'extraction de termes-clés. Pour faire de l'extraction
        de termes-clés, les n\oe{}uds du graphe sont les mots du document et les
        arêtes qui les connectent représentent leurs relations d'adjacence dans
        le document, c'est-à-dire leurs relations de cooccurrences dans une
        fenêtre de deux mots. Un score d'importance est calculé pour chaque mot
        à partir de l'algorithme PageRank~\cite{brin1998pagerank} qui est issu
        de la mesure de centralité des vecteurs propres. Le principe utilisé est
        celui de la recommandation (du vote)~: un mot est d'autant plus
        important s'il cooccurre avec un grand nombre de mots et si les mots
        avec lesquels il cooccurre sont eux aussi importants. Les mots les plus
        importants sont considérés comme des mots-clés, ces mots-clés sont
        marqués dans le document et les plus longues séquences de mots-clés
        adjacents sont extraites en tant que termes-clés.
      
        Soit $G = (N, A)$, le graphe non orienté de cooccurrences de mots où les
        n\oe{}uds $N$ représentent les mots du documents, et où les arêtes $A$
        les connectent lorsque les mots cooccurrent dans le document.
        L'importance de chaque mot $n_i$ est obtenue récursivement selon la
        formule suivante~:
        \begin{align}
          S(n_i) &= (1 - \lambda) + \lambda \times \sum_{n_j \in A(n_i)} \frac{S(n_j)}{|A(n_j)|} \label{math:textrank}
        \end{align}
        où $A(n_i)$ est l'ensemble des n\oe{}uds connectés au n\oe{}ud $n_i$ et
        où $\lambda$ est un facteur d'atténuation. Défini entre 0 et 1, ce
        dernier peut être considéré comme la probabilité pour que le n\oe{}ud
        $n_i$ soit atteint par recommandation.

        ~\\\newcite{wan2008expandrank} modifient TextRank et proposent
        SingleRank. Dans un premier temps, leur méthode augmente la précision de
        l'ordonnancement en utilisant une fenêtre de cooccurrences élargie
        (empiriquement) à dix et en pondérant les arêtes par le nombre de
        cooccurrences entre les deux mots qu'elles connectent (cf.
        équation~\ref{math:singlerank}). Dans un second
        temps, les termes-clés ne sont plus générés à partir des séquences de
        mots-clés dans le document, mais ordonnés à partir de
        la somme du score d'importance des mots qui les composent. Cette
        nouvelle méthode donne, dans la majorité des cas, des résultats
        meilleurs que ceux de TextRank.
        \begin{align}
          S(n_i) &= (1 - \lambda) + \lambda \times \sum_{n_j \in A(n_i)} \frac{\text{poids}_{ji} \times S(n_j)}{\mathlarger{\sum}_{n_k \in A(n_j)} \text{poids}_{jk}} \label{math:singlerank}
        \end{align}

        ~\\Toujours dans le but d'améliorer l'efficacité de l'ordonnancement
        proposé par \newcite{mihalcea2004textrank}, \newcite{wan2008expandrank}
        étendent SingleRank en utilisant des documents similaires au document
        analysé, selon la mesure de similarité vectorielle cosinus. Faisant
        l'hypothèse que ces documents similaires fournissent des informations
        supplémentaires relatives aux mots du document et aux relations qu'ils
        entretiennent, ils utilisent les relations de cooccurrences observées
        dans les documents similaires pour ajouter et renforcer des arêtes dans
        le graphe. Cette approche donne des résultats au-delà de ceux de
        SingleRank. Toutefois, ses performances sont fortement liées à la
        disponibilité de documents similaires. Leur usage peut aussi ajouter et
        renforcer des connexions qui ne devraient pas l'être s'ils ne sont pas
        suffisamment similaires. Pour pallier ce problème, les auteurs pondèrent
        l'impact des documents similaires à partir leur degré de similarité avec
        le document.

        ~\\\newcite{tsatsaronis2010semanticrank} tentent eux aussi d'améliorer
        TextRank. Dans leur méthode, ils créent et pondèrent une arête entre
        deux mots si et seulement si ceux-ci sont sémantiquement liés dans
        WordNet~\cite{miller1995wordnet} ou dans
        Wikipedia~\cite{milne2008wikipediasemanticrelatedness} (cf.
        équation~\ref{math:semanticrank}). WordNet est une base de données
        lexicales qui fournit un vecteur de synonymes pour les noms, les verbes,
        les adverbes et les adjectifs. Un vecteur de synonymes est ici considéré
        comme l'ensemble de tous les sens possibles pour un mot $n_i$, c'est son
        vecteur sémantique. À partir des vecteurs sémantiques de deux mots $n_i$
        et $n_j$, \newcite{tsatsaronis2010semanticrank} déterminent toutes les
        paires de sens $P_{ij}$ possibles, ainsi que tous les chemins $C_{i, j}$
        possible pour atteindre un sens de $n_j$ à partir d'un sens de $n_i$. Le
        score de similarité sémantique avec WordNet est obtenu en trouvant le
        couple paire sémantique/chemin sémantique pour lequel le produit des
        mesures sémantiques \textit{Semantic Compactness Measure}
        ($\textsc{Scm}$) et \textit{Semantic Path Elaboration} ($\textsc{Spe}$),
        introduites par \newcite{tsatsaronis2010textrelatedness}, est le plus
        élevé (cf. équation \ref{math:wordnetsemanticrelatedness}). Dans le cas
        où l'un des termes-clés candidats n'est pas présent dans WordNet, la
        similarité sémantique est calculée avec les données de Wikipédia (cf.
        équation~\ref{math:wikipediasemanticrelatedness}).
        \begin{align}
          \text{poids}_{j, i} &= \left\{\begin{array}{ll}
            1 & \text{si $n_i = n_j$}\\
            \text{Sim}_{WN}(n_i, n_j) & \text{sinon, si $n_i, n_j \in \text{WordNet}$}\\
             \text{Sim}_{W}(n_i, n_j) &  \text{sinon, si $n_i, n_j \in \text{Wikipedia}$}\\
            0 & \text{sinon}
          \end{array}\right. \label{math:semanticrank}\\
          \notag\\
          \text{Sim}_{WN}(n_i, n_j) &= \max_{p \in P_{i, j}}\left\{\max_{c \in C_{i, j}}\left\{\textsc{Scm}(p, c) \times \textsc{Spe}(p, c)\right\}\right\} \label{math:wordnetsemanticrelatedness}\\
          \notag\\
          \text{Sim}_{W}(n_i, n_j) &= \frac{\log(\max(|\text{art}(i)|, |\text{art}(j)|)) - \log(|\text{art}(i) \cup \text{art}(j)|)}{\log(|\text{Wikipedia}|) - \log(\min(|\text{art}(i)|, |\text{art}(j)|))} \label{math:wikipediasemanticrelatedness}\\
          \notag\\
          \text{art}(i) &= \left\{\text{\textit{article}} \in \text{Wikipedia}\ |\ n_i \in \text{\textit{article}}\right\} \notag
        \end{align}

        Cette modification seule donne de moins bons résultats que TextRank.
        Toutefois, elle améliore les résultats en combinaison avec un
        ordonnancement biaisé par le \textsc{Tf-Idf} des mots (cf
        équation~\ref{math:apw}) ou avec un ordonnancement dont le facteur
        $\lambda$ est  propre à chaque mot (cf équation~\ref{math:ppr}). Ce
        dernier est calculé selon l'apparition ou non du mot dans le titre du
        document.
        \begin{align}
          S_{\text{\textsc{Tf-Idf}}}(n_i) &= \frac{1}{2} \times \left(\frac{S(n_i)}{\mathlarger{\max}_{n_j \in N}(S(n_j))} + \frac{\text{\textsc{Tf-Idf}}(t_i)}{\mathlarger{\max}_{n_j \in N}(\text{\textsc{Tf-Idf}}(t_j))}\right) \label{math:apw}\\
          \notag\\
          S_{\lambda}(n_i) &= (1 - \lambda_i) + \lambda_i \times \sum_{n_j \in A(n_i)} \frac{\text{poids}_{j, i} \times S_{\lambda}(n_j)}{\mathlarger{\sum}_{n_k \in A(n_j)} p_{j, k}} \label{math:ppr}
        \end{align}

        ~\\\newcite{liu2010topicalpagerank} reprennent SingleRank et proposent
        TopicalPageRank (\textsc{Tpr}), une méthode qui cherche cette fois-ci à augmenter
        la couverture du document par les termes-clés extraits. Pour ce faire,
        ils détectent les sujets du document et ordonnent les mots en fonction
        de chaque sujet (un ordonnancement par sujet). En utilisant le modèle
        \textsc{Lda}~\cite{blei2003lda}, ils biaisent chaque ordonnancement avec
        un sujet (cf. équation~\ref{math:topicalpagerank}) et donnent le plus
        d'importance aux candidats dont les mots ont un meilleur rang pour le
        plus de sujets (cf. équation~\ref{math:topicalpagerankfinalscore}).
        \begin{align}
          S(n_i, \text{\textit{sujet}}) &= (1 - \lambda) \times p(\text{\textit{sujet}} | n_i) + \lambda \times \sum_{n_j \in A(n_i)} \frac{p_{j, i} \times S(n_j)}{\mathlarger{\sum}_{N_k \in A(N_j)} p_{j, k}} \label{math:topicalpagerank}\\
 \textsc{Tpr}(\text{\textit{candidat}}) &= \mathlarger{\sum}_{\text{\textit{sujet}}} \left[p(\text{\textit{sujet}} | \text{\textit{document}}) \times \sum_{n \in \text{\textit{candidat}}} \text{rang}_{\text{\textit{sujet}}}(n)\right] \label{math:topicalpagerankfinalscore}
        \end{align}

        ~\\Dans la continuité du travail de \newcite{liu2010topicalpagerank},
        \newcite{zhang2013wordtopicmultirank} ajoutent les sujets déterminés par
        \textsc{Lda} aux n\oe{}uds du graphe de cooccurrences et n'effectuent
        plus qu'un seul ordonnancement. Mots et sujets sont ordonnés
        simultanément, de sorte qu'un sujet soit d'autant plus important
        s'il est connecté à un grand nombre de mots importants et qu'un mot est
        d'autant plus important s'il cooccurre avec un grand nombre de mots
        importants et s'il est connecté à un grand nombre de sujets importants.

    \subsection{Approche supervisée}
    \label{subsec:main-state_of_the_art-automatic_keyphrase_extraction-supervised_keyphrase_extraction}
      Les méthodes supervisées apprennent principalement à classer les
      termes-clés en tant que \og{}terme-clé\fg{} ou \og{}non terme-clé\fg{}.
      Leur apprentissage se fait à partir d'un corpus d'apprentissage
      (d'entraînement) dont les documents sont annotés en termes-clés. Les
      termes-clés candidats sont sélectionnés dans ces documents et servent
      d'exemples et de contres-exemples.

      De nombreuses méthodes sont proposées, utilisant différents classifieurs
      et présentant de nouveaux traits (caractéristiques) discriminants pour
      l'extraction automatique de termes-clés. Dans la suite, nous présentons
      ces méthodes, regroupés par classifieurs, et soulignons les nouveaux
      traits proposés. Notez qu'il existe aussi une étude comparative des
      performances des différents classifieurs pour la tache d'extraction
      automatique de termes-clés~\cite{sarkar2012machinelearningcomparison}.

      \subsubsection{Classifieurs probabilistes}
      \label{subsubsec:main-state_of_the_art-automatic_keyphrase_extraction-supervised_keyphrase_extraction-probabilistic_models}
        Les classifieurs probabilistes utilisent des distributions de
        probabilités selon divers traits. Pour l'extraction des termes-clés d'un
        document, ces distributions sont combinées pour déterminer un score de
        vraisemblance de chaque terme-clé candidat en tant que termes-clés du
        document.

        ~\\\textsc{Kea}~\cite{witten1999kea} est la méthode d'extraction de
        termes-clés la plus populaire. Elle effectue une classification naïve
        bayesienne pour attribuer le score de vraisemblance de chaque terme-clé
        candidat. Elle combine les distributions probabilistes de deux traits~:
        la première position du candidat dans le document et son poids \textsc{Tf-Idf}.
        L'intuition de \newcite{witten1999kea} est que les termes-clés ont une
        certaine importance vis-à-vis du document (leur poids \textsc{Tf-Idf}) et qu'ils
        font leur première apparition dans des zones similaires du document.

        ~\\\textsc{Kea} est une approche très simple qui considère tous les
        traits comme indépendants (principe de la classification naïve
        bayésienne). Sa simplicité et ses bonnes performances ont suscité un
        grand intéret, il en existe de nombreuses variantes. C'est le cas de la
        méthode de \newcite{frank1999keafrequency}, qui utilise comme trait
        supplémentaire le nombre de fois qu'un terme-clé candidat est un exemple
        dans le corpus d'apprentissage. Cette méthode donne un plus fort score
        de vraisemblance aux candidats déjà utilisés comme termes-clés. Elle
        améliore les performances de \textsc{Kea}, mais nécessite plus de
        données d'apprentissage.
        
        ~\\\newcite{turney2003keacoherence} reprend lui aussi \textsc{Kea}.
        Estimant, comme \newcite{ding2011binaryintegerprogramming}, que les
        termes-clés d'un document doivent être cohérent entre eux (cf.
        section~\ref{subsubsec:main-state_of_the_art-automatic_keyphrase_extraction-unsupervised_keyphrase_extraction-statistical_approaches}
        page~\pageref{subsubsec:main-state_of_the_art-automatic_keyphrase_extraction-unsupervised_keyphrase_extraction-statistical_approaches:ilp}),
        il ajoute une deuxième classification naïve bayesienne après celle de
        \textsc{Kea}. La première classification sert à ordonner les candidats
        selon leur vraisemblance et la deuxième attribue un nouveau score de
        vraisemblance aux candidats, de sorte que les $L$ meilleurs candidats
        aient un meilleur score de vraisemblance s'ils ont un fort lien
        sémantique avec un ou plusieurs candidat(s) parmi les $K$ meilleurs ($K
        < L$). La force du lien sémantique est représenté par deux scores (soit
        $2 \times K$ traits)~: le nombre de pages Web contenant les deux
        candidats et le nombre de titres de pages Web contenant les deux
        candidats.

        ~\\\newcite{nguyen2007keadocumentstructure} améliorent \textsc{Kea} pour
        l'extraction de termes-clés à partir d'articles scientifiques. Faisant
        l'hypothèse que les termes-clés n'ont pas une répartition homogène dans
        les sections d'un article scientifique, ils notent les occurrences des
        termes-clés candidats dans les sections génériques d'un article
        scientifique (résumé, introduction, motivations, état de l'art et
        conclusion), puis utilisent le vecteur d'occurrences ainsi construit
        comme trait supplémentaire. De cette manière, les termes-clés
        apparaissant dans les sections les plus susceptibles de contenir des
        termes-clés ont un score de vraisemblance plus élevé.
        
        ~\\\newcite{caragea2014citationenhancedkeyphraseextraction} utilisent
        eux aussi un classifieur naïf bayesien. Leur méthode repose sur le même
        constat que \newcite{wan2008expandrank}~: l'extraction de termes-clés
        peut bénéficier des informations extraites dans des documents en liens
        avec le document à partir duquel les termes-clés doivent être extraits
        (cf.
        section~\ref{subsubsec:main-state_of_the_art-automatic_keyphrase_extraction-unsupervised_keyphrase_extraction-graph_based_approaches}
        page~\pageref{subsubsec:main-state_of_the_art-automatic_keyphrase_extraction-unsupervised_keyphrase_extraction-graph_based_approaches}).
        Travaillant avec des articles scientifiques,
        \newcite{caragea2014citationenhancedkeyphraseextraction} utilisent le
        réseau de citations des documents afin de déterminer leur influence sur
        les autres documents et inversement. En plus des traits existant, ils
        ajoutent un \textsc{Tf-Idf} calculés à partir de la fréquence de chaque
        candidat dans les contextes citationnels, ainsi que deux traits binaires
        indiquant si le candidat apparait dans une phrase du document qui cite
        un autre document ou s'il apparait dans une phrase, d'un autre document,
        qui cite le document. Bien que leur méthode obtient de meilleures
        performances que \textsc{Kea}, les auteurs mettent en évidence un défaut
        de leur approche. Concidérant les contextes citationnels, un document
        qui vient d'être publié ne peut pas avoir été cité par d'autres articles
        et leur extraction de termes-clés pour un document tant à s'améliorer
        dans le temps.

        ~\\Contrairement à \newcite{witten1999kea}, qui utilisent un classifieur
        naïf bayésien et considèrent que tous les traits sont indépendants,
        \newcite{sujian2003maximumentropy} proposent une méthode utilisant un
        classifieur d'entropie maximale. Ce classifieur cherche parmi plusieurs
        distributions (une pour chaque trait) laquelle a la plus forte entropie.
        La distribution ayant la plus forte entropie est par définition celle
        qui contient le moins d'information, ce qui la rend moins arbitraire et
        donc plus appropriée pour l'extraction automatique de termes-clés.
        Chaque trait se voit donc attribuer un poids, de sorte que les traits
        les moins arbitraires aient le plus de poids dans la classification. En
        plus des traits cités pour les méthodes précédentes, et à l'instar de
        \newcite{nguyen2007keadocumentstructure}, ils tirent parti d'autres
        traits liés à la nature des documents qu'ils traitent. Ainsi, pour des
        articles journalistiques ils utilisent leur type (information, sport,
        etc.) et la catégorie d'entité nommée des candidats s'ils en sont une
        (personne, pays, organisme, etc.).

        ~\\Plus récemment, le travail de \newcite{zhang2008crfkeywordextraction}
        montre l'applicapilité d'un \textsc{Crf} (\textit{Conditional Random
        Field}) à la tâche d'extraction de termes-clés. Ce classifieur a
        l'intéressante capacité à prédire des séquences de classes, soit à
        étiqueter tout un document en donnant les classes suivantes pour chaque
        mot~: \og{}terme-clé\fg{}, \og{}début d'un terme-clé\fg{}, \og{}partie
        d'un terme-clé\fg{} et \og{}non terme-clé\fg{} (\TODO{etiqueter
        l'exemple fil rouge}). \newcite{zhang2008crfkeywordextraction}
        reprend les traits présentés précédemment (\textsc{Tf-Idf}, première position,
        section d'article scientifique, etc.) et y ajoute le contexte de
        chaque mot. Nous trouvons cette notion de contexte dans les méthodes non
        supervisées à base de graphe (cf.
        section~\ref{subsubsec:main-state_of_the_art-automatic_keyphrase_extraction-unsupervised_keyphrase_extraction-graph_based_approaches}
        page~\pageref{subsubsec:main-state_of_the_art-automatic_keyphrase_extraction-unsupervised_keyphrase_extraction-graph_based_approaches}),
        mais il s'agit de la première fois que nous trouvons celle-ci dans une
        méthode supervisée. Le fonctionnement particulier du \textsc{Crf} se
        prête plus à l'utilisation du contexte que les autres classifieurs.

      \subsubsection{Arbres de décision}
      \label{subsubsec:main-state_of_the_art-automatic_keyphrase_extraction-supervised_keyphrase_extraction-decision_trees}
        Les arbres de décision sont des classifieurs dont les branches
        représentent des tests sur des traits des candidats. Ces tests routent
        les candidats vers les feuilles de l'arbre représentant leur classe
        respective (\og{}terme-clé\fg{} ou \og{}non terme-clé\fg{}).

        ~\\Dans son article sur l'apprentissage pour l'extraction automatique de
        termes-clés, \newcite{turney1999learningalgorithms} entraîne 50 arbres
        de décision (technique de \textit{random forest}) et réduit la tâche
        d'extraction de termes-clés à un vote. Les arbres de décision classent
        indépendemment chaque candidat et les candidats majoritairement
        classés \og{}terme-clé\fg{} sont extraits comme termes-clés. Parmi les
        nombreux traits qu'utilise \newcite{turney1999learningalgorithms}, les
        plus novateurs sont des traits binaires qui visent des catégories
        grammaticales précises~: \og{}contient un nom propre~?\fg{},
        \og{}contient un verbe usuel~?\fg{} et \og{}se termine par un
        adjectif~?\fg{}. Contrairement aux autres travaux, celui de
        \newcite{turney1999learningalgorithms} est aussi l'un des seuls à
        décliner les traits à des niveaux de granularité différents. Chaque
        trait concernent le candidat dans son ensemble et ses mots séparément.

        ~\\\newcite{ercan2007lexicalchains} utilisent eux aussi des arbres de
        décision pour extraire les termes-clés. Les termes-clés sont ici
        restreints aux mots-clés. L'aspect novateur de leur méthode est l'usage
        de chaînes lexicales pour la définition de nouveaux traits
        discriminants. Une chaîne lexicale est un graphe de mots liés entre eux
        hiérarchiquement (cf. figure~\ref{fig:lexical_chain}).
        \newcite{ercan2007lexicalchains} tiennent compte des relations
        hiérarchiques de méronymie\footnote{Méronyme~: mot dont le signifié est
        une sous-partie de celui d'un autre mot, son holonyme. Par exemple,
        \og{}bras\fg{} est un méronyme de
        \og{}corps\fg{}.}/holonymie\footnote{Holonyme~: mot dont le signifié est
        composé de celui d'un autre mot, son méronyme.},
        d'hyponymie\footnote{Hyponyme~: mot dont le signifié est
        plus spécifique que celui d'un autre mot, son
        hyperonyme.}/hyperonymie\footnote{Hyperonyme~: mot dont le sygnifié est
        plus général que celui d'un autre mot, son hyponyme.} et de synonymie,
        auxquelles ils donnent un poids (4 pour la méronymie/holonymie, 7 pour
        l'hyponymie/hyperonymie et 10 pour la synonymie). Chaque mot se voit
        attribuer quatre traits correspondant à quatre scores obtenus à partir
        des poids des relations~:
        \begin{enumerate}
          \item{Score de la chaîne lexicale~: somme du poids de toutes les
                relations de la chaîne lexicale~;}
          \item{Score du mot dans la chaîne lexicale~: somme du poids de toutes
                les relations du mot avec les autres mots de la chaîne
                lexicale~;}
          \item{Couverture de la chaîne lexicale~: différence entre la dernière
                occurrence, dans le document, d'un mot de la chaîne lexicale
                avec la première occurrence, dans le document, d'un mot de la
                chaîne lexicale~;}
          \item{Couverture du mot et de ses voisins dans la chaîne lexicale~:
                de même que la couverture de la chaîne lexicale, mais en tenant
                compte uniquement du mot et de ses voisins dans la chaîne.}
        \end{enumerate}
        \begin{figure}
          \centering
          \begin{tikzpicture}
            \node (programme) {programme};
            \node [below=of programme] (logiciel) {logiciel};
            \node [right=of logiciel, xshift=2em] (paquetage) {paquetage};
            \node [left=of logiciel, xshift=-2em] (application) {application};
            \node [below=of application] (app) {app};

            \draw (programme) -- (logiciel);
            \draw (programme) -- (application);
            \draw (programme) -- (paquetage);
            \draw (application) -- (logiciel);
            \draw (logiciel) -- (paquetage);
            \draw [dashed] (application) -- (app);

            % legend
            \node [scale=.75, right=of app, xshift=12em, yshift=3em] (legend_title) {\underline{Légende~:}};
            \node [scale=.75, below=of legend_title, xshift=-1.5em, yshift=3em] (begin_hyponym) {};
            \node [scale=.75, right=of begin_hyponym, xshift=-1em] (end_hyponym) {: hyponymie/hyperonymie};
            \node [scale=.75, below=of begin_hyponym, yshift=3em] (begin_synonym) {};
            \node [scale=.75, right=of begin_synonym, xshift=-1em] (end_synonym) {: synonymie};

            \draw (legend_title.north  -| end_hyponym.east) rectangle (end_synonym.south -| legend_title.west);

            \draw (begin_hyponym) -- (end_hyponym);
            \draw [dashed] (begin_synonym) -- (end_synonym);
          \end{tikzpicture}
          \caption{Exemple de chaîne lexicale~\cite{ercan2007lexicalchains}
                   \label{fig:lexical_chain}}
        \end{figure}

        ~\\Tirant aussi profit d'arbres de décision, \newcite{lopez2010humb}
        sont les vainqueurs de la campagne d'évaluation
        SemEval-2010~\cite{kim2010semeval}. Ils extraient les termes-clés en
        deux étapes. Tout d'abord, ils ordonnent les termes-clés candidats avec
        les arbres de décision, puis ils les ré-ordonnent à la manière de
        \newcite{turney2003keacoherence}~: les candidats bien classés
        initialement sont d'autant mieux classés qu'ils ont un fort lien
        sémantique avec d'autres candidats bien classés initialement.

      \subsubsection{Séparateurs à vastes marges}
      \label{subsubsec:main-state_of_the_art-automatic_keyphrase_extraction-supervised_keyphrase_extraction-svms}
        Les séparateurs à vastes marges (\textsc{Svm}) projettent les exemples
        et les contres-exemples du corpus d'entraînement sur un plan selon la
        valeur de leurs traits, puis construisent l'hyperplan qui les sépare.
        Pour classer les termes-clés candidats d'un document, il suffit ensuite
        de les projeter sur ce même plan et d'utiliser l'hyperplan appris.

        ~\\\newcite{zhang2006svm} utilisent un \textsc{Svm} pour extraire les
        termes-clés à partir de ce qu'ils appellent le contexte global et le
        contexte local des termes-clés candidats. Ils représentent le contexte
        global d'un candidats par son \textsc{Tf-Idf}, sa première position et
        ses occurrences dans différentes parties du document, tandis qu'ils
        représentent son contexte local par sa catégorie grammaticale et trois
        traits encore jamais utilisés auparavant. Les deux premiers traits sont
        déterminés à partir des dépendances entre les mots. À la manière des
        méthodes non supervisées à base de graphe (cf.
        section~\ref{subsubsec:main-state_of_the_art-automatic_keyphrase_extraction-unsupervised_keyphrase_extraction-graph_based_approaches}
        page~\pageref{subsubsec:main-state_of_the_art-automatic_keyphrase_extraction-unsupervised_keyphrase_extraction-graph_based_approaches})
        l'un des traits dénote le nombre de fois que le candidat modifie un mot
        et l'autre dénote le nombre de fois qu'un mot modifie le candidat. Le
        dernier trait s'appelle le \textsc{Tf-Idf} contextuel, il s'agit de la
        somme du \textsc{Tf-Idf} de tous les mots qui cooccurrent avec le
        candidat. Ce dernier trait indique si un candidat occurre dans un
        contexte important vis-à-vis du document.

        ~\\\newcite{jiang2009rankingsvm} extraient les termes-clés à partir d'un
        type particulier de \textsc{Svm}, baptisé
        \textsc{Svm}$^\textnormal{rank}$. \textsc{Svm}$^\textnormal{rank}$
        construit plusieurs hyperplans qui permettent d'ordonner les termes-clés
        candidats. Utilisant le score \textsc{Tf-Idf} des candidats, leur taille
        (en nombre de mots), leur première position, leur entropie et d'autres
        traits, le travail de \newcite{jiang2009rankingsvm} montre que le
        classifieur \textsc{Svm}$^\textnormal{rank}$ est plus performant qu'un
        \textsc{Svm} ou qu'un classifieur naïf bayesien utilisant les mêmes
        traits.

        ~\\\newcite{eichler2010keywe} extraient eux aussi les termes-clés à
        partir d'un \textsc{Svm}$^\textnormal{rank}$. Ils apprennent le
        \textsc{Svm}$^\textnormal{rank}$ avec trois valeurs pour le rang. La
        valeur maximale est attribuée aux exemples d'un document
        d'apprentissage, la valeur minimale aux contres-exemples du document et
        une valeur intermédiaire à ses contres-exemples qui sont des exemples
        d'autres documents d'apprentissage. Cette approche peut être assimilé à
        celle de \newcite{frank1999keafrequency}, qui estime qu'un terme-clé
        candidat fréquemment annoté comme terme-clé dans le corpus
        d'apprentissage est plus vraisemblablement un terme-clé. Quant aux
        traits utilisés pour entraîner le \textsc{Svm}$^\text{rank}$, le plus
        notable se réfère à Wikipedia. L'intuition des auteurs est que si un
        terme-clé candidat fait l'objet d'un article Wikipedia, alors il est
        plus vraisemblablement un terme-clé.

      \subsubsection{Perceptrons multicouches}
      \label{subsubsec:main-state_of_the_art-automatic_keyphrase_extraction-supervised_keyphrase_extraction-neural_network}
        Les perceptrons multicouches sont des classifieurs qui émulent la
        biologie de l'apprentissage des êtres vivants. Ce sont des réseaux de
        neurones répartis sur au moins trois couches. Les neurones de la
        première couche représentent les traits d'un candidat (un neurone par
        trait), ceux des couches intermédiaires (couches cachées) propagent des
        scores obtenus selon la valeur des traits et ceux de la dernière couche
        donnent un score final pour chaque classe \og{}terme-clé\fg{} et
        \og{}non terme-clé\fg{} (un neurone par classe). La classe ayant le plus
        haut score est celle du terme-clé candidat pour lequel correspond la
        valeur des traits.
        
        ~\\\newcite{sarkar2010neuralnetwork} utilisent un perceptron multicouche
        pour classer les termes-clés candidats selon leur \textsc{Tf-Idf}, leur
        position, leur taille (en nombre de mots) et celle de leurs mots (en
        nombre de caractères). L'une des principales contributions de leur
        travail est l'apprentissage d'un estimateur capable de donner le degré
        de confiance d'une classification selon le score attribué à chaque
        classe. Cet estimateur permet d'ordonner les termes-clés candidats,
        ceux classés \og{}terme-clé\fg{} en premier dans l'ordre décroissant de
        leur confiance et ceux classés \og{}non terme-clé\fg{} en dernier dans
        l'ordre croissant de leur confiance. Ainsi, la méthode de
        \newcite{sarkar2010neuralnetwork} peut extraire un nombre donné de
        termes-clés, en ajoutant des candidats classés \og{}non terme-clé\fg{}
        avec une faible confiance si nécessaire.

      \subsubsection{Algorithmes génétiques}
      \label{subsubsec:main-state_of_the_art-automatic_keyphrase_extraction-supervised_keyphrase_extraction-genex}
        Les algorithmes génétiques sont des algorithmes qui donnent une solution
        approchée à un problème d'optimisation. Ce type d'algorithme
        n'effectue pas de classification et n'est pas nécessairement supervisé.

        ~\\\newcite{turney1999learningalgorithms} propose une méthode
        supervisée, GenEx, dont les paramètres sont valués par un algorithme
        génétique, appelé \textit{Genitor}. Un algorithme d'extraction de
        termes-clés, appelé \textit{Extractor}, est appliqué sur le corpus
        d'apprentissage avec des paramètres initiaux, puis le \textit{Genitor}
        fait évoluer la valeur de ses paramètres jusqu'à trouver celle qui
        maximise les performances de l'extraction. L'extraction des termes-clés
        d'un document se fait ensuite avec l'\textit{Extractor} et ses
        paramètres configurés par le \textit{Genitor}. Les paramètres appris
        sont principalement des seuils limitant la taille des candidats, le
        nombre de mots importants à considérer pour filtrer les candidats, ou
        encore le nombre de termes-clés à extraire. Ce sont aussi des facteurs
        multiplicateurs utilisés notamment pour le calcul de l'importance des
        mots et des candidats.

  \section{Assignement automatique de termes-clés}
  \label{sec:main-state_of_the_art-automatic_keyphrase_assignment}
    L'assignement automatique de termes-clés fait l'objet de moins de travaux
    que l'extraction automatique de termes-clés. Il s'agit aussi d'une tâche
    plus difficile, car elle doit assigner des entrées d'un vocabulaire contrôlé
    en tant que termes-clés d'un document indépendemment de leur apparition dans
    celui-ci.

    ~\\\newcite{medelyan2008smalltrainingset} sont les premiers à proposer une
    méthode capable de faire de l'assignement automatique de termes-clés. Leur
    méthode, \textsc{Kea}++, améliore la méthode supervisée d'extraction de
    termes-clés \textsc{Kea}. Elle utilise un thésaurus\footnote{Thésaurus~:
    Liste de termes regroupés selon les concepts d'un domaine de connaissance
    qu'ils représentent.} du domaine des documents à traiter pour affiner la
    sélection des candidats et ajouter un trait au classifieur naïf bayesien.

    Lors de la sélection des candidats, les n-grammes qui ne font pas partie des
    entrées du thésaurus sont jugés mal formés et sont éliminés s'ils ne peuvent
    pas être remplacés par une autre forme. \textsc{Kea}++ accepte les entrées
    du thésaurus qui partagent la racine de leurs mots avec un n-gramme du
    document, ainsi que les entrées du thésaurus qui sont des synonymes d'un
    n-gramme du document. De cette manière, l'assignement peut s'effectuer. Tous
    les candidats font partie d'un vocabulaire contrôlé et certains d'entre eux
    n'apparaissent pas dans le document.

    \newcite{medelyan2008smalltrainingset} tirent aussi profit du thésaurus lors
    de la classification des candidats. Ils ajoutent un trait indiquant la
    relation qu'entretient chaque candidat avec les autres dans le thésaurus (le
    nombre de relations). De cette manière, ils imitent le travail de
    \newcite{turney2003keacoherence} qui tente d'améliorer la cohérence entre
    les termes-clés extraits par \textsc{Kea}.

    \textsc{Kea}++ obtient des performances une fois supérieures à celles de
    \textsc{Kea}, sans en
    augmenter grandement sa complexité. Le travail de
    \newcite{medelyan2008smalltrainingset} montre donc l'importance d'utiliser
    une méthode d'assignement de termes-clés lorsque les données disponibles le
    permettent.

    ~\\\newcite{liu2011vocabularygap} proposent une méthode pouvant effectuer de
    l'assignement de termes-clés. Leur approche consiste à transformer le
    problème d'indexation par termes-clés en un problème de traduction. Leur
    idée est la suivante~: un document et ses termes-clés expriment le même
    contenu dans deux langues différentes, l'une expressive et l'autre plus
    concise et plus synthétique. Les termes-clés produits sont limités par ceux
    utilisés lors de l'apprentissage  et n'apparaissent pas systématiquement
    dans le document. Dans leur travail, ils apprennent un modèle de traduction
    à partir des éléments du titre à la place des termes-clés annotés
    manuellement ou d'un vocabulaire contrôlé. Dans leur cas, il ne s'agit donc
    pas rigoureusement d'assignement de termes-clés. Leur approche y est
    toutefois appropriée.

  %-----------------------------------------------------------------------------

  \section{Évaluation automatique de l'indexation par termes-clés}
  \label{sec:main-state_of_the_art-automatic_evaluation_of_keyphrase_annotation}
    Pour montrer l'apport des nouvelles méthodes d'indexation par termes-clés,
    celles-ci sont comparées aux méthodes existantes dans un processus
    d'évaluation \og{}à la Cranfield\fg{} \citep{voorhees2002philosophy}. Chaque
    méthode est appliquée à un ensemble de documents de test (corpus de test) et
    les termes-clés qu'elle extrait sont comparés à un ensemble de termes-clés
    associés manuellement aux documents (jugements de référence). Un jugement de
    référence est supposé unique, la comparaison entre un terme-clé extrait
    et un terme-clé de référence est donc binaire. Le résultat des comparaisons
    pour chaque document est analysé selon différents critères (le moins
    d'erreurs, le plus de termes-clés correctes, etc.) et la méthode respectant
    en moyenne le mieux ces critères est jugée la plus performante.

    En indexation par termes-clés, il est courant d'évaluer une méthode en
    termes de précision, de rappel et de f-mesure. La précision capture la
    capacité d'une méthode à minimiser les erreurs (faux positifs, cf.
    tableau~\ref{tab:confusion_matrix} et équation~\ref{math:precision}). En
    opposition, le rappel ne considère pas la quantité d'erreurs et mesure la
    capacité de la méthode à fournir le plus possible de termes-clés correctes
    (vrais positifs, cf tableau~\ref{tab:confusion_matrix} et
    équation~\ref{math:recall}). Quant à la f-mesure, celle-ci mesure le
    compromis entre précision et rappel, soit la capacité de la méthode à
    extraire un maximum de termes-clés correctes en faisant un minimum d'erreurs
    (cf. équation~\ref{math:f_measure}).
    \begin{table}
      \begin{center}
        \begin{tabular}{cc|cc}
          \toprule
          \multicolumn{2}{c|}{} & \multicolumn{2}{c}{\textbf{Jugement de référence}}\\
          \multicolumn{2}{c|}{} & \og{}terme-clé\fg{} & \og{}non terme-clé\fg{}\\
          \hline
          \multirow{2}{*}{\textbf{Résultat}} & \og{}terme-clé\fg{} & vrai positif ($VP$) & faux positif ($FP$)\\
          & \og{}non terme-clé\fg{} & faux negatif ($FN$) & vrai negatif ($VN$)\\
          \bottomrule
        \end{tabular}
        \caption{Matrice de confusion pour l'évaluation des méthodes
                 d'indexation automatique par termes-clés
                \label{tab:confusion_matrix}}
      \end{center}
    \end{table}
    \begin{align}
      \text{précision} &= \frac{\#VP}{\#VP + \#FP} \label{math:precision}\\
      \notag\\
      \text{rappel} &= \frac{\#VP}{\#VP + \#FN} \label{math:recall}\\
      \notag\\
      \text{f-mesure} &= (1 + \beta^2) \times \frac{\text{précision} \times \text{rappel}}{(\beta^2 \times \text{précision}) + \text{rappel}} \label{math:f_measure}
    \end{align}
      
    En \textsc{Ri}, il est courant d'évaluer les méthodes selon la qualité de
    leur ordonnancement. Prenons l'exemple des moteurs de recherche, si deux
    moteurs de recherche doivent fournir dix documents répondant à une requête
    et que les deux systèmes n'ont que deux propositions pertinentes, alors ils
    ont tous les deux la même précision (20~\%) et le même rappel (20~\%).
    Toutefois, si le premier système classe ces documents en premier et que le
    second système les classe aux positions neuf et dix, alors le premier
    système est le meilleur. Lorsque les méthodes d'indexation par termes-clés
    le permettent, il est intéressant de mesurer leur capacité à classer en
    premier les vrais positifs. Dans la littérature, quatre mesures, dont
    certaines empruntées à la \textsc{Ri}, sont utilisées~: la \textsc{Map}
    (\textit{Mean Average Precision}), la Bpref (\textit{Binary Preference
    Measure}), la R-précision et la \textsc{Mrr} (\textit{Mean Reciprocal
    Rank}). La \textsc{Map} mesure pour chaque document la moyenne de la
    précision à chaque rang ($\textnormal{précision}@\textnormal{rang}$) d'un
    vrai positif (cf. équation~\ref{math:average_precision}). Avec la Bpref, la
    \textsc{Map} la seule mesure qui tient compte de tous les vrais positifs. La
    Bpref est une mesure similaire à la \textsc{Map} qui, contrairement à cette
    dernière, s'abstrait de la connaissance de tous les termes-clés de référence
    (cf. équation~\ref{math:bpref}). Elle peut donc, par exemple, s'appliquer
    dans le cadre d'évaluations manuelles où chaque terme-clé fourni par la
    méthode est jugé correcte ou non par un évaluateur humain et où tous les
    termes-clés du document ne sont pas connus. La R-précision est une variante
    de la précision. Elle mesure cette dernière dans le cas optimal (tous les
    termes-clés sont fournis et il n'y a aucune erreur), soit au rang $R$, où $R$
    est égale au nombre de termes-clés de référence du document (cf.
    équation~\ref{math:r_precision}). Quant à la \textsc{Mrr}, celle-ci est la
    moins précise, elle ne s'intéresse qu'au meilleur rang obtenu pour un vrai
    positif (cf. équation~\ref{math:reciprocal_rank})
    \begin{align}
      \text{\textit{average\_precision}} &= \frac{\mathlarger{\sum}_{\text{\textit{terme-clé}} = VP}\text{précision}@\text{rang}(\text{\textit{terme-clé}})}{\#VP + \#FN} \label{math:average_precision}\\
      \notag \\
      \text{Bpref} &= \sum_{\text{\textit{terme-clé}} = VP}{1 - \frac{|\text{\textit{terme-clé}'}\ \text{de meilleur rang que}\ \text{\textit{terme-clé}}|}{\#VP + \#FP}} \label{math:bpref}\\
      \notag \\
      \text{R-précision} &= \text{précision}@(\#VP + \#FN) \label{math:r_precision}\\
      \notag \\
      \text{\textit{reciprocal\_rank}} &= \frac{1}{\text{argmin}(\forall \text{\textit{terme-clé}} = VP, \text{rang}(\text{\textit{terme-clé}}))} \label{math:reciprocal_rank}
    \end{align}

    Toutes les mesures présentées précédemment respectent le paradigme de
    l'évaluation \og{}à la Cranfield\fg{}. Cependant, associer des termes-clés à
    un document est une tâche subjective~\cite{hasan2014state_of_the_art}, il
    n'y a donc pas une unique solution. Des travaux proposent d'apporter plus de
    souplesse à la comparaison entre termes-clés résultant d'une méthode et
    termes-clés de référence en tenant compte de leur
    chevauchement~\cite{zesch2009rprecision,kim2010rprecision}. Ces travaux
    ne sont toutefois pas utilisés pour l'évaluation des récents travaux.

  \section{Conclusion}
  \label{sec:main-state_of_the_art-automatic_evaluation_of_keyphrase_annotation-conclusion}
    Dans ce chapitre, nous présentons la tâche d'indexation par termes-clés, de
    la sélection des termes-clés candidats aux différentes méthodes d'extraction
    et d'assignement de termes-clés, sans oublier leur évaluation. Nos travaux
    de recherche s'intéressent à chacun de ces axes, sur la base d'observations
    critiques des précédents travaux.

    Les méthodes d'indexation par termes-clés sont presque toutes dépendantes
    de l'étape de sélection des termes-clés candidats. Bien que secondaire à la
    tâche, cette étape est d'autant plus importante que si elle fournit peu de
    candidats elle limite les performances d'une méthode d'indexation par
    termes-clés, soit le rappel maximum théorique, et que si elle fournit
    beaucoup de candidats elle augmente les risques de fournir des candidats
    erronés et donc de dégrader les performances de la méthode. Les méthodes de
    sélection de candidats présentées ont peu de fondements linguistiques.
    Utiliser des filtres linguistiques ou sélectionner les candidats avec des
    méthodes d'extraction
    terminologique~\cite{jacquemin1997fastr,daille2003acabit,rocheteau2011termsuite}
    sont deux solutions possibles pour extraire des candidats de bonne qualité
    et en nombre suffisant pour ne pas limiter la meilleure performance
    théorique.

    Parmi les méthodes d'indexation par termes-clés, la majorité extrait les
    termes-clés depuis le contenu du document. Cette extraction est réalisée de
    manière supervisée ou non supervisée. Les méthodes supervisées sont
    actuellement les plus performantes, cependant, celles-ci sont dépendantes
    d'un effort humain d'annotation manuelle des documents d'un corpus
    d'apprentissage et ne peuvent ensuite être appliquées qu'à des documents de
    même nature et de même domaine. Les méthodes non-supervisées ont l'avantage
    de s'abstraire de ce besoin, qu'elles pallient avec des représentations plus
    complexes des documents. Dans nos travaux de recherche, nous nous intéressons
    à la représentation sous forme de graphe du document. L'hypothèse selon
    laquelle un mot est important s'il apparait dans le contexte de mots
    importants est intéressante. Nous pensons toutefois qu'il serait plus
    pertinent d'ordonner les termes-clés candidats plutôt que les mots. Un autre
    point qui n'a jamais été abordé est la dispersion dans le graphe des
    informations relatives à un même concept, où sujet, représenté par plusieurs
    unités textuelles (plusieurs n\oe{}uds) dans le document.

    L'assignement de termes-clés est une catégorie d'indexation par termes-clés
    encore peut explorée. Celle-ci fournit des termes-clés d'un vocabulaire
    contrôlé. Ces termes-clés sont de meilleure qualité (mieux formés) que ceux
    qui sont extraits d'un document, mais le fait qu'ils n'apparaissent pas
    nécessairement dans le document rend la tâche plus difficile. De plus, se
    restreindre aux termes-clés d'un vocabulaire contrôlé ne permet pas de
    fournir des termes-clés représentant des concepts nouveaux qui ne sont pas
    encore dans le vocabulaire. Alors que l'assignement de termes-clés est
    considéré comme une indexation par termes-clés concurrente de l'extraction
    de termes-clés, il nous semble qu'elle devrait être utilisée en complément,
    lorsque les ressources disponibles le permettent.

    L'évaluation des méthodes d'indexation par termes-clés est une étape
    obligatoire pour démontrer leur apport vis-à-vis des précédentes méthodes.
    Parce qu'une évaluation manuelle est trop coûteuse, les chercheurs évaluent
    le plus souvent leurs méthodes automatiquement avec des corpus de test dont
    les documents sont associés à des termes-clés de référence annotés par des
    humains. Si un terme-clé est le même qu'un terme-clé de référence, alors
    il est jugé correcte, sinon il est jugé erroné. Ce jugement binaire permet
    de distinguer les meilleures méthodes, mais il ne permet pas de comparer
    deux méthodes sur la base de leurs erreurs. Prenons l'exemple extrême de
    deux méthodes n'ayant fournissent aucun terme-clé correct: l'une fournit
    des termes-clés qui n'ont aucun point commun avec les termes-clés de
    référence et l'autre fournit des termes-clés similaires, mais pas
    identiques, aux termes-clés de référence. Dans ce cas les deux méthodes sont
    considérées toutes aussi mauvaises. Une évaluation plus souple, prenant en
    compte la synonymie ou le recouvrement partiel des unités textuelles,
    pourrait indiquer que la seconde méthode est meilleure que la première.

