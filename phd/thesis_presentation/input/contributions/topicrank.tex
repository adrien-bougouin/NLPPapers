\begin{frame}{Contributions}\framesubtitle{TopicRank}
  Méthode à base de graphe d'extraction non supervisée de termes-clés

  \vspace{1em}

  \begin{alertblock}{Limites des méthodes à base de graphe}
    \begin{itemize}
      \item{Ordonnancement des mots}
      \item{Redondance des termes-clés extraits}
      \item{Contexte (fenêtre de mots) variable selon les méthodes}
    \end{itemize}
  \end{alertblock}

  \vspace{1em}

  \begin{block}{Propositions}
    \begin{itemize}
      \item{Ordonnancement des termes-clés candidats}
      \item{Groupement en sujets des termes-clés candidats}
      \item{Graphe complet $+$ pondération fine des arêtes}
    \end{itemize}
  \end{block}
\end{frame}

\begin{frame}{TopicRank}\framesubtitle{Aperçu}
  Méthode en 5 étapes~:
  \begin{enumerate}
    \item{Sélection des termes-clés candidats}
    \item{Groupement des candidats en sujets}
    \item{Création d'un graphe de sujets}
    \item{Ordonnancement des sujets}
    \item{Sélection du termes-clés représentatifs des sujets importants}
  \end{enumerate}
\end{frame}

\begin{frame}{TopicRank}\framesubtitle{Exemple}
  \alt<2->{
    \begin{exampleblock}{\small
      \textbf{Météo} du 19 \textbf{\underline{août\textvisiblespace 2012}}~:
      \textbf{\underline{alerte}} à la \textbf{\underline{canicule}} sur la
      \textbf{\underline{Belgique}} et le \textbf{\underline{Luxembourg}}
    }\justifying\small
      ~~~À l'\textbf{exception} de la \textbf{province} de
      \textbf{\underline{Luxembourg}}, en
      \textbf{\underline{alerte}\textvisiblespace jaune}, l'\textbf{ensemble} de
      la \textbf{\underline{Belgique}} est en
      \textbf{vigilance\textvisiblespace\underline{orange}} à la
      \textbf{\underline{canicule}}. Le \textbf{\underline{Luxembourg}} n'est
      pas épargné par la \textbf{vague} du \textbf{\underline{chaleur}}~: le
      \textbf{nord} du \textbf{pays} est en
      \textbf{\underline{alerte}\textvisiblespace\underline{orange}}, tandis que
      le \textbf{sud} a était placé en
      \textbf{\underline{alerte}\textvisiblespace rouge}.

      ~~~En \textbf{\underline{Belgique}}, la \textbf{\underline{température}}
      n'est pas descendue en dessous des \textbf{23\degre{}C} cette
      \textbf{nuit}, ce qui constitue la \textbf{deuxième\textvisiblespace nuit}
      \underline{la plus \textbf{chaude}} jamais enregistrée dans le
      \textbf{royaume}. Il se pourrait que ce \textbf{dimanche} soit la
      \textbf{journée} \underline{la plus \textbf{chaude}} de l'\textbf{année}.
      Les \textbf{\underline{températures}} seront comprises entre 33 et
      \textbf{38\degre{}C}. Une \textbf{légère\textvisiblespace brise} de
      \textbf{côte} pourra faiblement rafraichir l'\textbf{atmosphère}. Des
      \textbf{orages} de \textbf{\underline{chaleur}} sont a prévoir dans la
      \textbf{soirée} et en \textbf{début} de \textbf{nuit}.

      ~~~Au \textbf{\underline{Luxembourg}}, le \textbf{mercure} devrait
      atteindre \textbf{32\degre{}C} ce \textbf{dimanche} sur l'\textbf{Oesling}
      et jusqu'à \textbf{36\degre{}C} sur le \textbf{sud} du \textbf{pays}, et
      31 à 32\degre{}C \textbf{lundi}. Une \textbf{baisse} devrait intervenir
      pour le \textbf{reste} de la \textbf{semaine}. Néanmoins, le
      \textbf{record} d'\textbf{août 2003} (\textbf{37,9\degre{}C}) ne devrait
      pas être atteint.

      \begin{exampleblock}{\small Termes-clés}\justifying\small
        \underline{Août 2012}~; \underline{canicule}~;
        \underline{Belgique}~; \underline{Luxembourg}~; \underline{alerte}~;
        \underline{orange}~; \underline{chaleur}~; \underline{chaude}~;
        \underline{température}~; \underline{la plus chaude}
      \end{exampleblock}
    \end{exampleblock}
  }{
    \begin{exampleblock}{\small
      Météo du 19 \underline{août 2012}~: \underline{alerte} à la
      \underline{canicule} sur la \underline{Belgique} et le
      \underline{Luxembourg}
    }\justifying\small
      ~~~À l'exception de la province de \underline{Luxembourg}, en
      \underline{alerte} jaune, l'ensemble de la \underline{Belgique} est en
      vigilance \underline{orange} à la \underline{canicule}. Le
      \underline{Luxembourg} n'est pas épargné par la vague du \underline{chaleur}
      : le nord du pays est en \underline{alerte} \underline{orange}, tandis que
      le sud a était placé en \underline{alerte} rouge.

      ~~~En \underline{Belgique}, la \underline{température} n'est pas descendue
      en dessous des 23\degre{}C cette nuit, ce qui constitue la deuxième nuit
      \underline{la plus chaude} jamais enregistrée dans le royaume. Il se
      pourrait que ce dimanche soit la journée \underline{la plus chaude} de
      l'année. Les \underline{températures} seront comprises entre 33 et
      38\degre{}C. Une légère brise de côte pourra faiblement rafraichir
      l'atmosphère. Des orages de \underline{chaleur} sont a prévoir dans la
      soirée et en début de nuit.

      ~~~Au \underline{Luxembourg}, le mercure devrait atteindre 32\degre{}C ce
      dimanche sur l'Oesling et jusqu'à 36\degre{}C sur le sud du pays, et 31 à
      32\degre{}C lundi. Une baisse devrait intervenir pour le reste de la
      semaine. Néanmoins, le record d'août 2003 (37,9\degre{}C) ne devrait pas
      être atteint.

      \begin{exampleblock}{\small Termes-clés}\justifying\small
        \underline{Août 2012}~; \underline{canicule}~;
        \underline{Belgique}~; \underline{Luxembourg}~; \underline{alerte}~;
        \underline{orange}~; \underline{chaleur}~; \underline{chaude}~;
        \underline{température}~; \underline{la plus chaude}
      \end{exampleblock}
    \end{exampleblock}
  }
\end{frame}

\begin{frame}{TopicRank}\framesubtitle{Exemple}
  \begin{columns}
    \begin{column}{.45\linewidth}
      \begin{enumerate}\setlength{\itemindent}{-.75cm}
        \item<1->{Sélection des candidats}
          \begin{itemize}\setlength{\itemindent}{-1.1cm}
          \item{\texttt{/(NOM | ADJ)+/}}
        \end{itemize}
        \item<2->{Groupement des candidats}
        \begin{itemize}\setlength{\itemindent}{-1.1cm}
          \item{\textcolor{white}{$\frac{\text{racines}(c_1) \cap \text{racines}(c_2)}{\text{racines}(c_1) \cup \text{racines}(c_2)}$}}
          \begin{textblock*}{\textwidth}(.055\textwidth, -.055\textheight)
            $\text{sim}(c_1, c_2) = \frac{\text{racines}(c_1) \cap \text{racines}(c_2)}{\text{racines}(c_1) \cup \text{racines}(c_2)}$
          \end{textblock*}
        \end{itemize}
        \item<3->{Construction du graphe}
        \begin{itemize}\setlength{\itemindent}{-1.1cm}
          \item{
            Pondération selon la distance\\
            \hspace{-1.1cm}entre les sujets dans le document\\
            \hspace{-1.1cm}(en nombre de mots)}
        \end{itemize}
        \item<4->{Ordonnancement des sujets}
        \begin{itemize}\setlength{\itemindent}{-1.1cm}
          \item{TextRank}
        \end{itemize}
        \item<5->{Sélection des termes-clés}
        \begin{itemize}\setlength{\itemindent}{-1.1cm}
          \item{Un par sujet}
          \item{Le plus tôt dans le document}
        \end{itemize}
      \end{enumerate}
    \end{column}

    \begin{column}{.55\linewidth}
      \centering
      \alt<3->{
        \begin{overpic}[width=.95\linewidth]{include/44960_topicrank_graph.eps}
          \put (36, 101) {\scriptsize [soirée]}
          \put (53, 100) {\scriptsize [nord]}
          \put (20, 96) {\scriptsize [\OE{}sling]}
          \put (65, 95) {\scriptsize \alt<4->{\textcolor{termithorange}{\textbf{[Belgique]}}}{[Belgique]}}
          \put (38.5, 84) {\scriptsize [ensemble]}
          \put (-8, 86) {\scriptsize \alt<4->{\textcolor{termithorange}{\textbf{[août 2003~; août 2012]}}}{[août 2003~; août 2012]}}
          \put (58, 83) {\scriptsize [record]}
          \put (80, 83) {\scriptsize [36\degre{}C]}
          \put (22, 80) {\scriptsize [légère brise]}
          \put (-1, 74) {\scriptsize [37\degre{}C]}
          \put (13, 69) {\scriptsize [météo]}
          \put (29, 64) {\scriptsize \alt<4->{\textcolor{termithorange}{\textbf{[chaleur]}}}{[chaleur]}}
          \put (48, 69) {\scriptsize [reste]}
          \put (68, 72) {\scriptsize [orages]}
          \put (85, 71) {\scriptsize [royaume]}
          \put (-7, 57) {\scriptsize [année]}
          \put (10, 54) {\scriptsize \alt<4->{\textcolor{termithorange}{\textbf{[chaude]}}}{[chaude]}}
          \put (24, 47) {\scriptsize [province]}
          \put (40.5, 52) {\scriptsize [semaine]}
          \put (60, 56.5) {\scriptsize [vague]}
          \put (77, 58) {\scriptsize [baisse]}
          \put (90, 54) {\scriptsize \alt<4->{\textcolor{termithorange}{\textbf{[dimanche]}}}{[dimanche]}}
          \put (-15, 42) {\scriptsize [vigilance orange]}
          \put (-9, 37) {\scriptsize \alt<4->{\textcolor{termithorange}{\textbf{[température~; températures]}}}{[température~; températures]}}
          \put (40, 34) {\scriptsize [pays]}
          \put (54, 39) {\scriptsize [mercure]}
          \put (68, 43) {\scriptsize [atmosphère]}
          \put (83, 38) {\scriptsize \alt<4->{\textcolor{termithorange}{\textbf{[Luxembourg]}}}{[Luxembourg]}}
          \put (0, 25) {\scriptsize [début]}
          \put (24, 26) {\scriptsize [23\degre{}C]}
          \put (39, 18) {\scriptsize [lundi]}
          \put (53, 21) {\scriptsize [38\degre{}C]}
          \put (70, 27) {\scriptsize [sud]}
          \put (70, 23) {\scriptsize \alt<4->{\textcolor{termithorange}{\textbf{[nuit~; deuxième nuit]}}}{[nuit~; deuxième nuit]}}
          \put (9, 14) {\scriptsize [exception]}
          \put (24, 5) {\scriptsize [journée]}
          \put (59, 4.5) {\scriptsize [côte]}
          \put (70, 11) {\scriptsize \alt<4->{\textcolor{termithorange}{\textbf{[canicule]}}}{[canicule]}}
          \put (2.5, .5) {\scriptsize \alt<4->{\textcolor{termithorange}{\textbf{[alerte rouge~; alerte jaune~; alerte orange~; alerte]}}}{[alerte rouge~; alerte jaune~; alerte orange~; alerte]}}
        \end{overpic}
      }{
        \alt<2->{
          \begin{overpic}[width=.95\linewidth]{include/44960_topicrank_graph_blank.eps}
            \put (34, 98) {\scriptsize [soirée]}
            \put (51, 97) {\scriptsize [nord]}
            \put (18, 93) {\scriptsize [\OE{}sling]}
            \put (63, 92) {\scriptsize [Belgique]}
            \put (-10, 83) {\scriptsize [août 2003~; août 2012]}
            \put (36.5, 81) {\scriptsize [ensemble]}
            \put (56, 80) {\scriptsize [record]}
            \put (78, 80) {\scriptsize [36\degre{}C]}
            \put (20, 77) {\scriptsize [légère brise]}
            \put (-3, 71) {\scriptsize [37\degre{}C]}
            \put (11, 66) {\scriptsize [météo]}
            \put (27, 61) {\scriptsize [chaleur]}
            \put (46, 66) {\scriptsize [reste]}
            \put (66, 69) {\scriptsize [orages]}
            \put (83, 68) {\scriptsize [royaume]}
            \put (-9, 54) {\scriptsize [année]}
            \put (8, 51) {\scriptsize [chaude]}
            \put (22, 44) {\scriptsize [province]}
            \put (38.5, 49) {\scriptsize [semaine]}
            \put (58, 53.5) {\scriptsize [vague]}
            \put (75, 55) {\scriptsize [baisse]}
            \put (88, 51) {\scriptsize [dimanche]}
            \put (-17, 39) {\scriptsize [vigilance orange]}
            \put (-11, 34) {\scriptsize [température~; températures]}
            \put (38, 31) {\scriptsize [pays]}
            \put (52, 36) {\scriptsize [mercure]}
            \put (66, 40) {\scriptsize [atmosphère]}
            \put (81, 35) {\scriptsize [Luxembourg]}
            \put (-2, 22) {\scriptsize [début]}
            \put (22, 23) {\scriptsize [23\degre{}C]}
            \put (37, 15) {\scriptsize [lundi]}
            \put (51, 18) {\scriptsize [38\degre{}C]}
            \put (68, 24) {\scriptsize [sud]}
            \put (68, 20) {\scriptsize [nuit~; deuxième nuit]}
            \put (7, 11) {\scriptsize [exception]}
            \put (22, 2) {\scriptsize [journée]}
            \put (57, 1.5) {\scriptsize [côte]}
            \put (68, 8) {\scriptsize [canicule]}
            \put (0.5, -2.5) {\scriptsize [alerte rouge~; alerte jaune~; alerte orange~; alerte]}
          \end{overpic}
        }{
          \begin{overpic}[width=.95\linewidth]{include/44960_topicrank_graph_blank.eps}
            \put (34, 98) {\scriptsize \textcolor{white}{[}soirée\textcolor{white}{]}}
            \put (51, 97) {\scriptsize \textcolor{white}{[}nord\textcolor{white}{]}}
            \put (18, 93) {\scriptsize \textcolor{white}{[}\OE{}sling\textcolor{white}{]}}
            \put (63, 92) {\scriptsize \textcolor{white}{[}Belgique\textcolor{white}{]}}
            \put (-10, 83) {\scriptsize \textcolor{white}{[}août 2003~\textcolor{white}{;} août 2012\textcolor{white}{]}}
            \put (36.5, 81) {\scriptsize \textcolor{white}{[}ensemble\textcolor{white}{]}}
            \put (56, 80) {\scriptsize \textcolor{white}{[}record\textcolor{white}{]}}
            \put (78, 80) {\scriptsize \textcolor{white}{[}36\degre{}C\textcolor{white}{]}}
            \put (20, 77) {\scriptsize \textcolor{white}{[}légère brise\textcolor{white}{]}}
            \put (-3, 71) {\scriptsize \textcolor{white}{[}37\degre{}C\textcolor{white}{]}}
            \put (11, 66) {\scriptsize \textcolor{white}{[}météo\textcolor{white}{]}}
            \put (27, 61) {\scriptsize \textcolor{white}{[}chaleur\textcolor{white}{]}}
            \put (46, 66) {\scriptsize \textcolor{white}{[}reste\textcolor{white}{]}}
            \put (66, 69) {\scriptsize \textcolor{white}{[}orages\textcolor{white}{]}}
            \put (83, 68) {\scriptsize \textcolor{white}{[}royaume\textcolor{white}{]}}
            \put (-9, 54) {\scriptsize \textcolor{white}{[}année\textcolor{white}{]}}
            \put (8, 51) {\scriptsize \textcolor{white}{[}chaude\textcolor{white}{]}}
            \put (22, 44) {\scriptsize \textcolor{white}{[}province\textcolor{white}{]}}
            \put (38.5, 49) {\scriptsize \textcolor{white}{[}semaine\textcolor{white}{]}}
            \put (58, 53.5) {\scriptsize \textcolor{white}{[}vague\textcolor{white}{]}}
            \put (75, 55) {\scriptsize \textcolor{white}{[}baisse\textcolor{white}{]}}
            \put (88, 51) {\scriptsize \textcolor{white}{[}dimanche\textcolor{white}{]}}
            \put (-17, 39) {\scriptsize \textcolor{white}{[}vigilance orange\textcolor{white}{]}}
            \put (-11, 34) {\scriptsize \textcolor{white}{[}température~\textcolor{white}{;} températures\textcolor{white}{]}}
            \put (38, 31) {\scriptsize \textcolor{white}{[}pays\textcolor{white}{]}}
            \put (52, 36) {\scriptsize \textcolor{white}{[}mercure\textcolor{white}{]}}
            \put (66, 40) {\scriptsize \textcolor{white}{[}atmosphère\textcolor{white}{]}}
            \put (81, 35) {\scriptsize \textcolor{white}{[}Luxembourg\textcolor{white}{]}}
            \put (-2, 22) {\scriptsize \textcolor{white}{[}début\textcolor{white}{]}}
            \put (22, 23) {\scriptsize \textcolor{white}{[}23\degre{}C\textcolor{white}{]}}
            \put (37, 15) {\scriptsize \textcolor{white}{[}lundi\textcolor{white}{]}}
            \put (51, 18) {\scriptsize \textcolor{white}{[}38\degre{}C\textcolor{white}{]}}
            \put (68, 24) {\scriptsize \textcolor{white}{[}sud\textcolor{white}{]}}
            \put (68, 20) {\scriptsize \textcolor{white}{[}nuit~\textcolor{white}{;} deuxième nuit\textcolor{white}{]}}
            \put (7, 11) {\scriptsize \textcolor{white}{[}exception\textcolor{white}{]}}
            \put (22, 2) {\scriptsize \textcolor{white}{[}journée\textcolor{white}{]}}
            \put (57, 1.5) {\scriptsize \textcolor{white}{[}côte\textcolor{white}{]}}
            \put (68, 8) {\scriptsize \textcolor{white}{[}canicule\textcolor{white}{]}}
            \put (0.5, -2.5) {\scriptsize \textcolor{white}{[}alerte rouge~\textcolor{white}{;} alerte jaune~\textcolor{white}{;} alerte orange~\textcolor{white}{;} alerte\textcolor{white}{]}}
          \end{overpic}
        }
      }
    \end{column}
  \end{columns}

  \visible<6->{
    \begin{textblock*}{\textwidth}(0\textwidth, -.75\textheight)
      \begin{exampleblock}{\small
        \textbf{Météo} du 19 \textbf{\underline{août\textvisiblespace 2012}}~:
        \textbf{\underline{alerte}} à la \textbf{\underline{canicule}} sur la
        \textbf{\underline{Belgique}} et le \textbf{\underline{Luxembourg}}
      }\justifying\small
        ~~~À l'\textbf{exception} de la \textbf{province} de
        \textbf{\underline{Luxembourg}}, en
        \textbf{\underline{alerte}\textvisiblespace jaune}, l'\textbf{ensemble} de
        la \textbf{\underline{Belgique}} est en
        \textbf{vigilance\textvisiblespace\underline{orange}} à la
        \textbf{\underline{canicule}}. Le \textbf{\underline{Luxembourg}} n'est
        pas épargné par la \textbf{vague} du \textbf{\underline{chaleur}}~: le
        \textbf{nord} du \textbf{pays} est en
        \textbf{\underline{alerte}\textvisiblespace\underline{orange}}, tandis que
        le \textbf{sud} a était placé en
        \textbf{\underline{alerte}\textvisiblespace rouge}.

        ~~~En \textbf{\underline{Belgique}}, la \textbf{\underline{température}}
        n'est pas descendue en dessous des \textbf{23\degre{}C} cette
        \textbf{nuit}, ce qui constitue la \textbf{deuxième\textvisiblespace nuit}
        \underline{la plus \textbf{chaude}} jamais enregistrée dans le
        \textbf{royaume}. Il se pourrait que ce \textbf{dimanche} soit la
        \textbf{journée} \underline{la plus \textbf{chaude}} de l'\textbf{année}.
        Les \textbf{\underline{températures}} seront comprises entre 33 et
        \textbf{38\degre{}C}. Une \textbf{légère\textvisiblespace brise} de
        \textbf{côte} pourra faiblement rafraichir l'\textbf{atmosphère}. Des
        \textbf{orages} de \textbf{\underline{chaleur}} sont a prévoir dans la
        \textbf{soirée} et en \textbf{début} de \textbf{nuit}.

        ~~~Au \textbf{\underline{Luxembourg}}, le \textbf{mercure} devrait
        atteindre \textbf{32\degre{}C} ce \textbf{dimanche} sur l'\textbf{Oesling}
        et jusqu'à \textbf{36\degre{}C} sur le \textbf{sud} du \textbf{pays}, et
        31 à 32\degre{}C \textbf{lundi}. Une \textbf{baisse} devrait intervenir
        pour le \textbf{reste} de la \textbf{semaine}. Néanmoins, le
        \textbf{record} d'\textbf{août 2003} (\textbf{37,9\degre{}C}) ne devrait
        pas être atteint.

        \begin{exampleblock}{\small Termes-clés}\justifying\small
          \underline{Août 2012}~; \underline{canicule}~;
          \underline{Belgique}~; \underline{Luxembourg}~; \underline{alerte}~;
          \underline{orange}~; \underline{chaleur}~; \underline{chaude}~;
          \underline{température}~; \underline{la plus chaude}
        \end{exampleblock}
      \end{exampleblock}
    \end{textblock*}
  }
\end{frame}

\begin{frame}{TopicRank}\framesubtitle{Exemple}
  \begin{table}
    \centering
    \begin{tabular}{r|l|l|l}
      \toprule
      \textbf{Rang} & \multicolumn{1}{c|}{\textbf{TextRank}} &
      \multicolumn{1}{c|}{\textbf{SingleRank}} & \multicolumn{1}{c}{\textbf{TopicRank}} \\
      \hline
      01 & \cellcolor{termithorange!30}{août 2012} & alerte orange & \cellcolor{termithorange!30}{Luxembourg}\\
      02 & août 2003 & alerte jaune & \cellcolor{termithorange!30}{alerte} \\
      03 & alerte orange & alerte rouge & nuit \\
      04 & vigilance orange & \cellcolor{termithorange!30}{alerte} & \cellcolor{termithorange!30}{Belgique} \\
      05 & deuxième nuit & deuxième nuit & \cellcolor{termithorange!30}{août 2012}\\
      06 & légère brise & \cellcolor{termithorange!30}{août 2012} & \cellcolor{termithorange!30}{chaleur} \\
      07 & & août 2003 & \cellcolor{termithorange!30}{température} \\
      08 & & vigilance orange & \cellcolor{termithorange!30}{chaude} \\
      09 & & légère brise & \cellcolor{termithorange!30}{canicule} \\
      10 & & \cellcolor{termithorange!30}{Luxembourg} & dimanche \\
      \bottomrule
    \end{tabular}

    \caption{Termes-clés}
  \end{table}
\end{frame}

\begin{frame}{TopicRank}\framesubtitle{Évaluation}
  Trois méthodes de référence~:
  \begin{itemize}
    \item{TF-IDF~\cite{salton1975tfidf}}
    \item{TextRank~\cite{mihalcea2004textrank}}
    \item{SingleRank~\cite{wan2008expandrank}}
  \end{itemize}

  \vspace{1em}

  Évaluation à 10 termes-clés, en termes de~:
  \begin{itemize}
    \item{Précision (P)~: $\frac{|\text{corrects}|}{|\text{extraits}|}$}
    \item{Rappel (R)~: $\frac{|\text{corrects}|}{|\text{references}|}$}
    \item{F1-mesure (F)~: $2 \times \frac{\text{précision} \times \text{rappel}}{\text{précision} + \text{rappel}}$}
  \end{itemize}
\end{frame}

\begin{frame}{TopicRank}\framesubtitle{Résultats dans le contexte général}
  \begin{table}
    \resizebox{\linewidth}{!}{
      \begin{tabular}{@{~}l|c@{~~}c@{~~}c@{~}|c@{~~}c@{~~}c@{~}|c@{~~}c@{~~}c@{~}|c@{~~}c@{~~}c@{~}}
        \toprule
        \multirow{2}{*}[-2pt]{\textbf{Méthode}} & \multicolumn{3}{c|}{\textbf{DEft} \textit{(fr)}} & \multicolumn{3}{c|}{\textbf{Wikinews} \textit{(fr)}} & \multicolumn{3}{c|}{\textbf{SemEval} \textit{(en)}} & \multicolumn{3}{c}{\textbf{DUC} \textit{(en)}}\\
        \cline{2-4}\cline{5-7}\cline{8-10}\cline{11-13}
        & P & R & F & P & R & F & P & R & F & P & R & F\\
        \hline
        \textsc{Tf-Idf} & 10,3 & 19,1 & 13,2$^{~}$ & 33,9 & 35,9 & 34,3$^{~}$ & 13,2 & $~~$8,9 & 10,5$^{~}$ & \textbf{23,8} & \textbf{30,7} & \textbf{26,4}\\
        TextRank & $~~$4,9 & $~~$7,1 & $~~$5,7$^{~}$ & $~~$9,3 & $~~$8,3 & $~~$8,6$^{~}$ & $~~$7,9 & $~~$4,5 & $~~$5,6$^{~}$ & $~~$4,9 & $~~$5,4 & $~~$5,0\\
        SingleRank & $~~$4,5 & $~~$9,0 & $~~$5,9$^{~}$ & 19,4 & 20,7 & 19,7$^{~}$ & $~~$4,6 & $~~$3,2 & $~~$3,7$^{~}$ & 22,3 & 28,4 & 24,6\\
        \hline
        TopicRank & \textbf{11,7} & \textbf{21,7} & \textbf{15,1}$^\dagger$ & \textbf{35,0} & \textbf{37,5} & \textbf{35,6}$^\dagger$ & \textbf{14,9}$^{~}$ & \textbf{10,3} & \textbf{12,1}$^\dagger$ & 18,3 & 23,8 & 20,4\\
        \hline
        \uncover<2->{\textbf{Borne haute}} & \uncover<2->{\textbf{14,5}} & \uncover<2->{\textbf{27,0}} & \uncover<2->{\textbf{18,7}$^{~}$} & \uncover<2->{\textbf{41,8}} & \uncover<2->{\textbf{44,1}} & \uncover<2->{\textbf{42,2}$^{~}$} & \uncover<2->{\textbf{30,0}} & \uncover<2->{\textbf{20,7}} & \uncover<2->{\textbf{24,3}$^{~}$} & \uncover<2->{\textbf{30,5}} & \uncover<2->{\textbf{38,7}} & \uncover<2->{\textbf{33,7}}\\
        
        \bottomrule
      \end{tabular}
    }
  \end{table}
  \begin{block}{Observations}
    \begin{itemize}
      \item{Meilleure performance globale}
      \item{Significativement meilleur que TextRank et SingleRank ($^\dagger$)}
      \item{Plus faible performance sur DUC}
      \item<2->{Bonne perspective d'améliorarion (vers la borne haute)}
    \end{itemize}
  \end{block}

  \uncover<2->{
    \begin{exampleblock}{Exemple d'amélioration avec la borne haute}
      \begin{itemize}
        \item{Obama $\Rightarrow$ Barack Obama $\in$ [Obama~; Barack Obama]}
        \item{Romney $\Rightarrow$ Mitt Romney $\in$ [Romney~; Mitt Romney]}
      \end{itemize}
    \end{exampleblock}
  }
\end{frame}

%\begin{frame}{TopicRank}\framesubtitle{Résultats dans le contexte général}
%  \begin{table}
%    \resizebox{\linewidth}{!}{
%      \begin{tabular}{@{~}l|c@{~~}c@{~~}c@{~}|c@{~~}c@{~~}c@{~}|c@{~~}c@{~~}c@{~}|c@{~~}c@{~~}c@{~}}
%        \toprule
%        \multirow{2}{*}[-2pt]{\textbf{Méthode}} & \multicolumn{3}{c|}{\textbf{\textsc{De}ft} \textit{(fr)}} & \multicolumn{3}{c|}{\textbf{Wikinews} \textit{(fr)}} & \multicolumn{3}{c|}{\textbf{SemEval} \textit{(en)}} & \multicolumn{3}{c}{\textbf{\textsc{Duc}} \textit{(en)}}\\
%        \cline{2-4}\cline{5-7}\cline{8-10}\cline{11-13}
%        & P & R & F & P & R & F & P & R & F & P & R & F\\
%        \hline
%        SingleRank & $~~$4,5 & $~~$9,0 & $~~$5,9$^{~}$ & 19,4 & 20,7 & 19,7$^{~}$ & $~~$4,6 & $~~$3,2 & $~~$3,7$^{~}$ & \textbf{22,3} & \textbf{28,4} & \textbf{24,6}\\
%        + complet & $~~$4,4 & $~~$9,0 & $~~$5,8$^{~}$ & 20,0 & 21,4 & 20,3${~}$ & $~~$5,5 & $~~$3,8 & $~~$4,4$^{~}$ & 22,2 & 28,1 & 24,5\\
%        + candidats & 10,3 & 19,2 & 13,2$^\dagger$ & 28,5 & 30,0 & 28,8$^\dagger$ & $~~$9,4 & $~~$6,8 & $~~$7,8$^\dagger$ & 10,4 & 13,5 & 11,6\\
%        + sujets & 11,1 & 20,4 & 14,2$^\dagger$ & 30,7 & 32,6 & 31,1$^\dagger$ & 14,2 & $~~$9,9 & 11,6$^\dagger$ & 18,9 & 24,2 & 21,0\\
%        TopicRank & \textbf{11,7} & \textbf{21,7} & \textbf{15,1}$^\dagger$ & \textbf{35,0} & \textbf{37,5} & \textbf{35,6}$^\dagger$ & \textbf{14,9}$^{~}$ & \textbf{10,3} & \textbf{12,1}$^\dagger$ & 18,3 & 23,8 & 20,4\\
%        \bottomrule
%      \end{tabular}
%    }
%  \end{table}
%\end{frame}

\begin{frame}{TopicRank}\framesubtitle{Bilan}
  Méthode à base de graphe qui s'intéresse à ce que représentent les
  termes-clés candidats plutôt qu'à eux-mêmes ou à leurs mots

  \vspace{1em}

  \begin{block}{Avantages}
    \begin{itemize}
      \item{Générique}
      \item{Non redondant}
    \end{itemize}
  \end{block}

  \vspace{1em}

  \begin{alertblock}{Limites}
    \begin{itemize}
      \item{Groupement naïf des candidats en sujets}
      \item{Stratégie non optimale pour sélectionner le terme-clé d'un sujet}
      \item{Quelques instabilités}
    \end{itemize}
  \end{alertblock}
\end{frame}

\begin{frame}{TopicRank}\framesubtitle{Oui, mais\dots}
  \begin{table}
    \resizebox{\linewidth}{!}{
      \begin{tabular}{@{~}l|c@{~~}c@{~~}c@{~}|c@{~~~~~~~}c@{~~~~~~}c@{~}|c@{~~}c@{~~}c@{~}|c@{~~}c@{~~}c@{~}}
        \toprule
        \multirow{2}{*}[-2pt]{\textbf{Méthode}} & \multicolumn{3}{c|}{\textbf{Linguistique} \textit{(fr)}} & \multicolumn{3}{c|}{\textbf{Sciences de l'info.} \textit{(fr)}} & \multicolumn{3}{c|}{\textbf{Archéologie} \textit{(fr)}} & \multicolumn{3}{c}{\textbf{Chimie} \textit{(fr)}}\\
        \cline{2-4}\cline{5-7}\cline{8-10}\cline{11-13}
        & P & R & F & P & R & F & P & R & F & P & R & F\\
        \hline
        \textsc{Tf-Idf} & \textbf{13,0} & \textbf{15,4} & \textbf{13,9} & \textbf{13,4} & \textbf{14,0} & \textbf{13,2} & \textbf{28,1} & \textbf{19,1} & \textbf{22,2} & \textbf{14,1} & \textbf{11,1} & \textbf{11,9}\\
        TextRank & $~~$7,1 & $~~$6,1 & $~~$6,4 & $~~$5,8 & $~~$4,3 & $~~$4,8 & $~~$10,2 & $~~$5,3 & $~~$6,8 & $~~$9,4 & $~~$5,3 & $~~$6,5\\
        SingleRank & $~~$9.0 & 10,6 & $~~$9,6 & $~~$9,5 & 10,0 & $~~$9,4 & 12,7 & $~~$8,9 & 10,2 & 13,0 & 10,4 & 11,0\\
        TopicRank & 11,2 & 13,1 & 11,9 & 12,1 & 12,8 & 12,1 & 27,5 & 18,7 & 21,8 & 13,8 & 11,1 & 11,8\\
        \hline
        \textbf{Borne haute} & \textbf{14,5} & \textbf{17,0} & \textbf{15,4} & \textbf{15,0} & \textbf{15,6} & \textbf{14,9} & \textbf{32,5} & \textbf{22,2} & \textbf{25,8} & \textbf{15,8} & \textbf{12,5} & \textbf{13,3}\\
        \bottomrule
      \end{tabular}
    }
  \end{table}

  \vspace{1em}

  \begin{block}{Observations}
    \begin{itemize}
      \item{Meilleure performance face à TextRank et SingleRank}
      \item{Meilleure performance globale pour TF-IDF}
      \item{Plus faible perspective d'amélioration}
    \end{itemize}
  \end{block}
\end{frame}

