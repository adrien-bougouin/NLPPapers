\begin{frame}{Contributions}\framesubtitle{TopicRank}
  Méthode à base de graphe d'extraction non supervisée de termes-clés.

  \vspace{1em}

  \begin{alertblock}{Limites des méthodes à base de graphe}
    \begin{itemize}
      \item{Ordonnancement des mots}
      \item{Redondance des termes-clés extraits}
      \item{Fenêtre de cooccurrence}
    \end{itemize}
  \end{alertblock}

  \vspace{1em}

  \begin{block}{Proposition}
    \begin{itemize}
      \item{Ordonnancement des termes-clés candidats}
      \item{Groupement en sujet des termes-clés candidats}
      \item{Graphe complet $+$ pondération sémantique des arêtes}
    \end{itemize}
  \end{block}
\end{frame}

\begin{frame}{TopicRank}\framesubtitle{Exemple}
  \begin{exampleblock}{\small
    Météo du 19 août 2012~: alerte à la canicule sur la Belgique et le
    Luxembourg
  }\justifying\small
    ~~~À l'exception de la province de Luxembourg, en alerte jaune, l'ensemble
    de la Belgique est en vigilance orange à la canicule. Le Luxembourg n'est
    pas épargné par la vague du chaleur : le nord du pays est en alerte
    orange, tandis que le sud a était placé en alerte rouge.

    ~~~En Belgique, la température n'est pas descendue en dessous des
    23\degre{}C cette nuit, ce qui constitue la deuxième nuit la plus chaude
    jamais enregistrée dans le royaume. Il se pourrait que ce dimanche soit la
    journée la plus chaude de l'année. Les températures seront comprises entre
    33 et 38\degre{}C. Une légère brise de côte pourra faiblement rafraichir
    l'atmosphère. Des orages de chaleur sont a prévoir dans la soirée et en
    début de nuit.

    ~~~Au Luxembourg, le mercure devrait atteindre 32\degre{}C ce dimanche sur
    l'Oesling et jusqu'à 36\degre{}C sur le sud du pays, et 31 à 32\degre{}C
    lundi. Une baisse devrait intervenir pour le reste de la semaine.
    Néanmoins, le record d'août 2003 (37,9\degre{}C) ne devrait pas être
    atteint.

    \begin{exampleblock}{\small Termes-clés}\justifying\small
      \underline{Août 2012}~; \underline{canicule}~;
      \underline{Belgique}~; \underline{Luxembourg}~; \underline{alerte}~;
      \underline{orange}~; \underline{chaleur}~; \underline{chaude}~;
      \underline{température}~; \underline{la plus chaude}.
    \end{exampleblock}
  \end{exampleblock}
\end{frame}

\begin{frame}{TopicRank}\framesubtitle{Exemple}
  \begin{columns}
    \begin{column}{.45\linewidth}
      \begin{enumerate}
        \item{Sélection des candidats}
        \item{
          Groupement des candidats\\
          $\Rightarrow$ sujets
        }
        \item{Construction du graphe}
        \item{Ordonnancement des sujets}
        \item{Sélection des termes-clés}
      \end{enumerate}
    \end{column}

    \begin{column}{.55\linewidth}
      \centering
      \begin{overpic}[width=.95\linewidth]{include/44960_topicrank_graph.eps}
        \put (36, 101) {\scriptsize [soirée]}
        \put (53, 100) {\scriptsize [nord]}
        \put (20, 96) {\scriptsize [\OE{}sling]}
        \put (65, 95) {\scriptsize [Belgique]}
        \put (-8, 86) {\scriptsize [août 2003~; août 2012]}
        \put (38.5, 84) {\scriptsize [ensemble]}
        \put (58, 83) {\scriptsize [record]}
        \put (80, 83) {\scriptsize [36\degre{}C]}
        \put (22, 80) {\scriptsize [légère brise]}
        \put (-1, 74) {\scriptsize [37\degre{}C]}
        \put (13, 69) {\scriptsize [météo]}
        \put (29, 64) {\scriptsize [chaleur]}
        \put (48, 69) {\scriptsize [reste]}
        \put (68, 72) {\scriptsize [orages]}
        \put (85, 71) {\scriptsize [royaume]}
        \put (-7, 57) {\scriptsize [année]}
        \put (10, 54) {\scriptsize [chaude]}
        \put (24, 47) {\scriptsize [province]}
        \put (40.5, 52) {\scriptsize [semaine]}
        \put (60, 56.5) {\scriptsize [vague]}
        \put (77, 58) {\scriptsize [baisse]}
        \put (90, 54) {\scriptsize [dimanche]}
        \put (-16, 42) {\scriptsize [vigilance orange]}
        \put (-9, 37) {\scriptsize [température~; températures]}
        \put (40, 34) {\scriptsize [pays]}
        \put (54, 39) {\scriptsize [mercure]}
        \put (68, 43) {\scriptsize [atmosphère]}
        \put (83, 38) {\scriptsize [Luxembourg]}
        \put (0, 25) {\scriptsize [début]}
        \put (24, 26) {\scriptsize [23\degre{}C]}
        \put (39, 18) {\scriptsize [lundi]}
        \put (53, 21) {\scriptsize [38\degre{}C]}
        \put (70, 27) {\scriptsize [sud]}
        \put (70, 23) {\scriptsize [nuit~; deuxième nuit]}
        \put (9, 14) {\scriptsize [exception]}
        \put (24, 5) {\scriptsize [journée]}
        \put (59, 4.5) {\scriptsize [côte]}
        \put (70, 11) {\scriptsize [canicule]}
        \put (2.5, .5) {\scriptsize [alerte rouge~; alerte jaune~; alerte orange~; alerte]}
      \end{overpic}
    \end{column}
  \end{columns}
\end{frame}

\begin{frame}{TopicRank}\framesubtitle{Exemple}
  \begin{table}
    \centering
    \begin{tabular}{r|l|l|l}
      \toprule
      \textbf{Rang} & \multicolumn{1}{c|}{\textbf{TextRank}} &
      \multicolumn{1}{c|}{\textbf{SingleRank}} & \multicolumn{1}{c}{\textbf{TopicRank}} \\
      \hline
      01 & \cellcolor{termithorange!30}{août 2012} & alerte orange & \cellcolor{termithorange!30}{Luxembourg}\\
      02 & août 2003 & alerte jaune & \cellcolor{termithorange!30}{alerte} \\
      03 & alerte orange & alerte rouge & nuit \\
      04 & vigilance orange & \cellcolor{termithorange!30}{alerte} & \cellcolor{termithorange!30}{Belgique} \\
      05 & deuxième nuit & deuxième nuit & \cellcolor{termithorange!30}{août 2012}\\
      06 & légère brise & \cellcolor{termithorange!30}{août 2012} & \cellcolor{termithorange!30}{chaleur} \\
      07 & & août 2003 & \cellcolor{termithorange!30}{température} \\
      08 & & vigilance orange & \cellcolor{termithorange!30}{chaude} \\
      09 & & légère brise & \cellcolor{termithorange!30}{canicule} \\
      10 & & \cellcolor{termithorange!30}{Luxembourg} & dimanche \\
      \bottomrule
    \end{tabular}

    \caption{Termes-clés}
  \end{table}
\end{frame}

\begin{frame}{TopicRank}\framesubtitle{Évaluation}
  Trois méthodes de référence~:
  \begin{itemize}
    \item{TF-IDF~\cite{salton1975tfidf}}
    \item{TextRank~\cite{mihalcea2004textrank}}
    \item{SingleRank~\cite{wan2008expandrank}}
  \end{itemize}

  \vspace{1em}

  Évaluation à 10 termes-clés, en termes de~:
  \begin{itemize}
    \item{Précision~: $\frac{|\text{correctes}|}{|\text{extraits}|}$}
    \item{Rappel~: $\frac{|\text{correctes}|}{|\text{references}|}$}
    \item{F1-mesure~: $2 \times \frac{\text{précision} \times \text{rappel}}{\text{précision} + \text{rappel}}$}
  \end{itemize}
\end{frame}

\begin{frame}{TopicRank}\framesubtitle{Résultats dans le contexte général}
  \begin{table}
    \resizebox{\linewidth}{!}{
      \begin{tabular}{@{~}l|c@{~~}c@{~~}c@{~}|c@{~~}c@{~~}c@{~}|c@{~~}c@{~~}c@{~}|c@{~~}c@{~~}c@{~}}
        \toprule
        \multirow{2}{*}[-2pt]{\textbf{Méthode}} & \multicolumn{3}{c|}{\textbf{DEft} \textit{(fr)}} & \multicolumn{3}{c|}{\textbf{Wikinews} \textit{(fr)}} & \multicolumn{3}{c|}{\textbf{SemEval} \textit{(en)}} & \multicolumn{3}{c}{\textbf{DUC} \textit{(en)}}\\
        \cline{2-4}\cline{5-7}\cline{8-10}\cline{11-13}
        & P & R & F & P & R & F & P & R & F & P & R & F\\
        \hline
        \textsc{Tf-Idf} & 10,3 & 19,1 & 13,2$^{~}$ & 33,9 & 35,9 & 34,3$^{~}$ & 13,2 & $~~$8,9 & 10,5$^{~}$ & \textbf{23,8} & \textbf{30,7} & \textbf{26,4}\\
        TextRank & $~~$4,9 & $~~$7,1 & $~~$5,7$^{~}$ & $~~$9,3 & $~~$8,3 & $~~$8,6$^{~}$ & $~~$7,9 & $~~$4,5 & $~~$5,6$^{~}$ & $~~$4,9 & $~~$5,4 & $~~$5,0\\
        SingleRank & $~~$4,5 & $~~$9,0 & $~~$5,9$^{~}$ & 19,4 & 20,7 & 19,7$^{~}$ & $~~$4,6 & $~~$3,2 & $~~$3,7$^{~}$ & 22,3 & 28,4 & 24,6\\
        TopicRank & \textbf{11,7} & \textbf{21,7} & \textbf{15,1}$^\dagger$ & \textbf{35,0} & \textbf{37,5} & \textbf{35,6}$^\dagger$ & \textbf{14,9}$^{~}$ & \textbf{10,3} & \textbf{12,1}$^\dagger$ & 18,3 & 23,8 & 20,4\\
        \hline
        \textbf{Borne haute} & \textbf{14,5} & \textbf{27,0} & \textbf{18,7}$^{~}$ & \textbf{41,8} & \textbf{44,1} & \textbf{42,2}$^{~}$ & \textbf{30,0} & \textbf{20,7} & \textbf{24,3}$^{~}$ & \textbf{30,5} & \textbf{38,7} & \textbf{33,7}\\
        \bottomrule
      \end{tabular}
    }
  \end{table}

  \vspace{1em}

  \begin{block}{Observations}
    \begin{itemize}
      \item{Meilleure performance globale}
      \item{Significativement meilleur que TextRank et SingleRank ($^\dagger$)}
      \item{Bonne perspective d'améliorarion (vers la borne haute)}
      \item{Plus faible performance sur DUC\hfill$\rightarrow$}
    \end{itemize}
  \end{block}
\end{frame}

\begin{frame}{TopicRank}\framesubtitle{Résultats dans le contexte général}
  \begin{table}
    \resizebox{\linewidth}{!}{
      \begin{tabular}{@{~}l|c@{~~}c@{~~}c@{~}|c@{~~}c@{~~}c@{~}|c@{~~}c@{~~}c@{~}|c@{~~}c@{~~}c@{~}}
        \toprule
        \multirow{2}{*}[-2pt]{\textbf{Méthode}} & \multicolumn{3}{c|}{\textbf{\textsc{De}ft} \textit{(fr)}} & \multicolumn{3}{c|}{\textbf{Wikinews} \textit{(fr)}} & \multicolumn{3}{c|}{\textbf{SemEval} \textit{(en)}} & \multicolumn{3}{c}{\textbf{\textsc{Duc}} \textit{(en)}}\\
        \cline{2-4}\cline{5-7}\cline{8-10}\cline{11-13}
        & P & R & F & P & R & F & P & R & F & P & R & F\\
        \hline
        SingleRank & $~~$4,5 & $~~$9,0 & $~~$5,9$^{~}$ & 19,4 & 20,7 & 19,7$^{~}$ & $~~$4,6 & $~~$3,2 & $~~$3,7$^{~}$ & \textbf{22,3} & \textbf{28,4} & \textbf{24,6}\\
        + complet & $~~$4,4 & $~~$9,0 & $~~$5,8$^{~}$ & 20,0 & 21,4 & 20,3${~}$ & $~~$5,5 & $~~$3,8 & $~~$4,4$^{~}$ & 22,2 & 28,1 & 24,5\\
        + candidats & 10,3 & 19,2 & 13,2$^\dagger$ & 28,5 & 30,0 & 28,8$^\dagger$ & $~~$9,4 & $~~$6,8 & $~~$7,8$^\dagger$ & 10,4 & 13,5 & 11,6\\
        + sujets & 11,1 & 20,4 & 14,2$^\dagger$ & 30,7 & 32,6 & 31,1$^\dagger$ & 14,2 & $~~$9,9 & 11,6$^\dagger$ & 18,9 & 24,2 & 21,0\\
        TopicRank & \textbf{11,7} & \textbf{21,7} & \textbf{15,1}$^\dagger$ & \textbf{35,0} & \textbf{37,5} & \textbf{35,6}$^\dagger$ & \textbf{14,9}$^{~}$ & \textbf{10,3} & \textbf{12,1}$^\dagger$ & 18,3 & 23,8 & 20,4\\
        \bottomrule
      \end{tabular}
    }
  \end{table}
\end{frame}

\begin{frame}{TopicRank}\framesubtitle{Résultats en domaines de spécialité}
  \begin{table}
    \resizebox{\linewidth}{!}{
      \begin{tabular}{@{~}l|c@{~~}c@{~~}c@{~}|c@{~~~~~~~}c@{~~~~~~}c@{~}|c@{~~}c@{~~}c@{~}|c@{~~}c@{~~}c@{~}}
        \toprule
        \multirow{2}{*}[-2pt]{\textbf{Méthode}} & \multicolumn{3}{c|}{\textbf{Linguistique} \textit{(fr)}} & \multicolumn{3}{c|}{\textbf{Sciences de l'info.} \textit{(fr)}} & \multicolumn{3}{c|}{\textbf{Archéologie} \textit{(fr)}} & \multicolumn{3}{c}{\textbf{Chimie} \textit{(fr)}}\\
        \cline{2-4}\cline{5-7}\cline{8-10}\cline{11-13}
        & P & R & F & P & R & F & P & R & F & P & R & F\\
        \hline
        \textsc{Tf-Idf} & \textbf{13,0} & \textbf{15,4} & \textbf{13,9} & \textbf{13,4} & \textbf{14,0} & \textbf{13,2} & \textbf{28,1} & \textbf{19,1} & \textbf{22,2} & \textbf{14,1} & \textbf{11,1} & \textbf{11,9}\\
        TextRank & $~~$7,1 & $~~$6,1 & $~~$6,4 & $~~$5,8 & $~~$4,3 & $~~$4,8 & $~~$10,2 & $~~$5,3 & $~~$6,8 & $~~$9,4 & $~~$5,3 & $~~$6,5\\
        SingleRank & $~~$9.0 & 10,6 & $~~$9,6 & $~~$9,5 & 10,0 & $~~$9,4 & 12,7 & $~~$8,9 & 10,2 & 13,0 & 10,4 & 11,0\\
        TopicRank & 11,2 & 13,1 & 11,9 & 12,1 & 12,8 & 12,1 & 27,5 & 18,7 & 21,8 & 13,8 & 11,1 & 11,8\\
        \hline
        \textbf{Borne haute} & \textbf{14,5} & \textbf{17,0} & \textbf{15,4} & \textbf{15,0} & \textbf{15,6} & \textbf{14,9} & \textbf{32,5} & \textbf{22,2} & \textbf{25,8} & \textbf{15,8} & \textbf{12,5} & \textbf{13,3}\\
        \bottomrule
      \end{tabular}
    }
  \end{table}

  \vspace{1em}

  \begin{block}{Observations}
    \begin{itemize}
      \item{Meilleure performance face à TextRank et SingleRank}
      \item{Plus faible perspective d'amélioration}
      \item{Meilleure performance globale pour TF-IDF}
    \end{itemize}
  \end{block}
\end{frame}

\begin{frame}{TopicRank}\framesubtitle{Bilan}
  Méthode à base de graphe qui s'intéresse à ce que représentent les
  termes-clés candidats plutôt qu'à eux-mêmes ou à leurs mots.

  \vspace{1em}

  \begin{block}{Avantages}
    \begin{itemize}
      \item{Non redondant}
      \item{Bon ordonnancement des sujets (cf. borne haute)}
    \end{itemize}
  \end{block}

  \vspace{1em}

  \begin{alertblock}{Inconvénients}
    \begin{itemize}
      \item{Groupement naïf des candidats en sujets}
      \item{Quelques instabilités}
      \item{Plus faible en domaines de spécialité}
    \end{itemize}
  \end{alertblock}
\end{frame}

