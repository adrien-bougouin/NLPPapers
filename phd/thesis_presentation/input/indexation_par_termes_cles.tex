\section{Indexation par termes-clés}
  \begin{frame}{Indexation par termes-clés}
    \begin{figure}
      \centering
      \begin{tikzpicture}[scale=.35, node distance=3cm]
        \node [component] (extraction) {Extraction de termes-clés};
        \node [component, right=of extraction] (assignement) {Assignement de termes-clés};

        \node [io, above=of extraction] (document) {document};
        \node [io, above=of assignement] (vocabulaire_controle) {vocabulaire contrôlé};

        \node [io, below=of extraction] (termes_cles_extraits) {termes-clés $\subseteq$ document};
        \node [io, below=of assignement] (termes_cles_assignes) {termes-clés $\subseteq$ vocabulaire contrôlé};

        \node [above=of document, yshift=-2.85cm] (general) {\small Général};

        \draw [dashed] ($(extraction.north west)+(-.5cm, 5.5cm)$) rectangle ($(extraction.south east)+(.5cm, -5.5cm)$);
        \node [above=of vocabulaire_controle, yshift=-2.9cm] (domaine_de_specialite) {\small Domaines de spécialité};

        \draw [dashed] ($(assignement.north west)+(-1cm, 5.5cm)$) rectangle ($(assignement.south east)+(1cm, -5.5cm)$);

        \path [->] (document) edge (extraction);
        \path [->] (extraction) edge (termes_cles_extraits);
        \path [->] (document) edge (assignement);
        \path [->] (vocabulaire_controle) edge (assignement);
        \path [->] (assignement) edge (termes_cles_assignes);
      \end{tikzpicture}
    \end{figure}
  \end{frame}

  \subsection{Extraction de termes-clés}
    \begin{frame}{Indexation par termes-clés}
      \begin{figure}
        \centering
        \begin{tikzpicture}[scale=.35, node distance=3cm]
          \node [selectedcomponent] (extraction) {Extraction de termes-clés};
          \node [component, right=of extraction] (assignement) {Assignement de termes-clés};

          \node [io, above=of extraction] (document) {document};
          \node [io, above=of assignement] (vocabulaire_controle) {vocabulaire contrôlé};

          \node [io, below=of extraction] (termes_cles_extraits) {termes-clés $\subseteq$ document};
          \node [io, below=of assignement] (termes_cles_assignes) {termes-clés $\subseteq$ vocabulaire contrôlé};

          \node [above=of document, yshift=-2.85cm] (general) {\small Général};

          \draw [dashed] ($(extraction.north west)+(-.5cm, 5.5cm)$) rectangle ($(extraction.south east)+(.5cm, -5.5cm)$);
          \node [above=of vocabulaire_controle, yshift=-2.9cm] (domaine_de_specialite) {\small Domaines de spécialité};

          \draw [dashed] ($(assignement.north west)+(-1cm, 5.5cm)$) rectangle ($(assignement.south east)+(1cm, -5.5cm)$);

          \path [->] (document) edge (extraction);
          \path [->] (extraction) edge (termes_cles_extraits);
          \path [->] (document) edge (assignement);
          \path [->] (vocabulaire_controle) edge (assignement);
          \path [->] (assignement) edge (termes_cles_assignes);
        \end{tikzpicture}
      \end{figure}
    \end{frame}
    \begin{frame}{Indexation par termes-clés}\framesubtitle{Extraction de termes-clés}
  \begin{figure}
    \begin{tikzpicture}[scale=.3, node distance=2cm]
      \uncover<1->{
        \node [io] (document) {document};
      }
      \uncover<2->{
        \node [component] (preprocessing) [below=of document] {Prétraitement linguistique};
      }
      \uncover<3->{
        %\alt<9>{
        %  \node [selectedcomponent] (candidate_extraction) [below=of preprocessing] {Sélection des termes-clés candidats};
        %}{
          \node [component] (candidate_extraction) [below=of preprocessing] {Sélection des termes-clés candidats};
        %}
      }
      \uncover<4->{
        %\alt<9>{
        %  \node [selectedcomponent, yshift=-4cm] (candidate_classification_and_ranking) [below=of candidate_extraction] {Ordonnancement des candidats};
        %}{
          \node [component, yshift=-4cm] (candidate_classification_and_ranking) [below=of candidate_extraction] {Ordonnancement des candidats};
        %}
      }
      \uncover<7->{
        \node [component, minimum width=24cm, xshift=6.5cm] (keyphrase_selection) [below=of candidate_classification_and_ranking] {Sélection des termes-clés à extraire};
      }
      \uncover<8->{
        \node [io] (keyphrases) [below=of keyphrase_selection] {termes-clés};
      }
      %
      \uncover<6->{
        \node [component] (preprocessing2) [right=of preprocessing] {Prétraitement linguistique};
        \node [component] (candidate_extraction2) [right=of candidate_extraction] {Sélection des termes-clés candidats};
      }
      \uncover<5->{
        \node [component] (classification) [right=of candidate_classification_and_ranking] {Classification des candidats};
      }
      \uncover<6->{
        \node [component] (learning) [below=of candidate_extraction2]
        {Apprentissage du modèle de classification};
        \node [io] (documents) [above=of preprocessing2] {documents d'apprentissage};
      }

      \uncover<2->{
        \path [->] (document) edge (preprocessing);
      }
      \uncover<3->{
        \path [->] (preprocessing) edge (candidate_extraction);
      }
      \uncover<7->{
        \path [->] (candidate_classification_and_ranking) edge (keyphrase_selection);
      }
      \uncover<8->{
        \path [->] (keyphrase_selection) edge (keyphrases);
      }
      %
      \uncover<6->{
        \path [->] (documents) edge (preprocessing2);
        \path [->] (preprocessing2) edge (candidate_extraction2);
        \path [->] (candidate_extraction2) edge (learning);
        \path [->] (learning) edge (classification);
      }
      \uncover<7->{
        \path [->] (classification) edge (keyphrase_selection);
      }
      %
      \uncover<4->{
        \draw [->] (candidate_extraction) -- (candidate_classification_and_ranking) node [midway] (midway1) {};
      }
      \uncover<5->{
        \draw [->] (candidate_extraction) -- (classification.north west) node [midway] (midway2) {};
        \draw [dashed] (midway1) -- (midway2) node [midway, below, scale=.5] (xor) {\{ou\}};
      }

      \uncover<6->{
        \draw [dashed] ($(preprocessing2.north west)+(-.5cm,4.5cm)$) rectangle ($(learning.south east)+(.5cm,-.5cm)$);
        \node [above=of documents, yshift=-1.85cm, scale=.5] (apprentissage) {Apprentissage};
      }
    \end{tikzpicture}
  \end{figure}
\end{frame}

\subsubsection{Sélection des termes-clés candidats}
  \begin{frame}{Extraction de termes-clés}\framesubtitle{Sélection des termes-clés candidats}
    \begin{block}<1->{Terme-clé candidat}
      Unité textuelle qui possède des caractéristiques (linguistiques) d'un
      terme-clé.
    \end{block}

    \vspace{1em}

    \uncover<2->{
      Trois méthodes de sélection classiques~:
      \begin{itemize}
        \item<2->{$N$-grammes ($N = 1..3$)}
        \item<7->{Syntagmes nominaux}
        \item<12->{Séquences grammaticalement définies}
        \begin{itemize}
          \item{Séquences \texttt{NOM}~/~\texttt{ADJ} (générique)}
          \item{Séquences observées (spécifique)}
        \end{itemize}
      \end{itemize}
    }

    \vspace{1em}

    \uncover<15->{
      $\rightarrow$ Certaines méthodes sont $\pm$ exhaustives
    }

    \uncover<16->{
      $\Rightarrow$ Les résultats de l'extraction de termes-clés varient
    }

    \visible<3,4>{
      \begin{textblock*}{\textwidth}(0\textwidth, -.5\textheight)
        \setbeamertemplate{blocks}[rounded][shadow=true]

        \begin{exampleblock}{$3$-grammes}\small
          \og{}
            À l'exception de la province de Luxembourg, en alerte jaune,
            l'ensemble de la Belgique est en vigilance orange à la canicule.
          \fg{}

          \begin{multicols}{2}
            \begin{itemize}
              \item{\alt<4>{\sout{À l'exception}}{À l'exception}}
              \item{\alt<4>{\sout{l'exception de}}{l'exception de}}
              \item{\alt<4>{\sout{exception de la}}{exception de la}}
              \item{\alt<4>{\sout{de la province}}{de la province}}
              \item{\alt<4>{\sout{la province de}}{la province de}}
              \item{province de Luxembourg}
              \item{\alt<4>{\sout{en alerte jaune}}{en alerte jaune}}
              \item{\alt<4>{\sout{l'ensemble de}}{l'ensemble de}}
              \item{\alt<4>{\sout{ensemble de la}}{ensemble de la}}
              \item{\alt<4>{\sout{de la Belgique}}{de la Belgique}}
              \item{\alt<4>{\sout{la Belgique est}}{la Belgique est}}
              \item{\alt<4>{\sout{Belgique est en}}{Belgique est en}}
            \end{itemize}
          \end{multicols}
        \end{exampleblock}
      \end{textblock*}
    }

    \visible<5>{
      \begin{textblock*}{\textwidth}(0\textwidth, -.5\textheight)
        \setbeamertemplate{blocks}[rounded][shadow=true]

        \begin{exampleblock}{$\{1..3\}$-grammes}\small
          \og{}
            À l'exception de la province de Luxembourg, en alerte jaune,
            l'ensemble de la Belgique est en vigilance orange à la canicule.
          \fg{}

          \begin{multicols}{2}
            \begin{itemize}
              \item{exception}
              \item{province}
              \item{Luxembourg}
              \item{alerte}
              \item{jaune}
              \item{ensemble}
              \item{Belgique}
              \item{vigilance}
              \item{orange}
              \item{canicule}
              \item{alerte jaune}
              \item{vigilance orange}
              \item{province de Luxembourg}
            \end{itemize}
          \end{multicols}
        \end{exampleblock}
      \end{textblock*}
    }

    \visible<8,9,10>{
      \begin{textblock*}{\textwidth}(0\textwidth, -.5\textheight)
        \setbeamertemplate{blocks}[rounded][shadow=true]

        \begin{exampleblock}{Syntagmes nominaux}\small
          \og{}
            À l'exception de la province de Luxembourg, en alerte jaune,
            l'ensemble de la Belgique est en vigilance orange à la canicule.
          \fg{}

          \begin{multicols}{2}
            \begin{itemize}
              \item{\alt<9->{\sout{l'}}{l'}\alt<10>{[exception]}{exception} de
                    la \alt<10>{[province]}{province} de
                    \alt<10>{[Luxembourg]}{Luxembourg}}
              \item{alerte jaune}
              \item{\alt<9->{\sout{l'}}{l'}\alt<10>{[ensemble]}{ensemble} de la
                \alt<10>{[Belgique]}{Belgique}}
              \item{vigilance orange}
              \item{\alt<9->{\sout{la}}{la} canicule}
            \end{itemize}
          \end{multicols}
        \end{exampleblock}
      \end{textblock*}
    }

    \visible<13>{
      \begin{textblock*}{\textwidth}(0\textwidth, -.5\textheight)
        \setbeamertemplate{blocks}[rounded][shadow=true]

      \begin{exampleblock}{Syntagmes nominaux}\small
          \og{}
            À l'exception de la province de Luxembourg, en alerte jaune,
            l'ensemble de la Belgique est en vigilance orange à la canicule.
          \fg{}

          \begin{multicols}{2}
            \begin{itemize}
              \item{exception}
              \item{province}
              \item{Luxembourg}
              \item{alerte jaune}
              \item{ensemble}
              \item{Belgique}
              \item{vigilance orange}
              \item{canicule}
            \end{itemize}
          \end{multicols}
        \end{exampleblock}
      \end{textblock*}
    }
  \end{frame}

\subsubsection{Extraction non supervisée}
  \begin{frame}{Extraction de termes-clés}\framesubtitle{Extraction non supervisée}
    \begin{block}<+->{Objectif}
      Déterminer les termes-clés candidats les plus importants.
    \end{block}

    \uncover<+->{
      Ordonnancement des candidats selon diverses mesures d'importance~:
      \begin{itemize}
        \item{}
        \item{}
        \item{}
        \item{}
        \item{}
      \end{itemize}
    }
  \end{frame}

\subsubsection{Extraction supervisée}
  \begin{frame}{Extraction de termes-clés}\framesubtitle{Extraction supervisée}
    \begin{block}<+->{Objectif}
      Reconnaître les termes-clés parmi les termes-clés candidats.
    \end{block}

    \uncover<+->{
      Classification des candidats selon divers critères (traits)~:
      \begin{itemize}
        \item{}
        \item{}
        \item{}
        \item{}
        \item{}
      \end{itemize}
    }
  \end{frame}

\subsubsection{Bilan}
  \begin{frame}{Extraction de termes-clés}\framesubtitle{Bilan}
    \begin{block}{Avantages}
    \end{block}

    \begin{alertblock}{Inconvénients}
    \end{alertblock}
    
  \end{frame}



  \subsection{Assignement de termes-clés}
    \begin{frame}{Indexation par termes-clés}
      \begin{figure}
        \centering
        \begin{tikzpicture}[scale=.35, node distance=3cm]
          \node [component] (extraction) {Extraction de termes-clés};
          \node [selectedcomponent, right=of extraction] (assignement) {Assignement de termes-clés};

          \node [io, above=of extraction] (document) {document};
          \node [io, above=of assignement] (vocabulaire_controle) {vocabulaire contrôlé};

          \node [io, below=of extraction] (termes_cles_extraits) {termes-clés $\subseteq$ document};
          \node [io, below=of assignement] (termes_cles_assignes) {termes-clés $\subseteq$ vocabulaire contrôlé};

          \node [above=of document, yshift=-2.85cm] (general) {\small Général};

          \draw [dashed] ($(extraction.north west)+(-.5cm, 5.5cm)$) rectangle ($(extraction.south east)+(.5cm, -5.5cm)$);
          \node [above=of vocabulaire_controle, yshift=-2.9cm] (domaine_de_specialite) {\small Domaines de spécialité};

          \draw [dashed] ($(assignement.north west)+(-1cm, 5.5cm)$) rectangle ($(assignement.south east)+(1cm, -5.5cm)$);

          \path [->] (document) edge (extraction);
          \path [->] (extraction) edge (termes_cles_extraits);
          \path [->] (document) edge (assignement);
          \path [->] (vocabulaire_controle) edge (assignement);
          \path [->] (assignement) edge (termes_cles_assignes);
        \end{tikzpicture}
      \end{figure}
    \end{frame}
    \begin{frame}{Indexation par termes-clés}\framesubtitle{Assignement de termes-clés}
  \begin{block}{Objectif}
    Positionner le document vis-à-vis de son domaine.
  \end{block}

  \vspace{1em}

  \begin{block}{Vocabulaire contrôlé}
    Terminologie spécifique à un domaine~:
    \begin{itemize}
      \item{Liste de termes}
      \item{Thésaurus}
    \end{itemize}
  \end{block}
\end{frame}

\begin{frame}{Assignement de termes-clés}\framesubtitle{Méthodes}
  Classification des entrées d'un thésaurus~\cite[KEA++]{medelyan2006kea++}~:
  \begin{enumerate}
    \item{Termes-clés candidats $\subseteq$ thésaurus}
    \item{Trois critères de classification~:}
    \begin{itemize}
      \item{TF-IDF \tikz[overlay, remember picture] \node [xshift=4.15em, yshift=1.25em] (top) {};}
      \item{Première position \tikz[overlay, remember picture] \node [yshift=-.6em] (bottom) {};}
      \item{Nombre de relations avec d'autres candidats}
    \end{itemize}
  \end{enumerate}

  \begin{tikzpicture}[overlay, remember picture]
    \draw [decoration={brace, amplitude=0.5em}, decorate, thick]
    (top) -- (bottom) node [xshift=2em, yshift=1.5em] (kea) {KEA};
  \end{tikzpicture}

  Classification multi-étiquettes et multi-classes~\cite{partalas2013bioasq}~:
  \begin{itemize}
    \item{$n$ étiquettes $=$ $n$ termes-clés à assigner}
    \item{$m$ classes $=$ entrées du vocabulaire contrôlé}
  \end{itemize}
\end{frame}

%\begin{frame}{Assignement de termes-clés}\framesubtitle{Alternatives}
%  Indexation par termes-clés $\Leftrightarrow$ Traduction
%  automatique~\cite{liu2011vocabularygap}~:
%  \begin{itemize}
%    \item{Document $\Leftrightarrow$ Langage naturel}
%    \item{Termes-clés $\Leftrightarrow$ Langage synthétique}
%  \end{itemize}
%\end{frame}

%\subsubsection{Bilan}
%  \begin{frame}{Assignement de termes-clés}\framesubtitle{Bilan}
%    \begin{itemize}
%      \item{Une seule méthodes}
%      \item{Avec ressources supplémentaires}
%    \end{itemize}
%
%    \vspace{1em}
%
%    \begin{block}{Avantages}
%      \begin{itemize}
%        \item{Adapté au domaine de spécialité traité}
%        \item{Assigne des termes-clés terminologiquement validés}
%        \item{Indexation par termes-clés cohérente}
%      \end{itemize}
%    \end{block}
%
%    \begin{alertblock}{Inconvénients}
%      \begin{itemize}
%        \item{Scénarii d'utilisation limités}
%        \item{Impossibilité de trouver tous les termes-clés}
%      \end{itemize}
%    \end{alertblock}
%  \end{frame}



    \begin{frame}{Indexation par termes-clés}\framesubtitle{Bilan}
    Extraction Vs Assignement~:
    \begin{itemize}
      \item{Contexte général Vs domaines de spécialité}
      \item{Limation au document Vs Limitation au vocabulaire contrôlé}
    \end{itemize}

    \vspace{1em}

    \begin{alertblock}{Limite}
      Manque d'exhaustivité
    \end{alertblock}
  \end{frame}

  \begin{frame}{Indexation par termes-clés}\framesubtitle{Bilan}
    \vspace{-.33em}
    \begin{exampleblock}{\small
      Étude préliminaire de la \underline{céramique non tournée}
      \underline{micacée} du bas Languedoc occidental~: \underline{typologie},
      \underline{chronologie} et aire de \underline{diffusion}
    }\justifying\small
      ~~~L'étude présente une variété de \underline{céramique non tournée} dont la
      \underline{typologie} et l'analyse des \underline{décors} permettent de
      l'identifier facilement. La nature de l'argile enrichie de mica donne un
      aspect pailleté à la pâte sur laquelle le \underline{décor} effectué selon
      la méthode du brunissoir apparaît en traits brillant sur fond mat. Cette
      première approche se fonde sur deux séries issues de \underline{fouilles
      anciennes} menées sur les \underline{oppidums} \underline{du Cayla} à
      \underline{Mailhac} (\underline{Aude}) et de \underline{Mourrel-Ferrat} à
      \underline{Olonzac} (\underline{Hérault}). La carte de
      \underline{répartition} fait état d'\underline{échanges} ou de
      \underline{commerce} à l'échelon macrorégional rarement mis en évidence pour
      de la \underline{céramique non tournée}. S'il est difficile de statuer sur
      l'origine des \underline{décors}, il semble que la \underline{production}
      s'insère dans une ambiance celtisante. La \underline{chronologie} de cette
      \underline{production} se situe dans le deuxième \underline{âge du Fer}. La
      fourchette proposée entre la fin du IV$^\text{e}$ et la fin du II$^\text{e}$
      s. av. J.-C. reste encore à préciser.

      \begin{exampleblock}{\small Termes-clés}\justifying\small
        \underline{Mailhac}~; \underline{Aude}~; \underline{Mourrel-Ferrat}~;
        \underline{Olonzac}~; \underline{Hérault}~; \underline{céramique}~;
        \underline{typologie}~; \underline{décor}~; \underline{chronologie}~;
        \underline{diffusion}~; \underline{production}~; \underline{commerce}~;
        \underline{répartition}~; \underline{oppidum}~; \underline{analyse}~;
        \underline{fouille ancienne}~; \underline{le Cayla}~;
        \underline{micassé}~; \underline{céramique non-tournée}~;
        \underline{echange}~; \underline{age du} \underline{Fer}~; La Tène~;
        Europe~; France~; celtes~; distribution~; cartographie~; habitat~; site
        fortifié~; identification~; étude du matériel
      \end{exampleblock}
    \end{exampleblock}
  \end{frame}
