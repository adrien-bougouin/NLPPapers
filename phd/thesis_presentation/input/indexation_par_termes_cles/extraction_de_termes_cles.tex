\begin{frame}{Indexation par termes-clés}\framesubtitle{Extraction de termes-clés}
  \begin{figure}
    \centering
    \begin{tikzpicture}[scale=.35, node distance=2cm]
      \node [io] (document) {document};
      \node [component, below=of document] (pretraitement_et_selection_des_candidats) {Sélection des termes-clés candidats};
      \node [component, yshift=-4cm, below=of pretraitement_et_selection_des_candidats] (ordonnancement) {Ordonnancement des candidats};
      \node [io, below=of ordonnancement, xshift=7.5cm] (termes_cles) {termes-clés};
      \path [->] (document) edge (pretraitement_et_selection_des_candidats);
      \path [->] (pretraitement_et_selection_des_candidats) edge (ordonnancement);
      \path [->] (ordonnancement) edge (termes_cles);
      %
      \node [component, right=of pretraitement_et_selection_des_candidats] (pretraitement_et_selection_des_candidats2) {Sélection des termes-clés candidats};
      \node [io, above=of pretraitement_et_selection_des_candidats2] (documents) {documents d'entraînement};
      \node [component, below=of pretraitement_et_selection_des_candidats2] (apprentissage) {Apprentissage d'un modèle de classification};
      \node [component, below=of apprentissage] (classification) {Classification des candidats};
      \path [->] (documents) edge (pretraitement_et_selection_des_candidats2);
      \path [->] (pretraitement_et_selection_des_candidats2) edge (apprentissage);
      \path [->] (apprentissage) edge (classification);
      \path [->] (classification) edge (termes_cles);
      %
      \draw [->] (pretraitement_et_selection_des_candidats) -- (ordonnancement) node [midway] (midway1) {};
      \draw [->] (pretraitement_et_selection_des_candidats) -- (classification.north west) node [midway] (midway2) {};
      \draw [dashed] (midway1) -- (midway2) node [midway, below, scale=.5] (xor) {\Large \{ou\}};
      %
      \draw [dashed] ($(pretraitement_et_selection_des_candidats2.north west)+(-.5cm,4.5cm)$) rectangle ($(apprentissage.south east)+(.5cm,-.5cm)$);
      \node [above=of documents, yshift=-1.8cm, scale=.5] (titre_apprentissage) {\Large Apprentissage supervisé};
    \end{tikzpicture}
  \end{figure}
\end{frame}

\subsubsection{Sélection des termes-clés candidats}
  \begin{frame}{Indexation par termes-clés}\framesubtitle{Extraction de termes-clés}
    \begin{figure}
      \centering
      \begin{tikzpicture}[scale=.35, node distance=2cm]
        \node [io] (document) {document};
        \node [selectedcomponent, below=of document] (pretraitement_et_selection_des_candidats) {Sélection des termes-clés candidats};
        \node [component, yshift=-4cm, below=of pretraitement_et_selection_des_candidats] (ordonnancement) {Ordonnancement des candidats};
        \node [io, below=of ordonnancement, xshift=7.5cm] (termes_cles) {termes-clés};
        \path [->] (document) edge (pretraitement_et_selection_des_candidats);
        \path [->] (pretraitement_et_selection_des_candidats) edge (ordonnancement);
        \path [->] (ordonnancement) edge (termes_cles);
        %
        \node [selectedcomponent, right=of pretraitement_et_selection_des_candidats] (pretraitement_et_selection_des_candidats2) {Sélection des termes-clés candidats};
        \node [io, above=of pretraitement_et_selection_des_candidats2] (documents) {documents d'entraînement};
        \node [component, below=of pretraitement_et_selection_des_candidats2] (apprentissage) {Apprentissage d'un modèle de classification};
        \node [component, below=of apprentissage] (classification) {Classification des candidats};
        \path [->] (documents) edge (pretraitement_et_selection_des_candidats2);
        \path [->] (pretraitement_et_selection_des_candidats2) edge (apprentissage);
        \path [->] (apprentissage) edge (classification);
        \path [->] (classification) edge (termes_cles);
        %
        \draw [->] (pretraitement_et_selection_des_candidats) -- (ordonnancement) node [midway] (midway1) {};
        \draw [->] (pretraitement_et_selection_des_candidats) -- (classification.north west) node [midway] (midway2) {};
        \draw [dashed] (midway1) -- (midway2) node [midway, below, scale=.5] (xor) {\Large \{ou\}};
        %
        \draw [dashed] ($(pretraitement_et_selection_des_candidats2.north west)+(-.5cm,4.5cm)$) rectangle ($(apprentissage.south east)+(.5cm,-.5cm)$);
        \node [above=of documents, yshift=-1.8cm, scale=.5] (titre_apprentissage) {\Large Apprentissage supervisé};
      \end{tikzpicture}
    \end{figure}
  \end{frame}

  \begin{frame}{Extraction de termes-clés}\framesubtitle{Sélection des termes-clés candidats}
    Deux principales méthodes

    \begin{exampleblock}{Exemple}
      \og{}À l'exception de la province de Luxembourg, en alerte jaune,
      l'ensemble de la Belgique est en vigilance orange à la canicule.\fg{}

      \vspace{1em}

      Sélection des termes-clés candidats~:
      \begin{itemize}
        \item{$N$-grammes ($N = 1..3$)~:}
          \begin{itemize}
            \item{$\{1..3\}$-grammes~:}
            \begin{itemize}\setlength{\itemindent}{.5cm}
              \item[$(N = 1)$]{
                exception~; province~; Luxembourg~; alerte~; jaune~; ensemble~;\\
                \hspace{.5cm}Belgique~; vigilance~; orange~; canicule~;
              }
              \item[$(N = 2)$]{
                alerte jaune~; vigilance orange~;
              }
              \item[$(N = 3)$]{
                province de Luxembourg
              }
            \end{itemize}
          \end{itemize}
        \item{Séquences grammaticalement définies~:}
          \begin{itemize}
            \item{\texttt{/(NOM | ADJ)+/}~:}
            \begin{itemize}\setlength{\itemindent}{.5cm}
              \item{exception~; province~; Luxembourg~; alerte jaune~;
                    ensemble~;\\
                    \hspace{.5cm}Belgique~; vigilance orange~; canicule}
            \end{itemize}
          \end{itemize}
      \end{itemize}
    \end{exampleblock}
  \end{frame}

\subsubsection{Ordonancement des candidats}
  \begin{frame}{Indexation par termes-clés}\framesubtitle{Extraction de termes-clés}
    \begin{figure}
      \centering
      \begin{tikzpicture}[scale=.35, node distance=2cm]
        \node [io] (document) {document};
        \node [component, below=of document] (pretraitement_et_selection_des_candidats) {Sélection des termes-clés candidats};
        \node [selectedcomponent, yshift=-4cm, below=of pretraitement_et_selection_des_candidats] (ordonnancement) {Ordonnancement des candidats};
        \node [io, below=of ordonnancement, xshift=7.5cm] (termes_cles) {termes-clés};
        \path [->] (document) edge (pretraitement_et_selection_des_candidats);
        \path [->] (pretraitement_et_selection_des_candidats) edge (ordonnancement);
        \path [->] (ordonnancement) edge (termes_cles);
        %
        \node [component, right=of pretraitement_et_selection_des_candidats] (pretraitement_et_selection_des_candidats2) {Sélection des termes-clés candidats};
        \node [io, above=of pretraitement_et_selection_des_candidats2] (documents) {documents d'entraînement};
        \node [component, below=of pretraitement_et_selection_des_candidats2] (apprentissage) {Apprentissage d'un modèle de classification};
        \node [component, below=of apprentissage] (classification) {Classification des candidats};
        \path [->] (documents) edge (pretraitement_et_selection_des_candidats2);
        \path [->] (pretraitement_et_selection_des_candidats2) edge (apprentissage);
        \path [->] (apprentissage) edge (classification);
        \path [->] (classification) edge (termes_cles);
        %
        \draw [->] (pretraitement_et_selection_des_candidats) -- (ordonnancement) node [midway] (midway1) {};
        \draw [->] (pretraitement_et_selection_des_candidats) -- (classification.north west) node [midway] (midway2) {};
        \draw [dashed] (midway1) -- (midway2) node [midway, below, scale=.5] (xor) {\Large \{ou\}};
        %
        \draw [dashed] ($(pretraitement_et_selection_des_candidats2.north west)+(-.5cm,4.5cm)$) rectangle ($(apprentissage.south east)+(.5cm,-.5cm)$);
        \node [above=of documents, yshift=-1.8cm, scale=.5] (titre_apprentissage) {\Large Apprentissage supervisé};
      \end{tikzpicture}
    \end{figure}
  \end{frame}

  \begin{frame}{Extraction de termes-clés}\framesubtitle{Ordonnancement des candidats}
    \begin{block}{Objectif}
      Déterminer les termes-clés candidats les plus importants
    \end{block}

    \vspace{1em}

    Ordonnancement des candidats selon diverses approches~:
    \begin{itemize}
      \item{Statistiques~\cite[TF-IDF]{salton1975tfidf}}
      \item{Par groupement distributionnel~\cite{matsuo2004wordcooccurrence}}
      \item{\textbf{À base de graphe}~\cite[TextRank]{mihalcea2004textrank}}
    \end{itemize}
  \end{frame}

  \begin{frame}{Ordonnancement des candidats}\framesubtitle{TextRank -- SingleRank}
    \begin{exampleblock}{\small
      Météo du 19 \underline{août 2012}~: \underline{alerte} à la
      \underline{canicule} sur la \underline{Belgique} et le
      \underline{Luxembourg}
    }\justifying\small
      ~~~À l'exception de la province de \underline{Luxembourg}, en
      \underline{alerte} jaune, l'ensemble de la \underline{Belgique} est en
      vigilance \underline{orange} à la \underline{canicule}. Le
      \underline{Luxembourg} n'est pas épargné par la vague du \underline{chaleur}
      : le nord du pays est en \underline{alerte} \underline{orange}, tandis que
      le sud a était placé en \underline{alerte} rouge.

      ~~~En \underline{Belgique}, la \underline{température} n'est pas descendue
      en dessous des 23\degre{}C cette nuit, ce qui constitue la deuxième nuit
      \underline{la plus chaude} jamais enregistrée dans le royaume. Il se
      pourrait que ce dimanche soit la journée \underline{la plus chaude} de
      l'année. Les \underline{températures} seront comprises entre 33 et
      38\degre{}C. Une légère brise de côte pourra faiblement rafraichir
      l'atmosphère. Des orages de \underline{chaleur} sont a prévoir dans la
      soirée et en début de nuit.

      ~~~Au \underline{Luxembourg}, le mercure devrait atteindre 32\degre{}C ce
      dimanche sur l'Oesling et jusqu'à 36\degre{}C sur le sud du pays, et 31 à
      32\degre{}C lundi. Une baisse devrait intervenir pour le reste de la
      semaine. Néanmoins, le record d'août 2003 (37,9\degre{}C) ne devrait pas
      être atteint.

      \begin{exampleblock}{\small Termes-clés}\justifying\small
        \underline{Août 2012}~; \underline{canicule}~;
        \underline{Belgique}~; \underline{Luxembourg}~; \underline{alerte}~;
        \underline{orange}~; \underline{chaleur}~; \underline{chaude}~;
        \underline{température}~; \underline{la plus chaude}
      \end{exampleblock}
    \end{exampleblock}
  \end{frame}

  \begin{frame}{Ordonnancement des candidats}\framesubtitle{TextRank -- SingleRank}
    \begin{columns}[t]
      \begin{column}{.4\linewidth}\centering
        TextRank\\\cite{mihalcea2004textrank}

        \vspace{3em}

        \begin{overpic}[width=.95\linewidth]{include/44960_textrank_graph_termithgreen.eps}
          \put (19, 99) {\scriptsize \alt<2->{\textcolor{termithorange}{\textbf{deuxième}}}{\textcolor{Cerulean}{deuxième}}}
          \put (57, 101.5) {\scriptsize \alt<2->{\textcolor{termithorange}{\textbf{nuit}}}{\textcolor{Cerulean}{nuit}}}
          \put (-4, 61) {\scriptsize \alt<2->{\textcolor{termithorange}{\textbf{août}}}{\textcolor{Cerulean}{août}}}
          \put (20, 78) {\scriptsize \alt<2->{\textcolor{termithorange}{\textbf{2003}}}{\textcolor{Cerulean}{2003}}}
          \put (8, 14) {\scriptsize \alt<2->{\textcolor{termithorange}{\textbf{2012}}}{\textcolor{Cerulean}{2012}}}
          \put (48, 73) {\scriptsize \alt<2->{\textcolor{termithorange}{\textbf{alerte}}}{\textcolor{Cerulean}{alerte}}}
          \put (31, 33) {\scriptsize \textcolor{Cerulean}{jaune}}
          \put (60, 44) {\scriptsize \textcolor{Cerulean}{rouge}}
          \put (88, 46) {\scriptsize \alt<2->{\textcolor{termithorange}{\textbf{orange}}}{\textcolor{Cerulean}{orange}}}
          \put (80, 79) {\scriptsize \alt<2->{\textcolor{termithorange}{\textbf{vigilance}}}{\textcolor{Cerulean}{vigilance}}}
          \put (41, 3) {\scriptsize \alt<2->{\textcolor{termithorange}{\textbf{brise}}}{\textcolor{Cerulean}{brise}}}
          \put (71, 15) {\scriptsize \alt<2->{\textcolor{termithorange}{\textbf{légère}}}{\textcolor{Cerulean}{légère}}}
        \end{overpic}
      \end{column}
      \hspace{-1em}
      \vrule{}
      \hspace{1em}
      \begin{column}{.5\linewidth}\centering
        SingleRank\\\cite{wan2008expandrank}

        \vspace{2em}

        \begin{overpic}[width=\linewidth]{include/44960_singlerank_graph_termithgreen.eps}
          \put (46, 15.5) {\scriptsize \textcolor{Cerulean}{nord}}
          \put (55, 17) {\scriptsize \textcolor{Cerulean}{deuxième}}
          \put (43, 31) {\scriptsize \textcolor{Cerulean}{nuit}}
          \put (56.5, 32) {\scriptsize \textcolor{Cerulean}{août}}
          \put (29, 18.5) {\scriptsize \textcolor{Cerulean}{chaleur}}
          \put (18, 25) {\scriptsize \textcolor{Cerulean}{vague}}
          \put (0, 27.5) {\scriptsize \textcolor{Cerulean}{soirée}}
          \put (9, 15) {\scriptsize \textcolor{Cerulean}{orages}}
          \put (22, 6) {\scriptsize \textcolor{Cerulean}{début}}
          \put (37, 2.5) {\scriptsize \textcolor{Cerulean}{rouge}}
          \put (51.5, 1.5) {\scriptsize \textcolor{Cerulean}{record}}
          \put (67, 6) {\scriptsize \textcolor{Cerulean}{37,0\degre{}C}}
          \put (79.5, 15) {\scriptsize \textcolor{Cerulean}{2003}}
          \put (88.5, 27) {\scriptsize \textcolor{Cerulean}{23\degre{}C}}
          \put (71, 27.5) {\scriptsize \textcolor{Cerulean}{pays}}
          \put (-6, 42) {\scriptsize \textcolor{Cerulean}{année}}
          \put (9.5, 37.75) {\scriptsize \textcolor{Cerulean}{chaud}}
          \put (4.5, 51.5) {\scriptsize \textcolor{Cerulean}{journée}}
          \put (27, 36) {\scriptsize \textcolor{Cerulean}{météo}}
          \put (22, 50) {\scriptsize \textcolor{Cerulean}{2002}}
          \put (35, 48) {\scriptsize \textcolor{Cerulean}{alerte}}
          \put (49, 46.5) {\scriptsize \textcolor{Cerulean}{orange}}
          \put (62, 44) {\scriptsize \textcolor{Cerulean}{royaume}}
          \put (82, 38) {\scriptsize \textcolor{Cerulean}{sud}}
          \put (83, 43) {\scriptsize \textcolor{Cerulean}{température}}
          \put (94, 59) {\scriptsize \textcolor{Cerulean}{lundi}}
          \put (88.5, 73) {\scriptsize \textcolor{Cerulean}{baisse}}
          \put (81, 86.25) {\scriptsize \textcolor{Cerulean}{reste}}
          \put (56, 86.5) {\scriptsize \textcolor{Cerulean}{semaine}}
          \put (30, 100) {\scriptsize \textcolor{Cerulean}{atmosphère}}
          \put (54, 102) {\scriptsize \textcolor{Cerulean}{côte}}
          \put (45, 88) {\scriptsize \textcolor{Cerulean}{brise}}
          \put (68.5, 96) {\scriptsize \textcolor{Cerulean}{légère}}
          \put (-4.5, 60) {\scriptsize \textcolor{Cerulean}{38\degre{}C}}
          \put (-6, 73.4) {\scriptsize \textcolor{Cerulean}{températures}}
          \put (6, 86) {\scriptsize \textcolor{Cerulean}{province}}
          \put (17, 96) {\scriptsize \textcolor{Cerulean}{exception}}
          \put (9.5, 65) {\scriptsize \textcolor{Cerulean}{vigilance}}
          \put (17.5, 76) {\scriptsize \textcolor{Cerulean}{ensemble}}
          \put (30, 86) {\scriptsize \textcolor{Cerulean}{jaune}}
          \put (23, 62) {\scriptsize \textcolor{Cerulean}{Luxembourg}}
          \put (36, 73) {\scriptsize \textcolor{Cerulean}{dimanche}}
          \put (43, 59) {\scriptsize \textcolor{Cerulean}{canicule}}
          \put (50, 70) {\scriptsize \textcolor{Cerulean}{Belgique}}
          \put (67, 76.5) {\scriptsize \textcolor{Cerulean}{mercure}}
          \put (65.5, 59) {\scriptsize \textcolor{Cerulean}{19}}
          \put (75, 65) {\scriptsize \textcolor{Cerulean}{\OE{}sling}}
          \put (80, 51) {\scriptsize \textcolor{Cerulean}{38\degre{}C}}
        \end{overpic}

        \vspace{.5em}
      \end{column}
    \end{columns}

    \vspace{.5em}

    \begin{align}
      S(\textcolor{Cerulean}{n_i}) &= (1 - \lambda) + \lambda \times \sum_{\textcolor{Cerulean}{n_j} \in \textcolor{termithgreen}{A(n_i)}} \frac{\text{poids}(\textcolor{termithgreen}{n_j, n_i}) \times S(\textcolor{Cerulean}{n_j})}{\sum_{\textcolor{Cerulean}{n_k} \in \textcolor{termithgreen}{A(n_j)}} \text{poids}(\textcolor{termithgreen}{n_j, n_k})} \notag
    \end{align}
  \end{frame}

  \begin{frame}{Ordonnancement des candidats}\framesubtitle{TextRank -- SingleRank}
    \begin{table}
      \centering
      \begin{tabular}{r|l|l}
        \toprule
        \textbf{Rang} & \multicolumn{1}{c|}{\textbf{TextRank}} & \multicolumn{1}{c}{\textbf{SingleRank}} \\
        \hline
        01 & \cellcolor{termithorange!30}{août 2012} & alerte orange \\
        02 & août 2003 & alerte jaune \\
        03 & alerte orange & alerte rouge \\
        04 & vigilance orange & \cellcolor{termithorange!30}{alerte} \\
        05 & deuxième nuit & deuxième nuit \\
        06 & légère brise & \cellcolor{termithorange!30}{août 2012} \\
        07 & & août 2003 \\
        08 & & vigilance orange \\
        09 & & légère brise \\
        10 & & \cellcolor{termithorange!30}{Luxembourg} \\
        \bottomrule
      \end{tabular}

      \caption{Termes-clés}
    \end{table}

    \begin{block}{Observations}
      \begin{itemize}
        \item{Faibles performances}
        \item{Termes-clés redondants}
      \end{itemize}
    \end{block}
  \end{frame}

\subsubsection{Classification des candidats}
  \begin{frame}{Indexation par termes-clés}\framesubtitle{Extraction de termes-clés}
    \begin{figure}
      \centering
      \begin{tikzpicture}[scale=.35, node distance=2cm]
        \node [io] (document) {document};
        \node [component, below=of document] (pretraitement_et_selection_des_candidats) {Sélection des termes-clés candidats};
        \node [component, yshift=-4cm, below=of pretraitement_et_selection_des_candidats] (ordonnancement) {Ordonnancement des candidats};
        \node [io, below=of ordonnancement, xshift=7.5cm] (termes_cles) {termes-clés};
        \path [->] (document) edge (pretraitement_et_selection_des_candidats);
        \path [->] (pretraitement_et_selection_des_candidats) edge (ordonnancement);
        \path [->] (ordonnancement) edge (termes_cles);
        %
        \node [component, right=of pretraitement_et_selection_des_candidats] (pretraitement_et_selection_des_candidats2) {Sélection des termes-clés candidats};
        \node [io, above=of pretraitement_et_selection_des_candidats2] (documents) {documents d'entraînement};
        \node [component, below=of pretraitement_et_selection_des_candidats2] (apprentissage) {Apprentissage d'un modèle de classification};
        \node [selectedcomponent, below=of apprentissage] (classification) {Classification des candidats};
        \path [->] (documents) edge (pretraitement_et_selection_des_candidats2);
        \path [->] (pretraitement_et_selection_des_candidats2) edge (apprentissage);
        \path [->] (apprentissage) edge (classification);
        \path [->] (classification) edge (termes_cles);
        %
        \draw [->] (pretraitement_et_selection_des_candidats) -- (ordonnancement) node [midway] (midway1) {};
        \draw [->] (pretraitement_et_selection_des_candidats) -- (classification.north west) node [midway] (midway2) {};
        \draw [dashed] (midway1) -- (midway2) node [midway, below, scale=.5] (xor) {\Large \{ou\}};
        %
        \draw [dashed] ($(pretraitement_et_selection_des_candidats2.north west)+(-.5cm,4.5cm)$) rectangle ($(apprentissage.south east)+(.5cm,-.5cm)$);
        \node [above=of documents, yshift=-1.8cm, scale=.5] (titre_apprentissage) {\Large Apprentissage supervisé};
      \end{tikzpicture}
    \end{figure}
  \end{frame}

  \begin{frame}{Extraction de termes-clés}\framesubtitle{Classification des candidats}
    \begin{block}{Objectif}
      Apprendre à reconnaître les termes-clés parmi les candidats
    \end{block}

    \vspace{1em}

    Classification des candidats selon divers critères (traits)~:
    \begin{itemize}
      \item{Fréquenciels~:}
      \begin{itemize}
        \item{\alt<2->{\textbf{TF-IDF~\cite[KEA]{witten1999kea}}}{TF-IDF~\cite[KEA]{witten1999kea}}}
      \end{itemize}
      \item{Positionnels~:}
      \begin{itemize}
        \item{\alt<2->{\textbf{Première position~\cite[KEA]{witten1999kea}}}{Première position~\cite[KEA]{witten1999kea}}}
        \item{Apparition dans une section
              particulière~\cite{nguyen2007keadocumentstructure}}
      \end{itemize}
      \item{Linguistiques~:}
      \begin{itemize}
        \item{Catégorie grammaticale~\cite{nguyen2007keadocumentstructure}}
      \end{itemize}
    \end{itemize}

    \vspace{1em}

    Classification à l'aide de techniques d'apprentissage automatique~:
    \begin{itemize}
      \item{\alt<2->{\textbf{Classification naîve bayesienne~\cite[KEA]{witten1999kea}}}{Classification naîve bayesienne~\cite[KEA]{witten1999kea}}}
      \item{Réseau de neurones~\cite{sarkar2010neuralnetwork}}
    \end{itemize}
  \end{frame}

%\subsubsection{Bilan}
%  \begin{frame}{Extraction de termes-clés}\framesubtitle{Bilan}
%    \begin{itemize}
%      \item{Large panel de méthodes}
%      \item{Sans ressources suplémentaires}
%      \item{Avec ressources supplémentaires}
%    \end{itemize}
%
%    \vspace{1em}
%
%    \begin{block}{Avantages}
%      \begin{itemize}
%        \item{Multitude de scénarii d'utilisation}
%        \item{Adaptation possible par apprentissage supervisé}
%      \end{itemize}
%    \end{block}
%
%    \begin{alertblock}{Inconvénients}
%      \begin{itemize}
%        \item{Impossibilité de trouver tous les termes-clés}
%        \item{Extraction de termes-clés parfois mal formulés}
%        \item{Dépendence vis-à-vis des termes-clés candidats peu étudiée}
%      \end{itemize}
%    \end{alertblock}
%  \end{frame}

