\begin{frame}{Indexation par termes-clés}\framesubtitle{Extraction de termes-clés}
  \begin{figure}
    \begin{tikzpicture}[scale=.3, node distance=2cm]
      \uncover<1->{
        \node [io] (document) {document};
      }
      \uncover<2->{
        \node [component] (preprocessing) [below=of document] {Prétraitement linguistique};
      }
      \uncover<3->{
        %\alt<9>{
        %  \node [selectedcomponent] (candidate_extraction) [below=of preprocessing] {Sélection des termes-clés candidats};
        %}{
          \node [component] (candidate_extraction) [below=of preprocessing] {Sélection des termes-clés candidats};
        %}
      }
      \uncover<4->{
        %\alt<9>{
        %  \node [selectedcomponent, yshift=-4cm] (candidate_classification_and_ranking) [below=of candidate_extraction] {Ordonnancement des candidats};
        %}{
          \node [component, yshift=-4cm] (candidate_classification_and_ranking) [below=of candidate_extraction] {Ordonnancement des candidats};
        %}
      }
      \uncover<7->{
        \node [component, minimum width=24cm, xshift=6.5cm] (keyphrase_selection) [below=of candidate_classification_and_ranking] {Sélection des termes-clés à extraire};
      }
      \uncover<8->{
        \node [io] (keyphrases) [below=of keyphrase_selection] {termes-clés};
      }
      %
      \uncover<6->{
        \node [component] (preprocessing2) [right=of preprocessing] {Prétraitement linguistique};
        \node [component] (candidate_extraction2) [right=of candidate_extraction] {Sélection des termes-clés candidats};
      }
      \uncover<5->{
        \node [component] (classification) [right=of candidate_classification_and_ranking] {Classification des candidats};
      }
      \uncover<6->{
        \node [component] (learning) [below=of candidate_extraction2]
        {Apprentissage du modèle de classification};
        \node [io] (documents) [above=of preprocessing2] {documents d'apprentissage};
      }

      \uncover<2->{
        \path [->] (document) edge (preprocessing);
      }
      \uncover<3->{
        \path [->] (preprocessing) edge (candidate_extraction);
      }
      \uncover<7->{
        \path [->] (candidate_classification_and_ranking) edge (keyphrase_selection);
      }
      \uncover<8->{
        \path [->] (keyphrase_selection) edge (keyphrases);
      }
      %
      \uncover<6->{
        \path [->] (documents) edge (preprocessing2);
        \path [->] (preprocessing2) edge (candidate_extraction2);
        \path [->] (candidate_extraction2) edge (learning);
        \path [->] (learning) edge (classification);
      }
      \uncover<7->{
        \path [->] (classification) edge (keyphrase_selection);
      }
      %
      \uncover<4->{
        \draw [->] (candidate_extraction) -- (candidate_classification_and_ranking) node [midway] (midway1) {};
      }
      \uncover<5->{
        \draw [->] (candidate_extraction) -- (classification.north west) node [midway] (midway2) {};
        \draw [dashed] (midway1) -- (midway2) node [midway, below, scale=.5] (xor) {\{ou\}};
      }

      \uncover<6->{
        \draw [dashed] ($(preprocessing2.north west)+(-.5cm,4.5cm)$) rectangle ($(learning.south east)+(.5cm,-.5cm)$);
        \node [above=of documents, yshift=-1.85cm, scale=.5] (apprentissage) {Apprentissage};
      }
    \end{tikzpicture}
  \end{figure}
\end{frame}

\subsubsection{Sélection des termes-clés candidats}
  \begin{frame}{Extraction de termes-clés}\framesubtitle{Sélection des termes-clés candidats}
    \begin{block}<1->{Terme-clé candidat}
      Unité textuelle qui possède des caractéristiques (linguistiques) d'un
      terme-clé.
    \end{block}

    \vspace{1em}

    \uncover<2->{
      Trois méthodes de sélection classiques~:
      \begin{itemize}
        \item<2->{$N$-grammes ($N = 1..3$)}
        \item<7->{Syntagmes nominaux}
        \item<12->{Séquences grammaticalement définies}
        \begin{itemize}
          \item{Séquences \texttt{NOM}~/~\texttt{ADJ} (générique)}
          \item{Séquences observées (spécifique)}
        \end{itemize}
      \end{itemize}
    }

    \vspace{1em}

    \uncover<15->{
      $\rightarrow$ Certaines méthodes sont $\pm$ exhaustives
    }

    \uncover<16->{
      $\Rightarrow$ Les résultats de l'extraction de termes-clés varient
    }

    \visible<3,4>{
      \begin{textblock*}{\textwidth}(0\textwidth, -.5\textheight)
        \setbeamertemplate{blocks}[rounded][shadow=true]

        \begin{exampleblock}{$3$-grammes}\small
          \og{}
            À l'exception de la province de Luxembourg, en alerte jaune,
            l'ensemble de la Belgique est en vigilance orange à la canicule.
          \fg{}

          \begin{multicols}{2}
            \begin{itemize}
              \item{\alt<4>{\sout{À l'exception}}{À l'exception}}
              \item{\alt<4>{\sout{l'exception de}}{l'exception de}}
              \item{\alt<4>{\sout{exception de la}}{exception de la}}
              \item{\alt<4>{\sout{de la province}}{de la province}}
              \item{\alt<4>{\sout{la province de}}{la province de}}
              \item{province de Luxembourg}
              \item{\alt<4>{\sout{en alerte jaune}}{en alerte jaune}}
              \item{\alt<4>{\sout{l'ensemble de}}{l'ensemble de}}
              \item{\alt<4>{\sout{ensemble de la}}{ensemble de la}}
              \item{\alt<4>{\sout{de la Belgique}}{de la Belgique}}
              \item{\alt<4>{\sout{la Belgique est}}{la Belgique est}}
              \item{\alt<4>{\sout{Belgique est en}}{Belgique est en}}
            \end{itemize}
          \end{multicols}
        \end{exampleblock}
      \end{textblock*}
    }

    \visible<5>{
      \begin{textblock*}{\textwidth}(0\textwidth, -.5\textheight)
        \setbeamertemplate{blocks}[rounded][shadow=true]

        \begin{exampleblock}{$\{1..3\}$-grammes}\small
          \og{}
            À l'exception de la province de Luxembourg, en alerte jaune,
            l'ensemble de la Belgique est en vigilance orange à la canicule.
          \fg{}

          \begin{multicols}{2}
            \begin{itemize}
              \item{exception}
              \item{province}
              \item{Luxembourg}
              \item{alerte}
              \item{jaune}
              \item{ensemble}
              \item{Belgique}
              \item{vigilance}
              \item{orange}
              \item{canicule}
              \item{alerte jaune}
              \item{vigilance orange}
              \item{province de Luxembourg}
            \end{itemize}
          \end{multicols}
        \end{exampleblock}
      \end{textblock*}
    }

    \visible<8,9,10>{
      \begin{textblock*}{\textwidth}(0\textwidth, -.5\textheight)
        \setbeamertemplate{blocks}[rounded][shadow=true]

        \begin{exampleblock}{Syntagmes nominaux}\small
          \og{}
            À l'exception de la province de Luxembourg, en alerte jaune,
            l'ensemble de la Belgique est en vigilance orange à la canicule.
          \fg{}

          \begin{multicols}{2}
            \begin{itemize}
              \item{\alt<9->{\sout{l'}}{l'}\alt<10>{[exception]}{exception} de
                    la \alt<10>{[province]}{province} de
                    \alt<10>{[Luxembourg]}{Luxembourg}}
              \item{alerte jaune}
              \item{\alt<9->{\sout{l'}}{l'}\alt<10>{[ensemble]}{ensemble} de la
                \alt<10>{[Belgique]}{Belgique}}
              \item{vigilance orange}
              \item{\alt<9->{\sout{la}}{la} canicule}
            \end{itemize}
          \end{multicols}
        \end{exampleblock}
      \end{textblock*}
    }

    \visible<13>{
      \begin{textblock*}{\textwidth}(0\textwidth, -.5\textheight)
        \setbeamertemplate{blocks}[rounded][shadow=true]

      \begin{exampleblock}{Syntagmes nominaux}\small
          \og{}
            À l'exception de la province de Luxembourg, en alerte jaune,
            l'ensemble de la Belgique est en vigilance orange à la canicule.
          \fg{}

          \begin{multicols}{2}
            \begin{itemize}
              \item{exception}
              \item{province}
              \item{Luxembourg}
              \item{alerte jaune}
              \item{ensemble}
              \item{Belgique}
              \item{vigilance orange}
              \item{canicule}
            \end{itemize}
          \end{multicols}
        \end{exampleblock}
      \end{textblock*}
    }
  \end{frame}

\subsubsection{Extraction non supervisée}
  \begin{frame}{Extraction de termes-clés}\framesubtitle{Extraction non supervisée}
    \begin{block}<+->{Objectif}
      Déterminer les termes-clés candidats les plus importants.
    \end{block}

    \uncover<+->{
      Ordonnancement des candidats selon diverses mesures d'importance~:
      \begin{itemize}
        \item{}
        \item{}
        \item{}
        \item{}
        \item{}
      \end{itemize}
    }
  \end{frame}

\subsubsection{Extraction supervisée}
  \begin{frame}{Extraction de termes-clés}\framesubtitle{Extraction supervisée}
    \begin{block}<+->{Objectif}
      Reconnaître les termes-clés parmi les termes-clés candidats.
    \end{block}

    \uncover<+->{
      Classification des candidats selon divers critères (traits)~:
      \begin{itemize}
        \item{}
        \item{}
        \item{}
        \item{}
        \item{}
      \end{itemize}
    }
  \end{frame}

\subsubsection{Bilan}
  \begin{frame}{Extraction de termes-clés}\framesubtitle{Bilan}
    \begin{block}{Avantages}
    \end{block}

    \begin{alertblock}{Inconvénients}
    \end{alertblock}
    
  \end{frame}

