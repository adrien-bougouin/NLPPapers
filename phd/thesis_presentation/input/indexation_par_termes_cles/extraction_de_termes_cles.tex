\begin{frame}{Indexation par termes-clés}\framesubtitle{Extraction de termes-clés}
  \begin{figure}
    \centering
    \begin{tikzpicture}[scale=.35, node distance=2cm]
      \node [io] (document) {document};
      \node [component, below=of document] (pretraitement_et_selection_des_candidats) {Sélection de termes-clés candidats};
      \node [component, yshift=-4cm, below=of pretraitement_et_selection_des_candidats] (ordonnancement) {Ordonnancement des candidats};
      \node [io, below=of ordonnancement, xshift=6cm] (termes_cles) {termes-clés};
      \path [->] (document) edge (pretraitement_et_selection_des_candidats);
      \path [->] (pretraitement_et_selection_des_candidats) edge (ordonnancement);
      \path [->] (ordonnancement) edge (termes_cles);
      %
      \node [component, right=of pretraitement_et_selection_des_candidats] (pretraitement_et_selection_des_candidats2) {Sélection de termes-clés candidats};
      \node [io, above=of pretraitement_et_selection_des_candidats2] (documents) {documents d'apprentissage};
      \node [component, below=of pretraitement_et_selection_des_candidats2] (apprentissage) {Apprentissage d'un modèle de classification};
      \node [component, below=of apprentissage] (classification) {Classification des candidats};
      \path [->] (documents) edge (pretraitement_et_selection_des_candidats2);
      \path [->] (pretraitement_et_selection_des_candidats2) edge (apprentissage);
      \path [->] (apprentissage) edge (classification);
      \path [->] (classification) edge (termes_cles);
      %
      \draw [->] (pretraitement_et_selection_des_candidats) -- (ordonnancement) node [midway] (midway1) {};
      \draw [->] (pretraitement_et_selection_des_candidats) -- (classification.north west) node [midway] (midway2) {};
      \draw [dashed] (midway1) -- (midway2) node [midway, below, scale=.5] (xor) {\Large \{ou\}};
      %
      \draw [dashed] ($(pretraitement_et_selection_des_candidats2.north west)+(-.5cm,4.5cm)$) rectangle ($(apprentissage.south east)+(.5cm,-.5cm)$);
      \node [above=of documents, yshift=-1.8cm, scale=.5] (titre_apprentissage) {\Large Apprentissage supervisé};
    \end{tikzpicture}
  \end{figure}
\end{frame}

\subsubsection{Prétraitements\dots}
  \begin{frame}{Indexation par termes-clés}\framesubtitle{Extraction de termes-clés}
    \begin{figure}
      \centering
      \begin{tikzpicture}[scale=.35, node distance=2cm]
        \node [io] (document) {document};
        \node [selectedcomponent, below=of document] (pretraitement_et_selection_des_candidats) {Sélection de termes-clés candidats};
        \node [component, yshift=-4cm, below=of pretraitement_et_selection_des_candidats] (ordonnancement) {Ordonnancement des candidats};
        \node [io, below=of ordonnancement, xshift=6cm] (termes_cles) {termes-clés};
        \path [->] (document) edge (pretraitement_et_selection_des_candidats);
        \path [->] (pretraitement_et_selection_des_candidats) edge (ordonnancement);
        \path [->] (ordonnancement) edge (termes_cles);
        %
        \node [selectedcomponent, right=of pretraitement_et_selection_des_candidats] (pretraitement_et_selection_des_candidats2) {Sélection de termes-clés candidats};
        \node [io, above=of pretraitement_et_selection_des_candidats2] (documents) {documents d'apprentissage};
        \node [component, below=of pretraitement_et_selection_des_candidats2] (apprentissage) {Apprentissage d'un modèle de classification};
        \node [component, below=of apprentissage] (classification) {Classification des candidats};
        \path [->] (documents) edge (pretraitement_et_selection_des_candidats2);
        \path [->] (pretraitement_et_selection_des_candidats2) edge (apprentissage);
        \path [->] (apprentissage) edge (classification);
        \path [->] (classification) edge (termes_cles);
        %
        \draw [->] (pretraitement_et_selection_des_candidats) -- (ordonnancement) node [midway] (midway1) {};
        \draw [->] (pretraitement_et_selection_des_candidats) -- (classification.north west) node [midway] (midway2) {};
        \draw [dashed] (midway1) -- (midway2) node [midway, below, scale=.5] (xor) {\Large \{ou\}};
        %
        \draw [dashed] ($(pretraitement_et_selection_des_candidats2.north west)+(-.5cm,4.5cm)$) rectangle ($(apprentissage.south east)+(.5cm,-.5cm)$);
        \node [above=of documents, yshift=-1.8cm, scale=.5] (titre_apprentissage) {\Large Apprentissage supervisé};
      \end{tikzpicture}
    \end{figure}
  \end{frame}

  \begin{frame}{Extraction de termes-clés}\framesubtitle{Sélection des termes-clés candidats}
    Préparation préalable du document~:
    \begin{enumerate}
      \item{Segmentation en phrases}
      \item{Segmentation en mots}
      \item{Étiquetage grammatical}
    \end{enumerate}

    \vspace{1em}

    Sélection des termes-clés candidats~:
    \begin{itemize}
      \item{$N$-grammes}
      \item{Reconnaissance de forme}
    \end{itemize}
  \end{frame}

\subsubsection{Ordonancement des candidats}
  \begin{frame}{Indexation par termes-clés}\framesubtitle{Extraction de termes-clés}
    \begin{figure}
      \centering
      \begin{tikzpicture}[scale=.35, node distance=2cm]
        \node [io] (document) {document};
        \node [component, below=of document] (pretraitement_et_selection_des_candidats) {Sélection de termes-clés candidats};
        \node [selectedcomponent, yshift=-4cm, below=of pretraitement_et_selection_des_candidats] (ordonnancement) {Ordonnancement des candidats};
        \node [io, below=of ordonnancement, xshift=6cm] (termes_cles) {termes-clés};
        \path [->] (document) edge (pretraitement_et_selection_des_candidats);
        \path [->] (pretraitement_et_selection_des_candidats) edge (ordonnancement);
        \path [->] (ordonnancement) edge (termes_cles);
        %
        \node [component, right=of pretraitement_et_selection_des_candidats] (pretraitement_et_selection_des_candidats2) {Sélection de termes-clés candidats};
        \node [io, above=of pretraitement_et_selection_des_candidats2] (documents) {documents d'apprentissage};
        \node [component, below=of pretraitement_et_selection_des_candidats2] (apprentissage) {Apprentissage d'un modèle de classification};
        \node [component, below=of apprentissage] (classification) {Classification des candidats};
        \path [->] (documents) edge (pretraitement_et_selection_des_candidats2);
        \path [->] (pretraitement_et_selection_des_candidats2) edge (apprentissage);
        \path [->] (apprentissage) edge (classification);
        \path [->] (classification) edge (termes_cles);
        %
        \draw [->] (pretraitement_et_selection_des_candidats) -- (ordonnancement) node [midway] (midway1) {};
        \draw [->] (pretraitement_et_selection_des_candidats) -- (classification.north west) node [midway] (midway2) {};
        \draw [dashed] (midway1) -- (midway2) node [midway, below, scale=.5] (xor) {\Large \{ou\}};
        %
        \draw [dashed] ($(pretraitement_et_selection_des_candidats2.north west)+(-.5cm,4.5cm)$) rectangle ($(apprentissage.south east)+(.5cm,-.5cm)$);
        \node [above=of documents, yshift=-1.8cm, scale=.5] (titre_apprentissage) {\Large Apprentissage supervisé};
      \end{tikzpicture}
    \end{figure}
  \end{frame}

  \begin{frame}{Extraction de termes-clés}\framesubtitle{Ordonnancement des candidats}
    \begin{block}{Objectif}
      Déterminer les termes-clés candidats les plus importants.
    \end{block}

    \vspace{1em}

    Ordonnancement des candidats selon diverses approches~:
    \begin{itemize}
      \item{Statistiques~\cite[TF-IDF]{salton1975tfidf}}
      \item{Par groupement sémantique~\cite[KeyCluster]{liu2009keycluster}}
      \item{\textbf{À base de graphe}~\cite[TextRank]{mihalcea2004textrank}}
    \end{itemize}
  \end{frame}

  \begin{frame}{Ordonnancement des candidats}\framesubtitle{TextRank -- SingleRank}
    \begin{exampleblock}{\small
      Météo du 19 août 2012~: alerte à la canicule sur la Belgique et le
      Luxembourg
    }\justifying\small
      ~~~À l'exception de la province de Luxembourg, en alerte jaune, l'ensemble
      de la Belgique est en vigilance orange à la canicule. Le Luxembourg n'est
      pas épargné par la vague du chaleur : le nord du pays est en alerte
      orange, tandis que le sud a était placé en alerte rouge.

      ~~~En Belgique, la température n'est pas descendue en dessous des
      23\degre{}C cette nuit, ce qui constitue la deuxième nuit la plus chaude
      jamais enregistrée dans le royaume. Il se pourrait que ce dimanche soit la
      journée la plus chaude de l'année. Les températures seront comprises entre
      33 et 38\degre{}C. Une légère brise de côte pourra faiblement rafraichir
      l'atmosphère. Des orages de chaleur sont a prévoir dans la soirée et en
      début de nuit.

      ~~~Au Luxembourg, le mercure devrait atteindre 32\degre{}C ce dimanche sur
      l'Oesling et jusqu'à 36\degre{}C sur le sud du pays, et 31 à 32\degre{}C
      lundi. Une baisse devrait intervenir pour le reste de la semaine.
      Néanmoins, le record d'août 2003 (37,9\degre{}C) ne devrait pas être
      atteint.

      \begin{exampleblock}{\small Termes-clés}\justifying\small
        Août 2012~; alerte~; canicule~; Belgique~; Luxembourg~; orange~;
        chaleur~; température~; chaude~; la plus chaude.
      \end{exampleblock}
    \end{exampleblock}
  \end{frame}

  \begin{frame}{Ordonnancement des candidats}\framesubtitle{TextRank -- SingleRank}
    \begin{columns}[t]
      \begin{column}{.5\linewidth}\centering
        TextRank~\cite{mihalcea2004textrank}

        \vspace{2em}

        \begin{overpic}[width=.95\linewidth]{include/44960_textrank_graph.eps}
        \end{overpic}

        \vspace{2em}

        TODO formule
      \end{column}

      \begin{column}{.5\linewidth}\centering
        SingleRank~\cite{wan2008expandrank}

        \vspace{2em}

        \begin{overpic}[width=\linewidth]{include/44960_singlerank_graph.eps}
        \end{overpic}

        \vspace{2em}

        TODO formule
      \end{column}
    \end{columns}
  \end{frame}

  \begin{frame}{Ordonnancement des candidats}\framesubtitle{TextRank -- SingleRank}
    \begin{table}
      \centering
      \begin{tabular}{r|l|l}
        \toprule
        \textbf{Rang} & \multicolumn{1}{c|}{\textbf{TextRank}} & \multicolumn{1}{c}{\textbf{SingleRank}} \\
        \hline
        01 & \cellcolor{termithorange!30}{août 2012} & alerte orange \\
        02 & août 2003 & alerte jaune \\
        03 & alerte orange & alerte rouge \\
        04 & vigilance orange & \cellcolor{termithorange!30}{alerte} \\
        05 & deuxième nuit & deuxième nuit \\
        06 & légère brise & \cellcolor{termithorange!30}{août 2012} \\
        07 & & août 2003 \\
        08 & & vigilance orange \\
        09 & & légère brise \\
        10 & & \cellcolor{termithorange!30}{Luxembourg} \\
        \bottomrule
      \end{tabular}
    \end{table}
  \end{frame}

\subsubsection{Classification des candidats}
  \begin{frame}{Indexation par termes-clés}\framesubtitle{Extraction de termes-clés}
    \begin{figure}
      \centering
      \begin{tikzpicture}[scale=.35, node distance=2cm]
        \node [io] (document) {document};
        \node [component, below=of document] (pretraitement_et_selection_des_candidats) {Sélection de termes-clés candidats};
        \node [component, yshift=-4cm, below=of pretraitement_et_selection_des_candidats] (ordonnancement) {Ordonnancement des candidats};
        \node [io, below=of ordonnancement, xshift=6cm] (termes_cles) {termes-clés};
        \path [->] (document) edge (pretraitement_et_selection_des_candidats);
        \path [->] (pretraitement_et_selection_des_candidats) edge (ordonnancement);
        \path [->] (ordonnancement) edge (termes_cles);
        %
        \node [component, right=of pretraitement_et_selection_des_candidats] (pretraitement_et_selection_des_candidats2) {Sélection de termes-clés candidats};
        \node [io, above=of pretraitement_et_selection_des_candidats2] (documents) {documents d'apprentissage};
        \node [component, below=of pretraitement_et_selection_des_candidats2] (apprentissage) {Apprentissage d'un modèle de classification};
        \node [selectedcomponent, below=of apprentissage] (classification) {Classification des candidats};
        \path [->] (documents) edge (pretraitement_et_selection_des_candidats2);
        \path [->] (pretraitement_et_selection_des_candidats2) edge (apprentissage);
        \path [->] (apprentissage) edge (classification);
        \path [->] (classification) edge (termes_cles);
        %
        \draw [->] (pretraitement_et_selection_des_candidats) -- (ordonnancement) node [midway] (midway1) {};
        \draw [->] (pretraitement_et_selection_des_candidats) -- (classification.north west) node [midway] (midway2) {};
        \draw [dashed] (midway1) -- (midway2) node [midway, below, scale=.5] (xor) {\Large \{ou\}};
        %
        \draw [dashed] ($(pretraitement_et_selection_des_candidats2.north west)+(-.5cm,4.5cm)$) rectangle ($(apprentissage.south east)+(.5cm,-.5cm)$);
        \node [above=of documents, yshift=-1.8cm, scale=.5] (titre_apprentissage) {\Large Apprentissage supervisé};
      \end{tikzpicture}
    \end{figure}
  \end{frame}

  \begin{frame}{Extraction de termes-clés}\framesubtitle{Classification des candidats}
    \begin{block}{Objectif}
      Apprendre à reconnaître les termes-clés parmi les termes-clés candidats.
    \end{block}

    \vspace{1em}

    Classification des candidats selon divers critères (traits)~:
    \begin{itemize}
      \item{Fréquenciels~:}
      \begin{itemize}
        \item{TF-IDF~\cite[KEA]{witten1999kea}}
      \end{itemize}
      \item{Positionnels~:}
      \begin{itemize}
        \item{Première position~\cite[KEA]{witten1999kea}}
        \item{Apparition dans une section
              particulière~\cite{nguyen2007keadocumentstructure}}
      \end{itemize}
      \item{Linguistiques~:}
      \begin{itemize}
        \item{Catégorie grammaticale}
      \end{itemize}
    \end{itemize}

    \vspace{1em}

    Et divers méthodes d'apprentissage~:
    \begin{itemize}
      \item{Classification naîve bayesienne~\cite[KEA]{witten1999kea}}
      \item{Entraînement d'un réseau de neurones~\cite{sarkar2010neuralnetwork}}
    \end{itemize}
  \end{frame}

%\subsubsection{Bilan}
%  \begin{frame}{Extraction de termes-clés}\framesubtitle{Bilan}
%    \begin{itemize}
%      \item{Large panel de méthodes}
%      \item{Sans ressources suplémentaires}
%      \item{Avec ressources supplémentaires}
%    \end{itemize}
%
%    \vspace{1em}
%
%    \begin{block}{Avantages}
%      \begin{itemize}
%        \item{Multitude de scénarii d'utilisation}
%        \item{Adaptation possible par apprentissage supervisé}
%      \end{itemize}
%    \end{block}
%
%    \begin{alertblock}{Inconvénients}
%      \begin{itemize}
%        \item{Impossibilité de trouver tous les termes-clés}
%        \item{Extraction de termes-clés parfois mal formulés}
%        \item{Dépendence vis-à-vis des termes-clés candidats peu étudiée}
%      \end{itemize}
%    \end{alertblock}
%  \end{frame}

