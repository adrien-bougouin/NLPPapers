\begin{frame}{Introduction}\framesubtitle{Contexte}
  Accès à l'information des documents numériques

  \vspace{1em}

  \begin{figure}
    \begin{overpic}[width=\linewidth]{include/search.eps}
      \put (1, 3) {\huge \textcolor{black!50}{Recherche}}
    \end{overpic}
  \end{figure}
\end{frame}

\begin{frame}{Introduction}\framesubtitle{Contexte}
  Représentation de l'information numérique en domaines de spécialité

  \vspace{1em}

  \begin{block}{Termes-clés (mots-clés)}
    \begin{itemize}
      \item{Unités textuelles (mots et expressions)}
      \item{Décrivent le contenu principal d'un document}
      \item{Utiles pour la Recherche d'Information (RI)~:}
      \begin{itemize}
        \item{Indexation de document}
        \item{Expansion de requête}
        \item{Résumé automatique}
      \end{itemize}
    \end{itemize}
  \end{block}
\end{frame}

%\begin{frame}{Introduction}\framesubtitle{Contexte -- Exemple 1}
%  \begin{exampleblock}{\small
%    Météo du 19 \textbf{août 2012}~: \textbf{alerte} à la
%    \textbf{canicule} sur la \textbf{Belgique} et le
%    \textbf{Luxembourg}
%  }\justifying\small
%    ~~~À l'exception de la province de \textbf{Luxembourg}, en
%    \textbf{alerte} jaune, l'ensemble de la \textbf{Belgique} est en
%    vigilance \textbf{orange} à la \textbf{canicule}. Le
%    \textbf{Luxembourg} n'est pas épargné par la vague du \textbf{chaleur}
%    : le nord du pays est en \textbf{alerte} \textbf{orange}, tandis que
%    le sud a était placé en \textbf{alerte} rouge.
%
%    ~~~En \textbf{Belgique}, la \textbf{température} n'est pas descendue
%    en dessous des 23\degre{}C cette nuit, ce qui constitue la deuxième nuit
%    \textbf{la plus chaude} jamais enregistrée dans le royaume. Il se
%    pourrait que ce dimanche soit la journée \textbf{la plus chaude} de
%    l'année. Les \textbf{températures} seront comprises entre 33 et
%    38\degre{}C. Une légère brise de côte pourra faiblement rafraichir
%    l'atmosphère. Des orages de \textbf{chaleur} sont a prévoir dans la
%    soirée et en début de nuit.
%
%    ~~~Au \textbf{Luxembourg}, le mercure devrait atteindre 32\degre{}C ce
%    dimanche sur l'Oesling et jusqu'à 36\degre{}C sur le sud du pays, et 31 à
%    32\degre{}C lundi. Une baisse devrait intervenir pour le reste de la
%    semaine. Néanmoins, le record d'août 2003 (37,9\degre{}C) ne devrait pas
%    être atteint.
%
%    \begin{exampleblock}{\small Termes-clés de référence}\justifying\small
%      \textbf{Août 2012}~; \textbf{canicule}~;
%      \textbf{Belgique}~; \textbf{Luxembourg}~; \textbf{alerte}~;
%      \textbf{orange}~; \textbf{chaleur}~; \textbf{chaude}~;
%      \textbf{température}~; \textbf{la plus chaude}
%    \end{exampleblock}
%  \end{exampleblock}
%\end{frame}
%
%\begin{frame}{Introduction}\framesubtitle{Contexte -- Exemple 2}
%  \vspace{-.33em}
%  \begin{exampleblock}{\small
%    Étude préliminaire de la \textbf{céramique non tournée}
%    \textbf{micacée} du bas Languedoc occidental~: \textbf{typologie},
%    \textbf{chronologie} et aire de \textbf{diffusion}
%  }\justifying\small
%    ~~~L'étude présente une variété de \textbf{céramique non tournée} dont la
%    \textbf{typologie} et l'analyse des \textbf{décors} permettent de
%    l'identifier facilement. La nature de l'argile enrichie de mica donne un
%    aspect pailleté à la pâte sur laquelle le \textbf{décor} effectué selon
%    la méthode du brunissoir apparaît en traits brillant sur fond mat. Cette
%    première approche se fonde sur deux séries issues de \textbf{fouilles
%    anciennes} menées sur les \textbf{oppidums} \textbf{du Cayla} à
%    \textbf{Mailhac} (\textbf{Aude}) et de \textbf{Mourrel-Ferrat} à
%    \textbf{Olonzac} (\textbf{Hérault}). La carte de
%    \textbf{répartition} fait état d'\textbf{échanges} ou de
%    \textbf{commerce} à l'échelon macrorégional rarement mis en évidence pour
%    de la \textbf{céramique non tournée}. S'il est difficile de statuer sur
%    l'origine des \textbf{décors}, il semble que la \textbf{production}
%    s'insère dans une ambiance celtisante. La \textbf{chronologie} de cette
%    \textbf{production} se situe dans le deuxième \textbf{âge du Fer}. La
%    fourchette proposée entre la fin du IV$^\text{e}$ et la fin du II$^\text{e}$
%    s. av. J.-C. reste encore à préciser.
%
%    \begin{exampleblock}{\small Termes-clés de référence}\justifying\small
%      \textbf{Mailhac}~; \textbf{Aude}~; \textbf{Mourrel-Ferrat}~;
%      \textbf{Olonzac}~; \textbf{Hérault}~; \textbf{céramique}~;
%      \textbf{typologie}~; \textbf{décor}~; \textbf{chronologie}~;
%      \textbf{diffusion}~; \textbf{production}~; \textbf{commerce}~;
%      \textbf{répartition}~; \textbf{oppidum}~; \textbf{analyse}~;
%      \textbf{fouille ancienne}~; \textbf{le Cayla}~;
%      \textbf{micassé}~; \textbf{céramique non-tournée}~;
%      \textbf{echange}~; \textbf{age du} \textbf{Fer}~; La Tène~;
%      Europe~; France~; celtes~; distribution~; cartographie~; habitat~; site
%      fortifié~; identification~; étude du matériel
%    \end{exampleblock}
%  \end{exampleblock}
%\end{frame}

