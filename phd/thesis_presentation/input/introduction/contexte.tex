\begin{frame}{Introduction}\framesubtitle{Contexte}
  Accès à l'information des documents numériques.

  \vspace{1em}

  \begin{figure}
    \begin{overpic}[width=\linewidth]{include/search.eps}
      \put (1, 3) {\huge \textcolor{black!50}{Recherche}}
    \end{overpic}
  \end{figure}
\end{frame}

\begin{frame}{Introduction}\framesubtitle{Contexte}
  Représentation de l'information numérique en domaines de spécialité.

  \vspace{1em}

  \begin{block}{Termes-clés (mots-clés)}
    \begin{itemize}
      \item{Unités textuelles (mots et expressions)}
      \item{Décrivent le contenu principal d'un document}
      \item{Utiles pour la Recherche d'information (\textsc{Ri})~:}
      \begin{itemize}
        \item{Indexation}
        \item{Expansion de requête}
        \item{Résumé automatique}
      \end{itemize}
    \end{itemize}
  \end{block}
\end{frame}

\begin{frame}{Introduction}\framesubtitle{Contexte -- Exemple 1}
  \begin{exampleblock}{\small
    Météo du 19 \underline{août 2012}~: \underline{alerte} à la
    \underline{canicule} sur la \underline{Belgique} et le
    \underline{Luxembourg}
  }\justifying\small
    ~~~À l'exception de la province de \underline{Luxembourg}, en
    \underline{alerte} jaune, l'ensemble de la \underline{Belgique} est en
    vigilance \underline{orange} à la \underline{canicule}. Le
    \underline{Luxembourg} n'est pas épargné par la vague du \underline{chaleur}
    : le nord du pays est en \underline{alerte} \underline{orange}, tandis que
    le sud a était placé en \underline{alerte} rouge.

    ~~~En \underline{Belgique}, la \underline{température} n'est pas descendue
    en dessous des 23\degre{}C cette nuit, ce qui constitue la deuxième nuit
    \underline{la plus chaude} jamais enregistrée dans le royaume. Il se
    pourrait que ce dimanche soit la journée \underline{la plus chaude} de
    l'année. Les \underline{températures} seront comprises entre 33 et
    38\degre{}C. Une légère brise de côte pourra faiblement rafraichir
    l'atmosphère. Des orages de \underline{chaleur} sont a prévoir dans la
    soirée et en début de nuit.

    ~~~Au \underline{Luxembourg}, le mercure devrait atteindre 32\degre{}C ce
    dimanche sur l'Oesling et jusqu'à 36\degre{}C sur le sud du pays, et 31 à
    32\degre{}C lundi. Une baisse devrait intervenir pour le reste de la
    semaine. Néanmoins, le record d'août 2003 (37,9\degre{}C) ne devrait pas
    être atteint.

    \begin{exampleblock}{\small Termes-clés}\justifying\small
      \underline{Août 2012}~; \underline{canicule}~;
      \underline{Belgique}~; \underline{Luxembourg}~; \underline{alerte}~;
      \underline{orange}~; \underline{chaleur}~; \underline{chaude}~;
      \underline{température}~; \underline{la plus chaude}
    \end{exampleblock}
  \end{exampleblock}
\end{frame}

\begin{frame}{Introduction}\framesubtitle{Contexte -- Exemple 2}
  \vspace{-.33em}
  \begin{exampleblock}{\small
    Étude préliminaire de la \underline{céramique non tournée}
    \underline{micacée} du bas Languedoc occidental~: \underline{typologie},
    \underline{chronologie} et aire de \underline{diffusion}
  }\justifying\small
    ~~~L'étude présente une variété de \underline{céramique non tournée} dont la
    \underline{typologie} et l'analyse des \underline{décors} permettent de
    l'identifier facilement. La nature de l'argile enrichie de mica donne un
    aspect pailleté à la pâte sur laquelle le \underline{décor} effectué selon
    la méthode du brunissoir apparaît en traits brillant sur fond mat. Cette
    première approche se fonde sur deux séries issues de \underline{fouilles
    anciennes} menées sur les \underline{oppidums} \underline{du Cayla} à
    \underline{Mailhac} (\underline{Aude}) et de \underline{Mourrel-Ferrat} à
    \underline{Olonzac} (\underline{Hérault}). La carte de
    \underline{répartition} fait état d'\underline{échanges} ou de
    \underline{commerce} à l'échelon macrorégional rarement mis en évidence pour
    de la \underline{céramique non tournée}. S'il est difficile de statuer sur
    l'origine des \underline{décors}, il semble que la \underline{production}
    s'insère dans une ambiance celtisante. La \underline{chronologie} de cette
    \underline{production} se situe dans le deuxième \underline{âge du Fer}. La
    fourchette proposée entre la fin du IV$^\text{e}$ et la fin du II$^\text{e}$
    s. av. J.-C. reste encore à préciser.

    \begin{exampleblock}{\small Termes-clés}\justifying\small
      \underline{Mailhac}~; \underline{Aude}~; \underline{Mourrel-Ferrat}~;
      \underline{Olonzac}~; \underline{Hérault}~; \underline{céramique}~;
      \underline{typologie}~; \underline{décor}~; \underline{chronologie}~;
      \underline{diffusion}~; \underline{production}~; \underline{commerce}~;
      \underline{répartition}~; \underline{oppidum}~; \underline{analyse}~;
      \underline{fouille ancienne}~; \underline{le Cayla}~;
      \underline{micassé}~; \underline{céramique non-tournée}~;
      \underline{echange}~; \underline{age du} \underline{Fer}~; La Tène~;
      Europe~; France~; celtes~; distribution~; cartographie~; habitat~; site
      fortifié~; identification~; étude du matériel
    \end{exampleblock}
  \end{exampleblock}
\end{frame}

