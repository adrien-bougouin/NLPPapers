\begin{frame}{Introduction}\framesubtitle{Contexte}
  \visible<+->{
    Accès à l'information numérique en domaines de spécialité.
  }

  \begin{block}<+->{Information numérique}
    \begin{itemize}
      \item{Savoir~/~Connaissance}
      \begin{itemize}
        \item{Intellectuel}
        \item{Social}
      \end{itemize}
      \item{Stockée virtuellement}
      \begin{itemize}
        \item{document électronique}
        \item{base de données électronique}
      \end{itemize}
    \end{itemize}
  \end{block}

  \begin{block}<+->{Domaine de spécialité}
    \begin{itemize}
      \item{Champ de connaissance particulier}
      \item{Possède un vocabulaire spécifique}
    \end{itemize}
  \end{block}
\end{frame}

\begin{frame}{Introduction}\framesubtitle{Contexte}
  \visible<+->{
    Valorisation de l'information numérique à l'Inist.
  }

  \begin{block}<+->{Inist (Institut de l'information scientifique et technique)}
    \begin{itemize}
      \item{Savoir faire documentaire}
      \item{Bases de données bibliographiques~:}
      \begin{itemize}
        \item{\textsc{Francis}~: Sciences humaines et sociales}
        \item{\textsc{Pascal}~: Sciences exactes}
      \end{itemize}
    \end{itemize}
  \end{block}

  \begin{block}<+->{Base de données bibliographiques}
    Collection de notices bibliographiques~:
    \begin{itemize}
      \item{Titre}
      \item{Auteur(s)}
      \item{Résumé}
      \item{Descripteurs~/~Termes-clés (mots-clés)}
    \end{itemize}
  \end{block}
\end{frame}

\begin{frame}{Introduction}\framesubtitle{Contexte}
  \begin{block}{Termes-clés}
    \begin{itemize}
      \item<1->{Unités textuelles (mots et expressions)}
      \item<1->{Décrivent le contenu d'un document}
      \item<2->{Utiles pour la Recherche d'information (\textsc{Ri})~:}
      \begin{itemize}
        \item{Indexation}
        \item{Expansion de requête}
        \item{Résumé automatique}
      \end{itemize}
      \item<3->{Utiles pour d'autres tâches~:}
      \begin{itemize}
        \item{Lecture rapide}
        \item{Aide aux dyslexiques}
      \end{itemize}
    \end{itemize}
  \end{block}
\end{frame}

\begin{frame}{Introduction}\framesubtitle{Contexte -- Exemple 1}
  \begin{exampleblock}{\small
    \textbf{\normalsize Météo} du 19 \textbf{\normalsize août 2012}~:
    \textbf{\normalsize alerte} à la \textbf{\normalsize canicule} sur la
    \textbf{\normalsize Belgique} et le \textbf{\normalsize Luxembourg}
  }\justifying\small
    ~~~À l'exception de la province de \textbf{\normalsize Luxembourg}, en
    \textbf{\normalsize alerte} jaune, l'ensemble de la \textbf{\normalsize
    Belgique} est en vigilance \textbf{\normalsize orange} à la
    \textbf{\normalsize canicule}. Le \textbf{\normalsize Luxembourg} n'est pas
    épargné par la vague du \textbf{\normalsize chaleur} : le nord du pays est
    en \textbf{\normalsize alerte} \textbf{\normalsize orange}, tandis que le
    sud a était placé en \textbf{\normalsize alerte} rouge.

    ~~~En \textbf{\normalsize Belgique}, la \textbf{\normalsize température}
    n'est pas descendue en dessous des 23\degre{}C cette nuit, ce qui constitue
    la deuxième nuit la plus \textbf{\normalsize chaude} jamais enregistrée dans
    le royaume. Il se pourrait que ce dimanche soit la journée
    \textbf{\normalsize la plus chaude} de l'année. Les \textbf{\normalsize
    températures} seront comprises entre 33 et 38\degre{}C. Une légère brise de
    côte pourra faiblement rafraichir l'atmosphère. Des orages de
    \textbf{\normalsize chaleur} sont a prévoir dans la soirée et en début de
    nuit.

    ~~~Au \textbf{\normalsize Luxembourg}, le mercure devrait atteindre
    32\degre{}C ce dimanche sur l'Oesling et jusqu'à 36\degre{}C sur le sud du
    pays, et 31 à 32\degre{}C lundi. Une baisse devrait intervenir pour le reste
    de la semaine. Néanmoins, le record d'août 2003 (37,9\degre{}C) ne devrait
    pas être atteint.
  \end{exampleblock}
\end{frame}

\begin{frame}{Introduction}\framesubtitle{Contexte -- Exemple 2}
  \begin{exampleblock}{\small
    Étude préliminaire de la \textbf{\normalsize céramique non tournée}
    \textbf{\normalsize micacée} du bas Languedoc occidental~:
    \textbf{\normalsize typologie}, \textbf{\normalsize chronologie} et aire de
    diffusion
  }\justifying\small
    ~~~L'étude présente une variété de \textbf{\normalsize céramique non tournée}
    dont la \textbf{\normalsize typologie} et l'\textbf{\normalsize analyse} des
    \textbf{\normalsize décors} permettent de l'identifier facilement. La nature
    de l'argile enrichie de mica donne un aspect pailleté à la pâte sur laquelle
    le \textbf{\normalsize décor} effectué selon la méthode du brunissoir
    apparaît en traits brillant sur fond mat. Cette première approche se fonde
    sur deux séries issues de \textbf{\normalsize fouilles} anciennes menées sur
    les \textbf{\normalsize oppidums} du \textbf{\normalsize Cayla} à
    \textbf{\normalsize Mailhac} (\textbf{\normalsize Aude}) et de
    \textbf{\normalsize Mourrel-Ferrat} à \textbf{\normalsize Olonzac}
    (\textbf{\normalsize Hérault}). La carte de \textbf{\normalsize répartition}
    fait état d'\textbf{normalsize échanges} ou de \textbf{\normalsize commerce}
    à l'échelon macrorégional rarement mis en évidence pour de la
    \textbf{\normalsize céramique non tournée}. S'il est difficile de statuer
    sur l'origine des \textbf{\normalsize décors}, il semble que la
    \textbf{\normalsize production} s'insère dans une ambiance celtisante. La
    \textbf{\normalsize chronologie} de cette \textbf{\normalsize production} se
    situe dans le deuxième \textbf{\normalsize âge du Fer}. La fourchette
    proposée entre la fin du IV$^\text{e}$ et la fin du II$^\text{e}$ s. av.
    J.-C. reste encore à préciser.
  \end{exampleblock}
\end{frame}

