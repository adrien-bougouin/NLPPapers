\begin{frame}[label=selection_des_candidats]{Sélection des termes-clés candidats}\framesubtitle{Analyse linguistique}
  Approche~:
  \begin{itemize}
    \item{Analyse de trois collections de documents}
    \begin{itemize}
      \item{\textsc{De}ft \textit{(fr)}}
      \item{SemEval \textit{(en)}}
      \item{\textsc{Duc} \textit{(en)}}
    \end{itemize}
    \item{Analyse en surface}
    \begin{itemize}
      \item{Taille d'un termes-clés~?}
      \item{Classes grammaticales des mots d'un termes-clés~?}
    \end{itemize}
    \item{Analyse plus approfondie}
    \begin{itemize}
      \item{Quelles classes grammaticales sont intéressantes~?}
      \item{Pouvons-nous affiner la sélection à partir de ces classes~?}
    \end{itemize}
  \end{itemize}
\end{frame}
\begin{frame}{Sélection des termes-clés candidats}\framesubtitle{Analyse linguistique}
  Analyse en surface~:

  \begin{center}
    \resizebox{.79\linewidth}{!}{
      \begin{tabular}{ll|ccc}
        \toprule
        & & \textbf{\textsc{De}ft} \textit{(fr)} & \textbf{SemEval} \textit{(en)} & \textbf{\textsc{Duc}} \textit{(en)}\\
        \hline
        \multicolumn{2}{l|}{\textbf{Taux (en \%) de termes-clés~:}}\\
        & Uni-grammes & 60,2 & 20,2 & 17,1\\
        & Bi-grammes & 24,5 & 53,4 & 60,8\\
        & Tri-grammes & 8,8 & 21,3 & 17,8\\
        \hline
        \multicolumn{2}{l|}{\textbf{Taux (en \%) de termes-clés}} & & &\\
        \multicolumn{2}{l|}{\textbf{contenant au moins un(e)~:}} & & &\\
        & Nom commun & 93,1 & 98,7 & 94,5\\
        & Nom propre & $~~$6,9 & $~~$4,3 & 17,1\\
        & Adjectif & 65,5 & 50,2 & 50,0\\
        & Verbe & $~~$1,0 & $~~$4,0 & $~~$1,0\\
        & Adverbe & $~~$1,3 & $~~$0,7 & $~~$1,6\\
        & Préposition & 31,2 & $~~$1,5 & $~~$0,3\\
        & Déterminant & 20,4 & $~~$0,0 & $~~$0,0\\
        \bottomrule
      \end{tabular}
    }
  \end{center}

  \begin{block}{Observations}
    \begin{itemize}
      \item{Termes-clés rarement composés de plus de trois mots}
      \item{Nom omniprésent dans les termes-clés}
      \item{\textbf{Adjectif très utilisé (modification du nom)}}
      \item{Prépositions et déterminants spécifiques au français}
    \end{itemize}
  \end{block}
\end{frame}

\begin{frame}[t]{Sélection des termes-clés candidats}\framesubtitle{Analyse linguistique}
  Analyse des adjectifs~:

  \begin{block}{Adjectif relationnel}
    \begin{itemize}
      \item{Dérivé d'un nom}
      \begin{itemize}
        \item{\og{}culture\fg{} $\rightarrow$ \og{}culturel\fg{}}
      \end{itemize}
      \item{Établi une relation avec le nom dérivé}
      \begin{itemize}
        \item{\og{}héritage culturel\fg{} $\Leftrightarrow$ \og{}héritage de
              la culture\fg{}}
      \end{itemize}
      \item{Privilégié dans les noms de catégories}
      \begin{itemize}
        \item{Catégorie Wikipédia \og{}héritage culturel\fg{}}
      \end{itemize}
    \end{itemize}
  \end{block}

  \begin{block}{Adjectif composé}
    \begin{itemize}
      \item{Constitué de plusieurs mots}
      \item{Privilégié pour la formation de néologismes}
    \end{itemize}
  \end{block}
\end{frame}

\begin{frame}[t]{Sélection des termes-clés candidats}\framesubtitle{Analyse linguistique}
  Analyse des adjectifs~:

  \begin{table}
    \centering
    \resizebox{.7\linewidth}{!}{
      \begin{tabular}{l|ccc}
        \toprule
         & \textbf{\textsc{De}ft} \textit{(fr)} & \textbf{SemEval} \textit{(en)} & \textbf{\textsc{Duc}} \textit{(en)}\\
        \hline
        Adjectifs relationnels \hfill(\%) & 87,1 & 43,6 & 53,1\\
        Adjectifs composés \hfill(\%) & $~~$3,3 & 16,4 & 10,6\\
        Adjectifs qualificatifs \hfill(\%) & $~~$9,6 & 40,0 & 36,3\\
        \bottomrule
      \end{tabular}
    }
    \caption{Taux d'adjectifs dans les termes-clés}
  \end{table}

  \begin{table}
    \centering
    \resizebox{.7\linewidth}{!}{
      \begin{tabular}{l|ccc}
        \toprule
        & \textbf{\textsc{De}ft} \textit{(fr)} & \textbf{SemEval} \textit{(en)} & \textbf{\textsc{Duc}} \textit{(en)}\\
        \hline
        Adjectifs relationnels \hfill(\%) & 61,9 & 30,7 & 29,9\\
        Adjectifs composés \hfill(\%) & $~~$0,4 & $~~$7,9 & $~~$8,8\\
        Adjectifs qualificatifs \hfill(\%) & 37,7 & 61,4 & 61,3\\
        \bottomrule
      \end{tabular}
    }
    \caption{Taux d'adjectifs dans les documents}
  \end{table}

  \begin{block}{Observations}
    \begin{itemize}
      \item{Adjectifs relationnels très utilisés dans les termes-clés}
      \item{Adjectifs composés peu utilisés mais peu ambigus}
      \item{Ambigüité sur les adjectifs qualificatifs}
    \end{itemize}
  \end{block}
\end{frame}

\begin{frame}{Sélection des termes-clés candidats}\framesubtitle{Méthode}
  \begin{enumerate}
    \item{Présélection des candidats}
    \begin{itemize}
      \item{\texttt{/NOM+ ADJ?/} \textit{(fr)}}
      \item{\texttt{/ADJ? NOM+/} \textit{(en)}}
    \end{itemize}
    \item{Filtrage de adjectifs superflus}
    \begin{itemize}
      \item{Adjectifs relationnels~: OK}
      \item{Adjectifs composés~: OK}
      \item{\textbf{Adjectifs qualificatifs~: OK~/~KO}}
      \begin{itemize}
        \item{Si fréquence avec \texttt{ADJ} $>$ fréquence sans
                  \texttt{ADJ}~: OK}
      \end{itemize}
    \end{itemize}
  \end{enumerate}
\end{frame}

\begin{frame}{Sélection des termes-clés candidats}\framesubtitle{Évaluation}
  \begin{itemize}
    \item{Évaluation en deux temps~:}
    \begin{enumerate}
      \item{Intrinsèque}
      \item{Extrinsèque}
    \end{enumerate}
    \item{Trois collections de données~:}
    \begin{itemize}
      \item{\textsc{De}ft \textit{(fr)}}
      \item{SemEval \textit{(en)}}
      \item{\textsc{Duc} \textit{(en)}}
    \end{itemize}
    \item{Deux méthode d'extraction de termes-clés~:}
    \begin{itemize}
      \item{\textsc{Tf-Idf}}
      \item{KEA}
    \end{itemize}
  \end{itemize}
\end{frame}

\begin{frame}{Sélection des termes-clés candidats}\framesubtitle{Évaluation}
  Évaluation intrinsèque~:

  \begin{center}
    \resizebox{\linewidth}{!}{
      \begin{tabular}{l|cc|c|cc|c|cc|c}
        \toprule
        \multirow{2}{*}[-2pt]{\textbf{Méthode}} & \multicolumn{3}{c|}{\textbf{\textsc{De}ft} \textit{(fr)}} & \multicolumn{3}{c|}{\textbf{SemEval} \textit{(en)}} & \multicolumn{3}{c}{\textbf{\textsc{Duc}} \textit{(en)}}\\
        \cline{2-10}
        & Candidats & R$_{\text{max}}$ & $Q$ & Candidats & R$_{\text{max}}$ & $Q$ & Candidats & R$_{\text{max}}$ & $Q$\\
        \hline
        $N$-grammes & 2~610,4 & \textbf{74,1} & 0,03 & 1~652,3 & \textbf{71,7} & 0,04 & $~~$478,9 & \textbf{90,4} & 0,19\\
        \texttt{/(NOM~|~ADJ)+/} & $~~$810,3 & 61,1 & 0,08 & $~~$518,5 & 62,0 & 0,12 & $~~$147,4 & 88,3 & 0,60\\
        Syntagmes nominaux & $~~$736,5 & 63,0 & \textbf{0,09} & $~~$478,1 & 56,3 & 0,12 & $~~$141,4 & 75,6 & 0,54\\
        LR-NP & \textbf{$~~$658,2} & 60,1 & \textbf{0,09} & \textbf{$~~$423,8} & 59,0 & \textbf{0,14} & \textbf{$~~$135,3} & 84,8 & \textbf{0,63}\\
        \bottomrule
      \end{tabular}
    }
  \end{center}
\end{frame}

\begin{frame}{Sélection des termes-clés candidats}\framesubtitle{Évaluation}
  Évaluation extrinsèque~:

  \begin{center}
    \resizebox{\linewidth}{!}{
    \begin{tabular}{l@{~}|@{~}c@{~}|@{~}c@{~}|@{~}c@{~}|@{~}c@{~}|@{~}c@{~}|@{~}c}
      \toprule
      \multirow{2}{*}[-2pt]{\textbf{Méthode}} & \multicolumn{2}{c@{~}|@{~}}{\textbf{\textsc{De}ft} \textit{(fr)}} & \multicolumn{2}{c@{~}|@{~}}{\textbf{SemEval} \textit{(en)}} & \multicolumn{2}{c}{\textbf{\textsc{Duc}} \textit{(en)}}\\
      \cline{2-7}
      & \multicolumn{1}{c@{~}|@{~}}{\textsc{Tf-Idf}} & \multicolumn{1}{c@{~}|@{~}}{KEA} & \multicolumn{1}{c@{~}|@{~}}{\textsc{Tf-Idf}} & \multicolumn{1}{c@{~}|@{~}}{KEA} & \multicolumn{1}{c@{~}|@{~}}{\textsc{Tf-Idf}} & \multicolumn{1}{c}{KEA}\\
      \cline{2-7}
      & F & F & F & F & F & F\\
      \hline
      $N$-grammes & $~~$8,9 & \textbf{20,0} & $~~$7,7$~~$ & 16,2 & 17,7 & 14,3\\
      \texttt{/(NOM~|~ADJ)+/} & 13,2 & 18,4 & 10,5 & 17,1 & 27,3 & 16,8\\
      Syntagme nominaux & 12,8 & 18,7 & 10,6 & 16,9 & 24,1 & 15,7\\
      LR-NP & \textbf{13,3} & 19,2 & \textbf{10,9} & \textbf{17,7} & \textbf{27,4} & \textbf{16,9}\\
      \bottomrule
    \end{tabular}
  }
  \end{center}
\end{frame}

\begin{frame}{Sélection des termes-clés candidats}\framesubtitle{Bilan}
  \begin{itemize}
    \item{La sélection des termes-clés candidat influe $\pm$ sur
          l'extraction de termes-clés}
    \item{Certaines catégories de mots sont plus utiles dans les
          termes-clés}
    \item{Les adjectifs ne sont pas tous utiles dans les termes-clés}
  \end{itemize}
\end{frame}

