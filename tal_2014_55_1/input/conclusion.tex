\section{Conclusion et perspectives}
\label{sec:conclusion_et_perspectives}
  Dans ce travail, nous proposons une méthode à base de graphe pour
  l'extraction non-supervisée de termes-clés. Cette méthode groupe les
  termes-clés candidats par sujets, détermine quels sont ceux les plus
  importants, puis extrait le terme-clé candidat qui représente le mieux chacun
  des sujets les plus importants. Cette nouvelle méthode offre plusieurs
  avantages vis-à-vis des précédentes à base de graphe. Dans un premier temps,
  le groupement des termes-clés potentiels en sujets distincts permet le
  rassemblement  d'informations auparavant éparpillées. Dans un second temps,
  le choix d'un seul terme-clé pour représenter l'un des sujets les plus
  importants permet d'extraire un ensemble de termes-clés non redondant -- pour
  $k$ termes-clés extraits, exactement $k$ sujets sont couverts. Finalement, le
  graphe est désormais complet et ne requiert plus le paramètre de fenêtre de
  co-occurrences.

  Les bons résultats de notre méthode montrent la pertinence d'un groupement en
  en sujets des candidats pour ensuite les ordonner. Les expériences
  supplémentaires montrent aussi que la stratégie de sélection du terme-clé
  candidat le plus représentatif d'un sujet joue un rôle crucial. La stratégie
  actuellement utilisée pourrait ainsi être améliorée de sorte que les résultats
  soient significativement améliorés (pour un gain maximum allant de 4,2 à 22,1
  points de f-score).

  Dans de futurs travaux, il est envisagé d'améliorer le groupement en sujets et
  la sélection du terme-clé candidat le plus représentatif pour chacun d'eux.
  Ces deux points sont cruciaux et nécessitent un travail plus approfondit
  linguistiquement.
  
  Le groupement actuellement effectué est un groupement naïf qui ne prend en
  compte ni l'ambiguïté d'un mot, ni la relation de synonymie entre deux mots.
  L'ajout de connaissances concernant les synonymes permettrait de créer des
  sujets plus consistants et la désambiguïsation éviterait un groupement
  systématique des termes-clés candidats ayant un ou plusieurs mots en commun.
  D'un point de vu plus technique, il est aussi envisagé d'explorer différentes
  techniques de groupement, dont le groupement spectral (\textit{spectral
  clustering}) qui, dans d'autres travaux portant sur l'extraction
  automatique de termes-clés~\cite{liu2009keycluster}, montre de meilleures
  performances que le groupement hiérarchique agglomératif.

  En ce qui concerne la stratégie de sélection des termes-clés candidats les
  plus représentatifs des sujets, une étude détaillée des caractéristiques des
  termes-clés pourrait orienter notre travail vers des critères plus efficaces
  que la première position des candidats dans le document. Un apprentissage
  supervisé à partir de certains critère est aussi envisageable, au même titre
  que l'usage de méthodes d'optimisation telles que celle utilisée par
  \newcite{ding2011binaryintegerprogramming} dans leur méthode d'extraction
  automatique de termes-clés.

