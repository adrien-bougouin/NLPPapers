\section{État de l'art}
\label{sec:etat_de_l_art}
  TopicRank faisant partie des méthodes non-supervisées, l'état de l'art
  présenté ici se focalise sur cette catégorie de méthodes.
  % Quel est le fonctionnement général des méthodes d'extraction automatique de
  % termes-clés ?
  L'extraction automatique non-supervisée de termes-clés est une tâche répartie
  en quatre étapes. Les documents sont traités un par un. Ils sont tout d'abord
  enrichis linguistiquement (segmentés en phrases, segmentés en mots, étiquetés
  en parties du discours, etc.), des termes-clés candidats en sont extraits,
  puis ordonnés afin de ne sélectionner que les plus pertinents (voir la
  figure~\ref{fig:etapes_de_l_extraction_de_termes_cles}). L'extraction des
  termes-clés candidats et leur ordonnancement sont les deux étapes auxquelles
  nous nous intéressons dans cet article. En effet, l'ordonnancement des
  termes-clés candidats est le c\oe{}ur de la tâche d'extraction de termes-clés
  et ses performances dépendent de la qualité des candidats préalablement
  extraits.
  \begin{figure}
    \tikzstyle{io}=[
      ellipse,
      minimum width=5cm,
      minimum height=2cm,
      fill=green!20,
      draw=green!33,
      transform shape,
      font={\huge}
    ]
    \tikzstyle{component}=[
      text centered,
      thick,
      rectangle,
      minimum width=11cm,
      minimum height=2cm,
      fill=cyan!20,
      draw=cyan!33,
      transform shape,
      font={\huge\bfseries}
    ]

    \centering
    \begin{tikzpicture}[thin,
                        align=center,
                        scale=.45,
                        node distance=2cm,
                        every node/.style={text centered, transform shape}]
      \node[io] (document) {document};
      \node[component] (preprocessing) [right=of document] {Prétraitement Linguistique};
      \node[component] (candidate_extraction) [below=of preprocessing] {Extraction des candidats};
      \node[component] (candidate_classification_and_ranking) [below=of candidate_extraction] {Ordonnancement des candidats};
      \node[component] (keyphrase_selection) [below=of candidate_classification_and_ranking] {Sélection des termes-clés};
      \node[io] (keyphrases) [right=of keyphrase_selection] {termes-clés};

      \path[->, thick] (document) edge (preprocessing);
      \path[->, thick] (preprocessing) edge (candidate_extraction);
      \path[->, thick] (candidate_extraction) edge (candidate_classification_and_ranking);
      \path[->, thick] (candidate_classification_and_ranking) edge (keyphrase_selection);
      \path[->, thick] (keyphrase_selection) edge (keyphrases);
    \end{tikzpicture}
    \caption{Les quatre principales étapes de l'extraction automatique de
             termes-clés. \label{fig:etapes_de_l_extraction_de_termes_cles}}
  \end{figure}

  \subsection{Extraction de termes-clés candidats}
  \label{subsec:extraction_de_termes_cles_candidats}
    % Quel est l'objectif ?
    L'objectif de l'extraction de termes-clés candidats est de réduire l'espace
    des solutions possibles aux seules unités textuelles ayant des
    particularités semblables à celles des termes-clés tels qu'ils peuvent être
    donnés par des humains. Deux avantages à cela sont la réduction du temps de
    calcul nécessaire à l'extraction des termes-clés et la suppression d'unités
    textuelles non pertinentes pouvant engendrer du bruit, celui-ci pouvant
    affecter les performances de l'extraction. Pour distinguer les différents
    candidats extraits, nous définissons deux catégories~: les candidats
    positifs, qui sont présents dans les références de nos collection de
    données, et les candidats non positifs, qui n'y sont pas. Parmi les
    candidats non positifs, nous distinguons aussi deux catégories~: les
    candidats porteurs d'informations utiles à la promotions de candidats
    positifs et les candidats non pertinents, que nous considérons comme des
    erreurs d'extraction.

    % Quels sont les différentes méthodes utilisées pour extraire les
    % termes-clés candidats ?
    Dans les travaux précédents concernant l'extraction automatique de
    termes-clés, trois méthodes d'extraction de candidats sont classiquement
    utilisées~: l'extraction de n-grammes, l'extraction de chunks nominaux et
    l'extraction d'unités textuelles à partir de patrons syntaxiques.

    L'extraction de n-grammes consiste en l'extraction de toutes les séquences
    ordonnées de $n$ mots. Cette extraction est très exhaustive, elle fournit
    un grand nombre de termes-clés candidats, maximisant ainsi la quantité de
    candidats positifs, la quantité de candidats porteurs d'informations, mais
    aussi la quantité de candidats non pertinents. Pour pallier en partie ce
    problème, il est courant d'utiliser une liste de mots outils. Ces mots
    outils sont les conjonctions (et, ou, soit, car, etc.), les prépositions (à,
    de, en, avec, etc.) et les mots usuels (particulier, beaucoup, près, etc.).
    Un n-gramme contenant un mot outil au début ou à la fin ne doit pas être
    considéré comme un terme-clé candidat. Malgré son aspect bruité, cette
    extraction est encore largement utilisée parmi les méthodes
    supervisées~\cite{witten1999kea,turney1999learningalgorithms,hulth2003keywordextraction}.
    En effet, la phase d'apprentissage de celles-ci les rend plus robustes et
    moins sensibles aux éventuels bruits, comparées aux méthodes supervisées.

    L'extraction de chunks nominaux consiste en l'extraction d'unités
    minimales de sens ayant pour tête un nom. Les chunks nominaux sont
    constitués d'un unique nom ou pronom et de ses déterminants et
    modificateurs. D'un point de vue linguistique, la syntaxe clairement définie
    des chunks nominaux les rend plus justifiée que les n-grammes. Les
    expériences menées par \newcite{hulth2003keywordextraction} et
    \newcite{eichler2010keywe} avec les chunks nominaux montrent une
    amélioration des performances vis-à-vis de l'usage des n-grammes.
    Cependant, \newcite{hulth2003keywordextraction} constate qu'en tirant partie
    de l'étiquetage en parties du discours des termes-clés candidats,
    l'extraction supervisée de termes-clés à partir de n-grammes donne des
    performances au-dessus de celles obtenues avec les chunks nominaux. Cela
    montre que les n-grammes contiennent plus de candidats positifs que les
    chunks nominaux et que la syntaxe des candidats est une propriété
    discriminante des termes-clés.

    L'extraction d'unités textuelles à partir de patrons syntaxiques prédéfinis
    permet de contrôler avec précision la nature des candidats extraits (par
    exemple, les chunks nominaux). De ce fait cette extraction est, tout comme
    l'extraction de chunks nominaux, plus fondée linguistiquement que la simple
    extraction de n-grammes filtrés. Comparée aux chunks nominaux, elle permet
    aussi de pallier le problème du manque de candidats positifs extraits. Dans
    ses travaux, \newcite{hulth2003keywordextraction} choisie d'extraire des
    candidats avec les patrons des termes-clés les plus fréquents (plus de 10
    occurrences) dans sa collection d'apprentissage, tandis que d'autres
    chercheurs tels que \newcite{wan2008expandrank} et
    \newcite{hassan2010conundrums} se concentrent uniquement sur les plus
    longues séquences de noms (noms propres inclus) et d'adjectifs. Pour des
    méthodes non-supervisées telles que la notre, l'extraction des séquences de 
    noms et d'adjectifs est intéressante, car elle ne nécessite ni données
    supplémentaires, ni adaptation particulière pour une langue donnée (tel que
    c'est le cas pour l'extraction des chunks nominaux, par exemple).

    % Que veut-on apporter ?
    Dans le but d'améliorer la qualité des candidats extraits à partir
    d'articles scientifique, \newcite{kim2009termextraction} proposent un
    filtrage des candidats en fonction de leur spécificité vis-à-vis du document
    analysé. Cette spécificité est déterminée par rapport à la fréquence d'un
    candidat dans le document et le nombre de documents, d'une collection, dans
    lesquels il apparaît~\cite[TF-IDF]{jones1972tfidf}. Intuitivement, un
    candidat très fréquent dans le document analysé est d'autant plus spécifique
    à celui-ci s'il est présent dans très peu d'autres documents. Cette approche
    est intéressant, mais elle requière des documents  supplémentaires et la
    définition d'un seuil pour le filtrage. Dans le cas de TopicRank, nous
    tentons de nous abstraire de l'usage d'autres documents que celui qui est
    analysé, cette méthode d'extraction de candidats n'est donc pas consistante
    avec nos objectifs.

  \subsection{Ordonnancement des termes-clés candidats}
  \label{subsec:ordonnancement_des_termes_cles_candidats}
    % Quel est l'objectif ?
    L'étape d'ordonnancement intervient après l'extraction des termes-clés
    candidats. Son rôle est de déterminer quels sont parmi les candidats les
    termes-clés du document analysé.
    % Quels sont les différentes méthodes non-supervisées existantes pour
    % l'extraction de termes-clés ?
    % Quels sont les inconvénients des méthodes actuelles ?
    % Que veut-on apporter ?
    Les méthodes non-supervisées d'extraction automatique de termes-clés
    emploient des techniques très différentes, allant du simple usage de mesures
    statistiques~\cite{jones1972tfidf,paukkeri2010likey} à l'utilisation de
    modèles de langues obtenus à partir de données
    non-annotées~\cite{tomokiyo2003languagemodel}, en passant par la
    construction d'un graphe de co-occurrences~\cite{mihalcea2004textrank}.
    Puisque la méthode que nous présentons dans cet article est une méthode dite
    \og à base de graphe~\fg, nous ne nous intéressons ici qu'à cette dernière
    catégorie de méthodes.

    \newcite{mihalcea2004textrank} proposent TextRank, une méthode
    d'ordonnancement d'unités textuelles à partir d'un graphe. Utilisés dans de
    nombreuses applications du TAL~\cite{kozareva2013textgraphs}, les graphes
    ont l'avantage de présenter de manière simple et efficace les unités
    textuelles d'un document et les relations entre elles. De plus, ils sont
    l'assurance de bénéficier de nombreux outils et algorithmes issus de la
    théorie des graphes pouvant répondre à de nombreux problèmes. Dans le cas de
    TextRank, les n\oe{}uds du graphe sont les mots du document et les arrêtes
    sont leurs relations de co-occurrences dans une fenêtre de 2 mots. Un score
    d'importance est ensuite calculé pour chaque mot à partir de l'algorithme
    PageRank~\cite{brin1998pagerank} qui est issu de la mesure de centralité
    eigenvector. Le principe utilisé est celui de la recommandation, c'est à
    dire un mot est d'autant plus important qu'il co-occurre avec un grand
    nombre de mots différents et le plus ces mots sont importants, le plus il
    est important. Les mots les plus importants sont considérés comme des
    mots-clés. Ces mots clés sont marqués dans le document et les plus longues
    séquences de mots-clés sont extraites en tant que termes-clés. Dans cette
    méthode, la précision de l'ordonnancement dépend de la qualité du graphe qui
    elle même dépend de la fenêtre de co-occurrences. La définition de cette
    fenêtre est sujette à une intervention manuelle et peut dépendre des
    propriétés des documents analysés. Dans nos travaux, nous tentons de nous
    abstraire de cette fenêtre.

    \newcite{wan2008expandrank} proposent la méthode SingleRank qui présente
    deux améliorations à TextRank. Dans un premier temps, les auteurs améliorent
    la précision de l'ordonnancement en pondèrant les liens de co-occurrence par
    le nombre de co-occurrences (par exemple, un mot co-occurrent deux fois avec
    un autre est relié à celui-ci par un poids de 2) et en utilisant ce poids
    lors du transfère d'importance.
    %Ce nouvel ordonnancement utilise plus d'informations présentes dans le
    %document analysé et est dotant plus efficace quand le document analysé est
    %de grande taille.
    Dans un second temps, les termes-clés ne sont plus générés, mais ordonnés à
    partir de la somme du score d'importance des mots qu'ils contiennent. La
    méthode SingleRank donne, dans la majorité des cas, des résultats meilleurs
    que ceux de TextRank. Cependant, faire la somme du score d'importance des
    mots pour ordonner les candidats est une approche maladroite. En effet, cela
    a pour effet de faire monter dans le classement des candidats qui se
    recouvrent. Ainsi, dans le document \textit{as\_2002\_000700ar} de la
    collection DEFT (voir la section~\ref{sec:evaluation}), le candidat positif
    \og bio-politique~\fg\ est classé neuvième, alors que les autres candidats
    contenant \og bio-politique~\fg\ plus d'autres mots non importants occupent
    les classements 2 à 8. Dans nos travaux nous ordonnons les termes-clés
    candidats en tenant compte de l'importance du sujet qu'ils représentent puis
    choisissons un représentant par sujet, nous évitons ainsi le problème
    rencontré avec SingleRank.

    Toujours dans l'optique d'utiliser plus d'informations pour améliorer
    l'efficacité de l'ordonnancement, \newcite{wan2008expandrank} étendent
    SingleRank en utilisant des documents voisins (similaires) du document en
    cours d'analyse. Leur méthode, ExpandRank, consiste à observer les
    co-occurrences dans les documents similaires afin de renforcer ou ajouter
    des liens dans le graphe initial. Cette approche donne des résultats au delà
    de ceux de SingleRank, mais il est important de noter que ses performances
    sont fortement liées à la disponibilité de documents similaires à celui qui
    est analysé.

    A l'instar de \newcite{wan2008expandrank},
    \newcite{tsatsaronis2010semanticrank} tentent d'améliorer TextRank en
    modifiant le processus de création des liens entre les n\oe{}uds du graphe.
    Dans leur approche, un lien entre deux mots est créé et pondéré en fonction
    du lien sémantique de ces derniers selon WordNet~\cite{miller1995wordnet} ou
    Wikipedia~\cite{milne2008wikipediasemanticrelatedness}. Les expériences
    menées par les auteurs montrent de moins bons résultats que TextRank.
    Toutefois, en biaisant l'ordonnancement en faveur des mots apparaissant dans
    le titre du document analysé ou bien en ajoutant le poids TF-IDF des mots
    dans le calcul de l'importance des mots, leur méthode est capable de donner
    de meilleurs résultats que TextRank.

    L'usage de sujets dans le processus d'ordonnancement de TextRank est à
    l'origine proposé par \newcite{liu2010topicalpagerank}. Reposant sur un
    modèle LDA~\cite[Latent Dirichlet Allocation]{blei2003lda}, leur méthode
    effectue des ordonnancements biaisés par les sujets du document, puis
    fusionne les rangs des mots dans ces différents ordonnancement afin
    d'obtenir un rang global pour chaque mot. Dans notre travail, nous émettons
    aussi l'hypothèse que le sujet auquel appartient une unité textuelle doit
    jouer un rôle majeur dans le processus d'ordonnancement. Cependant, nous
    tentons de nous abstraire de l'usage de documents supplémentaires et
    n'utilisons donc pas le modèle LDA. En addition, il nous semble plus
    judicieux d'effectuer un seul ordonnancement.

