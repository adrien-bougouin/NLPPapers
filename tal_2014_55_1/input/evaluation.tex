\section{Évaluation}
\label{sec:evaluation}
  TODO introduction

  \subsection{Environnement expérimental}
  \label{subsec:environnement_experimental}
    TODO introduction

    \subsubsection{Données de test}
    \label{subsubsec:donnees_de_test}
      Les données utilisées lors des expériences correspondes aux ensembles de
      test fournis avec les collections DUC, SemEval et DEFT. Le
      tableau~\ref{tab:donnees_de_test} montre les statistiques sur ces données
      de test. Ces statistiques sont cohérentes avec celles montrées dans la
      section~\ref{sec:presentation_et_analyse_des_donnees}. Nous pouvons donc
      en déduire que les propriétés inférées précédemment sont satisfaites dans
      les données de test.
      \begin{table}
        \centering
        \begin{tabular}{@{~~}rccc@{~~}}
          \toprule
          \multirow{2}{*}[-2pt]{\textbf{Statistique}} & \multicolumn{3}{c}{\textbf{Corpus}}\\
          \cmidrule{2-4}
          & DUC & SemEval & DEFT\\
          \midrule
          Langue & Anglais & Anglais & Français\\
          Nature & Journalistique & Scientifique & Scientifique\\
          Documents & 100 & 100 & 93\\
          Mos/document & 877.3 & 5177.7 & 6839.4\\
          Termes-clés/document & 7.9 & 14.7 & 5.2\\
          Mots/termes-clés & 2.1 & 2.1 & 1.6\\
          Termes-clés manquants & 2.8\% & 22.1\% & 21.1\% \\
          \bottomrule
        \end{tabular}
        \caption{Statistiques sur les données de test utilisées. En accord avec
                 l'évaluation effectuée lors de nos expériences, la proportion
                 de termes-clés manquant est calculée à partir de leur racine.
                 \label{tab:donnees_de_test}}
      \end{table}

    \subsubsection{Prétraitement}
    \label{subsubsec:pretraitement}
      Chaque document des collections de données utilisées ont subis les mêmes
      prétraitements, celà pour toutes les expériences. Les documents sont tout
      d'abord segmentés en phrases, puis en mots et enfin étiquetés en parties
      du discours. La segmentation en mots est effectuée par le
      TreeBankWordTokenizer disponible avec la librairie python
      NLTK~\cite[\textit{Natural Language ToolKit}]{bird2009nltk}, pour
      l'anglais, et par l'outil Bonsai du Bonsai PCFG-LA
      parser\footnote{\url{http://alpage.inria.fr/statgram/frdep/fr\_stat\_dep\_parsing.html}},
      pour le français. Quant à l'étiquetage en parties du discours, il est
      réalisé avec le Stanford POS tagger~\cite{toutanova2003stanfordpostagger},
      pour l'anglais, et avec MElt~\cite{denis2009melt}, pour le français.

    \subsubsection{Mesures d'évaluation}
    \label{subsubsec:mesures_d_evaluation}
      Les performances des méthodes d'extraction de termes-clés sont exprimées
      en termes de précision (P), rappel (R) et f-score (f1-mesure, F). Afin de
      réduire le problème de termes-clés manquants, les comparaisons sont
      effectuées avec la forme non flexionnelle des mots des termes-clés
      extraits et des termes-clés de référence. Afin de montrer la limite que
      permettent d'atteindre les différentes méthodes d'extraction de
      termes-clés candidats, les ensembles de termes-clés extraits sont évalués
      avec le rappel maximum (Rmax).

    \subsubsection{Systèmes de référence pour l'extraction de termes-clés}
    \label{subsubsec:systemes_de_reference_pour_l_extraction_de_termes_cles}

  \subsection{Analyse intrinsèque des méthodes d'extraction de candidats}
  \label{subsec:analyse_intrinseque_des_methodes_d_extraction_de_candidats}
    \begin{table}
      \centering
      \begin{tabular}{@{~}r@{~~}c@{~~}cc@{~~}cc@{~~}c@{~}}
        \toprule
        \multirow{2}{*}[-2pt]{\textbf{Méthode}} & \multicolumn{2}{c}{\textbf{DUC}} & \multicolumn{2}{c}{\textbf{SemEval}} & \multicolumn{2}{c}{\textbf{DEFT}}\\
        \cmidrule(r){2-3}\cmidrule(lr){4-5}\cmidrule(l){6-7}
        & Cand./Doc. & Rmax & Cand./Doc. & Rmax & Cand./Doc. & Rmax\\
        \midrule
        \{1..2\}-grammes & $~~$491.0 & 76.6 & 1633.6 & 61.0 & 2566.4 & 67.3\\
        \{1..3\}-grammes & $~~$596.2 & \textbf{90.8} & 2580.5 & \textbf{72.2} & 4070.2 & 74.1\\
        Patrons appris & $~~$317.6 & 90.6 & 1227.4 & 69.8 & 2148.3 & \textbf{76.5}\\
        Séquences (Nc | A)+ & $~~$155.6 & 88.7 & $~~$646.5 & 62.4 & $~~$914.5 & 61.1\\
        +sous-composants & $~~$171.8 & 90.6 & $~~$712.2 & 64.4 & $~~$934.0 & 61.1\\
        Chunks nominaux & $~~$149.9 & 76.0 & $~~$598.4 & 56.6 & $~~$812.3 & 63.0\\
        TermSuite & $~~$161.8 & 46.1 & $~~$498.6 & 32.4 & $~~$647.0 & 52.8\\
        \bottomrule
      \end{tabular}
      \caption{Résultats de l'extraction de termes-clés candidats pour les
               collections de données DUC, SemEval et DEFT.
               \label{tab:resultats_de_l_extraction_de_candidats}}
    \end{table}

  \subsection{Comparaison de TopicRank avec l'existant}
  \label{subsec:comparaison_de_topicrank_avec_l_existant}
    \begin{table}
      \centering
      \begin{tabular}{@{~~}rc@{~~~}c@{~~~}cc@{~~~}c@{~~~}cc@{~~~}c@{~~~}c@{~~}}
        \toprule
        \multirow{2}{*}[-2pt]{\textbf{Méthode}} & \multicolumn{3}{c}{\textbf{DUC}} & \multicolumn{3}{c}{\textbf{SemEval}} & \multicolumn{3}{c}{\textbf{DEFT}}\\
        \cmidrule(r){2-4}\cmidrule(lr){5-7}\cmidrule(l){8-10}
        & P & R & F & P & R & F & P & R & F\\
        \midrule
        \{1..2\}-grammes & 14.7 & 19.5 & 16.5 & 10.3 & $~~$7.0 & $~~$8.3 & $~~$8.1 & 15.1 & 10.4\\
        \{1..3\}-grammes & 14.3 & 19.0 & 16.1 & $~~$9.0 & $~~$6.0 & $~~$7.2 & $~~$6.7 & 12.5 & $~~$8.6\\
        Patrons appris & 19.1 & 25.4 & 21.5 & 10.7 & $~~$7.3 & $~~$8.6 & $~~$7.0 & 13.1 & $~~$9.0\\
        Séquences (Nc | A)+ & 24.2 & 31.7 & \textbf{27.0} & 11.7 & $~~$7.9 & $~~$9.3 & $~~$9.5 & 17.6 & 12.1\\
        +sous-composants & 22.8 & 29.9 & 25.5 & 10.8 & $~~$7.2 & $~~$8.6 & $~~$9.2 & 17.2 & 11.9\\
        Chunks nominaux & 21.1 & 28.1 & 23.8 & 11.9 & $~~$8.0 & \textbf{$~~$9.5} & $~~$9.6 & 17.9 & 12.3\\
        TermSuite & 17.4 & 23.2 & 19.6 & 11.2 & $~~$8.1 & $~~$9.3 & 11.3 & 21.1 & \textbf{14.5}\\
        \bottomrule
      \end{tabular}
      \caption{Résultats de l'extraction de $10$ termes-clés selon
               \textbf{TF-IDF}, pour DUC, SemEval et DEFT, en fonction des types
               de candidats utilisés. 
               \label{tab:resultats_de_tfidf}}
    \end{table}
    \begin{table}
      \centering
      \begin{tabular}{@{~~}rc@{~~~}c@{~~~}cc@{~~~}c@{~~~}cc@{~~~}c@{~~~}c@{~~}}
        \toprule
        \multirow{2}{*}[-2pt]{\textbf{Méthode}} & \multicolumn{3}{c}{\textbf{DUC}} & \multicolumn{3}{c}{\textbf{SemEval}} & \multicolumn{3}{c}{\textbf{DEFT}}\\
        \cmidrule(r){2-4}\cmidrule(lr){5-7}\cmidrule(l){8-10}
        & P & R & F & P & R & F & P & R & F\\
        \midrule
        \{1..2\}-grammes & 15.5 & 20.5 & 17.4 & 10.4 & $~~$7.0 & \textbf{$~~$8.3} & $~~$3.0 & $~~$6.2 & $~~$4.0\\
        \{1..3\}-grammes & 13.7 & 18.0 & 15.3 & $~~$3.4 & $~~$2.3 & $~~$2.7 & $~~$1.9 & $~~$4.2 & $~~$2.6\\
        Patrons appris & 18.7 & 24.3 & 20.8 & $~~$4.6 & $~~$3.1 & $~~$1.2 & $~~$1.9 & $~~$4.2 & $~~$2.6\\
        Séquence (Nc | A)+ & 22.8 & 29.5 & \textbf{25.3} & $~~$3.7 & $~~$2.5 & $~~$3.0 & $~~$4.6 & $~~$9.2 & $~~$6.1\\
        +sous-composants & 21.2 & 27.3 & 23.5 & $~~$3.5 & $~~$2.4 & $~~$2.8 & $~~$4.6 & $~~$9.2 & $~~$6.1\\
        Chunks nominaux & 20.6 & 27.3 & 23.1 & $~~$8.4 & $~~$5.7 & $~~$6.7 & $~~$4.9 & $~~$9.7 & $~~$6.4\\
        TermSuite & 17.0 & 22.4 & 19.1 & $~~$7.5 & $~~$5.4 & $~~$6.2 & $~~$6.8 & 13.3 & \textbf{$~~$8.8}\\
        \bottomrule
      \end{tabular}
      \caption{Résultats de l'extraction de $10$ termes-clés selon
               \textbf{SingleRank}, pour DUC, SemEval et DEFT, en fonction des
               types de candidats utilisés. 
               \label{tab:resultats_de_singlerank}}
    \end{table}
    \begin{table}
      \centering
      \begin{tabular}{@{~~}rc@{~~~}c@{~~~}cc@{~~~}c@{~~~}cc@{~~~}c@{~~~}c@{~~}}
        \toprule
        \multirow{2}{*}[-2pt]{\textbf{Méthode}} & \multicolumn{3}{c}{\textbf{DUC}} & \multicolumn{3}{c}{\textbf{SemEval}} & \multicolumn{3}{c}{\textbf{DEFT}}\\
        \cmidrule(r){2-4}\cmidrule(lr){5-7}\cmidrule(l){8-10}
        & P & R & F & P & R & F & P & R & F\\
        \midrule
        \{1..2\}-grammes & 10.2 & 14.1 & 11.7 & 11.9 & $~~$8.2 & $~~$9.6 & $~~$5.8 & 11.0 & $~~$7.5\\
        \{1..3\}-grammes & $~~$7.8 & 10.7 & $~~$8.9 & $~~$9.5 & $~~$6.7 & $~~$7.7 & $~~$6.2 & 11.4 & $~~$8.0\\
        Patrons appris & 14.9 & 19.8 & 16.7 & 12.2 & $~~$4.6 & \textbf{$~~$9.9} & $~~$8.8 & 16.1 & 11.3\\
        Séquences (Nc | A)+ & 17.7 & 23.2 & 19.8 & 11.6 & $~~$7.9 & $~~$9.3 & 11.6 & 21.5 & \textbf{14.9}\\
        +sous-composants & 18.3 & 24.0 & \textbf{20.5} & 11.3 & $~~$7.7 & $~~$9.0 & 11.6 & 21.5 & \textbf{14.9}\\
        Chunks nominaux& 13.3 & 21.5 & 18.3 & 11.7 & $~~$8.0 & $~~$9.4 & 11.1 & 20.7 & 14.4\\
        TermSuite & 10.4 & 13.9 & 11.7 & $~~$8.9 & $~~$6.5 & $~~$7.5 & $~~$9.6 & 18.5 & 12.4\\
        \bottomrule
      \end{tabular}
      \caption{Résultats de l'extraction de $10$ termes-clés selon
               \textbf{TopicRank}, pour DUC, SemEval et DEFT, en fonction des
               types de candidats utilisés. 
               \label{tab:resultats_de_topicrank}}
    \end{table}

    \begin{table}
      \centering
      \begin{tabular}{@{~~}rc@{~~~}c@{~~~}cc@{~~~}c@{~~~}cc@{~~~}c@{~~~}c@{~~}}
        \toprule
        \multirow{2}{*}[-2pt]{\textbf{Méthode}} & \multicolumn{3}{c}{\textbf{DUC}} & \multicolumn{3}{c}{\textbf{SemEval}} & \multicolumn{3}{c}{\textbf{DEFT}}\\
        \cmidrule(lr){2-4}\cmidrule(lr){5-7}\cmidrule(lr){8-10}
        & P & R & F & P & R & F & P & R & F\\
        \midrule
        SingleRank & 22.8 & 29.5 & 25.3 & $~~$3.7 & $~~$2.5 & $~~$3.0 & $~~$4.6 & $~~$9.2 & $~~$6.1\\
        +candidats & & & & & & & & &\\
        +sujets & & & & & & & & &\\
        +complet & & & & & & & & &\\
        TopicRank & 17.7 & 23.2 & 19.8 & 11.6 & $~~$7.9 & 9.3 & $~~$6.8 & 13.3 & $~~$8.8\\
        \bottomrule
      \end{tabular}
      \caption{Évaluation individuelle des apports de TopicRank vis-à-vis de
               SingleRank, pour l'extraction de $10$ termes-clés, avec les plus
               longues séquences de noms et d'adjectifs comme candidats.
               \label{tab:ameliorations_de_singlerank}}
    \end{table}

  \subsection{Efficacité du groupement en sujets}
  \label{subsec:efficacite_du_groupement_en_sujets}
    \begin{itemize}
      \item{Comparer les résultats avec les différentes stratégies de
            groupement, uniquement pour TRIGRAMS, LNP et TERMESUITE}
      \item{Évaluation manuelle des groupes construits avec LNP}
      \item{Évaluation des clusters ordonnés}
    \end{itemize}

    \begin{table*}
      \centering
      \begin{tabular}{@{~~}rc@{~~~}c@{~~~}cc@{~~~}c@{~~~}cc@{~~~}c@{~~~}c@{~~}}
        \toprule
        \multirow{2}{*}[-2pt]{\textbf{Méthode}} & \multicolumn{3}{c}{\textbf{DUC}} & \multicolumn{3}{c}{\textbf{SemEval}} & \multicolumn{3}{c}{\textbf{DEFT}}\\
        \cmidrule(lr){2-4}\cmidrule(lr){5-7}\cmidrule(lr){8-10}
        & P & R & F & P & R & F & P & R & F\\
        \midrule
        \{1..2\}-grammes & 17.7 & 23.9 & 20.0 & 24.1 & 16.6 & 19.5 & 9.2 & 17.3 & 11.9\\
        \{1..3\}-grammes & 17.9 & 23.6 & 20.0 & 28.5 & 19.8 & 23.1 & 10.2 & 19.5 & 13.2\\
        Patrons appris & 26.9 & 34.6 & 29.8 & 30.8 & 21.3 & \textbf{24.9} & 13.1 & 24.7 & 16.9\\
        Séquence (NC | A)+ & 38.5 & 49.4 & 42.6 & 28.0 & 19.0 & 22.4 & 14.5 & 27.1 & 18.7\\
        +sous-composants & 39.9 & 51.2 & \textbf{44.2} & 27.5 & 18.4 & 21.8 & 14.5 & 27.0 & 18.7\\
        Chunks nominaux & 31.9 & 41.5 & 35.5 & 26.4 & 18.3 & 21.4 & 15.1 & 28.1 & \textbf{19.4}\\
        TermSuite & 17.4 & 23.1 & 19.6 & 16.5 & 11.8 & 13.6 & 11.3 & 22.3 & 14.8\\
        \bottomrule
      \end{tabular}
      \caption{Évaluation de l'ordonnancement des sujets. Parmi les $10$
               meilleurs sujets, les termes-clés candidats sélectionnés sont
               ceux présents dans l'ensemble des termes-clés de référence.
               \label{tab:evaluation_de_l_ordonnancement_des_clusters}}
    \end{table*}

