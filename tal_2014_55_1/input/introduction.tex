\section{Introduction}
\label{sec:introduction}
  % Que sont les termes-clés ?
  Les termes-clés sont des mots ou des expressions polylexicales qui représentent les sujets principaux du document auquel ils se réfèrent.
  % À quoi sont-ils utiles ?
  Du fait de leur propriété synthétique, les termes-clés sont utilisés dans de nombreuses applications telles que l'indexation de documents~\cite{medelyan2008smalltrainingset}, le résumé automatique~\cite{litvak2008graphbased} ou encore la classification de documents~\cite{han2007webdocumentclustering}.
    % Avec l'augmentation exponentielle de la quantité de documents à notre disposition et le besoin d'accéder toujours plus rapidement à ceux-ci, les enrichissements susmentionnés deviennent une nécessité.
  Avec la croissance exponentielle du nombre de documents disponibles, en particulier sur le web, les termes-clés constituent un moyen efficace pour accéder rapidement aux informations pertinentes.
  Cependant, la plupart des documents ne sont pas pourvus de termes-clés et l'annotation manuelle de ces derniers est une tâche beaucoup trop coûteuse pour être envisagée.
  % Que faut-il faire pour les obtenir ?
    % De ce fait, nous observons un intérêt grandissant pour la tâche d'extraction automatique de termes-clés.
  C'est la raison pour laquelle la problématique de l'extraction automatique de termes-clés suscite de plus en plus l'intéret des chercheurs.

  % Comment cela a-t-il déjà été fait ?
  L'extraction automatique de termes-clés peut être perçue soit comme une tâche
  de classification binaire de termes-clés candidats~\cite{witten1999kea}, soit
  comme une tâche d'ordonnancement de termes-clés
  candidats~\cite{mihalcea2004textrank}. Dans le premier cas, l'extraction est
  le plus souvent supervisée, elle nécessite une collection de documents annotés
  en termes-clés pour une phase d'apprentissage préliminaire. Dans le second
  cas, l'extraction est le plus souvent non-supervisée, elle ne nécessite pas de
  documents préalablement annotés. Les méthodes non-supervisées ont des
  performances plus faibles que les méthodes supervisées actuelles. Ceci
  s'explique par le fait que les méthodes supervisées apprennent les critères
  discriminants pour l'extraction de termes-clés à partir de documents annotés.
  Cependant, la dépendance de ces méthodes au domaine des documents utilisés
  lors de l'apprentissage pousse de plus en plus de chercheurs à s'intéresser à
  l'extraction non-supervisée de termes-clés.

  % Que proposons-nous dans cet article ?
  % Comment le fait-on ?
  Dans cet article, nous présentons TopicRank, une méthode d'extraction
  non-supervisée de termes-clés qui se fonde sur les travaux de
  \newcite[TextRank]{mihalcea2004textrank} pour l'ordonnancement à base de
  graphe des unités textuelles d'un document. TopicRank groupe les termes-clés
  candidats selon leur appartenance à un sujet, ordonne les sujets par
  importance dans le document, puis sélectionne pour chacun des meilleurs sujets
  le candidat qui le représente le mieux (son terme-clé associé).
  % Quelle est la nouveauté ?
  % En quoi cela semble-t-il meilleur ?
  Contrairement à l'ordonnancement des mots tel qu'il est fait avec TextRank,
  le groupement des candidats en sujets utilisés pour l'ordonnancement permet de
  tirer partie d'informations complémentaires extraites de différents candidats
  d'un même sujet. De plus, le fait de ne sélectionner qu'un seul candidat par
  sujet permet d'éviter l'extraction de termes-clés redondants.

  % Que peut-on dire sur les évaluation réalisées ?
  Pour évaluer TopicRank, nous utilisons quatre collections de données dont les
  propriétés diffèrent (nature, langue, taille des documents, etc.) afin de
  mieux observer ses avantages et ses faiblesses~\cite{hassan2010conundrums}. En
  addition, nous comparons TopicRank avec trois méthodes non-supervisées, l'une
  extrayant des statistiques à partir de documents supplémentaires et les deux
  autres appartenant à la catégorie des méthodes à base de graphe. Pour trois
  des collections utilisées, TopicRank donne de meilleurs résultats que les
  autres méthodes. De plus, cette amélioration est significative vis-à-vis des
  deux méthodes à base de graphe.

  % Quel est notre plan ?
  L'article est structuré comme suit. Après un état de l'art des méthodes
  d'extraction automatique de termes-clés en
  section~\ref{sec:etat_de_l_art}, nous décrivons le fonctionnement de TopicRank
  en section~\ref{sec:extraction_de_termes_cles_avec_topicrank} et présentons
  son évaluation approfondie en section~\ref{sec:evaluation}. Enfin, nous
  concluons et discutons des futurs travaux dans la
  section~\ref{sec:conclusion_et_perspectives}.

