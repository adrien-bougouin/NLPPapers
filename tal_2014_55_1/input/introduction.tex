\section{Introduction}
\label{sec:introduction}
  % Que sont les termes-clés ?
  Les termes-clés sont des mots ou des expressions polylexicales qui donnent,
  ensembles, un aperçu du contenu principal d'un document auquel ils se
  réfèrent.
  % À quoi sont-ils utiles ?
  Du fait de leur propriété synthétique, les termes-clés sont utilisés dans
  diverses tâches d'enrichissement telles que l'indexation de
  documents~\cite{medelyan2008smalltrainingset}, le résumé
  automatique~\cite{litvak2008graphbased} ou encore la classification de
  documents~\cite{han2007webdocumentclustering}. Avec l'accroissement de la
  quantité de documents à notre disposition et le besoin d'accéder efficacement
  à ceux-ci, les enrichissements susmentionnés deviennent une nécessité.
  Cependant, tous les documents ne sont pas pourvus de termes-clés et l'ajout
  manuel de ces derniers est une tâche trop coûteuse pour être encore envisagée.
  % Que faut-il faire pour les obtenir ?
  De ce fait, nous observons un intérêt grandissant pour la tâche d'extraction
  automatique de termes-clés.

  % Comment cela a-t-il déjà été fait ?
  L'extraction automatique de termes-clés peut être perçue soit comme une tâche
  de classification binaire de termes-clés candidats~\cite{witten1999kea}, soit
  comme une tâche d'ordonnancement de termes-clés
  candidats~\cite{mihalcea2004textrank}. Dans le premier cas l'extraction est le
  plus souvent supervisée, elle nécessite une collection de documents annotés en
  termes-clés pour une phase d'apprentissage préliminaire. Dans le second cas
  l'extraction est le plus souvent non-supervisée, elle ne nécessite pas de
  documents préalablement annotés. Les méthodes non-supervisées existantes ont
  des performances plus faibles que les méthodes supervisées actuelles, mais la
  nécessité pour ces dernier d'avoir une collection de documents annotés, ainsi
  que la dépendance au domaine des documents résultante de la phase
  d'apprentissage, poussent de plus en plus les chercheurs à s'intéresser à
  l'extraction non-supervisée de termes-clés.

  % Que proposons-nous dans cet article ?
  Les termes-clés candidats étant un point commun à toutes les méthodes
  d'extraction automatique de termes-clés, nous nous intéressons dans un premier
  temps aux propriétés des termes-clés donnés par des humains. En effet, bien
  que l'annotation humaine soit subjective, certaines propriétés relatives à
  l'informativité et à la syntaxe des termes-clés émergent et nous permettent de
  comparer les méthodes d'extraction de termes-clés candidats existantes. Ces
  méthodes consistent en l'extraction de n-grammes filtrés avec une liste de
  mots outils, l'extraction de chunks nominaux ou encore l'extraction de
  séquences de mots respectant un patron syntaxique donné. \og Quels sont les
  méthodes qui extraient des candidats de qualité~?~\fg, \og N'est-il pas
  possible d'utiliser les méthodes d'extraction terminologique pour fournir des
  candidats de bonne qualité~?~\fg\ et \og La quantité de candidats extraits
  prime-t-elle sur la qualité de l'ensemble de candidats~?~\fg\ sont autant de
  questions que nous nous posons.

  % Que proposons-nous dans cet article ?
  % Comment le fait-on ?
  Dans un second temps nous présentons TopicRank, une méthode d'extraction
  non-supervisée de termes-clés. Celle-ci se fonde sur les travaux de
  \newcite[TextRank]{mihalcea2004textrank} pour l'ordonnancement à base de
  graphe des unités textuelles d'un document. Elle groupe les termes-clés
  candidats selon leur appartenance à un sujet, ordonne les sujets par
  importance dans le document, puis sélectionne pour chacun des meilleurs sujets
  le candidat qui le représente le mieux (son terme-clé associé).
  % Quelle est la nouveauté ?
  % En quoi cela semble-t-il meilleur ?
  Contrairement à l'ordonnancement des mots, tel qu'il est fait avec TextRank,
  le groupement des candidats par sujets, puis leur ordonnancement, permet de
  tirer partie des informations complémentaires entre plusieurs unités
  textuelles représentant le même sujet, ou la même idée. De plus, le fait de ne
  sélectionner qu'un seul candidat par sujet permet d'éviter l'extraction de
  termes-clés redondants.

  % Que peut-on dire sur les évaluation réalisées ?
  TODO parler des évaluations.

  % Quel est notre plan ?
  L'article est structuré comme suit. Après un état de l'art des méthodes
  existantes en section~\ref{sec:etat_de_l_art}, nous présentons et analysons en
  section~\ref{sec:presentation_et_analyse_des_donnees} les données que nous
  utilisons, puis nous décrivons le fonctionnement de TopicRank en
  section~\ref{sec:extraction_de_termes_cles_avec_topicrank}. En
  section~\ref{sec:evaluation} nous présentons notre évaluation approfondie des
  différentes étapes de l'extraction de termes-clés, avant de conclure sur les
  résultats obtenus et les futurs travaux dans la
  section~\ref{sec:conclusion_et_perspectives}.

