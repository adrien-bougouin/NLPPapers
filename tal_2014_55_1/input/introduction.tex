\section{Introduction}
\label{sec:introduction}
  Un terme-clé est un mot ou une expression polylexicale permettant de
  caractériser le contenu d'un document. Un ensemble de termes-clés permet ainsi
  de définir les principales thématiques abordées dans un document. Du fait de
  cette propriété synthétique, les ensembles de termes-clés sont utilisés dans
  de nombreuses applications du Traitement Automatique des Langues (TAL) telles
  que l'indexation automatique de documents~\cite{medelyan2008smalltrainingset},
  le résumé automatique~\cite{avanzo2005keyphrase}, la compression
  multi-phrase~\cite{boudin2013multisentencecompression} ou encore la
  classification de documents~\cite{han2007webdocumentclustering}.
  Pourtant, de nombreux documents, tels que ceux accessibles par Internet, ou
  collections de documents, telles que les actes de conférences, n'en sont pas
  accompagnées. La quantité de document à traiter est aujourd'hui trop
  importante pour que l'annotation de leurs termes-clés soit effectuée
  manuellement. C'est pourquoi de nombreux chercheurs se penchent sur la
  problématique de l'extraction automatique de termes-clés.

  L'extraction automatique de termes-clés consiste à sélectionner dans un
  document les unités textuelles les plus importantes, c'est-à-dire celles qui
  représentent le contenu principale du document. Bien que des documents
  supplémentaires peuvent être utilisés, l'extraction de termes-clés ne concerne
  qu'un seul document à la fois. Parmi les différentes méthodes d'extraction
  automatique de termes-clés proposées dans la littérature, deux grandes
  catégories émergent~: les méthodes supervisées et les méthodes
  non-supervisées. Les premières réduisent la tâche d'extraction de termes-clés
  en une tâche de classification binaire~\cite{witten1999kea}, où il s'agit
  d'attribuer la classe \og{}\textit{terme-clé}\fg{} ou \og{}\textit{non
  terme-clé}\fg{} aux différents candidats extraits à partir du document. Une
  collection de documents annotés en termes-clés est alors nécessaire pour
  l'apprentissage du modèle de classification. Les méthodes non-supervisées,
  quant à elles, attribuent un score d'importance à chaque candidat en fonction
  de divers indicateurs comme par exemple la fréquence, le nombre de
  co-occurrences ou la position dans le document. Bien que les méthodes
  supervisées soient en général plus performantes, la faible quantité de
  documents annotés en termes-clés disponibles couplée à la forte dépendance des
  modèles de classification vis-à-vis du type de documents sur lesquels ils ont
  été appris, poussent les chercheurs à s'intéresser de plus en plus aux
  méthodes non-supervisées.

  Les méthodes d'extraction de termes-clés non-supervisées les plus étudiées
  sont sans conteste celles basée sur TextRank~\cite{mihalcea2004textrank}, qui
  est une méthode d'ordonnancement d'unités textuelles à partir de graphe. Les
  graphes sont un moyen naturel de représenter les unités textuelles et les
  relations qui les relient, et ils sont utilisés dans de nombreuses
  applications du TAL~\cite{kozareva2013textgraphs}. Pour l'extraction de
  termes-clés, l'idée est de représenter le document sous la forme d'un graphe
  dans lequel les n\oe{}uds correspondent aux mots et les arêtes correspondent à
  des relations de co-occurrences dans une fenêtre de mots. Un score
  d'importance est alors calculé pour chaque mot selon un principe de
  recommandation, c'est-à-dire un mot est d'autant plus important s'il
  co-occurre avec un grand nombre de mots et si les mots avec lesquels il
  co-occurre sont eux aussi importants. Les mots les plus importants sont
  ensuite assemblés pour générer les termes-clés.

  Dans cet article, nous présentons TopicRank, une méthode non-supervisée
  d'extraction de termes-clés basée sur TextRank. TopicRank groupe les
  termes-clés candidats selon leur appartenance à un sujet, représente le
  document sous la forme d'un graphe complet de sujets, ordonne les sujets selon
  leur importance, puis sélectionne pour chacun des meilleurs sujets son
  candidat le plus représentatif. La notion de sujet est vague, tant elle peut
  exprimer un thème ou un domaine général (par exemple, \og le traitement
  automatique des langues~\fg) ou plus spécifique (par exemple, \og l'extraction
  non-supervisée de termes-clés~\fg). Ici, nous nous intéressons aux sujets les
  plus spécifiques, car ils caractérisent avec plus de précision le contenu d'un
  document. Notre approche possède plusieurs avantages, par rapport à TextRank,
  que nous détaillons ci-dessous~:
  \begin{enumerate}
    \item{Le regroupement des termes-clés candidats en sujets supprime en amont
          les problèmes de redondance dans les termes-clés extraits.}
    \item{Le fait d'utiliser des sujets à la place des mots permet de construire
          un graphe plus compact, de renforcer le poids des arêtes dans le
          graphe et d'améliorer la qualité de l'ordonnancement.}
    \item{La construction d'un graphe complet permet de supprimer le paramètre
          de la fenêtre de mots et de capturer de manière plus précise le niveau
          de relation entre les sujets.}
  \end{enumerate}

  Pour évaluer notre méthode, nous utilisons quatre collections de test aux
  propriétés différentes (nature des documents, taille des documents, langue,
  etc.). Nous comparons TopicRank à trois autres méthodes non-supervisées et
  détaillons l'impact de chacune des contributions que nous proposons.

  % Quel est notre plan ?
  L'article est structuré comme suit. Après un état de l'art des méthodes
  d'extraction automatique de termes-clés en section~\ref{sec:etat_de_l_art},
  nous décrivons le fonctionnement de TopicRank en
  section~\ref{sec:extraction_de_termes_cles_avec_topicrank} et présentons son
  évaluation approfondie en section~\ref{sec:evaluation}. Enfin, nous concluons
  et discutons des travaux futurs dans la
  section~\ref{sec:conclusion_et_perspectives}.
