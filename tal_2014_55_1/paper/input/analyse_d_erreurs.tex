\section{Analyse d'erreurs}
\label{sec:analyse_d_erreurs}
  Dans cette section, nous proposons d'analyser les erreurs de TopicRank. Dans
  un premier temps, nous analysons les sujets que détecte TopicRank, puis dans
  un second temps, nous analysons les termes-clés de référence qui ne sont pas
  extraits par Topic\-Rank.

  \subsection{Analyse des sujets détectés}
  \label{subsec:analyse_des_sujets_générés}
    Dans cette section, nous analysons les groupements en sujets effectués par
    Topic\-Rank afin de déterminer quelles sont les principales causes
    d'erreurs.

    Nous observons des erreurs liées à la sélection des
    termes-clés candidats. Lors de cette étape, certaines unités textuelles sont
    sélectionnées comme candidats à cause des erreurs lors de l'étiquetage
    grammatical. Ces erreurs concernent principalement la détection des
    prépositions composées et la détection des participes. Par exemple, dans la
    phrase \og{}[\dots] elles ne cessent de se développer à travers le monde et
    particulièrement dans les pays dits ``du sud'' [\dots]\fg{}\footnote{Exemple
    issu de l'article d'anthropologie \textit{Le marché parallèle du médicament
    en milieu rural au Sénégal} (\url{http://id.erudit.org/iderudit/014935ar})
    de la collection DEFT.}, le participe passé \og{}dits\fg{} est considéré
    comme un adjectif par l'outils MElt, ce qui entraîne la sélection erronée du
    termes-clés candidat \og{}pays dits\fg{}.

    Nous observons également de nombreuses erreurs lorsque les
    groupements sont déclenchés par un adjectif. Ce sont particulièrement les
    expansions nominales s'effectuant à gauche qui en sont la cause (p.~ex.
    \og{}même langue\fg{} groupé avec \og{}même représentation\fg{}). Parmi les
    expansions nominales s'effectuant à droite, les adjectifs relationnels sont
    moins sujets aux erreurs que les autres adjectifs. Notons tout de même que
    lorsque ces adjectifs sont liés au contexte général du document, ils sont
    très fréquemment utilisés et beaucoup de candidats les contenant sont
    groupés par erreur (p.~ex. \og{}forces économiques\fg{} peut être groupé 
    avec \og{}délabrement économique\fg{} dans un document en économie). Outres
    ces groupements erronés, nous observons aussi de mauvais groupements lorsque
    les candidats ne contiennent que très peu de mots. Pour les candidats de
    deux mots, il ne suffit que d'un seul mot en commun pour les grouper. Ces
    candidats étant très fréquents, ils sont la cause de nombreuses erreurs.

  \subsection{Analyse des faux négatifs}
  \label{subsec:analyse_faux_négatifs}
    Dans cette section, nous analysons les termes-clés de référence qui n'ont
    pas été extraits par TopicRank. Plus particulièrement, nous nous intéressons
    à ceux qui sont présents dans les 10 sujets jugés les plus importants de
    chaque document, mais qui n'ont pas été sélectionnés pour les représenter.
    Nous observons deux sources d'erreurs~:

    La première source d'erreurs est le groupement en sujets. Lorsqu'un sujet
    détecté contient en réalité des termes-clés candidats représentant des
    sujets différents, la stratégie de sélection du meilleur terme-clé dans le
    sujet parvient à sélectionner le termes-clés correct dans certains cas, mais
    elle échoue parfois.

    La seconde source d'erreurs est la spécialisation des termes-clés de
    référence. Nous observons deux problèmes de sous- et sur-spécialisation de
    certains termes-clés extraits vis-à-vis des termes-clés de référence. Dans
    le cas de la sous-spécialisation, nous pouvons citer, par exemple,
    \og{}papillons\fg{} qui est extrait à la place de \og{}papillons
    mutants\fg{}\footnote{Exemple issue de l'article journalistique
    \textit{Fukushima fait muter les papillons}
    (\url{http://fr.wikinews.org/w/index.php?oldid=432477}) de la collection
    WikiNews.}. Bien que ce problème de sous-spécialisation soit identifié,
    l'existance du problème inverse le rend plus difficile à résoudre. Dans le
    cas de la sur-spécialisation, nous pouvons citer, par exemple, \og{}député
    Antoni Pastor\fg{} qui est extrait à la place de \og{}Antoni
    Pastor\fg{}\footnote{Exemple issu de l'article journalistique \textit{Îles
    Baléares : le Parti populaire exclut le député Antoni Pastor pour avoir
    défendu la langue catalane}
    (\url{http://fr.wikinews.org/w/index.php?oldid=479948}) de la collection
    WikiNews.}. La raison principale de ce problème est l'aspect libre de
    l'annotation manuelle des termes-clés. Toutefois, privilégier les
    modifications adjectivales (p.~ex. \og{}mutants\fg{}) et, au contraire,
    éviter les modifications nominales (p.~ex. \og{}député\fg{}) semble être une
    hypothèse à vérifier.

