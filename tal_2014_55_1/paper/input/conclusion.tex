\section{Conclusion et perspectives}
\label{sec:conclusion_et_perspectives}
  Dans ce travail, nous proposons une méthode à base de graphe pour
  l'extraction non supervisée de termes-clés. Cette méthode groupe les
  termes-clés candidats par sujet, détermine quels sont ceux les plus
  importants, puis extrait le terme-clé candidat qui représente le mieux chacun
  des sujets les plus importants. Cette nouvelle méthode offre plusieurs
  avantages vis-à-vis des précédentes à base de graphe. Dans un premier temps,
  le groupement des termes-clés potentiels en sujets distincts permet de
  rassembler des indices utiles auparavant éparpillés. Dans un second temps,
  le choix d'un seul terme-clé pour représenter un des sujets les plus
  importants permet d'extraire un ensemble de termes-clés non redondants ---
  pour $k$ termes-clés extraits, exactement $k$ sujets sont couverts.
  Finalement, le graphe est complet et ne requiert plus le paramétrage d'une
  fenêtre de cooccurrences, contrairement aux autres méthodes à base de graphe.

  Les bons résultats de notre méthode montrent la pertinence d'un groupement en
  sujets des candidats pour ensuite les ordonner. Les expériences
  supplémentaires montrent aussi qu'il est encore possible d'améliorer notre
  méthode en proposant une nouvelle stratégie de sélection du terme-clé candidat
  le plus représentatif d'un sujet (pour un gain maximum allant de 4,2 à 15
  points de f-score).

  Nous avons aussi effectué une analyse d'erreurs à partir de laquelle trois
  perspectives de travaux futurs émergent~:

  Nous avons pour objectif d'améliorer la sélection des
  termes-clés candidats. Aussi, des méthodes empruntées à d'autres domaines du
  TAL peuvent être appliquées. Il semble, par exemple, pertinent d'évaluer
  l'apport des méthodes d'extraction
  terminologiques~\cite{castellvi2001automatictermdetection} pour la sélection
  des termes-clés candidats.
  
  Nous envisageons également d'améliorer le groupement en sujets,
  car celui-ci est très naïf et ne tient compte ni de la synonymie, ni de
  l'ambiguïté des mots. Aussi, l'usage du
  radical~\cite{porter1980suffixstripping} des mots n'est pas sans introduire du
  bruit lié à certains faux positifs (p.~ex. \og{}\underline{empir}e\fg{} et
  \og{}\underline{empir}ique\fg{}). L'ajout de connaissances concernant les
  synonymes permettrait de créer des sujets plus complets et une étape de
  désambiguïsation éviterait un groupement systématique des termes-clés
  candidats ayant un ou plusieurs mots en commun. Quant à la méthode de
  \newcite{porter1980suffixstripping}, nous envisageons de la remplacer par une
  méthode de lemmatisation. Enfin, d'un point de vue plus technique, il faudrait
  explorer différentes méthodes de groupement, dont le groupement spectral
  (\textit{spectral clustering}) qui, dans d'autres travaux portant sur
  l'extraction automatique de termes-clés~\cite{liu2009keycluster}, montre de
  meilleures performances que le groupement hiérarchique agglomératif.

  Enfin, une étude détaillée des caractéristiques des termes-clés pourrait
  orienter notre travail vers des critères plus efficaces pour la définition
  d'une stratégie \og{}optimale\fg{} de sélection du terme-clé le plus
  représentatif d'un sujet. Un apprentissage supervisé à partir de certains
  critères est aussi envisageable, au même titre que l'usage de méthodes
  d'optimisation, telles que celle utilisée par
  \newcite{ding2011binaryintegerprogramming} dans leur méthode d'extraction
  automatique de termes-clés.

