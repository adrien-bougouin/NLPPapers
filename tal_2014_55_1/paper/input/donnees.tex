\section{Présentation et analyse des données}
\label{sec:presentation_et_analyse_des_donnees}
  Les termes-clés candidats sont les unités textuelles pouvant être extraites
  comme termes-clés d'un document. Leurs propriétés doivent alors correspondre
  aux propriétés caractéristiques des termes-clés donnés par des humains. Avant
  de présenter notre méthode d'extraction de termes-clés en
  section~\ref{sec:extraction_de_termes_cles_avec_topicrank}, nous présentons
  les données utilisées et en déduisons les propriétés des termes-clés annotés
  manuellement pour ces données.

  \subsection{Données pour l'extraction de termes-clés}
  \label{subsec:corpus_pour_l_extraction_de_termes_cles}
    Les collections de documents annotés en termes-clés sont utilisées soit pour
    l'apprentissage, soit pour l'évaluation (test) des méthodes d'extraction
    automatique de termes-clés. Les annotations en termes-clés sont fournies par
    les auteurs des documents, des lecteurs (novices ou experts) ou bien les
    deux. Dans les travaux présentés dans cet article, nous utilisons trois
    collections dont le type d'annotateur, le domaine, la langue et la taille
    des documents diverge.

    \textbf{DUC}~\cite{over2001duc} est une collection issue des données (en
    anglais) de la campagne d'évaluation DUC-2001. Cette campagne d'évaluation
    concerne les méthodes de résumé automatique, elle ne contient donc
    originellement pas d'annotations en termes-clés. Cependant, les 308 articles
    journalistiques de la partie test de DUC-2001 sont annotés par
    \newcite{wan2008expandrank}. Selon nos besoin, nous divisons ces 308
    documents en un ensemble de 208 documents d'entraînement et 100 documents de
    test.

    \textbf{SemEval}~\cite{kim2010semeval} est la collection fournie lors de la
    campagne d'évaluation SemEval-2010 pour la tâche d'extraction automatique de
    termes-clés. Cette collection contient 284 articles scientifiques
    (conférences et ateliers issus de la librairie numérique ACM (en anglais).
    La collection est répartie en trois sous-ensembles, un ensemble de 40
    documents d'essais (que nous n'utilisons pas), un ensemble de 144 documents
    d'entraînement et un ensemble de 100 documents de test. Les termes-clés
    manuellement associés à ces documents sont donnés par les auteurs et des
    lecteurs.

    \textbf{DEFT}~\cite{paroubek2012deft} est la collection fournie lors de la
    campagne d'évaluation DEFT-2012 pour la tâche d'extraction automatique de
    termes-clés. Celle-ci contient 234 documents français issus de quatre revues
    de Sciences Humaines et Sociales. La collection est divisée en deux
    sous-ensembles, un ensemble d'entraînement contenant 141 documents et un
    ensemble de test contenant 93 documents. Seuls les termes-clés d'auteurs
    sont disponibles.

  \subsection{Termes-clés de référence}
  \label{subsec:termes_cles_de_reference}
    % Que fait-on ?
    % Quelles statistiques extrait-on ?
    En utilisant les collections de données présentées plus haut, nous inférons
    maintenant des propriétés relatives aux termes-clés donnés par des humains.
    Pour ce faire, nous utilisons des statistiques extraites à partir des
    données d'entraînement\footnote{Inférer des propriétés seulement à partir
    des données d'entraînement permet de ne pas biaiser nos expériences avec des
    connaissances extraites à partir des documents analysés.}. Le
    tableau~\ref{tab:donnees_d_entrainement} présente les statistiques extraites
    des données d'entraînement. Il comporte les informations concernant les
    documents (langue, nature, taille, etc.) et celles concernant leurs
    termes-clés (nombre de mots, parties du discours, etc.). Pour le calcul de
    la proportion de termes-clés contenant un mot d'une catégorie grammaticale
    donnée, les termes-clés sont automatiquement étiquetés en parties du
    discours. Cet étiquetage ayant lieu hors contexte, il comporte de nombreuses
    erreurs. Dans un soucis d'exactitude, tous les termes-clés étiquetés sont
    manuellement vérifiés et corrigés si nécessaire.
    \begin{table}
      \centering
      \begin{tabular}{@{~~}rrccc@{~~}}
        \toprule
        & \multirow{2}{*}[-2pt]{\textbf{Statistique}} & \multicolumn{3}{c}{\textbf{Collection de documents}}\\
        \cmidrule{3-5}
        & & DUC & SemEval & DEFT\\
        \midrule
        \multirow{6}{*}[-2pt]{\begin{sideways}\textbf{Documents}\end{sideways}} & Langue & Anglais & Anglais & Français\\
        & Nature & Journalistique & Scientifique & Scientifique\\
        & Documents & 208 & 144 & 141\\
        & Mots/document & 912.0 & 5134.6 & 7276.7\\
        & Termes-clés/document & 8.1 & 15.4 & 5.4\\
        & Termes-clés manquants & 3.9\% & 13.5\% & 18.2\%\\
        \addlinespace[1.5\defaultaddspace]
        \multirow{5}{*}[-2pt]{\begin{sideways}\textbf{Termes-clés}\end{sideways}} & Uni-grammes & 17.1\% & 20.2\% & 60.2\%\\
        & Bi-grammes & 60.8\% & 53.4\% & 24.5\%\\
        & Tri-grammes & 17.8\% & 21.3\% & $~~$8.8\%\\
        & Quadri-grammes & $~~$3.0\% & $~~$3.9\% & $~~$4.2\%\\
        & N-grammes (N $\geq$ 5) & $~~$1.3\% & $~~$1.3\% & $~~$2.4\%\\
        \addlinespace[1.5\defaultaddspace]
        \multirow{8}{*}[-1.5pt]{\begin{sideways}\textbf{Termes-clés complexes}\end{sideways}} & Contient un nom & 94.5\% & 98.7\% & 93.3\%\\
        & Contient un nom propre & 17.1\% & $~~$4.3\% & $~~$6.9\%\\
        & Contient un adjectif & 50.0\% & 50.2\% & 65.5\%\\
        & Contient un verbe & $~~$1.0\% & $~~$4.0\% & $~~$1.0\%\\
        & Contient un adverbe & $~~$1.6\% & $~~$0.7\% & $~~$1.3\%\\
        & Contient une préposition & $~~$0.3\% & $~~$1.5\% & 31.2\%\\
        & Contient un déterminant & $~~$0.0\% & $~~$0.0\% & 20.4\%\\
        & Autres catégories & $~~$1.5\% & $~~$2.5\% & 11.8\%\\
        \addlinespace[.5\defaultaddspace]
        \bottomrule
      \end{tabular}
      \caption{Statistiques sur les données d'entraînement utilisées. En accord
               avec l'évaluation effectuée lors de nos expériences, la
               proportion de termes-clés manquant est calculée à partir de leur
               racine. \label{tab:donnees_d_entrainement}}
    \end{table}

    % Quelles propriétés observe-t-on ?
    La deuxième partie du tableau~\ref{tab:donnees_d_entrainement} montre que la
    majorité ($\simeq$ 80\%) des  termes-clés associés aux documents sont des
    uni-grammes ou des bi-grammes. Ceci indique que malgré le fait qu'ils
    doivent être suffisamment informatifs pour véhiculer un sujet ou une idée,
    les termes-clés comportent le moins d'informations possibles.

    \begin{property}
      Les termes-clés contiennent le moins d'informations possibles permettant
      de représenter un sujet ou une idée (par exemple, \og littérature
      canadienne-française~\fg\ à la place de \og littérature
      canadienne-française de la seconde moitié du XXe siècle~\fg).
      % meta_2003_006958ar
      \label{prop:informativite}
    \end{property}

    La dernière partie du tableau~\ref{tab:donnees_d_entrainement} donne la
    proportion de termes-clés complexes (bi-grammes ou plus) qui contiennent au
    moins un mot ayant une classe grammaticale supposée pertinente dans un
    terme-clé\footnote{Sachant que les termes-clés ne contenant qu'un seul mot
    sont majoritairement des noms ou des noms propres, nous ne nous focalisons
    que sur les termes-clés de plus d'un mot.}. Que ce soit en anglais, ou en
    français, presque tous les termes-clés contiennent des noms et, dans la
    moitié des cas, ceux-ci sont modifiés par un ou plusieurs adjectifs.
    Toujours dans les deux langues, l'emploi de verbes ou d'adverbes est très
    rare. Le français ce différencie de l'anglais par l'usage de prépositions
    (parfois accompagnées de déterminants). Ceci s'explique par le fait que,
    contrairement au français, l'anglais permet de former des variantes qui sont
    privilégiées aux formes prépositionnelles (par exemple, \og conservation de
    la nature~\fg\ se traduit \og conservation of nature~\fg\ ou \og nature
    conservation~\fg). Enfin, le domaine particulier de DEFT fait que la
    catégorie grammaticale (inattendue) \textit{mot étranger} n'est pas rare.
    Pour plus d'information sur les classes grammaticales des mots composant un
    terme-clé, le tableau~\ref{tab:patrons_syntaxiques} donne les cinq patrons
    syntaxiques les plus fréquents pour l'anglais et le français.
    \begin{table}
      \centering
      \begin{tabular}{@{~~}rl@{~~}l@{~~}l@{~~}l@{~~}p{.65\linewidth}@{~~}}
        \toprule
        & \multicolumn{4}{l}{\textbf{Patron syntaxique}} & \textbf{Exemple}\\
        \midrule
        \multirow{5}{*}[-2pt]{\begin{sideways}\textbf{Anglais}\end{sideways}} & Nc & Nc & & & hurricane expert -- expert en ouragans\\ % AP880409-0015
        & A & Nc & & & turbulent summer -- été agité\\ % AP88049-0015
        & Nc & & & & storms -- tempête\\ % AP880409-0015
        & A & Nc & Nc & & annual hurricane forecast -- prévision annuelle d'ouragans\\ % AP880409-0015
        & Nc & Nc & Nc & & hurricane reconnaissance flights -- vols de reconnaissance des ouragans\\ % AP890529-0030
        \addlinespace[1.5\defaultaddspace]
        \multirow{5}{*}[-2pt]{\begin{sideways}\textbf{Français}\end{sideways}} & Nc & & & & patrimoine\\ % as_2002_007048ar
                                                                                        & Nc & A & & & tradition orale\\ % as_2002_007048ar
        & Np & & & & Indonésie\\ % as_2001_000235ar
        & Nc & Sp & D & Nc & conservation de la nature\\ % as_2005_011742ar
        & Nc & Sp & Nc & & traduction en anglais\\ % meta_2003_006958ar
        \bottomrule
      \end{tabular}
      \caption{Séquences de parties du discours (au format Multex) les plus
               fréquentes, pour l'anglais et le français.
               \label{tab:patrons_syntaxiques}}
    \end{table}

    \begin{property}
      Les termes-clés sont principalement des noms (par exemple,
      \og patrimoine~\fg, ou des noms propres (par exemple, \og Indonésie~\fg),
      modifiés par des adjectifs lorsque nécessaire (par exemple, \og tradition
      \underline{orale}~\fg).
      \label{prop:syntaxe}
    \end{property}

    % Quelles sont, parmi les méthodes d'extraction de candidats présentées,
    % celles qui respectent le plus ces propriétés ?
    Parmi les types de termes-clés candidats présentées dans la
    section~\ref{sec:etat_de_l_art}, les chunks nominaux, de part leur
    minimalisme et leur syntaxe, ainsi que les termes, lorsque les verbes et
    adverbes sont exclus, sont les candidats qui respectent le mieux les
    propriétés~\ref{prop:informativite}~et~\ref{prop:syntaxe}. Les n-grammes,
    quant à eux, contiennent de nombreux candidats satisfaisant à la fois les
    propriétés~\ref{prop:informativite}~et~\ref{prop:syntaxe}, mais leur
    exhaustivité est telle que de nombreux candidats sont inconsistant avec
    ces deux propriétés.

