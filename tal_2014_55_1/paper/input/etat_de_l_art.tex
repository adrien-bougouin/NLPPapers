\section{État de l'art}
\label{sec:etat_de_l_art}
  TopicRank faisant partie des méthodes non-supervisées, l'état de l'art
  présenté ici se focalise sur cette catégorie de méthodes.
  % Quel est le fonctionnement général des méthodes d'extraction automatique de
  % termes-clés ?
  L'extraction automatique non-supervisée de termes-clés est une tâche
  décomposée en général en quatre étapes. Les méthodes non-supervisées traitent
  les documents généralement un à un. Ils sont tout d'abord enrichis
  linguistiquement, c'est-à-dire segmentés en phrases, segmentés en mots 
  et étiquetés en parties du discours. Des termes-clés candidats en sont
  extraits, puis ordonnés afin de ne sélectionner que les plus pertinents (voir
  la figure~\ref{fig:etapes_de_l_extraction_de_termes_cles}). L'extraction des
  termes-clés candidats et leur ordonnancement sont les deux étapes auxquelles
  nous nous intéressons dans cet article. L'ordonnancement des termes-clés
  candidats est le c\oe{}ur de la tâche d'extraction de termes-clés, et ses
  performances dépendent de la qualité des candidats préalablement extraits.
  \begin{figure}
    \tikzstyle{io}=[
      ellipse,
      minimum width=5cm,
      minimum height=2cm,
      fill=green!20,
      draw=green!33,
      transform shape,
      font={\huge}
    ]
    \tikzstyle{component}=[
      text centered,
      thick,
      rectangle,
      minimum width=11cm,
      minimum height=2cm,
      fill=cyan!20,
      draw=cyan!33,
      transform shape,
      font={\huge\bfseries}
    ]

    \centering
    \begin{tikzpicture}[thin,
                        align=center,
                        scale=.45,
                        node distance=2cm,
                        every node/.style={text centered, transform shape}]
      \node[io] (document) {document};
      \node[component] (preprocessing) [right=of document] {Prétraitement Linguistique};
      \node[component] (candidate_extraction) [below=of preprocessing] {Extraction des candidats};
      \node[component] (candidate_classification_and_ranking) [below=of candidate_extraction] {Ordonnancement des candidats};
      \node[component] (keyphrase_selection) [below=of candidate_classification_and_ranking] {Sélection des termes-clés};
      \node[io] (keyphrases) [right=of keyphrase_selection] {termes-clés};

      \path[->, thick] (document) edge (preprocessing);
      \path[->, thick] (preprocessing) edge (candidate_extraction);
      \path[->, thick] (candidate_extraction) edge (candidate_classification_and_ranking);
      \path[->, thick] (candidate_classification_and_ranking) edge (keyphrase_selection);
      \path[->, thick] (keyphrase_selection) edge (keyphrases);
    \end{tikzpicture}
    \caption{Les quatre principales étapes de l'extraction automatique de
             termes-clés. \label{fig:etapes_de_l_extraction_de_termes_cles}}
  \end{figure}

  \subsection{Extraction de termes-clés candidats}
  \label{subsec:extraction_de_termes_cles_candidats}
    % Quel est l'objectif ?
    L'objectif de l'extraction de termes-clés candidats est de réduire l'espace
    des solutions possibles, c'est-à-dire toutes les unités textuelles pouvant
    être extraites du document analysé, aux seules unités textuelles ayant des
    particularités similaires à celles des termes-clés tels qu'ils peuvent être
    donnés par des humains. Il y a deux avantages à cela. Le premier, très
    évident, est la réduction du temps de calcul nécessaire à l'extraction des
    termes-clés. Le second avantage est la suppression en amont d'unités
    textuelles non pertinentes, ces dernières pouvant très négativement affecter
    les performances de l'ordonnancement. Pour distinguer les différents
    candidats extraits, nous définissons deux catégories~: les candidats
    positifs, qui sont présents en tant que termes-clés de référence dans nos
    collections de données, et les candidats non positifs. Parmi les candidats
    non positifs, nous distinguons deux sous-catégories~: les candidats porteurs
    d'indices de différentes natures pouvant influencer la promotion de
    candidats positifs (par exemple, la présence des candidats \og alerte
    rouge~\fg, \og alerte jaune~\fg\ et \og alerte orange~\fg\ influence
    l'extraction du candidat positif \og alerte~\fg\ en tant que terme-clé,
    dans l'article \textit{44960} de notre collection WikiNews -- voir la
    section~\ref{sec:evaluation}) et les candidats non pertinents, que nous
    considérons comme des erreurs.

    % Quels sont les différentes méthodes utilisées pour extraire les
    % termes-clés candidats ?
    Dans les travaux précédents, trois méthodes d'extraction de candidats sont
    classiquement utilisées~: l'extraction de n-grammes, de chunks nominaux, et
    d'unités textuelles respectant certains patrons grammaticaux.

    Les n-grammes sont toutes les séquences ordonnées de $n$ mots, avec
    $n \in 1..m$, où $m$ vaut généralement
    3~\cite{witten1999kea,turney1999learningalgorithms,hulth2003keywordextraction}.
    Leur extraction est très exhaustive, elle fournit un grand nombre de
    termes-clés candidats, maximisant ainsi la quantité de candidats positifs,
    la quantité de candidats porteurs d'indices utiles, mais aussi la quantité
    de candidats non pertinents. Pour pallier en partie ce problème, il est
    courant d'utiliser une liste de mots outils pour filtrer les candidats. Les
    mots outils regroupent les mots fonctionnels de la langue (conjonctions,
    prépositions, etc.) et les mots courants (\og particulier~\fg, \og près~\fg,
    \og beaucoup~\fg, etc.). Ainsi, un n-gramme contenant un mot outil en début
    ou en fin n'est pas considéré comme un terme-clé candidat. Malgré son aspect
    bruité, ce type d'extraction est encore largement utilisé parmi les méthodes
    supervisées~\cite{witten1999kea,turney1999learningalgorithms,hulth2003keywordextraction}.
    En effet, la phase d'apprentissage de celles-ci les rend moins sensibles aux
    éventuels candidats erronés (bruit) par rapport aux méthodes supervisées.

    Les chunks nominaux sont des syntagmes non récursifs dont la tête est un
    nom, accompagné de ses éventuels déterminants et modifieurs usuels. Ce sont
    des segments linguistiquement définis rendant leur extraction plus fiable
    que celle des n-grammes. Les expériences menées par
    \newcite{hulth2003keywordextraction} et \newcite{eichler2010keywe} avec les
    chunks nominaux montrent une amélioration des performances vis-à-vis de
    l'usage des n-grammes. Cependant, \newcite{hulth2003keywordextraction}
    constate qu'en tirant parti de l'étiquetage en parties du discours des
    termes-clés candidats, l'extraction supervisée de termes-clés à partir de
    n-grammes donne des performances au-dessus de celles obtenues avec les
    chunks nominaux. L'usage de ce trait supplémentaire a pour effet de filtrer
    les n-grammes grammaticalement incorrects, favorisant alors l'extraction des
    candidats positifs, qui sont plus nombreux que ceux parmi les chunks
    nominaux.

    L'extraction d'unités textuelles à partir de patrons grammaticaux prédéfinis
    permet de contrôler avec précision la nature et la grammaticalité des
    candidats extraits. À l'instar des chunks nominaux, leur extraction est plus
    fondée linguistiquement que celle des n-grammes filtrés, et comparée à eux,
    elle fournit un plus grand nombre de candidats positifs. Dans ses travaux,
    \newcite{hulth2003keywordextraction} choisie d'extraire des candidats avec
    les patrons des termes-clés de références les plus fréquents (plus de 10
    occurrences) dans sa collection d'apprentissage, tandis que d'autres
    chercheurs tels que \newcite{wan2008expandrank} et
    \newcite{hassan2010conundrums} se concentrent uniquement sur les plus
    longues séquences de noms (noms propres inclus) et d'adjectifs. Pour des
    méthodes non-supervisées telles que la nôtre, l'extraction des séquences de 
    noms et d'adjectifs est intéressante, car elle ne nécessite ni données
    supplémentaires, ni adaptation particulière pour une langue donnée (tel que
    c'est le cas pour l'extraction des chunks nominaux, par exemple).

    % Que veut-on apporter ?
    Dans le but d'améliorer la qualité des candidats extraits à partir
    d'articles scientifiques, \newcite{kim2009termextraction} proposent un
    filtrage des candidats en fonction de leur spécificité vis-à-vis du document
    analysé. Cette spécificité est déterminée par rapport à la fréquence d'un
    candidat dans le document et le nombre de documents d'une collection dans
    lesquels il apparaît~\cite[TF-IDF]{jones1972tfidf}. Intuitivement, un
    candidat très fréquent dans le document analysé est d'autant plus spécifique
    à celui-ci s'il est présent dans très peu d'autres documents. Cette approche
    est intéressante, mais elle requiert des documents  supplémentaires et la
    définition d'un seuil pour le filtrage. Dans le cas de TopicRank, nous
    tentons de nous abstraire de l'usage d'autres documents que celui qui est
    analysé, cette méthode d'extraction de candidats n'est donc pas compatible
    avec nos objectifs.

  \subsection{Ordonnancement des termes-clés candidats}
  \label{subsec:ordonnancement_des_termes_cles_candidats}
    % Quel est l'objectif ?
    L'étape d'ordonnancement intervient après l'extraction des termes-clés
    candidats. Son rôle est de déterminer quels sont, parmi les candidats, les
    termes-clés d'un document.
    % Quels sont les différentes méthodes non-supervisées existantes pour
    % l'extraction de termes-clés ?
    % Quels sont les inconvénients des méthodes actuelles ?
    % Que veut-on apporter ?
    Les méthodes non-supervisées d'extraction automatique de termes-clés
    emploient des techniques très différentes, allant du simple usage de mesures
    fréquentielles~\cite{jones1972tfidf,paukkeri2010likey} à l'utilisation de
    modèles de langues obtenus à partir de données
    non-annotées~\cite{tomokiyo2003languagemodel}, en passant par la
    construction d'un graphe de co-occurrences~\cite{mihalcea2004textrank}.
    Puisque la méthode que nous présentons dans cet article est une méthode dite
    \og à base de graphe~\fg, nous décrivons ici uniquement les travaux
    effectués au sujet de cette catégorie de méthodes.

    \newcite{mihalcea2004textrank} proposent TextRank, une méthode
    d'ordonnancement d'unités textuelles à partir d'un graphe. Utilisés dans de
    nombreuses applications du TAL~\cite{kozareva2013textgraphs}, les graphes
    ont l'avantage de présenter de manière simple et efficace les unités
    textuelles d'un document et les relations qu'elles entretiennent entre
    elles. De plus, ils bénéficient de nombreuses études théoriques donnant lieu
    à des outils et algorithmes capables de résoudre de nombreux problèmes du
    TAL, tels que le résumé automatique~\cite{wan2007iterativereinforcement}, la
    compression multi-phrase~\cite{boudin2013multisentencecompression} et la
    désambiguïsassion de texte~\cite{schwab2013desambiguisation}. Dans le cas de
    TextRank, les n\oe{}uds du graphe sont les mots du document et les arrêtes
    sont leurs relations de co-occurrences dans une fenêtre de 2 mots. Un score
    d'importance est calculé pour chaque mot à partir de l'algorithme 
    PageRank~\cite{brin1998pagerank} qui est issu de la mesure de centralité des
    vecteurs propres. Le principe utilisé est celui de la recommandation, ou du
    vote, c'est-à-dire un mot est d'autant plus important qu'il co-occurre avec
    un grand nombre de mots et si les mots avec lesquels il co-occurre sont eux
    aussi important. Les mots les plus importants sont considérés comme des
    mots-clés. Ces mots-clés sont marqués dans le document et les plus longues
    séquences de mots-clés adjacents sont extraites en tant que termes-clés.
    Dans cette méthode, la précision de l'ordonnancement dépend de la qualité du
    graphe qui elle-même dépend de la fenêtre de co-occurrence. Dans nos
    travaux, nous créons un graphe dont la qualité ne dépend pas d'un paramètre
    tel que cette fenêtre de co-occurrence.

    \newcite{wan2008expandrank} modifient TextRank et proposent SingleRank. Dans
    un premier temps, leur méthode augmente la précision de l'ordonnancement
    grâce à une pondération des liens de co-occurrence avec le nombre de
    co-occurrences des mots liés (deux mots qui co-occurrent deux fois, par
    exemple, sont liés par une arête dont le poids vaut 2).
    %Ce nouvel ordonnancement utilise plus d'informations présentes dans le
    %document analysé et est dotant plus efficace quand le document analysé est
    %de grande taille.
    Dans un second temps, les termes-clés ne sont plus générés, mais ordonnés à
    partir de la somme du score d'importance des mots qui les composent. Cette
    nouvelle méthode donne dans la majorité des cas des résultats meilleurs que
    ceux de TextRank. Cependant, faire la somme du score d'importance des
    mots pour ordonner les candidats est une approche maladroite. En effet, cela
    a pour effet de faire monter dans le classement des candidats redondants.
    Ainsi, dans le document \textit{as\_2002\_000700ar} de la collection DEFT
    (voir la section~\ref{sec:evaluation}), le candidat positif
    \og bio-politique~\fg\ est classé neuvième, alors que les autres candidats
    qui contiennent entre autres \og bio-politique~\fg\ occupent les
    classements 2 à 8. Dans nos travaux, nous n'ordonnons pas les termes-clés
    candidats en fonction des mots qu'ils contiennent et évitons ainsi le
    problème rencontré avec SingleRank.

    Toujours dans l'optique d'améliorer l'efficacité de l'ordonnancement,
    \newcite{wan2008expandrank} étendent SingleRank en utilisant des documents
    similaires au document analysé. Sélectionnés à partir d'une vaste collection
    couvrant plusieurs thématiques, les documents similaires fournissent plus de
    données relatives aux mots du document analysé et aux relations qu'ils
    entretiennent. La méthode consiste à utiliser les co-occurrences observées
    dans les documents similaires pour ajouter ou renforcer des liens dans le
    graphe. Cette approche donne des résultats au delà de ceux de SingleRank,
    mais il est important de noter que ses performances sont fortement liées à
    la disponibilité de documents similaires à celui qui est analysé.

    À l'instar de \newcite{wan2008expandrank},
    \newcite{tsatsaronis2010semanticrank} tentent d'améliorer TextRank. Dans
    leur méthode, ils créent et pondèrent une arrête entre deux mots si et
    seulement si ceux-ci sont sémantiquement liés selon deux mesures définies à
    partir de WordNet~\cite{miller1995wordnet} et de
    Wikipedia~\cite{milne2008wikipediasemanticrelatedness}. Les expériences
    menées par les auteurs montrent de moins bons résultats que TextRank.
    Toutefois, en biaisant l'ordonnancement en faveur des mots apparaissant dans
    le titre du document analysé ou en ajoutant le poids TF-IDF des mots dans le
    calcul de l'importance des mots, leur méthode est capable de donner de
    meilleurs résultats que TextRank. Ceci suggère qu'il existe des traits, tels
    que la première apparition dans le texte et la spécificité des candidats,
    qui peuvent influencer positivement l'ordonnancement. Dans notre approche,
    nous estimons que l'ordonnancement des termes-clés candidats vis-à-vis des
    sujets auxquels ils appartiennent est un moyen de prendre en compte leur
    spécificité dans le document analysé. En effet, un candidat appartenant à un
    sujet important d'un document est intuitivement plus spécifique à ce
    document qu'un candidat appartenant à un sujet moins important.

    L'usage de sujets dans le processus d'ordonnancement de TextRank est à
    l'origine proposé par \newcite{liu2010topicalpagerank}. Reposant sur un
    modèle LDA~\cite[Latent Dirichlet Allocation]{blei2003lda}, leur méthode
    effectue des ordonnancements biaisés par les sujets du document, puis
    fusionne les rangs des mots dans ces différents ordonnancements afin
    d'obtenir un rang global pour chaque mot. Dans notre travail, nous émettons
    aussi l'hypothèse que le sujet auquel appartient une unité textuelle doit
    jouer un rôle majeur dans le processus d'ordonnancement. Cependant, nous
    tentons de nous abstraire de l'usage de documents supplémentaires et
    n'utilisons pas le modèle LDA. De plus, il nous semble plus judicieux, d'un
    point de vue complexité, d'effectuer un seul ordonnancement.

