\section{Introduction}
\label{sec:introduction}
  Un terme-clé, couramment appelé mot-clé, est un mot ou une expression
  polylexicale permettant de caractériser une partie du contenu d'un document.
  Groupés ensemble, les termes-clés d'un document permettent de définir les
  principaux sujets abordés dans un document (cf. exemple
  figure~\ref{fig:recurrent_example}). Ils sont alors utiles dans de
  nombreuses applications du Traitement Automatique des Langues (TAL), telles
  que  le résumé automatique~\cite{avanzo2005keyphrase}, la compression
  multi-phrase~\cite{boudin2013multisentencecompression}, la classification de
  documents~\cite{han2007webdocumentclustering}, ou l'indexation automatique de
  documents~\cite{medelyan2008smalltrainingset}, qui nous intéresse plus
  particulièrement. Pourtant, de nombreux documents, tels que ceux disponibles
  sur Internet, n'en sont pas accompagnées et la quantité de documents à traiter
  est aujourd'hui trop importante pour que l'annotation de leurs termes-clés
  soit effectuée manuellement. C'est pourquoi de nombreux chercheurs se penchent
  sur la problématique de l'extraction automatique de termes-clés, en témoigne
  la quantité grandissante de travaux scientifiques à ce
  sujet~\cite{hasan2014state_of_the_art} ainsi que l'émergence de  de campagnes
  d'évaluation des méthodes d'extraction automatique de
  termes-clés~\cite{kim2010semeval,paroubek2012deft}.

  \begin{figure}[h]
    \begin{minipage}{.975\linewidth}
      \cornersize{0.1}\Ovalbox{
        \parbox{.99\linewidth}{\small
          \textit{\textbf{Météo} du 19 \textbf{août 2012} : \textbf{\large
          alerte} à la \textbf{\normalsize canicule} sur la \textbf{\normalsize
          Belgique} et le \textbf{\large Luxembourg}}\\

          \setlength\parindent{12pt}\vspace{-0.5em}
          A l'exception de la province de \textbf{\large Luxembourg}, en
          \textbf{\large alerte} jaune, l'ensemble de la \textbf{\normalsize
          Belgique} est en vigilance \textbf{\normalsize orange} à la
          \textbf{\normalsize canicule}. Le \textbf{\large Luxembourg} n'est pas
          épargné par la vague du \textbf{\normalsize chaleur} : le nord du pays
          est en \textbf{\large alerte} \textbf{\normalsize orange}, tandis que
          le sud a était placé en \textbf{\large alerte} rouge.

          En \textbf{\normalsize Belgique}, la \textbf{\normalsize température}
          n'est pas descendue en dessous des 23°C cette nuit, ce qui constitue
          la deuxième nuit \textbf{\normalsize la plus chaude} jamais
          enregistrée dans le royaume. Il se pourrait que ce dimanche soit la
          journée \textbf{\normalsize la plus chaude} de l'année. Les
          \textbf{\normalsize températures} seront comprises entre 33 et 38°C.
          Une légère brise de côte pourra faiblement rafraichir l'atmosphère.
          Des orages de \textbf{\normalsize chaleur} sont a prévoir dans la
          soirée et en début de nuit.

          Au \textbf{\large Luxembourg}, le mercure devrait atteindre 32°C ce
          dimanche sur l'Oesling et jusqu'à 36°C sur le sud du pays, et 31 à
          32°C lundi. Une baisse devrait intervenir pour le reste de la semaine.
          Néanmoins, le record d'août 2003 (37,9°C) ne devrait pas être
          atteint.%\\
%
%          \setlength\parindent{0pt}\vspace{-0.5em}
%          \underline{Termes-clés de référence~:} Luxembourg~; alerte~; météo~;
%          Belgique~; août 2012~; chaleur~; température~; chaude~; canicule~;
%          orange~; la plus chaude.
        \setlength\parindent{0pt}}
      }
    \end{minipage}
    \caption{Exemple de termes-clés mis en évidence dans un article
             journalistique du site Web WikiNews.}
             \label{fig:recurrent_example}
  \end{figure}

  L'extraction automatique de termes-clés consiste à sélectionner dans un
  document les unités textuelles les plus importantes parmi un ensemble de
  termes-clés candidats sélectionnés dans le document. Les termes-clés candidats
  doivent être de nature similaire à celles des termes-clés assignés par des
  humains. Ils sont sélectionnés à partir d'hypothèses simples (contient des
  noms, contient des adjectifs, etc.) et ne remplissent pas nécessairement les
  critères d'un terme tel que défini en extraction terminologique. Parmi les
  différentes méthodes d'extraction automatique de termes-clés proposées dans la
  littérature, deux grandes catégories émergent~: les méthodes supervisées et
  les méthodes non-supervisées. Les premières tirent profit d'une collection de
  documents annotés en termes-clés et apprennent, en amont, un modèle de
  classification binaire permettant, par la suite, de déterminer quels sont
  parmi les termes-clés candidats d'un document ceux qui sont des termes-clés et
  ceux qui n'en sont pas. Quant aux secondes, les méthodes non-supervisées,
  celles-ci attribuent un score d'importance à chaque terme-clé candidat en
  fonction de divers indicateurs comme par exemple la fréquence, les relations
  de cooccurrences ou la position dans le document. Du fait de leur phase
  d'apprentissage, les méthodes supervisées sont en général plus performantes
  que les méthodes non-supervisées. Cependant, la faible quantité de documents
  annotés en termes-clés disponibles couplée à la forte dépendance des modèles
  de classification vis-à-vis du type de documents à partir desquels ils sont
  appris, poussent les chercheurs à s'intéresser de plus en plus aux méthodes
  non-supervisées~\cite{hassan2010conundrums}.

  Les méthodes d'extraction de termes-clés non-supervisées les plus étudiées
  sont sans conteste celles basée sur TextRank~\cite{mihalcea2004textrank}, qui
  est une méthode d'ordonnancement d'unités textuelles à partir d'un graphe. Un
  graphe est un moyen naturel de représenter les unités textuelles et les
  relations qu'elles entretiennent. Ils sont utilisés dans de nombreuses
  applications du TAL~\cite{kozareva2013textgraphs}. Dans le cadre de
  l'extraction de termes-clés, le principe est de représenter le document sous
  la forme d'un graphe dans lequel les n\oe{}uds correspondent aux mots et les
  arêtes correspondent à leurs relations de co-occurrences dans une fenêtre de
  mots. Un score d'importance est alors calculé pour chaque mot selon un
  principe de recommandation~: un mot est d'autant plus important
  s'il co-occurre avec un grand nombre de mots et si les mots avec lesquels il
  co-occurre sont eux aussi importants. Enfin, les mots les plus importants
  servent à générer des termes-clés pour le document.

  Dans cet article, nous présentons TopicRank \footnote{Cette article est une
  version étendu de \cite{bougouin2013topicrank}.}, une méthode non-supervisée
  d'extraction de termes-clés basée sur TextRank. TopicRank groupe les
  termes-clés candidats selon leur appartenance à un sujet, représente le
  document sous la forme d'un graphe complet de sujets, ordonne les sujets par
  importance selon le principe de recommandation, puis sélectionne pour chacun
  des meilleurs sujets son candidat le plus représentatif. La notion de sujet
  est vague, tant elle peut exprimer un thème ou un domaine général (par
  exemple, \og le traitement automatique des langues~\fg) ou plus spécifique
  (par exemple, \og l'extraction non-supervisée de termes-clés~\fg). Ici, nous
  nous intéressons aux sujets les plus spécifiques, car ils caractérisent avec
  plus de précision le contenu d'un document. Notre approche possède plusieurs
  avantages, par rapport à TextRank, que nous détaillons ci-dessous~:
  \begin{enumerate}
    \item{Le regroupement des termes-clés candidats en sujets supprime en amont
          les problèmes de redondance dans les termes-clés extraits.}
    \item{L'usage de sujets à la place de mots permet de construire un graphe
          plus compact, de renforcer le poids des arêtes dans le graphe et
          d'améliorer la qualité de l'ordonnancement.}
    \item{La construction d'un graphe complet permet de supprimer le paramètre
          de la fenêtre de mots et de capturer de manière plus précise le niveau
          de relation entre les sujets.}
  \end{enumerate}

  Pour évaluer notre méthode, nous utilisons quatre collections de test aux
  propriétés différentes (nature des documents, taille des documents, langue,
  etc.). Nous comparons TopicRank à trois autres méthodes non-supervisées et
  détaillons l'impact de chacune des contributions que nous proposons.

  % Quel est notre plan ?
  L'article est structuré comme suit. Après un état de l'art des méthodes
  d'extraction automatique de termes-clés en section~\ref{sec:etat_de_l_art},
  nous décrivons le fonctionnement de TopicRank en
  section~\ref{sec:extraction_de_termes_cles_avec_topicrank} et présentons son
  évaluation approfondie en section~\ref{sec:evaluation}. Enfin, nous analysons
  les erreurs de TopicRank dans la section~\ref{sec:analyse_d_erreurs}, puis
  nous concluons et discutons des travaux futurs dans la
  section~\ref{sec:conclusion_et_perspectives}.
