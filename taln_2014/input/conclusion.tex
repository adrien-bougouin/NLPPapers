\section{Conclusion et perspectives}
\label{sec:conclusion_et_perspectives}
  Dans cet article nous étudions la difficulté de la tâche d'extraction
  automatique de termes-clés appliquée à différents domaines scientifiques. Nous
  utilisons cinq collections disciplinaires de notices bibliographiques annotées
  en termes-clés dans des conditions réelles d'indexation. Pour l'Archéologie,
  les Sciences de l'Information, la Linguistique, la Psychologie et la Chimie
  nous observons l'échelle de difficulté suivante (de la discipline la plus
  difficile à la plus facile)~:
  \begin{enumerate}
    \item{Chimie}
    \item{Psychologie}
    \item{Linguistique}
    \item{Sciences de l'Information}
    \item{Archéologie}
  \end{enumerate}

  À l'issue de nos expériences et d'observations à partir du contenu des
  notices, nous constatons deux facteurs ayant un impact sur la difficulté de la
  tâche d'extraction automatique de termes-clés. Tout d'abord, le vocabulaire
  utilisé dans une discipline peut influer sur la difficulté à extraire des
  termes-clés à partir de documents de cette discipline. Si le vocabulaire
  spécifique contient des composés syntagmatiques dont certains éléments sont
  généraux dans la discipline, alors il peut être plus difficile
  d'extraire des termes-clés de documents de cette discipline. Ensuite, nous
  observons que l'organisation du discours peut aider l'extraction de
  termes-clés. Un discours riche en explications et en mises en relations des
  différents éléments est moins difficile à traiter qu'un discours énumératif.

  De ces deux facteurs identifiés émergent deux perspectives de travaux futurs.
  La nature différente des unités textuelles définies comme termes-clés selon
  les disciplines implique un besoin d'adapter l'extraction des candidats à la
  discipline traitée. L'une des ressources disciplinaire utilisées par les
  indexeurs professionnels étant la spécification précise du type d'information
  que doivent représenter les termes-clés, il serait intéressant d'utiliser une
  méthode d'extraction d'information grâce à des formulaires (e.g. champs
  périodes et lieus, en Archéologie). Une autre perspective serait d'analyser le
  discours afin de mesurer, en amont, le degré de difficulté de l'extraction de
  termes-clés. Avec une telle connaissance, nous pourrions proposer une méthode
  capable de s'adapter au degré de difficulté d'un document, en ajustant
  automatiquement ses différents paramètres.

