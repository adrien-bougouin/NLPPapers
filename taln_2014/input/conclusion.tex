\section{Conclusion et perspectives}
\label{sec:conclusion_et_perspectives}
  Dans cet article nous étudions la difficulté de la tâche d'extraction
  automatique de termes-clés appliquée à différents domaines scientifiques. Nous
  utilisons cinq collections de documents appartenant à cinq disciplines
  différentes dont les documents ont été annotés dans des conditions réelles
  afin de les d'indexer. Pour l'Archéologie, les Sciences de l'Information, la
  Linguistique, la Psychologie et la Chimie nous observons l'échelle de
  difficulté suivante (du plus difficile au plus facile)~:
  \begin{enumerate}
    \item{Chimie}
    \item{Psychologie}
    \item{Linguistique}
    \item{Sciences de l'Information}
    \item{Archéologie}
  \end{enumerate}

  À l'issue de nos expériences, nous observons différents facteurs ayant un
  impact sur la difficulté de la tâche d'extraction de termes-clés. \TODO{Ce
  référer à la section discussions}

  \TODO{Donner des perspectives~: adaptation des méthodes d'extraction de
        candidats~; utiliser des méthodes d'extraction d'information~; permettre
        l'auto-adaptation d'une méthode en fonction de la difficulté (degré de
        difficulté à calculer)}

