\section{État de l'art}
\label{sec:etat_de_l_art}
  Qu'elles soient conçues pour des documents de nature ou de domaine spécifique,
  les méthodes d'extraction de termes-clés fonctionnent toujours selon le même
  principe~: préparation des données, puis extraction des termes-clés candidats,
  puis classification ou ordonnancement des candidats, puis sélection des
  candidats à extraire comme termes-clés. Dans cette section, nous présentons
  les travaux relatifs à l'extraction automatique de termes-clés dans des
  documents de nature scientifique. Pour plus de détails sur les différentes
  méthodes d'extraction automatique de termes-clés, un état de l'art est
  disponible dans~\cite{bougouin2013stateoftheart}.

  Les travaux précédents exploitent la structure des documents et la position
  des termes-clés en leur sein. En effet, les articles scientifiques sont
  divisés en sections qui structurent le contenu. Ces sections ont des fonctions
  différentes (résumé, introduction, conclusion, etc.) et le contenu de
  certaines d'entre elles est donc plus représentatif du contenu du document que
  le contenu d'autres sections~\cite{shah2003wherearethekeywords}. Dans le cas
  supervisé, \newcite{nguyen2007keadocumentstructure} définissent 14 sections
  génériques (résumé, introduction, conclusion, état de l'art, motivations,
  etc.) qu'ils utilisent comme traits supplémentaire pour améliorer la méthode
  KEA~\cite{witten1999kea}. Dans le cas non-supervisé,
  \newcite{hofmann2009impactofdocumentstructure} définissent six sections
  génériques (introduction, contexte, méthode, résultats, discussion,
  conclusion) et expérimentent en extrayant les termes-clés à partir de l'une ou
  l'autre de ces sections. Leurs résultats montrent que la section de
  discussion, contenant en moyenne 79\% de tous les termes-clés candidats, est
  la section la plus utile pour l'extraction de termes-clés.

  \newcite{kim2009termextraction} s'intéressent à l'usage de la spécificité et
  proposent d'utiliser la pondération TF-IDF~\cite{jones1972tfidf} pour filtrer
  les unités textuelles lors de l'extraction des candidats.

  \newcite{bertels2012etudesemantiquedesmotscles} proposent aussi une étude des
  termes-clés en domaine de spécialité. Celle-ci porte sur la sémantique des
  termes-clés dans une collection de documents techniques traitant des machines
  outils pour l'usinage des métaux. Dans cet article, nous n'étudions pas la
  sémantique des termes-clés dans les domaines représentés par nos collections,
  mais une telle étude est intéressante. En effet, une forte tendance à la
  polysémie implique une forte ambiguïté des termes-clés candidats et donc une
  augmentation de la difficulté de l'extraction des termes-clés.

