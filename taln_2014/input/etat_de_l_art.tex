\section{État de l'art}
\label{sec:etat_de_l_art}
  \TODO{Référence à l'état de l'art RECITAL + présentation des travaux en
        domaine de spécialité ou pour des documents de nature spécifique
        (scientifique).}
  \textcolor{red}{\lipsum[1-6]}
%  \begin{figure}
%    \tikzstyle{io}=[
%      ellipse,
%      minimum width=5cm,
%      minimum height=2cm,
%      fill=green!20,
%      draw=green!33,
%      transform shape,
%      font={\huge}
%    ]
%    \tikzstyle{component}=[
%      text centered,
%      thick,
%      rectangle,
%      minimum width=11cm,
%      minimum height=2cm,
%      fill=cyan!20,
%      draw=cyan!33,
%      transform shape,
%      font={\huge\bfseries}
%    ]
%
%    \centering
%    \begin{tikzpicture}[thin,
%                        align=center,
%                        scale=.45,
%                        node distance=2cm,
%                        every node/.style={text centered, transform shape}]
%      \node[io] (document) {document};
%      \node[component] (preprocessing) [right=of document] {Prétraitement Linguistique};
%      \node[component] (candidate_extraction) [below=of preprocessing] {Extraction des candidats};
%      \node[component] (candidate_classification_and_ranking) [below=of candidate_extraction] {
%        \begin{tabular}{r|lcl}
%          Classification & \multirow{2}{*}[-1.5pt]{des candidats}\\
%          Ordonnancement &
%        \end{tabular}
%      };
%      \node[component] (keyphrase_selection) [below=of candidate_classification_and_ranking] {Sélection des termes-clés};
%      \node[io] (keyphrases) [right=of keyphrase_selection] {termes-clés};
%
%      \path[->, thick] (document) edge (preprocessing);
%      \path[->, thick] (preprocessing) edge (candidate_extraction);
%      \path[->, thick] (candidate_extraction) edge (candidate_classification_and_ranking);
%      \path[->, thick] (candidate_classification_and_ranking) edge (keyphrase_selection);
%      \path[->, thick] (keyphrase_selection) edge (keyphrases);
%    \end{tikzpicture}
%    \caption{Principales étapes de l'extraction automatique de termes-clés.
%             \label{fig:etapes_de_l_extraction_de_termes_cles}}
%  \end{figure}

