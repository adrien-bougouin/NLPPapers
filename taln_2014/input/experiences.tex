\section{Expériences}
\label{sec:experiences}
  Dans cette section, nous présentons les expériences menées dans le but
  d'observer l'échelle de difficulté pour l'extraction automatique de
  termes-clés en domaine de spécialité à partir des méthodes TF-IDF et
  TopicRank, en fonction des candidats qui sont extrait~: $\{1..3\}$-grammes,
  groupes nominaux et candidats termes non filtrés.

  \subsection{Mesure d'évaluation}
  \label{subsec:mesure_d_evaluation}
    Afin mesurer l'échelle de difficulté pour l'extraction automatique de
    termes-clés en domaines de spécialité, nous utilisons la MAP (\textit{Mean
    Average Precision}) pour mesurer la capacité des méthodes TF-IDF et
    TopicRank à ordonner correctement les termes-clés candidats, i.e.~leur
    capacité à extraire en premier des candidats qui sont présents dans la
    référence. Dans notre évaluation, nous considérons correcte l'extraction
    d'une variante flexionnelle d'un terme-clé de référence. Les opérations de
    comparaison entre les termes-clés de référence et les termes-clés extraits
    sont donc effectuées à partir de la racine des mots qui les composent.

  \subsection{Résultats}
  \label{subsec:resultats}
    La figure~\ref{fig:resultats} montre la performance des méthodes
    d'extraction de termes-clés lorsque les candidats extraits sont soit des
    $\{1..3\}$-grammes, soit des groupes nominaux, soit des candidats termes.
    %
    Notre hypothèse de dépard par laquelle il existe un degré de difficulté
    différent de la tâche d'extraction de termes-clés appliquée à des documents
    de disciplines scientifiques différentes se vérifie.
    %
    %Nous distinguons effectivement une différence de difficulté pour
    %l'extraction de termes-clés dans les différentes disciplines.
    %
    L'Archéologie
    est la discipline pour laquelle la tâche d'extraction automatique de
    termes-clés est la moins difficile et la Chimie est la discipline la plus
    difficile, précédée par la Psychologie, la Linguistique et les Sciences de
    l'Information.
    %
    Nous observons aussi que le choix des candidats a une forte influence sur
    certaines méthodes. En effet, avec les $\{1..3\}$-grammes, TopicRank obtient
    des résultats deux à trois fois inférieurs que ceux obtenus avec les groupes
    nominaux ou les candidats termes. Ceci est dû à l'exhaustivité de l'ensemble
    de n-grammes, faisant de celui-ci un ensemble comportant de nombreux
    candidats non pertinents qui dégradent les performances de TopicRank
    (dégradation du groupement en sujets, renforcement de liens non pertinents
    dans le graphe, etc.). Dans le cas de la méthode TF-IDF, cette dégradation
    est moins conséquente, car les candidats non pertinents ne sont pas des
    unités textuelles spécifiques~\cite{kim2009termextraction}, ils se trouvent
    donc en queue du classement.
    %
    %Nous observons aussi que, dans le cas de TopicRank, la
    %performance n'est pas stable selon l'ensemble de candidats utilisé. Les
    %$\{1..3\}$-grammes dégradent significativement les résultats de TopicRank,
    %en comparaison avec les autres types de candidats, alors que ce n'est pas le
    %cas pour la méthode TF-IDF. Ceci est dû au fait que l'ensemble de
    %$\{1..3\}$-grammes contient un grand nombre de candidats non pertinents qui
    %ont pour effet de dégrader les résultats de TopicRank, mais pas ceux de la
    %méthode TF-IDF, car les candidats non pertinents ne sont pas
    %spécifiques~\cite{kim2009termextraction}. \TODO{Compléter pour les autres
    %types de candidats.}
    %
    %Deux
    %disciplines se distinguent sur l'échelle de difficulté, l'Archéologie se
    %veut être la discipline la plus facile pour la tâche d'extraction de
    %termes-clés, en opposition avec la Chimie qui, elle, est la plus difficile.
    %Quant aux Sciences de l'Information, à la Linguistique et à la Psychologie,
    %celles-ci ont sensiblement le même degré de difficulté. Les deux méthodes
    %d'extraction de termes-clés se distinguent aussi par leur comportement. La
    %méthode TF-IDF est plus stable que TopicRank lorsque l'on compare les
    %résultats obtenus à partir des tri-grammes et des termes. En effet, en
    %comparaison avec l'ensemble de termes, l'ensemble de tri-grammes contient
    %plus de candidats non pertinants qui dégradent la qualité des groupements en
    %sujets de TopicRank, tandis que la notion de spécificité incorporée dans
    %TF-IDF permet de placer ces candidats en queue du classement et donc de
    %diminuer leur effet négatif.
    \begin{figure}
      \centering
      \subfigure[$\{1..3\}$-grammes]{
        \begin{tikzpicture}%[scale=.75]
          \pgfkeys{/pgf/number format/.cd, use comma, fixed, fixed zerofill, precision=3}
          \begin{axis}[axis lines=left,
                       symbolic x coords={Archéologie, Sciences de l'Information, Linguistique, Psychologie, Chimie},
                       xtick=data,
                       enlarge x limits=0.125,
                       x=.1\linewidth,
                       xticklabel style={anchor=east, xshift=.5em, yshift=-.25em, rotate=22.5},
                       nodes near coords,
                       nodes near coords align={vertical},
                       every node near coord/.append style={font=\scriptsize},
                       ytick={0, 0.100, 0.200, 0.300, 0.400, 0.500},
                       y=\linewidth,
                       ymin=0,
                       ymax=0.22,
                       ybar=10pt,
                       ylabel=MAP,
                       ylabel style={at={(ticklabel* cs:1)},
                                     anchor=south,
                                     rotate=270}]
            \addplot[black!66,
                     pattern=north east lines,
                     pattern color=black!40] coordinates{
              (Archéologie, 0.156)
              (Sciences de l'Information, 0.103)
              (Linguistique, 0.095)
              (Psychologie, 0.068)
              (Chimie, 0.052)
            };
            \addplot[black!66,
                     pattern=north west lines,
                     pattern color=black!66] coordinates{
              (Archéologie, 0.035)
              (Sciences de l'Information, 0.043)
              (Linguistique, 0.038)
              (Psychologie, 0.029)
              (Chimie, 0.018)
            };
            \legend{TF-IDF, TopicRank}
          \end{axis}
        \end{tikzpicture}
      }
      \subfigure[Groupes nominaux]{
        \begin{tikzpicture}%[scale=.75]
          \pgfkeys{/pgf/number format/.cd, use comma, fixed, fixed zerofill, precision=3}
          \begin{axis}[axis lines=left,
                       symbolic x coords={Archéologie, Sciences de l'Information, Linguistique, Psychologie, Chimie},
                       xtick=data,
                       enlarge x limits=0.125,
                       x=.1\linewidth,
                       xticklabel style={anchor=east, xshift=.5em, yshift=-.25em, rotate=22.5},
                       nodes near coords,
                       nodes near coords align={vertical},
                       every node near coord/.append style={font=\scriptsize},
                       ytick={0, 0.100, 0.200, 0.300, 0.400, 0.500},
                       y=\linewidth,
                       ymin=0,
                       ymax=0.22,
                       ybar=10pt,
                       ylabel=MAP,
                       ylabel style={at={(ticklabel* cs:1)},
                                     anchor=south,
                                     rotate=270}]
            \addplot[black!66,
                     pattern=north east lines,
                     pattern color=black!40] coordinates{
              (Archéologie, 0.180)
              (Sciences de l'Information, 0.120)
              (Linguistique, 0.098)
              (Psychologie, 0.079)
              (Chimie, 0.063)
            };
            \addplot[black!66,
                     pattern=north west lines,
                     pattern color=black!66] coordinates{
              (Archéologie, 0.150)
              (Sciences de l'Information, 0.128)
              (Linguistique, 0.095)
              (Psychologie, 0.078)
              (Chimie, 0.055)
            };
            \legend{TF-IDF, TopicRank}
          \end{axis}
        \end{tikzpicture}
      }
      \subfigure[Candidats termes]{
        \begin{tikzpicture}%[scale=.75]
          \pgfkeys{/pgf/number format/.cd, use comma, fixed, fixed zerofill, precision=3}
          \begin{axis}[axis lines=left,
                       symbolic x coords={Archéologie, Sciences de l'Information, Linguistique, Psychologie, Chimie},
                       xtick=data,
                       enlarge x limits=0.125,
                       x=.1\linewidth,
                       xticklabel style={anchor=east, xshift=.5em, yshift=-.25em, rotate=22.5},
                       nodes near coords,
                       nodes near coords align={vertical},
                       every node near coord/.append style={font=\scriptsize},
                       ytick={0, 0.100, 0.200, 0.300, 0.400, 0.500},
                       y=\linewidth,
                       ymin=0,
                       ymax=0.22,
                       ybar=10pt,
                       ylabel=MAP,
                       ylabel style={at={(ticklabel* cs:1)},
                                     anchor=south,
                                     rotate=270}]
            \addplot[black!66,
                     pattern=north east lines,
                     pattern color=black!40] coordinates{
              (Archéologie, 0.166)
              (Sciences de l'Information, 0.110)
              (Linguistique, 0.104)
              (Psychologie, 0.071)
              (Chimie, 0.060)
            };
            \addplot[black!66,
                     pattern=north west lines,
                     pattern color=black!66] coordinates{
              (Archéologie, 0.173)
              (Sciences de l'Information, 0.096)
              (Linguistique, 0.104)
              (Psychologie, 0.076)
              (Chimie, 0.052)
            };
            \legend{TF-IDF, TopicRank}
          \end{axis}
        \end{tikzpicture}
      }
%      \subfigure[Candidats termes et entités nommées (NEMESIS)]{
%        \begin{tikzpicture}[scale=.75]
%          \pgfkeys{/pgf/number format/.cd, use comma, fixed, fixed zerofill, precision=3}
%          \begin{axis}[axis lines=left,
%                       symbolic x coords={Archéologie, Sciences de l'Information, Linguistique, Psychologie, Chimie},
%                       xtick=data,
%                       enlarge x limits=0.125,
%                       x=.1\linewidth,
%                       xticklabel style={anchor=east, xshift=.5em, yshift=-.25em, rotate=22.5},
%                       nodes near coords,
%                       nodes near coords align={vertical},
%                       every node near coord/.append style={font=\scriptsize},
%                       ytick={0, 0.100, 0.200, 0.300, 0.400, 0.500},
%                       y=\linewidth,
%                       ymin=0,
%                       ymax=0.22,
%                       ybar=10pt,
%                       ylabel=MAP,
%                       ylabel style={at={(ticklabel* cs:1)},
%                                     anchor=south,
%                                     rotate=270}]
%            \addplot[black!66,
%                     pattern=north east lines,
%                     pattern color=black!40] coordinates{
%              (Archéologie, 0.171)
%              (Sciences de l'Information, 0.111)
%              (Linguistique, 0.104)
%              (Psychologie, 0.071)
%              (Chimie, 0.060)
%            };
%            \addplot[black!66,
%                     pattern=north west lines,
%                     pattern color=black!66] coordinates{
%              (Archéologie, 0.152)
%              (Sciences de l'Information, 0.092)
%              (Linguistique, 0.097)
%              (Psychologie, 0.072)
%              (Chimie, 0.050)
%            };
%            \legend{TF-IDF, TopicRank}
%          \end{axis}
%        \end{tikzpicture}
%      }
%      \subfigure[Candidats termes et Np+]{
%        \begin{tikzpicture}[scale=.75]
%          \pgfkeys{/pgf/number format/.cd, use comma, fixed, fixed zerofill, precision=3}
%          \begin{axis}[axis lines=left,
%                       symbolic x coords={Archéologie, Sciences de l'Information, Linguistique, Psychologie, Chimie},
%                       xtick=data,
%                       enlarge x limits=0.125,
%                       x=.1\linewidth,
%                       xticklabel style={anchor=east, xshift=.5em, yshift=-.25em, rotate=22.5},
%                       nodes near coords,
%                       nodes near coords align={vertical},
%                       every node near coord/.append style={font=\scriptsize},
%                       ytick={0, 0.100, 0.200, 0.300, 0.400, 0.500},
%                       y=\linewidth,
%                       ymin=0,
%                       ymax=0.22,
%                       ybar=10pt,
%                       ylabel=MAP,
%                       ylabel style={at={(ticklabel* cs:1)},
%                                     anchor=south,
%                                     rotate=270}]
%            \addplot[black!66,
%                     pattern=north east lines,
%                     pattern color=black!40] coordinates{
%              (Archéologie, 0.170)
%              (Sciences de l'Information, 0.110)
%              (Linguistique, 0.103)
%              (Psychologie, 0.071)
%              (Chimie, 0.058)
%            };
%            \addplot[black!66,
%                     pattern=north west lines,
%                     pattern color=black!66] coordinates{
%              (Archéologie, 0.175)
%              (Sciences de l'Information, 0.097)
%              (Linguistique, 0.104)
%              (Psychologie, 0.076)
%              (Chimie, 0.051)
%            };
%            \legend{TF-IDF, TopicRank}
%          \end{axis}
%        \end{tikzpicture}
%      }
      \caption{Performace des méthodes d'extraction de termes-clés en domaines
               de spécialité à partir de différents type de candidats.
               \label{fig:resultats}}
    \end{figure}

