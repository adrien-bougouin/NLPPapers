\section{Expériences}
\label{sec:experiences}
  Dans cette section, nous présentons les expériences menées dans le but
  d'observer l'échelle de difficulté pour l'extraction automatique de
  termes-clés en domaine de spécialité. Deux méthodes non-supervisées
  d'extraction automatique de termes-clés sont sélectionnées pour extraire les
  termes-clés puis leur performance est évaluée pour chacune des cinq
  collections (cf. section~\ref{sec:presentation_des_donnees}).

  \subsection{Mesure d'évaluation}
  \label{subsec:mesure_d_evaluation}
    La performance des méthodes d'extraction automatique de termes-clés est
    exprimée avec la F1-mesure. Cette mesure est un compromis entre précision et
    rappel, qui mesurent la capacité d'une méthode à, respectivement, minimiser
    le nombre de faux positifs et maximiser le nombre de vrais positifs. Dans
    notre évaluation, nous concidérons correcte l'extraction d'une variante
    flexionnelle d'un terme-clé de référence. Les opérations de comparaison sont
    donc effectuées à partir de la forme radicale des mots des termes-clés
    (extraits et de référence).

  \subsection{Préparation des données}
  \label{subsec:preparation_des_donnees}
    Les documents des collections de données utilisées subissent les mêmes
    prétraitements. Ils sont tout d'abord segmentés en phrases, puis en mots et
    enfin étiquetés en parties du discours. La segmentation en mots est
    effectuée avec l'outil Bonsai, du Bonsai PCFG-LA
    parser\footnote{\url{http://alpage.inria.fr/statgram/frdep/fr_stat_dep_parsing.html}}
    et l'étiquetage en parties du discours est réalisé avec
    MElt~\cite{denis2009melt}. Tous ces outils sont utilisés sans modification
    de leurs paramètres par défaut.

  \subsection{Extraction des termes-clés candidats}
  \label{subsec:extraction_des_termes_cles_candidats}
    %Après la préparation des données, l'étape à ne pas négliger est celle
    %de l'extraction des candidats. Dans notre cas, seules les unités textuelles
    %présentes dans le document peuvent être extraites en tant que termes-clés.
    %Cependant, extraire toutes les séquences de mots possibles n'est pas la
    %meilleure stratégie. En effet, plus il y a de candidats, plus il est
    %difficile d'extraire les termes-clés et plus le temps de calcul est
    %important. De ce fait, nous répétons nos expériences avec trois méthodes
    %d'extraction de candidats différentes, l'une extrayant les tri-grammes, une
    %autre les syntagmes nominaux et la dernière les termes. En adéquation avec
    %les travaux précédents, les tri-grammes sont filtrés avec une liste de mots
    %outils, qui regroupe les mots fonctionnels de la langue (conjonctions,
    %prépositions, etc.) et les mots courants (e.g. \og{}beaucoup\fg{},
    %\og{}près\fg{}, etc.). L'extraction de syntagmes nominaux est effectuée par
    %reconnaissance de formes, avec les patrons suivants~: \texttt{NP+~|
    %(A?~Nc~A+)~| (A~Nc)~| Nc+}. Quand aux termes, ceux-ci sont extraits avec
    %l'outil TermSuite~\cite{rocheteau2011termsuite}. Tout d'abord, une
    %terminologie par corpus est extraite par TermSuite (32 119 termes en
    %Archéologie, 16 557 termes en Sciences de l'Information, 21 330 termes en
    %linguistique, 24 680 termes en Psychologie et 21 020 termes en chimie), puis
    %seules les unités textuelles appartenant à la terminologie correspondante
    %sont extraites en tant que termes-clés candidats des documents.
    Après la préparation des données, l'étape à ne pas négliger est celle
    de l'extraction des candidats. Dans notre cas, seules les unités textuelles
    présentes dans le document peuvent être extraites en tant que termes-clés.
    Cependant, utiliser toutes les séquences de mots possibles en tant que
    termes-clés candidats n'est pas la meilleure stratégie. En effet, plus il y
    a de candidats, plus il peut être difficile d'extraire les termes-clés et
    plus le temps de calcul est important. De ce fait, nous répétons nos
    expériences avec deux méthodes d'extraction de candidats différentes, l'une
    extrayant des tri-grammes et l'autre des termes. En adéquation avec les
    travaux précédents~\cite{witten1999kea}, les tri-grammes sont filtrés avec
    une liste de mots outils, qui regroupe les mots fonctionnels de la langue
    (conjonctions, prépositions, etc.) et les mots courants (e.g.
    \og{}beaucoup\fg{}, \og{}près\fg{}, etc.). Quand aux termes, ceux-ci sont
    extraits avec l'outil d'extraction terminologique mono- et multilingue
    TermSuite~\cite{rocheteau2011termsuite}. Tout d'abord, une terminologie par
    corpus est extraite (32 119 termes en Archéologie, 16 557 termes en Sciences
    de l'Information, 21 330 termes en linguistique, 24 680 termes en
    Psychologie et 21 020 termes en chimie), puis seules les unités textuelles
    appartenant à la terminologie correspondante sont extraites en tant que
    termes-clés candidats des documents.

  \subsection{Extraction de termes-clés}
  \label{subsec:extraction_de_termes_cles}
    Afin d'avoir un meilleur aperçu des différences d'extraction de termes-clés
    en fonction des disciplines, nous utilisons deux méthodes non-supervisées
    employant des techniques différentes, une méthode reposant sur la
    pondération TF-IDF~\cite{jones1972tfidf} et la méthode
    TopicRank~\cite{bougouin2013topicrank}.

    Le principe de la méthode utilisant la pondération TF-IDF consiste à
    extraite en tant que termes-clés les candidats contenant les mots les plus
    importants (fort poids TF-IDF). Un mot est considéré important s'il est
    fréquent dans le document et s'il est spécifique à celui-ci. La spécificité
    est déterminée à partir de la collection de documents et un mot est
    considéré comme spécifique lorsqu'il apparaît dans très peu de documents.

    La méthode TopicRank extrait les termes-clés d'un document à partir d'une
    représentation sous forme de graphe de celui-ci. Tout d'abord, les
    termes-clés candidats sont groupés par sujets, puis ces sujets sont utilisés
    pour construire un graphe complet dans lequel ils représentent chacun un
    n\oe{}ud. L'algorithme d'ordonnancement
    TextRank~\cite{mihalcea2004textrank} est ensuite appliqué afin d'obtenir un
    score d'importance pour chaque n\oe{}ud du graphe. Enfin, les $k$ sujets les
    plus importants, selon TextRank, sont sélectionnés et, pour chacun d'eux, le
    candidat le plus représentatif est extrait en tant que terme-clé. Dans la
    méthode originale, le groupement en sujet est effectué avec une similarité
    lexicale. Lorsque les termes-clés candidats sont les termes extraits par
    TermSuite, nous utilisons le groupement terme/variantes de TermSuite à la
    place de la similarité lexicale.

  \subsection{Résultats}
  \label{subsec:resultats}
    La figure~\ref{fig:resultats} montre la performance des méthodes
    d'extraction de termes-clés lorsqu'elles extraient 10 termes-clés à partir
    des tri-grammes ou des termes des documents. Moins la F1-mesure est élevée,
    plus nous supposons que l'extraction de termes-clés est difficulté.
    \begin{figure}
      \centering
      \subfigure[Tri-grammes]{
        \begin{tikzpicture}[scale=.75]
          \begin{axis}[axis lines=left,
                       symbolic x coords={Archéologie, Sciences de l'Information, Linguistique, Psychologie, Chimie},
                       xtick=data,
                       enlarge x limits=0.125,
                       x=.1\linewidth,
                       xticklabel style={anchor=east, xshift=.5em, yshift=-.25em, rotate=22.5},
                       nodes near coords,
                       nodes near coords align={vertical},
                       every node near coord/.append style={font=\scriptsize},
                       ytick={0, 10, 20, 30, 40, 50},
                       y=0.01\linewidth,
                       ymin=0,
                       ymax=25,
                       ybar=7.5pt,
                       ylabel=F1-mesure (\%),
                       ylabel style={at={(ticklabel* cs:1)},
                                     anchor=south,
                                     rotate=270}]
            \addplot[black!66,
                     pattern=north east lines,
                     pattern color=black!40] coordinates{
              (Archéologie, 17.8)
              (Sciences de l'Information, 10.2)
              (Linguistique, 11.2)
              (Psychologie, 9.5)
              (Chimie, 8.0)
            };
            \addplot[black!66,
                     pattern=north west lines,
                     pattern color=black!66] coordinates{
              (Archéologie, 6.9)
              (Sciences de l'Information, 5.1)
              (Linguistique, 6.4)
              (Psychologie, 5.3)
              (Chimie, 3.7)
            };
            \legend{TF-IDF, TopicRank}
          \end{axis}
        \end{tikzpicture}
      }
    %\begin{figure}
    %  \centering
    %  \begin{tikzpicture}[scale=.75]
    %    \begin{axis}[axis lines=left,
    %                 symbolic x coords={Archéologie, Sciences de l'Information, Linguistique, Psychologie, Chimie},
    %                 xtick=data,
    %                 enlarge x limits=0.125,
    %                 x=.1\linewidth,
    %                 xticklabel style={anchor=east, xshift=.5em, yshift=-.25em, rotate=22.5},
    %                 nodes near coords,
    %                 nodes near coords align={vertical},
    %                 every node near coord/.append style={font=\scriptsize},
    %                 ytick={0, 10, 20, 30, 40, 50},
    %                 y=0.01\linewidth,
    %                 ymin=0,
    %                 ymax=25,
    %                 ybar=7.5pt,
    %                 ylabel=F1-mesure (\%),
    %                 ylabel style={at={(ticklabel* cs:1)},
    %                               anchor=south,
    %                               rotate=270}]
    %      \addplot[black!66,
    %               pattern=north east lines,
    %               pattern color=black!40] coordinates{
    %        (Archéologie, 22.7)
    %        (Sciences de l'Information, 12.9)
    %        (Linguistique, 12.8)
    %        (Psychologie, 12.2)
    %        (Chimie, 10.5)
    %      };
    %      \addplot[black!66,
    %               pattern=north west lines,
    %               pattern color=black!66] coordinates{
    %        (Archéologie, 21.2)
    %        (Sciences de l'Information, 12.6)
    %        (Linguistique, 12.5)
    %        (Psychologie, 12.1)
    %        (Chimie, 9.6)
    %      };
    %      \legend{TF-IDF, TopicRank}
    %    \end{axis}
    %  \end{tikzpicture}
    %  \caption{Résultats de l'extraction de 10 termes-clés à partir des
    %           syntagmes nominaux.
    %           \label{fig:resultats_syntagmes_nominaux}}
    %\end{figure}
      \subfigure[Termes]{
        \begin{tikzpicture}[scale=.75]
          \begin{axis}[axis lines=left,
                       symbolic x coords={Archéologie, Sciences de l'Information, Linguistique, Psychologie, Chimie},
                       xtick=data,
                       enlarge x limits=0.125,
                       x=.1\linewidth,
                       xticklabel style={anchor=east, xshift=.5em, yshift=-.25em, rotate=22.5},
                       nodes near coords,
                       nodes near coords align={vertical},
                       every node near coord/.append style={font=\scriptsize},
                       ytick={0, 10, 20, 30, 40, 50},
                       y=0.01\linewidth,
                       ymin=0,
                       ymax=25,
                       ybar=7.5pt,
                       ylabel=F1-mesure (\%),
                       ylabel style={at={(ticklabel* cs:1)},
                                     anchor=south,
                                     rotate=270}]
            \addplot[black!66,
                     pattern=north east lines,
                     pattern color=black!40] coordinates{
              (Archéologie, 19.2)
              (Sciences de l'Information, 11.1)
              (Linguistique, 12.7)
              (Psychologie, 10.3)
              (Chimie, 9.7)
            };
            \addplot[black!66,
                     pattern=north west lines,
                     pattern color=black!66] coordinates{
              (Archéologie, 20.9)
              (Sciences de l'Information, 9.9)
              (Linguistique, 13.7)
              (Psychologie, 11.5)
              (Chimie, 8.8)
            };
            \legend{TF-IDF, TopicRank}
          \end{axis}
        \end{tikzpicture}
      }
      \caption{Résultats de l'extraction de 10 termes-clés.
               \label{fig:resultats}}
    \end{figure}

    Deux disciplines se distinguent sur l'échelle de difficulté, l'Archéologie
    se veut être la discipline la plus facile pour la tâche d'extraction de
    termes-clés, en opposition avec la Chimie qui, elle, est la plus difficile.
    Quant aux Sciences de l'Information, à la Linguistique et à la Psychologie,
    celles-ci ont sensiblement le même degré de difficulté. En comparant les
    résultats obtenus en fonction de différentes méthodes d'extraction de
    candidats et d'extraction de termes-clés, nous observons que l'usage de
    tri-grammes rend la tâche plus difficile, surtout lorsqu'une méthode telle
    que TopicRank ne tient en aucun cas compte de la spécificité des candidats.
    Cependant, lorsque les candidats sont contrôlés d'un point de vu
    terminologique, comme c'est le cas avec les termes extraits par TermSuite,
    TopicRank donne de meilleurs résultat que la pondération TF-IDF.
    \TODO{À développer}

