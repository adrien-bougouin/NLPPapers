\section{Expériences}
\label{sec:experiences}
  Dans cette section, nous présentons les expériences menées dans le but
  d'observer l'échelle de difficulté pour l'extraction automatique de
  termes-clés en domaine de scpécialité. Deux méthodes non-supervisées
  d'extraction automatique de termes-clés sont sélectionnées pour extraire les
  termes-clés de chacune des collections présentées en
  section~\ref{sec:presentation_des_donnees}, puis leur performance est évaluée
  de sorte que moins les méthodes sont performante pour une discipline, plus
  cette discipline est jugée difficile pour la tâche d'extraction automatique de
  termes-clés.

  \subsection{Mesure d'évaluation}
  \label{subsec:mesure_d_evaluation}
    La performance des méthodes d'extraction automatique de termes-clés est
    exprimée avec la F1-mesure. Cette mesure est un compromis entre précision et
    rappel, qui mesurent la capacité d'une méthode à respectivement minimiser le
    nomber de faux positifs et maximiser le nombre de vrais positifs. Plus la
    F1-mesure est élevée, meilleure est l'extraction de termes-clés. Afin de ne
    pas considérer fausse l'extraction d'une variante flexionnelle d'un
    terme-clé de référence, les opérations de comparaison sont effectuées à
    partir de la forme radicale des mots.

  \subsection{Préparation des données}
  \label{subsec:preparation_des_donnees}
    Chaque document des collections de données utilisées subit les mêmes
    prétraitements. Il est tout d'abord segmenté en phrases, puis en mots et
    enfin étiqueté en parties du discours. La segmentation en mots est effectuée
    par l'outil Bonsai, du Bonsai PCFG-LA
    parser\footnote{\url{http://alpage.inria.fr/statgram/frdep/fr_stat_dep_parsing.html}}
    et l'étiquetage en parties du discours est réalisé avec
    MElt~\cite{denis2009melt}. Tous ces outils sont utilisés sans aucune
    modification de leurs paramètres par défaut.

  \subsection{Extraction des termes-clés candidats}
  \label{subsec:extraction_des_termes_cles_candidats}
    Après la préparation des données, l'étape à ne pas négliger est celle
    de l'extraction des candidats. Dans notre cas, seules les unités textuelles
    présentes dans le document peuvent être extraites en tant que termes-clés.
    Cependant, extraire toutes les séquences de mots possibles n'est pas la
    meilleure stratégie. En effet, plus il y a de candidats, plus il est
    difficile d'extraire les termes-clés et plus le temps de calcul est
    important. De ce fait, nous répétons nos expériences avec trois méthodes
    d'extraction de candidats différentes, l'une extrayant les tri-grammes, une
    autre les syntagmes nominaux et la dernière les termes. En adéquation avec
    les travaux précédents, les tri-grammes sont filtrés avec une liste de mots
    outils, qui regroupe les mots fonctionnels de la langue (conjonctions,
    prépositions, etc.) et les mots courants (e.g. \og{}beaucoup\fg{},
    \og{}près\fg{}, etc.). L'extraction de syntagmes nominaux est effectuée par
    reconnaissance de formes, avec les patrons suivants~: \texttt{NP+~|
    (A?~Nc~A+)~| (A~Nc)~| Nc+}. Quand aux termes, ceux-ci sont extraits avec
    l'outil TermSuite~\cite{rocheteau2011termsuite}. Tout d'abord, une
    terminologie par corpus est extraite par TermSuite (32 119 termes en
    Archéologie, 16 557 termes en Sciences de l'Information, 21 330 termes en
    linguistique, 24 680 termes en Psychologie et 21 020 termes en chimie), puis
    seules les unités textuelles appartenant à la terminologie correspondante
    sont extraites en tant que termes-clés candidats des documents.

  \subsection{Extraction de termes-clés}
  \label{subsec:extraction_de_termes_cles}
    Afin d'avoir un meilleur aperçu des différences d'extraction de termes-clés
    en fonction des disciplines, nous utilisons deux méthodes non-supervisées
    employant des techniques différentes, la pondération
    TF-IDF~\cite{jones1972tfidf} et la méthode
    TopicRank~\cite{bougouin2013topicrank}.

    Le principe de la méthode utilisant la pondération TF-IDF consiste à
    extraite en tant que termes-clés les candidats contenant les mots les plus
    importants (fort poids TF-IDF). Dans ce cas, un mot est considéré important
    s'il est fréquent dans le document et s'il est spécifique à celui-ci. La
    spécificité est déterminée à partir de la collection de documents. Un mot
    est considéré comme spécifique lorsqu'il apparaît dans très peu de document.

    La méthode TopicRank extrait les termes-clés d'un document à partir d'une
    représentation sous forme de graphe de celui-ci. Tout d'abord, les
    termes-clés candidats sont groupés par sujets, puis les sujets sont utilisés
    pour construire un graphe complet dans lequel ils représentent chacun un
    n\oe{}ud. L'algorithme d'ordonnancement
    TextRank~\cite{mihalcea2004textrank} est ensuite appliqué au graphe afin
    d'obtenir un score d'importance pour chaque sujet (n\oe{}ud). Enfin, les $k$
    sujets les plus importants sont sélectionnés et, pour chacun d'eux, le
    candidat le plus représentatif est extrait en tant que terme-clé. Dans la
    méthode classique, le groupement en sujet est effectué avec une similarité
    de Jaccard. Dans le cas de l'usage des termes comme termes-clés candidats,
    nous décidons de tirer profit du groupement terme/variantes de TermSuite.

  \subsection{Résultats}
  \label{subsec:resultats}
    Les figures~\ref{fig:resultats_tri_grammes},
    \ref{fig:resultats_syntagmes_nominaux} et \ref{fig:resultats_termes}
    montrent la performance des méthodes d'extraction de termes-clés
    lorsqu'elles extraient 10 termes-clés à partir, respectivement, des
    tri-grammes, des syntagmes nominaux et des termes présents dans les
    documents. Plus le f-score est élevé, plus nous supposons que l'extraction
    de termes-clés est aisée.
    \begin{figure}
      \centering
      \begin{tikzpicture}[scale=.75]
        \begin{axis}[axis lines=left,
                     symbolic x coords={Archéologie, Sciences de l'Information, Linguistique, Psychologie, Chimie},
                     xtick=data,
                     enlarge x limits=0.125,
                     x=.1\linewidth,
                     xticklabel style={anchor=east, xshift=.5em, yshift=-.25em, rotate=22.5},
                     nodes near coords,
                     nodes near coords align={vertical},
                     every node near coord/.append style={font=\scriptsize},
                     ytick={0, 10, 20, 30, 40, 50},
                     y=0.01\linewidth,
                     ymin=0,
                     ymax=25,
                     ybar=7.5pt,
                     ylabel=F1-mesure (\%),
                     ylabel style={at={(ticklabel* cs:1)},
                                   anchor=south,
                                   rotate=270}]
          \addplot[black!66,
                   pattern=north east lines,
                   pattern color=black!40] coordinates{
            (Archéologie, 17.8)
            (Sciences de l'Information, 10.2)
            (Linguistique, 11.2)
            (Psychologie, 9.5)
            (Chimie, 8.0)
          };
          \addplot[black!66,
                   pattern=north west lines,
                   pattern color=black!66] coordinates{
            (Archéologie, 6.9)
            (Sciences de l'Information, 5.1)
            (Linguistique, 6.4)
            (Psychologie, 5.3)
            (Chimie, 3.7)
          };
          \legend{TF-IDF, TopicRank}
        \end{axis}
      \end{tikzpicture}
      \caption{Résultats de l'extraction de 10 termes-clés à partir des
               tri-grammes.
               \label{fig:resultats_tri_grammes}}
    \end{figure}
    \begin{figure}
      \centering
      \begin{tikzpicture}[scale=.75]
        \begin{axis}[axis lines=left,
                     symbolic x coords={Archéologie, Sciences de l'Information, Linguistique, Psychologie, Chimie},
                     xtick=data,
                     enlarge x limits=0.125,
                     x=.1\linewidth,
                     xticklabel style={anchor=east, xshift=.5em, yshift=-.25em, rotate=22.5},
                     nodes near coords,
                     nodes near coords align={vertical},
                     every node near coord/.append style={font=\scriptsize},
                     ytick={0, 10, 20, 30, 40, 50},
                     y=0.01\linewidth,
                     ymin=0,
                     ymax=25,
                     ybar=7.5pt,
                     ylabel=F1-mesure (\%),
                     ylabel style={at={(ticklabel* cs:1)},
                                   anchor=south,
                                   rotate=270}]
          \addplot[black!66,
                   pattern=north east lines,
                   pattern color=black!40] coordinates{
            (Archéologie, 22.7)
            (Sciences de l'Information, 12.9)
            (Linguistique, 12.8)
            (Psychologie, 12.2)
            (Chimie, 10.5)
          };
          \addplot[black!66,
                   pattern=north west lines,
                   pattern color=black!66] coordinates{
            (Archéologie, 21.2)
            (Sciences de l'Information, 12.6)
            (Linguistique, 12.5)
            (Psychologie, 12.1)
            (Chimie, 9.6)
          };
          \legend{TF-IDF, TopicRank}
        \end{axis}
      \end{tikzpicture}
      \caption{Résultats de l'extraction de 10 termes-clés à partir des
               syntagmes nominaux.
               \label{fig:resultats_syntagmes_nominaux}}
    \end{figure}
    \begin{figure}
      \centering
      \begin{tikzpicture}[scale=.75]
        \begin{axis}[axis lines=left,
                     symbolic x coords={Archéologie, Sciences de l'Information, Linguistique, Psychologie, Chimie},
                     xtick=data,
                     enlarge x limits=0.125,
                     x=.1\linewidth,
                     xticklabel style={anchor=east, xshift=.5em, yshift=-.25em, rotate=22.5},
                     nodes near coords,
                     nodes near coords align={vertical},
                     every node near coord/.append style={font=\scriptsize},
                     ytick={0, 10, 20, 30, 40, 50},
                     y=0.01\linewidth,
                     ymin=0,
                     ymax=25,
                     ybar=7.5pt,
                     ylabel=F1-mesure (\%),
                     ylabel style={at={(ticklabel* cs:1)},
                                   anchor=south,
                                   rotate=270}]
          \addplot[black!66,
                   pattern=north east lines,
                   pattern color=black!40] coordinates{
            (Archéologie, 19.2)
            (Sciences de l'Information, 11.1)
            (Linguistique, 12.7)
            (Psychologie, 10.3)
            (Chimie, 9.7)
          };
          \addplot[black!66,
                   pattern=north west lines,
                   pattern color=black!66] coordinates{
            (Archéologie, 20.9)
            (Sciences de l'Information, 9.9)
            (Linguistique, 13.7)
            (Psychologie, 11.5)
            (Chimie, 8.8)
          };
          \legend{TF-IDF, TopicRank}
        \end{axis}
      \end{tikzpicture}
      \caption{Résultats de l'extraction de 10 termes-clés à partir des termes
               extraits par TermSuite.
               \label{fig:resultats_termes}}
    \end{figure}

