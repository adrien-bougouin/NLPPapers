\section{Expériences}
\label{sec:experiences}
  \subsection{Préparation des données}
  \label{subsec:preparation_des_donnees}

  \subsection{Mesure d'évaluation}
  \label{subsec:mesure_d_evaluation}

  \subsection{Systèmes d'extraction de termes-clés}

  \subsection{Résultats}
  \label{subsec:resultats}
    %\begin{figure}
    %  \centering
    %  \begin{tikzpicture}
    %    \begin{axis}[axis lines=left,
    %                 symbolic x coords={Archéologie, Sciences de l'Information, Linguistique, Psychologie, Chimie},
    %                 xtick=data,
    %                 enlarge x limits=0.125,
    %                 x=.075\linewidth,
    %                 xticklabel style={anchor=east, rotate=22.5},
    %                 nodes near coords,
    %                 nodes near coords align={vertical},
    %                 every node near coord/.append style={font=\footnotesize},
    %                 ytick={0, 10, 20, 30, 40, 50, 60, 70, 75, 85, 95, 105},
    %                 y=0.01\linewidth,
    %                 ymin=75,
    %                 ymax=100,
    %                 ybar,
    %                 ylabel=Difficulté,
    %                 ylabel style={at={(ticklabel* cs:1)},
    %                               anchor=south,
    %                               rotate=270}]
    %      \addplot[red!66, fill=red!40] coordinates{
    %        (Archéologie, 77.3)
    %        (Sciences de l'Information, 87.1)
    %        (Linguistique, 87.2)
    %        (Psychologie, 87.8)
    %        (Chimie, 89.5)
    %      };
    %    \end{axis}
    %  \end{tikzpicture}
    %  \caption{Difficulté de la tâche d'extraction de 10 termes-clés avec
    %           \textbf{TF-IDF}.}
    %\end{figure}
    %\begin{figure}
    %  \centering
    %  \begin{tikzpicture}
    %    \begin{axis}[axis lines=left,
    %                 symbolic x coords={Archéologie, Sciences de l'Information, Linguistique, Psychologie, Chimie},
    %                 xtick=data,
    %                 enlarge x limits=0.125,
    %                 x=.075\linewidth,
    %                 xticklabel style={anchor=east, rotate=22.5},
    %                 nodes near coords,
    %                 nodes near coords align={vertical},
    %                 every node near coord/.append style={font=\footnotesize},
    %                 ytick={0, 10, 20, 30, 40, 50, 60, 70, 75, 85, 95, 105},
    %                 y=0.01\linewidth,
    %                 ymin=75,
    %                 ymax=100,
    %                 ybar,
    %                 ylabel=Difficulté,
    %                 ylabel style={at={(ticklabel* cs:1)},
    %                               anchor=south,
    %                               rotate=270}]
    %      \addplot[red!66, fill=red!40] coordinates{
    %        (Archéologie, 89.4)
    %        (Sciences de l'Information, 89.3)
    %        (Linguistique, 90.7)
    %        (Psychologie, 90.5)
    %        (Chimie, 91)
    %      };
    %    \end{axis}
    %  \end{tikzpicture}
    %  \caption{Difficulté de la tâche d'extraction de 10 termes-clés avec
    %           \textbf{SingleRank}.}
    %\end{figure}
    %\begin{figure}
    %  \centering
    %  \begin{tikzpicture}
    %    \begin{axis}[axis lines=left,
    %                 symbolic x coords={Archéologie, Sciences de l'Information, Linguistique, Psychologie, Chimie},
    %                 xtick=data,
    %                 enlarge x limits=0.125,
    %                 x=.075\linewidth,
    %                 xticklabel style={anchor=east, rotate=22.5},
    %                 nodes near coords,
    %                 nodes near coords align={vertical},
    %                 every node near coord/.append style={font=\footnotesize},
    %                 ytick={0, 10, 20, 30, 40, 50, 60, 70, 75, 85, 95, 105},
    %                 y=0.01\linewidth,
    %                 ymin=75,
    %                 ymax=100,
    %                 ybar,
    %                 ylabel=Difficulté,
    %                 ylabel style={at={(ticklabel* cs:1)},
    %                               anchor=south,
    %                               rotate=270}]
    %      \addplot[red!66, fill=red!40] coordinates{
    %        (Archéologie, 78.8)
    %        (Sciences de l'Information, 87.4)
    %        (Linguistique, 87.5)
    %        (Psychologie, 87.9)
    %        (Chimie, 90.4)
    %      };
    %    \end{axis}
    %  \end{tikzpicture}
    %  \caption{Difficulté de la tâche d'extraction de 10 termes-clés avec
    %           \textbf{TopicRank}.}
    %\end{figure}
    \begin{figure}
      \centering
      \begin{tikzpicture}
        \begin{axis}[axis lines=left,
                     symbolic x coords={Archéologie, Sciences de l'Information, Linguistique, Psychologie, Chimie},
                     xtick=data,
                     enlarge x limits=0.125,
                     x=.1\linewidth,
                     xticklabel style={anchor=east, xshift=.5em, yshift=-.25em, rotate=22.5},
                     nodes near coords,
                     nodes near coords align={vertical},
                     every node near coord/.append style={font=\scriptsize},
                     ytick={0, 10, 20, 30, 40, 50, 60, 70, 75, 80, 85, 90, 95, 100},
                     y=0.01\linewidth,
                     ymin=0,
                     ymax=25,
                     ybar=7.5pt,
                     ylabel=F1-mesure (\%),
                     ylabel style={at={(ticklabel* cs:1)},
                                   anchor=south,
                                   rotate=270}]
          \addplot[black!66,
                   pattern=north east lines,
                   pattern color=black!40] coordinates{
            (Archéologie, 22.7)
            (Sciences de l'Information, 12.9)
            (Linguistique, 12.8)
            (Psychologie, 12.2)
            (Chimie, 10.5)
          };
          \addplot[black!66,
                   pattern=north west lines,
                   pattern color=black!66] coordinates{
            (Archéologie, 21.2)
            (Sciences de l'Information, 12.6)
            (Linguistique, 12.5)
            (Psychologie, 12.1)
            (Chimie, 9.6)
          };
          \legend{TF-IDF, TopicRank}
        \end{axis}
      \end{tikzpicture}
      \caption{Difficulté de la tâche d'extraction de 10 termes-clés.}
    \end{figure}

