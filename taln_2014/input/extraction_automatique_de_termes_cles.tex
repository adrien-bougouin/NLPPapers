\section{Extraction automatique de termes-clés}
\label{sec:extraction_automatique_de_termes_cles}
  L'extraction automatique de termes-clés fonctionne en quatre étapes qui sont~:
  \begin{enumerate}
    \item{Préparation des données (segmentation, étiquetage morphosyntaxique,
          etc.)
          \label{item:preprocessing}}
    \item{Extraction des termes-clés candidats
          \label{item:candidate_extraction}}
    \item{Ordonnancement (ou classification) des termes-clés candidats
          \label{item:ranking}}
    \item{Sélection/Extraction des termes-clés à partir des termes-clés
          candidats
          \label{item:extraction}}
  \end{enumerate}
  Les étapes les plus importantes d'un système d'extraction automatique de
  termes-clés sont les étapes~\ref{item:candidate_extraction}
  et~\ref{item:ranking}. Intuitivement, l'ordonnancement (ou la classification)
  des candidats est le c\oe{}ur du système, mais la performance de celui-ci est
  limitée par la qualité de l'ensemble de termes-clés candidats qui lui est
  fourni. Nous estimons qu'un ensemble de candidats est de bonne qualité
  lorsqu'il fournit le plus possible de candidats qui sont effectivement des
  termes-clés, i.e.~des candidats parmi les termes-clés de référence, et
  lorsqu'il fournit peu de candidats non-pertinents, i.e.~des candidats qui ne
  sont pas des termes-clés de référence et qui peuvent dégrader la performance
  du système d'extraction de termes-clés.

  \subsection{Préparation des données}
  \label{subsec:preparation_des_donnees}
    Les documents des collections de données utilisées subissent les mêmes
    prétraitements. Ils sont tout d'abord segmentés en phrases, puis en mots et
    enfin étiquetés en parties du discours. La segmentation en mots est
    effectuée avec l'outil Bonsai, du Bonsai PCFG-LA
    parser\footnote{\url{http://alpage.inria.fr/statgram/frdep/fr_stat_dep_parsing.html}}
    et l'étiquetage en parties du discours est réalisé avec
    MElt~\cite{denis2009melt}. Tous ces outils sont utilisés sans modification
    de leurs paramètres par défaut.

  \subsection{Extraction des termes-clés candidats}
  \label{subsec:extraction_de_termes_cles_candidats}
    Dans les travaux précédents, deux approches sont utilisées. Soit les
    candidats sont extraits à partir des n-grammes filtrés soit ils sont
    extraits par reconnaissance de formes~\cite{hulth2003keywordextraction}.
    Dans ce travail, nous expérimentons avec une méthode pour chacune de ces
    deux approches et une méthode qui extrait les candidats avec un extracteur
    terminologique.

    L'extraction des \textbf{n-grammes} filtrés consiste à extraire toutes les
    séquences ordonnées de $n$ mots, puis à les filtrer avec une liste
    regroupant les mots fonctionnels de la langue (conjonctions, prépositions,
    etc.), ainsi que les mots courants (\og{}près\fg{}, \og{}beaucoup\fg{},
    etc.). Dans ce travail, les n-grammes de taille $n \in {1..3}$ (tri-grammes)
    sont extrait lorsque leurs mots en tête et en queue ne sont pas présents
    dans la liste de mots outils fourni par l'université de
    Neuchâtel\footnote{\url{http://members.unine.ch/jacques.savoy/clef/index.html}}
    (UniNE).

    La reconnaissance de formes consiste à extraire les unités textuelles qui
    respectent des patrons définis. Dans ce travail, nous définissons quatre
    patrons morphosyntaxiques afin d'extraire, comme termes-clés candidats, tous
    les \textbf{syntagmes nominaux}~:
    \begin{itemize}
      \item[]{Np+}
      \item[|]{(A? Nc A+)}
      \item[|]{(A Nc)}
      \item[|]{Nc+}
    \end{itemize}

    L'extraction de \textbf{candidats termes} consiste à extraire les unités
    textuelles qui sont potentiellement des termes. En terminologie, un terme
    est un mot, ou une séquence de mots, représentant un concept spécifique à un
    domaine (ou une discipline) et, par conséquent, traité comme une
    \verb+~?~+unité simple\verb+~?~+. Pour ce travail, nous utilisons
    l'extracteur terminologique TermSuite~\cite{rocheteau2011termsuite}. Une
    terminologie par corpus est construite automatiquement par TermSuite (32~119
    termes en Archéologie, 16~557 termes en Sciences de l'Information, 21~330
    termes en Linguistique, 24~680 termes en Psychologie et 21~020 termes en
    Chimie) et uniquement les unités textuelles se trouvant dans cette
    terminologie sont extraites comme termes-clés candidats.

  \subsection{Extraction de termes-clés}
  \label{subsec:extraction_de_termes_cles}
    Dans les travaux précédents, de nombreuses méthodes sont proposées pour
    extraire les termes-clés~\cite{bougouin2013stateoftheart}. Dans la
    catégorie des méthodes non-supervisées, un grand nombre de techniques
    différentes est proposé, dont la pondération TF-IDF~\cite{jones1972tfidf} et
    l'approche à base de graphe~\cite{mihalcea2004textrank}. De part sa
    simplicité et sa robustesse, le TF-IDF s'impose comme la méthode de
    référence, tandis que les méthodes à base de graphe suscitent un intérêt
    grandissant du fait qu'un graphe permet de représenter simplement les
    relations qu'entretiennent les unités textuelles d'un document et aussi du
    fait de l'important travail théorique autour des graphes.

    La méthode \textbf{TF-IDF} consiste à extraire, comme termes-clés, les
    candidats dont les mots sont les plus importants. Un mot est considéré
    important, dans un document, s'il est fréquent dans le document et s'il est
    spécifique à celui-ci. La spécificité est déterminée à partir de la
    collection de documents~: un mot est considéré spécifique lorsqu'il apparaît
    dans très peu de documents.

    Parmi les méthodes à base de graphe, nous utilisons
    \textbf{TopicRank}~\cite{bougouin2013topicrank}, une méthode qui extrait les
    termes-clés qui représentent les sujets les plus important d'un document.
    Tout d'abord, les termes-clés candidats sont groupés par sujets, puis les
    relations entre ces sujets sont modélisées par un graphe dans lequel les
    sujets sont les n\oe{}uds. Ensuite, l'algorithme
    TextRank~\cite{mihalcea2004textrank} ordonne les sujets par importance, puis
    pour chaque sujet, le candidat le plus représentatif est extrait.

