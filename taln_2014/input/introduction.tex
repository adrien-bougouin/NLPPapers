\section{Introduction}
\label{sec:introduction}
  % * définition de terme-clé, applications et enjeux
  Un terme-clé est un mot ou une expression polylexicale qui représente un
  concept important d'un document auquel il est associé. En pratique, plusieurs
  termes-clés, représentant des concepts différents, sont associés à un même
  document. Ils forment alors un ensemble de termes-clés à partir duquel il est
  possible de déduire le contenu principal du document. Du fait de leur capacité
  à synthétiser le contenu d'un document, les termes-clés sont utilisés dans
  diverses applications en Recherche d'Information (RI)~: résumé
  automatique~\cite{avanzo2005keyphrase}, classification de
  documents~\cite{han2007webdocumentclustering}, indexation
  automatique~\cite{medelyan2008smalltrainingset}, etc. Avec l'essor du
  numérique, de plus en plus de documents (articles scientifiques, articles
  journalistiques, etc.) sont accessibles depuis des médiums d'informations tels
  que Internet. Afin d'efficacement stocker les documents pour permettre à un
  utilisateur de rapidement trouver des documents et aussi de fournir un bref
  aperçu de leur contenu, les tâches sus-mentionnées sont nécessaires.
  Cependant, la majorité des documents ne sont pas associés avec des termes-clés
  et l'ajout manuel de ces derniers n'est pas envisageable, compte tenu du
  nombre important de documents numériques. Pour pallier ce problème, de plus en
  plus de chercheurs s'intéressent à l'extraction automatique de termes-clés et
  certaines campagnes d'évaluations, telles que DEFT~\cite{paroubek2012deft} et
  SemEval~\cite{kim2010semeval}, proposent des tâches d'extraction automatique
  de termes-clés.

  % * qu'est-ce que l'extraction automatique de termes-clés
  % * deux écoles : indexation libre et indexation contrôlée (assignation de
  %                 termes-clés)
  %   -> nous sommes de la première école
  % * deux catégories de méthodes : supervisées et non-supervisées
  %    -> en supervisé ils utilisent la structure des documents
  %    -> très peu de travaux en non-supervisé (filtrage des candidats)
  L'extraction automatique de termes-clés, ou indexation libre, est la tâche qui
  consiste à extraire les unités textuelles les plus importantes d'un document,
  en opposition à la tâche d'assignation automatique de termes-clés, ou
  indexation contrôlée, qui consiste à assigner des termes-clés à partir d'une
  terminologie donnée~\cite{paroubek2012deft}. Parmi les méthodes d'extraction
  automatique de termes-clés existantes, nous distinguons deux catégories~: les
  méthodes supervisées et les méthodes non-supervisées. Dans le cas supervisé,
  la tâche d'extraction de termes-clés est considérée comme une tâche de
  classification binaire~\cite{witten1999kea}, où il s'agit d'attribuer la
  classe \og{}\textit{terme-clé}\fg{} ou \og{}\textit{non terme-clé}\fg{} aux
  termes-clés candidats extraits du document. Une collection de documents
  annotés en termes-clés est alors nécessaire pour l'apprentissage d'un modèle
  de classification utilisant divers traits, allant de la simple fréquence aux
  informations structurelles du document (titre, résumé, introduction,
  conclusion, etc.). Dans le cas non-supervisé, les méthodes attribuent un
  score d'importance à chaque candidat en fonction de divers indicateurs tels
  que la fréquence et la position de la première occurrence dans le document.
  Bien que les méthodes supervisées soient en général plus performantes, la
  faible quantité de documents annotés en termes-clés disponibles, ainsi que la
  forte dépendance des modèles de classification au type des documents à partir
  desquels ils sont appris, poussent les chercheurs à s'intéresser de plus en
  plus aux méthodes non-supervisées.

  % * ici, on cherche à identifier l'échelle de difficulté d'indexation des
  %   documents en Sciences Humaines et Sociales (SHS)
  % * on dispose de 4 collections de notices de 4 disciplines différentes de
  %   SHS + 1 collection de notices de chimie (science dure)
  Dans cette article, nous nous intéressons à l'extraction non-supervisée de
  termes-clés dans les articles scientifiques, et plus particulièrement à la
  performance des méthodes état de l'art dans des domaines de spécialité. Au
  moyen de cinq corpus disciplinaires, notre objectif est d'observer et
  d'analyser l'échelle de difficulté pour l'extraction automatique de
  termes-clés dans des articles scientifiques appartenant à cinq disciplines
  différents~: Archéologie, Sciences de l'Information, Linguistique, Psychologie
  et Chimie. Les quatre premières disciplines appartenant toutes au domaine des
  Sciences Humaines et Sociales (SHS), qui est considéré comme une science
  molle, notre travail permet aussi une comparaison des performances
  d'extraction de termes-clés en science molle et en science dure.

  % * annonce du plan
  L'article est structuré comme suit. Un bref état de l'art des méthodes
  d'extraction de termes-clés est donné dans la
  section~\ref{sec:etat_de_l_art}, les données utilisées sont présentées dans la
  section~\ref{sec:presentation_des_donnees} et les expériences menées, ainsi
  que les résultats obtenus, sont décrits dans la section~\ref{sec:experiences}.
  Enfin, une analyse des résultats est donnée dans la
  section~\ref{sec:discussion}, puis une conclusion générale et des perspectives
  de travaux futurs sont présentés en
  section~\ref{sec:conclusion_et_perspectives}.

