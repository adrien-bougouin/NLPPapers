\section{Introduction}
\label{sec:introduction}
  Un terme-clé est un mot ou une expression polylexicale qui représente un
  concept important d'un document auquel il est associé. En pratique, plusieurs
  termes-clés représentant des concepts différents sont associés à un même
  document. Ils forment alors un ensemble à partir duquel il est
  possible de caractériser, synthétiser, le contenu principal du document. Du
  fait de cette capacité de synthèse, les termes-clés sont utilisés dans de
  nombreuses applications telles que le résumé
  automatique~\cite{avanzo2005keyphrase}, la classification de
  documents~\cite{han2007webdocumentclustering} ou l'indexation
  automatique~\cite{medelyan2008smalltrainingset}. Avec l'essor du numérique et,
  en particulier, le développement des bibliothèques numériques sur Internet, le
  nombre d'articles scientifiques auxquels les chercheurs ont accès ne cesse
  d'augmenter. Les termes-clés, le plus souvent assignés par les auteurs,
  facilitent l'indexation pour la recherche d'articles scientifiques et donnent
  un bref aperçu de leur contenu. Cependant, de nombreux articles n'ont pas de
  termes-clés associés et, compte tenu de leur nombre, l'annotation manuelle de
  ces derniers n'est pas envisageable. Pour pallier ce problème, de plus en plus
  de chercheurs s'intéressent à l'extraction automatique de termes-clés à partir
  d'articles scientifiques et certaines campagnes d'évaluation, telles que
  DEFT~\cite{paroubek2012deft} et SemEval~\cite{kim2010semeval}, proposent des
  tâches sur cette problématique.

  L'extraction automatique de termes-clés, ou indexation libre,
  consiste à extraire les unités textuelles les plus importantes d'un document,
  en opposition à l'assignation automatique de termes-clés, ou
  indexation contrôlée, qui consiste à assigner des termes-clés à partir d'un
  référentiel donné~\cite{paroubek2012deft}. Parmi les méthodes d'extraction
  automatique de termes-clés existantes, nous distinguons deux catégories~: les
  méthodes supervisées et les méthodes non-supervisées. Dans le cadre supervisé,
  la tâche d'extraction de termes-clés est considérée comme une tâche de
  classification~\cite{witten1999kea}, où il s'agit d'attribuer la classe
  \og{}\textit{terme-clé}\fg{} ou \og{}\textit{non terme-clé}\fg{} à des
  termes-clés candidats extraits du document. Une collection de documents
  annotés en termes-clés est alors nécessaire pour l'apprentissage d'un modèle
  de classification reposant sur divers traits tels que la fréquence du
  terme-clé candidat ou sa position dans le document. Dans le cadre
  non-supervisé, les méthodes attribuent un score d'importance aux candidats
  en fonction de divers indicateurs comme leur spécificité par rapport au
  document~\cite{paukkeri2010likey} ou les relations de coocurrence que leurs
  mots entretiennent~\cite{mihalcea2004textrank}. Les méthodes supervisées sont
  plus performantes que les méthodes non-supervisées, mais leur besoin en
  données annotées (pour l'apprentissage) pousse les chercheurs à proposer des
  méthodes non-supervisées compétitives avec les méthodes supervisées.

  Dans cet article, nous nous plaçons dans le contexte de l'extraction
  non-supervisée de termes-clés à partir de documents de nature scientifique.
  Les documents de cette nature appartiennent à des disciplines variées, chacune
  possédant une terminologie qui lui est spécifique. Certains chercheurs
  s'adaptent à ces documents en réduisant l'ensemble des termes-clés candidats à
  ceux les plus spécifiques~\cite{kim2009termextraction}, ou en tenant compte de
  la structure des documents pour extraire les termes-clés apparaissant dans les
  sections les plus favorables~\cite{hofmann2009impactofdocumentstructure}.
  D'autres chercheurs proposent une analyse de ces documents~:
  \newcite{shah2003wherearethekeywords} étudient les sections génériques des
  articles scientifiques et déterminent celles les plus favorable à
  l'introduction de termes-clés, tandis que
  \newcite{bertels2012etudesemantiquedesmotscles} analysent la sémantique des
  termes-clés en domaine de spécialité. À l'instar de
  \newcite{shah2003wherearethekeywords} et de
  \newcite{bertels2012etudesemantiquedesmotscles},
  nous proposons une analyse de l'influence des domaines de spécialité dans
  l'extraction de termes-clés et supposons que chaque discipline est traitée
  avec un degré de difficulté différent. Au moyen de cinq corpus disciplinaires
  (section~\ref{sec:presentation_des_donnees}), nous utilisons différentes
  méthodes d'extraction automatique de termes-clés
  (section~\ref{sec:extraction_automatique_de_termes_cles}) et observons leur
  performance en domaine de spécialité afin de déduire une échelle de difficulté
  disciplinaire (section~\ref{sec:experiences}). Enfin, nous proposons une
  analyse des résultats et déterminons les causes potentielles d'une
  augmentation de la difficulté (section~\ref{sec:discussion}).

%  Dans cet article, nous nous intéressons à la problématique de l'extraction
%  non-supervisée de termes-clés à partir d'articles scientifiques, et plus
%  particulièrement à l'impact que peut avoir la discipline scientifique sur les
%  performances des méthodes. En effet, certaines disciplines comme la
%  linguistique ou la psychologie sont caractérisées par un fort degré de
%  recouvrement et d'ambiguïté entre langue de spécialité et langue générale, ce
%  qui rend la tâche d'extraction de termes-clés plus complexe.
%  Au moyen de cinq corpus disciplinaires (Archéologie, Sciences de
%  l'Information, Linguistique, Psychologie et Chimie), nous comparons les
%  performances de différentes méthodes d'extraction de termes-clés et nous en
%  déduisons une échelle de difficulté disciplinaire.
%
%  L'article est structuré comme suit. Un bref état de l'art est donné dans la
%  section~\ref{sec:etat_de_l_art}, les données utilisées sont présentées dans la
%  section~\ref{sec:presentation_des_donnees} et les expériences menées, ainsi
%  que les résultats obtenus, sont décrits dans la section~\ref{sec:experiences}.
%  Enfin, une analyse des résultats est donnée dans la
%  section~\ref{sec:discussion}, puis une conclusion générale et des perspectives
%  de travaux futurs sont présentés en
%  section~\ref{sec:conclusion_et_perspectives}.

