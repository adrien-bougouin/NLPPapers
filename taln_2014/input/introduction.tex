\section{Introduction}
\label{sec:introduction}
  % * définition de terme-clé, applications et enjeux
  Un terme-clé est un mot ou une expression polylexicale qui représente un
  concept important d'un document auquel il est associé. En pratique, plusieurs
  termes-clés représentant des concepts différents sont associés à un même
  document. Ils forment alors un ensemble à partir duquel il est
  possible d'inférer le contenu principal du document. Du fait de leur capacité
  à synthétiser le contenu d'un document, les termes-clés sont utilisés dans
  de nombreuses applications telles que le résumé
  automatique~\cite{avanzo2005keyphrase}, la classification de
  documents~\cite{han2007webdocumentclustering} ou l'indexation
  automatique~\cite{medelyan2008smalltrainingset}.
  %
  Avec l'essor du numérique, et en particulier le développement des bibliothèques numériques sur Internet, le nombre d'articles scientifiques auxquels les chercheurs ont accès ne cesse d'augmenter.
  %
  Les termes-clés, le plus souvent assignés par les auteurs, facilitent l'indexation et la recherche d'articles scientifiques, et donnent un bref aperçu de leur contenu.
  %
  Cependant, la plupart des articles n'ont pas de termes-clés associés et, compte tenu de leur nombre, l'annotation manuelle de ces derniers n'est pas envisageable.
  %
  Pour pallier à ce problème, de plus en plus de chercheurs s'intéressent à l'extraction automatique de termes-clés à partir d'articles scientifiques et certaines campagnes d'évaluations, telles que DEFT~\cite{paroubek2012deft} et SemEval~\cite{kim2010semeval}, proposent des tâches sur cette problématique.
  %
  % Avec l'essor du
  % numérique, de plus en plus de documents (articles scientifiques, articles
  % journalistiques, etc.) sont accessibles depuis des médiums d'informations tels
  % que Internet. Afin de permettre à un utilisateur de rapidement trouver des
  % documents, ainsi que d'avoir un bref aperçu de leur contenu, les tâches
  % sus-mentionnées sont nécessaires.
  % %
  % Cependant, la majorité des documents ne sont pas associés avec des termes-clés
  % et, compte tenu du nombre important de documents numériques, l'ajout manuel de
  % ces derniers n'est pas envisageable. Pour pallier ce problème, de plus en plus
  % de chercheurs s'intéressent à l'extraction automatique de termes-clés et
  % certaines campagnes d'évaluations, telles que DEFT~\cite{paroubek2012deft} et
  % SemEval~\cite{kim2010semeval}, proposent des tâches d'extraction automatique
  % de termes-clés.

  % * qu'est-ce que l'extraction automatique de termes-clés
  % * deux écoles : indexation libre et indexation contrôlée (assignation de
  %                 termes-clés)
  %   -> nous sommes de la première école
  % * deux catégories de méthodes : supervisées et non-supervisées
  %    -> en supervisé ils utilisent la structure des documents
  %    -> très peu de travaux en non-supervisé (filtrage des candidats)
  L'extraction automatique de termes-clés, ou indexation libre,
  consiste à extraire les unités textuelles les plus importantes d'un document,
  en opposition à l'assignation automatique de termes-clés, ou
  indexation contrôlée, qui consiste à assigner des termes-clés à partir d'une
  terminologie donnée~\cite{paroubek2012deft}. Parmi les méthodes d'extraction
  automatique de termes-clés existantes, nous distinguons deux catégories~: les
  méthodes supervisées et les méthodes non-supervisées. Dans le cas supervisé,
  la tâche d'extraction de termes-clés est considérée comme une tâche de
  classification binaire~\cite{witten1999kea}, où il s'agit d'attribuer la
  classe \og{}\textit{terme-clé}\fg{} ou \og{}\textit{non terme-clé}\fg{} aux
  termes-clés candidats extraits du document.
  %
  Une collection de documents annotés en termes-clés est alors nécessaire pour l'apprentissage d'un modèle de classification reposant sur divers traits tels que la fréquence du terme-clé candidat ou sa position dans le document.
  %
  Dans le cas non-supervisé, les méthodes attribuent un score d'importance à chaque candidat en fonction de divers indicateurs comme sa spécificité par rapport au document~\cite{paukkeri2010likey} ou le nombre de termes avec lesquels il cooccurre~\cite{matsuo2004wordcooccurrence}.
  %
  % Une collection de documents
  % annotés en termes-clés est alors nécessaire pour l'apprentissage d'un modèle
  % de classification reposant sur divers traits, allant de la simple fréquence
  % aux informations structurelles du document (titre, résumé, introduction,
  % conclusion, etc.). Dans le cas non-supervisé, les méthodes attribuent un
  % score d'importance à chaque candidat en fonction de divers indicateurs tels
  % que la fréquence et la position de la première occurrence dans le document.
  Bien que les méthodes supervisées soient en général plus performantes, la
  faible quantité de documents annotés en termes-clés disponibles, ainsi que la
  forte dépendance des modèles de classification au type des documents à partir
  desquels ils sont appris, poussent les chercheurs à s'intéresser de plus en
  plus aux méthodes non-supervisées.

  Dans cet article, nous nous intéressons à la problématique de l'extraction non-supervisée de termes-clés à partir d'articles scientifiques, et plus particulièrement à l'impact que peut avoir la discipline scientifique sur les performances des méthodes.
  %
  En effet, certaines disciplines comme la linguistique ou la psychologie sont caractérisées par un fort degrée de recouvrement et d'ambiguïté entre langue de spécialité et langue générale, ce qui rend la tâche d'extraction de termes-clés plus complexe.
  %
  Au moyen de cinq corpus disciplinaires (Archéologie, Sciences de l'Information, Linguistique, Psychologie et Chimie), nous comparons les performances de différentes méthodes d'extraction de termes-clés et nous en déduisons un échelle de difficulté disciplinaire.
  %
  Nous montrons ensuite qu'il est possible d'utiliser un lexique transdisciplinaire pour résoudre les problèmes d'ambiguïté et améliorer la qualité des termes-clés extraits.

  % * ici, on cherche à identifier l'échelle de difficulté d'indexation des
  %   documents en Sciences Humaines et Sociales (SHS)
  % * on dispose de 4 collections de notices de 4 disciplines différentes de
  %   SHS + 1 collection de notices de chimie (science dure)
  % Dans cet article, nous nous intéressons à l'extraction non-supervisée de
  % termes-clés dans les articles scientifiques, et plus particulièrement à la
  % performance des méthodes d'extraction de termes-clés dans des domaines de
  % spécialité. Au moyen de cinq corpus disciplinaires, notre objectif est
  % d'observer et d'analyser l'échelle de difficulté pour l'extraction
  % automatique de termes-clés dans des articles scientifiques appartenant à cinq
  % disciplines différentes~: Archéologie, Sciences de l'Information,
  % Linguistique, Psychologie et Chimie.
  % \TODO{Dire pourquoi nous nous intéressons aux méthodes non-supervisées}
  % \TODO{Dire pourquoi nous nous intéressons aux articles scientifiques}

  % * annonce du plan
  L'article est structuré comme suit. Un bref état de l'art est donné dans la
  section~\ref{sec:etat_de_l_art}, les données utilisées sont présentées dans la
  section~\ref{sec:presentation_des_donnees} et les expériences menées, ainsi
  que les résultats obtenus, sont décrits dans la section~\ref{sec:experiences}.
  Enfin, une analyse des résultats est donnée dans la
  section~\ref{sec:discussion}, puis une conclusion générale et des perspectives
  de travaux futurs sont présentés en
  section~\ref{sec:conclusion_et_perspectives}.

