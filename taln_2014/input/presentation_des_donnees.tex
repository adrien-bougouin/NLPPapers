\section{Présentation des données}
\label{sec:presentation_des_donnees}
  Nous disposons de cinq collections de notices bibliographiques (Archéologie,
  Sciences de l'Information, Linguistique, Psychologie et Chimie) fournies par
  l'Institut de l’Information Scientifique et
  Technique\footnote{\url{http://www.inist.fr}} (INIST). Chacune de ces notices
  INIST contient le titre et le résumé d'un article, ainsi que ses termes-clés
  associés. Ces derniers sont obtenus semi-automatiquement~: des indexeurs
  professionnels vérifient, corrigent et complètent les sorties d'un système
  dont les entrées sont un document et un (ou plusieurs) référentiel(s)
  terminologique(s). Un référentiel terminologique est une liste disciplinaire
  de termes associés manuellement à des déclencheurs, i.e. des unités textuelles
  qui, lorsqu'elles sont présentes dans le document, impliquent l'usage du terme
  associé en tant que terme-clé. En addition, les indexeurs disposent de règles
  d'indexation (non formatées) qui précisent quels types d'informations doivent
  être présents dans l'ensemble de termes-clés (e.g. en Archéologie il est
  important de connaître la période et la localisation de ce qui fait l'objet de
  l'article).

  Le corpus d'\textbf{Archéologie} est composé de 718 notices INIST. Celles-ci
  représentent des articles parus entre 2001 et 2012 dans 22 revues différentes
  (\textit{Paléo}, \textit{Le bulletin de la Société préhistorique française},
  etc.).

  Le corpus de \textbf{Sciences de l'Information} contient 706 notices INIST
  d'articles publiés entre 2001 et 2012 dans six revues différentes
  (\textit{Documentaliste -- Sciences de l'information}, \textit{Document
  numérique}, etc.).

  Le corpus de \textbf{Linguistique} est constitué de 716 notices INIST
  d'articles parus entre 2000 à 2012 dans 12 revues différentes
  (\textit{Linx -- Revue des linguistes de l'Université Paris Ouest Nanterre La
  Défense}, \textit{Travaux de linguistique}, etc.).

  Le corpus de \textbf{Psychologie} contient 720 notices INIST d'articles
  publiés entre 2001 et 2012 dans sept revues différentes
  (\textit{Enfance}, \textit{Revue internationale de psychologie et de gestion
  des comportements organisationnels}, etc.).

  Le corpus de \textbf{Chimie} est composé de 782 notices INIST d'articles
  publiés entre 1983 et 2012 dans quatre revues (\textit{Comptes Rendus de
  l'Académie des Sciences}, \textit{Comptes Rendus Chimie}, etc.).

  Le tableau~\ref{tab:statistiques_des_corpus} présente les statistiques des
  cinq collections de données présentées ci-dessus. Nous observons tout d'abord
  une différence concernant la taille des documents. Les notices archéologiques
  contiennent plus d'informations que les autres notices et les notices de
  Chimie en contiennent moins. Ceci est dû au fait que les notices
  archéologiques font l'objet d'un effort de présentation du contexte historique
  auquel s'intéresse l'article, tandis que dans les notices de Chimie,
  représentant en grande partie des comptes rendus, ne donnent parfois que le
  noms de l'expériences réalisée et les noms des éléments chimiques qui entrent
  en jeu. Nous remarquons aussi qu'en fonction des collections, le nombre moyen
  de termes-clés assignés aux documents varie de 5,8 termes-clés en Sciences de
  l'Information à 17,7 en Archéologie. Cette variation est due à différents
  facteurs~:
  \begin{itemize}
    \item{taille des notices~: plus une notice contient d'informations, alors
          plus il peut y avoir de termes-clés extraits~;}
    \item{disponibilité du contenu de l'article~: les indexeurs ont parfois
          accès aux articles intégraux afin de recueillir plus d'informations
          utiles à l'extraction de termes-clés~;}
    \item{référentiel terminologique~: plus un référentiel est précis, plus il
          contient de termes et de déclencheurs, alors plus le nombre de
          termes-clés extraits peut être important~;}
    \item{règles d'indexation~: plus il y a de types d'informations nécessaires,
          alors plus il doit y avoir de termes-clés à extraire.}
  \end{itemize}
  Enfin, le nombre de mots qui composent un terme-clé varie aussi selon les
  disciplines. En effet, il est fréquent que les termes-clés en Archéologie ne
  soient composés que d'un seul mot, en général une entité nommée de type
  \textit{période} (\og{}Paléolithique\fg{}, \og{}Mésolithique\fg{},
  \og{}Néolithique\fg{}, etc.) ou \textit{lieu} (\og{}Asie\fg{},
  \og{}Europe\fg{}, \og{}France\fg{}, etc.), tandis que les termes-clés en
  Chimie sont principalement composés de plusieurs mots, du fait de la
  spécialisation systématique de certains termes tels que \og{}composé\fg{}, qui
  peut être \og{}organique\fg{}, \og{}aliphatique\fg{}, \og{}éthylénique\fg{},
  etc.
  \begin{table}
    \centering
    \begin{tabular}{@{~}r|ccccc@{~}}
      \toprule
        & & \textbf{Sciences} & & &\\
        \textbf{Statistique} & \textbf{Archéologie} & \textbf{de} & \textbf{Linguistique} & \textbf{Psychologie} & \textbf{Chimie}\\
        & & \textbf{l'Information} & & &\\
      \hline
        Documents & 718 & 706 & 716 & 720 & 782\\
        Mots/doc. & 219,1 & 119,7 & 156,4 & 185,8 & 104,9\\
        Termes-clés/doc. & 17,7 & 5,8 & 8,0 & 11,0 & 12,9\\
        Mots/terme-clé & 1,3 & 1,7 & 1,7 & 1,6 & 2,2\\
      \bottomrule
    \end{tabular}
    \caption{Statistiques des corpus disciplinaires.
             \label{tab:statistiques_des_corpus}}
  \end{table}

