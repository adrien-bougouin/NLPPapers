\section{Collections de données}
\label{sec:presentation_des_donnees}
  Nous disposons de cinq collections de notices bibliographiques~: Archéologie,
  Sciences de l'Information, Linguistique, Psychologie et Chimie. Ces notices
  sont fournies par l'Institut de l’Information Scientifique et
  Technique\footnote{\url{http://www.inist.fr}} (INIST). Chacune d'elles
  contient le titre, le résumé et les termes-clés associés à un article. Les
  termes-clés sont obtenus semi-automatiquement %~:
  %
  à partir des textes intégraux (non disponibles pour nos travaux) et à partir
  de ressources disciplinaires telles qu'une terminologie ou des spécifications
  précises quant aux types d'informations que les termes-clés doivent
  représenter (e.g.~lieu, période et autre, en Archéologie).
  %des indexeurs
  %professionnels vérifient, corrigent et complètent les sorties d'un système
  %dont les entrées sont un document et un (ou plusieurs) référentiel(s)
  %terminologique(s). Un référentiel terminologique est une liste disciplinaire
  %de termes associés manuellement à des déclencheurs, i.e. des unités textuelles
  %qui, lorsqu'elles sont présentes dans le document, impliquent l'usage du terme
  %associé en tant que terme-clé. En addition, les indexeurs disposent de règles
  %d'indexation (non formatées) qui précisent quels types d'informations doivent
  %être présents dans l'ensemble de termes-clés (e.g. en Archéologie il est
  %important de connaître la période et la localisation de ce qui fait l'objet de
  %l'article).

  Le corpus d'\textbf{Archéologie} est composé de 718 notices INIST. Celles-ci
  représentent des articles parus entre 2001 et 2012 dans 22 revues différentes
  (\textit{Paléo}, \textit{Le bulletin de la Société préhistorique française},
  etc.).

  Le corpus de \textbf{Sciences de l'Information} contient 706 notices INIST
  d'articles publiés entre 2001 et 2012 dans six revues différentes
  (\textit{Documentaliste -- Sciences de l'information}, \textit{Document
  numérique}, etc.).

  Le corpus de \textbf{Linguistique} est constitué de 716 notices INIST
  d'articles parus entre 2000 à 2012 dans 12 revues différentes
  (\textit{Linx -- Revue des linguistes de l'Université Paris Ouest Nanterre La
  Défense}, \textit{Travaux de linguistique}, etc.).

  Le corpus de \textbf{Psychologie} contient 720 notices INIST d'articles
  publiés entre 2001 et 2012 dans sept revues différentes
  (\textit{Enfance}, \textit{Revue internationale de psychologie et de gestion
  des comportements organisationnels}, etc.).

  Le corpus de \textbf{Chimie} est composé de 782 notices INIST d'articles
  publiés entre 1983 et 2012 dans quatre revues (\textit{Comptes Rendus de
  l'Académie des Sciences}, \textit{Comptes Rendus Chimie}, etc.).

  Le tableau~\ref{tab:statistiques_des_corpus} présente les caractéristiques des
  cinq collections de données présentées ci-dessus.
  %
  Les notices INIST sont de petite taille et sont rédigées différemment selon
  les disciplines (voir la figure~\ref{fig:exemple_notice_inist}). Ainsi, les
  notices d'Archéologie font l'objet d'un effort de présentation des contextes
  historiques liés aux travaux présentés, tandis que les notices de Chimie,
  principalement des comptes rendus d'expériences, ne décrive que sommairement
  les expériences réalisées (noms des expériences et éléments chimiques
  impliqués). Les termes-clés associés aux documents varient en nombre et en
  complexité. Il est fréquent que les termes-clés en Archéologie soient composés
  d'un unique mot, tandis que les termes-clés en Chimie sont principalement des
  composés syntagmatiques. En effet, en Archéologie, nous observons qu'un grand
  nombre de termes-clés sont des entités nommées de type \textit{période}
  (e.g.~\og{}Paléolithique\fg{}, \og{}Mésolithique\fg{}, \og{}Néolithique\fg{},
  etc.) ou \textit{lieu} (e.g.~\og{}Asie\fg{}, \og{}Europe\fg{},
  \og{}France\fg{}, etc.) comptant principalement un seul mot, tandis qu'en
  Chimie, nous observons un usage fréquent de notions générales nécessitant une
  spécialisation (e.g.~\og{}\underline{réaction} solvothermale\fg{},
  \og{}\underline{réaction} électrochimique\fg{}, \og{}\underline{réaction}
  thermique\fg{}, etc.). Enfin, il est important de noter le faible rappel
  maximum pouvant être obtenu pour ces corpus. Par exemple, dans le corpus de
  Sciences de l'Information, uniquement 1,3 termes-clés peuvent être extraits
  des notices parmi les 5,8 associés aux notices, en moyenne.
  %
  %Nous observons tout d'abord
  %une différence concernant la taille des documents. Les notices archéologiques
  %ont plus de contenu que les autres notices et les notices de Chimie en ont
  %moins. Ceci est dû au fait que les notices archéologiques font l'objet d'un
  %effort de présentation du contexte historique auquel s'intéresse l'article,
  %tandis que les notices de Chimie, représentant en grande partie des comptes
  %rendus, ne donnent parfois que le nom de l'expérience réalisée et les noms
  %des éléments chimiques qui entrent en jeu. %Nous remarquons aussi qu'en
  %%fonction des collections, le nombre moyen de termes-clés assignés aux
  %%documents varie de 5,8 termes-clés en Sciences de l'Information à 17,7 en
  %%Archéologie. Cette variation est due à différents facteurs %~:
  %
  %dont les ressources disciplinaires.
  %
  %\begin{itemize}
  %  \item{taille des notices~: \TODO{plus de contenu = potentiellement plus de
  %        termes-clés}~;}
  %  \item{disponibilité du contenu de l'article~: les indexeurs ont parfois
  %        accès aux articles intégraux pour compléter l'ensemble de termes-clés
  %        extrait automatiquement~;}
  %  \item{référentiel terminologique~: plus un référentiel est précis, plus il
  %        contient de termes et de déclencheurs, alors plus le nombre de
  %        termes-clés extraits peut être important~;}
  %  \item{règles d'indexation~: plus il y a de types d'information nécessaires,
  %        alors plus il doit y avoir de termes-clés à extraire.}
  %\end{itemize}
  %
  %La longueur des termes-clés varie aussi selon les disciplines. Il est fréquent
  %que les termes-clés en Archéologie ne soient que des mots, en général des
  %entités nommées (49,6\% des termes-clés de références présents dans les
  %notices contiennent des noms propres) de type \textit{période}
  %(\og{}Paléolithique\fg{}, \og{}Mésolithique\fg{}, \og{}Néolithique\fg{}, etc.)
  %ou \textit{lieu} (\og{}Asie\fg{}, \og{}Europe\fg{}, \og{}France\fg{}, etc.),
  %tandis que les termes-clés en Chimie sont principalement des composés
  %syntagmatiques, du fait de la spécialisation systématique de certains termes
  %tels que \og{}composé\fg{}, qui peut être \og{}organique\fg{},
  %\og{}aliphatique\fg{}, \og{}éthylénique\fg{}, etc. Enfin, nous observons
  %d'importantes différences entre les proportions de termes-clés ne pouvant être
  %extraits (sous une quelconque forme fléchie). Cette différence, due à l'usage
  %de ressources externes lors de l'indexation, a une influence sur les résultats
  %des méthodes qui extraient uniquement des termes-clés à partir d'unités
  %textuelles présentes dans les documents.
  \begin{table}
    \centering
    \begin{tabular}{@{~}r|ccccc@{~}}
      \toprule
        & & \textbf{Sciences} & & &\\
        \textbf{Statistique} & \textbf{Archéologie} & \textbf{de} & \textbf{Linguistique} & \textbf{Psychologie} & \textbf{Chimie}\\
        & & \textbf{l'Information} & & &\\
      \hline
        Documents & 718 & 706 & 716 & 720 & 782\\
        Mots/doc. & 219,1 & 119,7 & 156,4 & 185,8 & 104,9\\
        Termes-clés/doc. & 17,7 & 5,8 & 8,0 & 11,0 & 12,9\\
        Mots/terme-clé & 1,3 & 1,7 & 1,7 & 1,6 & 2,2\\
        Termes-clés extractibles (Rappel max.) & 64.3\% & 21.9\% & 61.2\% & 27.6\% & 40.2\%\\
        %Termes-clés extractibles avec Np & 49,6\% & 18,7\% & 9,8\% & 12,5\% & 9,2\%\\
        Termes-clés ne contenant que des Np & 41,3\% & 13,9\% & 7,7\% & 9,3\% & 6,7\%\\
      \bottomrule
    \end{tabular}
    \caption{Caractéristiques des corpus disciplinaires. Le pourcentage de
             termes-clés ne contenant que des Np correspond au pourcentage de
             termes-clés de références qui contiennent uniquement des mots
             étiquetés Np (nom propre) par notre outils d'étiquetage
             morphosyntaxique (MElt).
             \label{tab:statistiques_des_corpus}}
  \end{table}
  \begin{figure}
    %\subfigure[Archéologie]{
      % Archeologie_09-0054907_TEI_final.xml
      \framebox[\linewidth]{
        \parbox{.99\linewidth}{
          \textbf{Variabilité du \underline{gravettien} de \underline{Kostienki}
                  (bassin moyen du Don) et des territoires associés}
          \hfill\underline{\textit{Archéologie}}\\

          Dans la région de Kostienki-\underline{Borschevo}, on observe
          l'expression, à ce jour, la plus orientale du modèle européen de
          l'évolution du \underline{Paléolithique supérieur}. Elle est
          différente à la fois du modèle Sibérien et du modèle de l'Asie
          centrale. Comme ailleurs en \underline{Europe}, le Gravettien
          apparaît à Kostienki vers 28 ka (Kostienki 8 /II/). Par la suite,
          entre 24-20 ka, les techno-complexes gravettiens sont représentés au
          moins par quatre faciès dont deux, ceux de Kostienki 21/III/ et
          Kostienki 4 /II/, ressemblent au Gravettien occidental et deux autres,
          Kostienki-\underline{Avdeevo} et Kostienki 11/II/, sont des faciès
          propres à l'Europe de l'Est, sans analogie à l'Ouest.
        }
      }
    %}
    ~\\~\\
    %\subfigure[Chimie]{
      % Chimie_88-0340321_TEI_final.xml
      \framebox[\linewidth]{
        \parbox{.99\linewidth}{
          \textbf{Etude de la metallation des carbamates d'hydroxy-2,3,4
                  quinoléines}
          \hfill\underline{\textit{Chimie}}\\

          \underline{Lithiations} régiosélectives en position ortho (C3 et C4)
          puis réactions électrophiles.
        }
      }
    %}
    \caption{Exemple de notice INIST.
             \label{fig:exemple_notice_inist}}
  \end{figure}

