\section{Conclusion et perspectives}
\label{sec:conclusion_et_perspectives}
  Dans cet article, nous nous intéressons à la tâche d'extraction automatique de
  termes-clés dans les documents scientifiques et émettons l'hypothèse que sa
  difficulté est variable selon la discipline des documents traités. Pour
  vérifier cette hypothèse, nous disposons de notices bibliographiques réparties
  dans cinq disciplines (archéologie, linguistique, sciences de l'information,
  psychologie et chimie) auxquelles nous appliquons six systèmes d'extraction
  automatique de termes-clés différents. En comparant les termes-clés extraits
  par chaque système avec les termes-clés de référence assignés aux notices dans
  des conditions réels d'indexation, notre hypothèse se vérifie et nous
  observons l'échelle suivante (de la discipline la plus facile à la plus
  difficile)~:
  \begin{enumerate*}
    \item{Archéologie~;}
    \item{Linguistique~;}
    \item{Sciences de l'information~;}
    \item{Psychologie~;}
    \item{Chimie.}
  \end{enumerate*}

  À l'issue de nos expériences et de nos observations du contenu des notices,
  nous constatons deux facteurs ayant un impact sur la difficulté de la tâche
  d'extraction automatique de termes-clés. Tout d'abord, nous observons que
  l'organisation du résumé peut aider l'extraction de termes-clés. Un résumé
  riche en explications et en mises en relations des différents concepts est
  moins difficile à traiter qu'un résumé énumératif pauvre en explications.
  Ensuite, le vocabulaire utilisé dans une discipline peut influer sur la
  difficulté à extraire les termes-clés des documents de cette discipline. Si le
  vocabulaire spécifique contient des composés syntagmatiques dont certains
  éléments sont courants dans la discipline, alors il peut être plus difficile
  d'extraire les termes-clés des documents de cette discipline.

  Des deux facteurs identifiés émergent plusieurs perspectives de travaux
  futurs. Il peut être intéressant d'analyser le discours des documents afin de
  mesurer, en amont, le degré de difficulté de l'extraction de termes-clés. Avec
  une telle connaissance, nous pourrions proposer une méthode capable de
  s'adapter au degré de difficulté en ajustant automatiquement son paramètrage.
  Cependant, l'analyse que nous proposons dans cet article se fonde uniquement
  sur le contenu de notices appartenant à cinq disciplines. Il serait pertinent
  d'étendre cette analyse au contenu intégral des documents scientifiques, ainsi
  que d'élargir le panel de disciplines utilisées dans ce travail, afin
  d'établir des catégories de disciplines plus ou moins difficiles à traiter
  (p.~ex. la chimie fait partie des disciplines expérimentales, qui sont
  difficiles à traiter). Nous observons aussi que le vocabulaire utilisé dans
  une discipline, en particulier celui utilisé pour les termes-clés, peut rendre
  la tâche d'extraction automatique de termes-clés plus difficile. Il est donc
  important de bénéficier de resources telles que des thésaurus pour permettre à
  une méthode d'extraction de termes-clés de s'adapter au domaine. Pour
  TopicRank, par exemple, avoir connaissance de la terminologie utilisée dans
  une discipline peut améliorer le choix du terme-clé le plus représentatif d'un
  sujet. Enfin, il serait intéressant de penser la tâche d'extraction de
  termes-clés comme une tâche d'extraction d'information pour le remplissage
  d'un formulaire. En archéologie, par exemple, il pourrait s'agir d'extraire
  les informations géographiques (pays, régions, etc.), chronologiques (période,
  culture, etc.), ou encore environnementales (animaux, végétaux, etc.).

