\section{Introduction}
\label{sec:introduction}
  Un terme-clé est un mot ou une expression polylexicale qui représente un
  concept important d'un document auquel il est associé. En pratique, plusieurs
  termes-clés représentant des concepts différents sont associés à un même
  document. Ils forment alors un ensemble à partir duquel il est possible de
  caractériser, synthétiser, le contenu du document. Du fait de cette capacité
  de synthèse, les termes-clés sont utilisés dans de nombreuses applications
  telles que le résumé automatique~\cite{avanzo2005keyphrase}, la classification
  de documents~\cite{han2007webdocumentclustering} ou l'indexation
  automatique~\cite{medelyan2008smalltrainingset}. Cependant, tous les documents
  ne sont pas accompagnés de termes-clés et leur assignation manuelle est une
  tâche coûteuse. Pour pallier ce problème, de plus en plus de chercheurs
  s'intéressent à l'extraction automatique de termes-clés, en témoignent les
  récentes campagnes d'évaluation~\cite{paroubek2012deft,kim2010semeval}, ainsi
  que les nombreux travaux à ce sujet~\cite{hasan2014state_of_the_art}.

  L'extraction automatique de termes-clés consiste à extraire du contenu d'un
  document les unités textuelles les plus importantes, celles qui permettent de
  le résumer. Parmi les méthodes d'extraction automatique de termes-clés
  existantes, nous distinguons deux catégories~: les méthodes supervisées et les
  méthodes non-supervisées. Dans le cadre supervisé, la tâche d'extraction de
  termes-clés est considérée comme une tâche de
  classification~\cite{witten1999kea} où il s'agit d'attribuer la classe
  \og{}\textit{terme-clé}\fg{} ou \og{}\textit{non terme-clé}\fg{} aux
  termes-clés candidats du document. Une collection de
  documents annotés en termes-clés est utilisée pour l'apprentissage d'un
  modèle de classification reposant sur divers traits tels que la fréquence du
  terme-clé candidat ou sa position dans le document. Dans le cadre
  non-supervisé, les méthodes attribuent un score d'importance aux candidats
  selon divers indicateurs tels que leur degré de
  spécificité~\cite{jones1972tfidf} ou les relations de coocurrence que leurs
  mots entretiennent~\cite{mihalcea2004textrank}. En général, les méthodes
  supervisées sont plus performantes que les méthodes non-supervisées, mais leur
  besoin en données d'apprentissage annotées et leur dépendance vis-à-vis du
  domaine de ces données d'apprentissage poussent les chercheurs à s'intéresser
  aux méthodes non-supervisées.

  Dans cet article, nous nous plaçons dans le contexte de l'extraction
  non-supervisée de termes-clés à partir de documents de nature scientifique.
  Faisant l'hypothèse que certaines disciplines sont plus difficiles à traiter
  que d'autres, nous présentons diverses stratégies d'extraction de termes-clés
  puis comparons leurs différences de performance. Nous déterminons ensuite
  quels sont les facteurs qui influent sur la difficulté de la tâche
  d'extraction automatique de termes-clés. De la connaissance de ces facteurs
  peut émerger le besoin d'utiliser des ressources externes, telles que des
  thésaurus, souvent mises de côté dans les travaux portant sur l'extraction
  non-supervisée de termes-clés. Cela peut aussi permettre de détecter la
  difficulté en amont de l'extraction de termes-clés afin d'affiner le
  paramétrage de la méthode utilisée.

  Le reste de cet article est organisé comme suit. Dans un premier, temps nous
  présentons les collections de données
  (section~\ref{sec:presentation_des_donnees}) et les méthodes d'extraction de
  termes-clés (section~\ref{sec:extraction_automatique_de_termes_cles}) que nous
  utilisons. Dans un second temps, nous appliquons ces méthodes à nos
  collections de données (section~\ref{sec:experiences}), puis nous discutons
  des différents facteurs observables (section~\ref{sec:discussion}) avant de
  conclure (section~\ref{sec:conclusion_et_perspectives}).

