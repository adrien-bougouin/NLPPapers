\section{Collections de données}
\label{sec:presentation_des_donnees}
  Pour ce travail, nous disposons de cinq corpus disciplinaire de notices
  bibliographiques fournies par l'Inist\footnote{Institut de l'Information
  Scientifique et Technique~: \url{http://www.inist.fr}} dans le cadre du projet
  \textsc{Anr} Termith\footnote{TERMinologie et Indexation de Textes en sciences
  Humaines~: \url{http://www.atilf.fr/ressources/termith/}}~: archéologie,
  linguistique, sciences de l'information, psychologie et chimie. Chaque notice
  contient le titre, le résumé et les termes-clés d'un document auquel elle est
  associée. Les termes-clés sont classés en deux catégories~:
  \begin{itemize}
    \item{les termes-clés d'auteurs, assignés librement par les auteurs afin
          d'attirer le lecteur~;}
    \item{les termes-clés Inist (en français, en anglais ou en espagnole),
          assignés par des ingénieurs documentalistes (indexeurs professionnels)
          selon des règles très précises destinées à améliorer la recherche
          d'information et à homogénéiser l'indexation des notices~:}
    \begin{itemize}
      \item{les termes-clés doivent être du même niveau de spécificité que celui
            du document et peuvent parfois être accompagnés d'un terme-clé plus
            générique pour le restituer dans son contexte~;}
      \item{les termes-clés doivent respecter, autant que possible, le langage
            de la discipline à laquelle appartient le document (termes-clés
            contrôlés)~;}
      \item{pour tous les documents d'une même discipline, un même concept doit
            être représenté par le même terme-clé~;}
      \item{les termes-clés d'un document doivent présenter tous les concepts
            qui y sont importants, même ceux qui sont implicites.}
    \end{itemize}
  \end{itemize}
  Parmi ces différents ensembles de termes-clés, nous utilisons les termes-clés
  français assignés par l'Inist.

  Le corpus d'\textbf{archéologie} est composé de 718 notices. Celles-ci
  représentent des articles parus entre 2001 et 2012 dans 22 revues différentes
  (\textit{Paléo}, \textit{Le bulletin de la Société préhistorique française},
  etc.).

  Le corpus de \textbf{linguistique} est constitué de 716 notices d'articles
  parus entre 2000 à 2012 dans 12 revues différentes (\textit{Linx -- Revue des
  linguistes de l'Université Paris Ouest Nanterre La Défense}, \textit{Travaux
  de linguistique}, etc.).

  Le corpus de \textbf{sciences de l'information} contient 706 notices
  d'articles publiés entre 2001 et 2012 dans six revues différentes
  (\textit{Documentaliste -- Sciences de l'information}, \textit{Document
  numérique}, etc.).

  Le corpus de \textbf{psychologie} contient 720 notices d'articles parus
  entre 2001 et 2012 dans sept revues différentes (\textit{Enfance},
  \textit{Revue internationale de psychologie et de gestion des comportements
  organisationnels}, etc.).

  Le corpus de \textbf{chimie} est composé de 782 notices d'articles publiés
  entre 1983 et 2012 dans quatre revues (\textit{Comptes Rendus de l'Académie
  des Sciences}, \textit{Comptes Rendus Chimie}, etc.).


  Le tableau~\ref{tab:statistiques_des_corpus} présente les caractéristiques des
  cinq collections de données dont nous disposons. Les notices sont de petite
  taille et sont rédigées différemment selon les disciplines
  (cf.~figure~\ref{fig:exemple_notice_inist}). Les notices d'archéologie, par
  exemple, font l'objet d'un effort de présentation du contexte historique lié
  aux travaux présentés, tandis que les notices de chimie, principalement des
  comptes rendus d'expériences, décrivent sommairement (énumèrent) les
  expériences réalisées (noms des expériences, éléments chimiques impliqués,
  etc.). Les termes-clés associés aux documents varient en nombre (de 8,5 à
  16,6) et en complexité. Par exemple, en archéologie, nous observons qu'un
  grand nombre de termes-clés sont des entités nommées principalement composées
  d'un seul mot (p.~ex.~\og{}Paléolithique\fg{}, \og{}Europe\fg{}, etc.), tandis
  qu'en chimie, nous observons un usage fréquent de notions centrales (dans le
  langage de chimie) nécessitant une spécialisation systématique
  (p.~ex.~\og{}\underline{réaction} topotactique\fg{}, \og{}\underline{réaction}
  sonochimique\fg{}, \og{}\underline{réaction} électrochimique\fg{}, etc.).
  Nous remarquons aussi une diversité variable selon les disciplines (de 23,0~\%
  à 40,6~\% de termes-clés différents). En chimie, la diversité plus importante
  que pour les autres disciplines indique une difficulté a priori plus
  importante. Enfin, il est important de noter la faible proportion de
  termes-clés apparaissant dans les notices --- rappel maximum pouvant être
  obtenu. Par exemple, dans le corpus de chimie, uniquement 3,0 termes-clés
  peuvent être extraits des notices parmi les 12,8 associés aux notices, en
  moyenne. Ce dernier point concerne principalement les termes-clés contrôlés,
  qui peuvent être assignés à un document à partir de règles concernant les
  unités textuelles présentent dans le document. Ces règles, dites de
  déclenchement, sont définies manuellement par les indexeurs professionnels et
  ne sont pas disponibles pour ce travail.
  \begin{table}
    \centering
    \begin{tabular}{@{~}r|ccccc@{~}}
      \toprule
        & & & \textbf{Sciences} & &\\
        \textbf{Statistique} & \textbf{Archéologie} & \textbf{Linguistique} & \textbf{de} & \textbf{Psychologie} & \textbf{Chimie}\\
        & & & \textbf{l'information} & &\\
      \hline
        Documents & 718 & 715 & 706 & 720 & 782\\
        Mots/doc. & 219,1 & 156,7 & 119,7 & 185,7 & 105,2\\
        Termes-clés/doc. & 16,6 & 8,0 & 8,5 & 11,6 & 12,8\\
        Mots/terme-clé & 1,3 & 1,8 & 1,7 & 1,6 & 2,2\\
        Diversité des termes-clés & 25,5~\% & 23,0~\% & 25,0~\% & 17,4~\% & 40,6~\% \\~\vspace{-0.75em}\\
        Termes-clés contrôlés & 79,8~\% & 86,9\% & 85,8~\% & 90,9\% & 83,0~\% \\
        Termes-clés non contrôlés & 20,2~\% & 13,1\% & 14,2~\% & ~~9,1\% & 17,0~\% \\~\vspace{-0.75em}\\
        Termes-clés extractibles (Rappel max.) & 62,9~\% & 38,8~\% & 32,4~\% & 27,1~\% & 23,7~\%\\
        $\hookrightarrow$\hfill\small Termes-clés contrôlés extractibles & \small 48,8~\% & \small 34,9~\% & \small 27,9~\% & \small 24,9~\% & \small 21,7~\% \\
        $\hookrightarrow$\hfill\small Termes-clés non contrôlés extractibles & \small 14,1~\% & \small ~~3,9~\% & \small ~~4,5~\% & \small ~~2,2~\% & \small ~~2,0~\% \\
      \bottomrule
    \end{tabular}
    \caption{Caractéristiques des corpus disciplinaires. La diversité des
             termes-clés représente la proportion de termes-clés différents dans
             la discipline $\left(\frac{\mbox{nombre de termes-clés
             différents}}{\mbox{nombre total de termes-clés}}\right)$. Les
             termes-clés extractibles sont les termes-clés pouvant être extraits
             du contenu des documents.
             \label{tab:statistiques_des_corpus}}
  \end{table}

  \begin{figure}
    \begin{minipage}{\linewidth}
      \centering
      % Archeologie_09-0054907_TEI_final.xml
      \framebox[\linewidth]{
        \parbox{.99\linewidth}{
          \textbf{Variabilité du gravettien de Kostienki (bassin moyen du Don)
          et des territoires
          associés\footnote{\url{http://cat.inist.fr/?aModele=afficheN&cpsidt=20563716}}}
          \hfill\underline{\textit{Archéologie}}\\

          \vspace{-0.5em}
          Dans la région de Kostienki-Borschevo, on observe l'expression, à ce
          jour, la plus orientale du modèle européen de l'évolution du
          Paléolithique supérieur. Elle est différente à la fois du modèle
          Sibérien et du modèle de l'Asie centrale. Comme ailleurs en Europe, le
          Gravettien apparaît à Kostienki vers 28 ka (Kostienki 8 /II/). Par la
          suite, entre 24-20 ka, les techno-complexes gravettiens sont
          représentés au moins par quatre faciès dont deux, ceux de Kostienki
          21/III/ et Kostienki 4 /II/, ressemblent au Gravettien occidental et
          deux autres, Kostienki-Avdeevo et Kostienki 11/II/, sont des faciès
          propres à l'Europe de l'Est, sans analogie à l'Ouest.\\

          \vspace{-0.5em}
          \textbf{Termes-clés de référence~:} \underline{Europe$^*$}, Kostienko,
          \underline{Borschevo}, variation$^*$, typologie$^*$, industrie
          osseuse$^*$, industrie lithique$^*$, Europe centrale$^*$,
          \underline{Avdeevo$^*$}, \underline{Paléolithique supérieur$^*$},
          \underline{Gravettien$^*$}.
        }
      }
      ~\\~\\
      % Linguistique_08-0265302_TEI_final.xml
      \framebox[\linewidth]{
        \parbox{.99\linewidth}{
          \textbf{Termes techniques et marqueurs d'argumentation : pour
          débusquer l'argumentation cachée dans}
          \hfill\underline{\textit{Linguistique}}\\
          \textbf{les articles de
          recherche\footnote{\url{http://cat.inist.fr/?aModele=afficheN&cpsidt=17395748}}}\\

          \vspace{-0.5em}
          Les articles de recherche présentent les résultats d'une expérience
          qui modifie l'état de la connaissance dans le domaine concerné. Le
          lecteur néophyte a tendance à considérer qu'il s'agit d'une simple
          description et à passer à côté de l'argumentation au cours de laquelle
          le scientifique cherche à convaincre ses pairs de l'innovation et de
          l'originalité présentées dans l'article et du bien-fondé de sa
          démarche tout en respectant la tradition scientifique dans laquelle il
          s'insère. Ces propriétés spécifiques du discours scientifique peuvent
          s'avérer un obstacle supplémentaire à la compréhension, surtout
          lorsqu'il s'agit d'un article en langue étrangère. C'est pourquoi il
          peut être utile d'incorporer dans l'enseignement des langues de
          spécialité une sensibilisation aux marqueurs linguistiques
          (terminologiques et argumentatifs), qui permettent de dépister le
          développement de cette rhétorique. Les auteurs s'appuient sur deux
          articles dans le domaine de la microbiologie.\\

          \vspace{-0.5em}
          \textbf{Termes-clés de référence~:} Langue scientifique$^*$,
          \underline{argumentation$^*$},  \underline{rhétorique$^*$}, 
          \underline{langue de spécialité$^*$}, \underline{enseignement des
          langues$^*$}, linguistique appliquée$^*$,  \underline{discours
          scientifique$^*$},  \underline{article de recherche}.
        }
      }
      ~\\~\\
      % Chimie_90-0137940_TEI_final.xml
      \framebox[\linewidth]{
        \parbox{.99\linewidth}{
          \textbf{Etude d'un condensat acide
          isocyanurique-urée-formaldéhyde}\footnote{\url{http://cat.inist.fr/?aModele=afficheN&cpsidt=6719275}}
          \hfill\underline{\textit{Chimie}}\\

          \vspace{-0.5em}
          La synthèse d'un condensat acide isocyanurique-urée-formaldéhyde
          utilisant la pyridine en tant que solvant a été effectuée par réaction
          sonochimique.\\

          \vspace{-0.5em}
          \textbf{Termes-clés de référence~:} \underline{Réaction
          sonochimique$^*$}, hétérocycle azote$^*$, cycle 6 chaînons$^*$,
          ether$^*$.
        }
      }
    \end{minipage}
    \caption{Exemples de notices Inist. Les termes-clés soulignés représentent
             les concepts pouvant être extraits depuis le titre ou le résumé de
             la notice. Les termes-clés marqués d'une $^*$ font partie des
             termes-clés contrôlés. Conformément au processus d'évaluation
             standard pour les méthodes d'extraction automatique de termes-clés
             (cf. section~\ref{subsec:mesure_d_evaluation}), les variantes
             flexionnelles d'un terme-clé de référence sont jugées correctes
             (p.~ex. \og{}langue\underline{s} de  spécialité\fg{} peut être
             extrait à la place de \og{}langue de spécialité\fg{}).
             \label{fig:exemple_notice_inist}}
  \end{figure}

