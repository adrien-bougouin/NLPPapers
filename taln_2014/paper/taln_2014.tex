\documentclass[10pt,a4paper,twoside]{article}

\usepackage{times}
\usepackage[utf8]{inputenc}
\usepackage[T1]{fontenc}
\usepackage{graphicx}
\usepackage{amsmath}
\usepackage{relsize}

% faire les \usepackage dont vous avez besoin AVANT le \usepackage{jeptaln2012} 
% add the \usepackage for you packages BEFORE the \usepackage{jeptaln2012}
\usepackage{ifthen}
\usepackage{color}
\usepackage{tikz}
\usepackage{pgfplots}
\usepackage{booktabs}
\usepackage{multirow}
\usepackage{lipsum}
\usepackage{subfigure}
\usepackage[normalem]{ulem} % normalem pour ne pas souligner les et al. des citations
\usepackage[inline]{enumitem}

\usepackage{taln2014}
\usepackage[frenchb]{babel}

\pgfplotsset{compat=1.8}
\usetikzlibrary{calc, positioning, topaths, shapes, arrows, patterns}

\newcommand\newcite[2][]{\ifthenelse{\equal{#1}{}}{\shortcite{#2}}{\shortcite[#1]{#2}}}
\newcommand\TODO[1]{\textcolor{red}{[TODO #1]}}
%\renewcommand\TODO[1]{}
%\renewcommand\lipsum[1][]{}

\title{Influence des domaines de spécialité dans l'extraction de termes-clés}

\author{Adrien Bougouin \quad Florian Boudin \quad Béatrice Daille\\
        LINA -- UMR CNRS 6241, 2 rue de la Houssinière 44322 Nantes Cedex 3, France\\ 
        \texttt{<prenom.nom>@univ-nantes.fr}}

% Titre qui apparait en en-tête (1 ligne maxi)
\fancyhead[CO]{Influence des domaines de spécialité dans l'extraction de termes-clés}
% Auteurs qui apparaissent en en-tête (1 ligne maxi)
\fancyhead[CE]{Adrien Bougouin, Florian Boudin, Béatrice Daille}

\begin{document}
  \maketitle

  \resume{
    Les termes-clés sont les mots ou les expressions polylexicales qui
    représentent le contenu principal d'un document. Ils sont utiles pour
    diverses applications telles que l'indexation automatique ou le résumé
    automatique, mais ne sont pas toujours disponibles. De ce fait, nous nous
    intéressons à la tâche d'extraction automatique de termes-clés et, plus
    particulièrement, à la difficulté de cette tâche lors du traitement de
    documents appartenant à certaines disciplines scientifiques. Au moyen de
    cinq corpus représentant cinq disciplines différentes (archéologie,
    linguistique, sciences de l'information, psychologie et chimie), nous
    déduisons une échelle de difficulté disciplinaire et analysons les facteurs
    qui influent sur cette difficulté.
  }

  \abstract{
    Keyphrases are single or multi-word expressions that represent the main
    content of a document. Keyphrases are useful in many applications such as
    document indexing or text summarization, which are very useful for
    researchers. However, most documents are not provided with keyphrases. To
    tackle this problem, researchers propose methods to automatically extract
    keyphrases from documents of various nature. In this paper, we focus on the
    difficulty of the automatic keyphrase extraction from scientific papers from
    various areas. Using five corpora representing five areas (archaeology,
    linguistics, information sciences, psychology and chemistry), we observe the
    difficulty scale and analyze factors inducing a higher or a lower
    difficulty.
  }

  \motsClefs{Extraction de termes-clés, articles scientifiques, domaines de
  spécialité, méthodes non-supervisées}{Keyphrase extraction, scientific papers,
  specific domain, unsupervised methods}

  \section{Introduction}
\label{sec:introduction}
  % * définition de terme-clé, applications et enjeux
  Un terme-clé est un mot ou une expression polylexicale qui représente un
  concept important d'un document auquel il est associé. En pratique, plusieurs
  termes-clés représentant des concepts différents sont associés à un même
  document. Ils forment alors un ensemble de termes-clés à partir duquel il est
  possible de déduire le contenu principal du document. Du fait de leur capacité
  à synthétiser le contenu d'un document, les termes-clés sont utilisés dans
  diverses applications en Recherche d'Information (RI)~: résumé
  automatique~\cite{avanzo2005keyphrase}, classification de
  documents~\cite{han2007webdocumentclustering}, indexation
  automatique~\cite{medelyan2008smalltrainingset}, etc. Avec l'essor du
  numérique, de plus en plus de documents (articles scientifiques, articles
  journalistiques, etc.) sont accessibles depuis des médiums d'informations tels
  que Internet. Afin de permettre à un utilisateur de rapidement trouver des
  documents, ainsi que d'avoir un bref aperçu de leur contenu, les tâches
  sus-mentionnées sont nécessaires.
  Cependant, la majorité des documents ne sont pas associés avec des termes-clés
  et, compte tenu du nombre important de documents numériques, l'ajout manuel de
  ces derniers n'est pas envisageable. Pour pallier ce problème, de plus en plus
  de chercheurs s'intéressent à l'extraction automatique de termes-clés et
  certaines campagnes d'évaluations, telles que DEFT~\cite{paroubek2012deft} et
  SemEval~\cite{kim2010semeval}, proposent des tâches d'extraction automatique
  de termes-clés.

  % * qu'est-ce que l'extraction automatique de termes-clés
  % * deux écoles : indexation libre et indexation contrôlée (assignation de
  %                 termes-clés)
  %   -> nous sommes de la première école
  % * deux catégories de méthodes : supervisées et non-supervisées
  %    -> en supervisé ils utilisent la structure des documents
  %    -> très peu de travaux en non-supervisé (filtrage des candidats)
  L'extraction automatique de termes-clés, ou indexation libre, est la tâche qui
  consiste à extraire les unités textuelles les plus importantes d'un document,
  en opposition à la tâche d'assignation automatique de termes-clés, ou
  indexation contrôlée, qui consiste à assigner des termes-clés à partir d'une
  terminologie donnée~\cite{paroubek2012deft}. Parmi les méthodes d'extraction
  automatique de termes-clés existantes, nous distinguons deux catégories~: les
  méthodes supervisées et les méthodes non-supervisées. Dans le cas supervisé,
  la tâche d'extraction de termes-clés est considérée comme une tâche de
  classification binaire~\cite{witten1999kea}, où il s'agit d'attribuer la
  classe \og{}\textit{terme-clé}\fg{} ou \og{}\textit{non terme-clé}\fg{} aux
  termes-clés candidats extraits du document. Une collection de documents
  annotés en termes-clés est alors nécessaire pour l'apprentissage d'un modèle
  de classification reposant sur divers traits, allant de la simple fréquence
  aux informations structurelles du document (titre, résumé, introduction,
  conclusion, etc.). Dans le cas non-supervisé, les méthodes attribuent un
  score d'importance à chaque candidat en fonction de divers indicateurs tels
  que la fréquence et la position de la première occurrence dans le document.
  Bien que les méthodes supervisées soient en général plus performantes, la
  faible quantité de documents annotés en termes-clés disponibles, ainsi que la
  forte dépendance des modèles de classification au type des documents à partir
  desquels ils sont appris, poussent les chercheurs à s'intéresser de plus en
  plus aux méthodes non-supervisées.

  % * ici, on cherche à identifier l'échelle de difficulté d'indexation des
  %   documents en Sciences Humaines et Sociales (SHS)
  % * on dispose de 4 collections de notices de 4 disciplines différentes de
  %   SHS + 1 collection de notices de chimie (science dure)
  Dans cette article, nous nous intéressons à l'extraction non-supervisée de
  termes-clés dans les articles scientifiques, et plus particulièrement à la
  performance des méthodes d'extraction de termes-clés dans des domaines de
  spécialité. Au moyen de cinq corpus disciplinaires, notre objectif est
  d'observer et d'analyser l'échelle de difficulté pour l'extraction
  automatique de termes-clés dans des articles scientifiques appartenant à cinq
  disciplines différentes~: Archéologie, Sciences de l'Information,
  Linguistique, Psychologie et Chimie.
  \TODO{Dire pourquoi nous nous intéressons aux méthodes non-supervisées}
  \TODO{Dire pourquoi nous nous intéressons aux articles scientifiques}

  % * annonce du plan
  L'article est structuré comme suit. Un bref état de l'art est donné dans la
  section~\ref{sec:etat_de_l_art}, les données utilisées sont présentées dans la
  section~\ref{sec:presentation_des_donnees} et les expériences menées, ainsi
  que les résultats obtenus, sont décrits dans la section~\ref{sec:experiences}.
  Enfin, une analyse des résultats est donnée dans la
  section~\ref{sec:discussion}, puis une conclusion générale et des perspectives
  de travaux futurs sont présentés en
  section~\ref{sec:conclusion_et_perspectives}.


  %\section{État de l'art}
\label{sec:etat_de_l_art}
  % Quel est le fonctionnement général des méthodes d'extraction automatique de
  % termes-clés ?
  L'extraction automatique de termes-clés est une tâche répartie en quatre
  étapes. Les documents sont considérés un par un, ils sont tout d'abord
  enrichis linguistiquement (segmentés en phrases, segmentés en mots, étiquetés
  en parties du discours, etc.), puis des termes-clés candidats en sont extraits
  et classifiés, ou ordonnés, afin de pouvoir sélectionner leurs termes-clés
  (cf. figure~\ref{fig:etapes_de_l_extraction_de_termes_cles}). L'extraction des
  termes-clés candidats et leur classification, ou ordonnancement, sont les deux
  étapes auxquels nous nous intéressons dans cet article. En effet, la
  classification, ou l'ordonnancement, des termes-clés candidats est le c\oe{}ur
  de la tâche d'extraction de termes-clés, mais ses performances dépendent de la
  qualité des candidats préalablement extraits.
  \begin{figure}
    \tikzstyle{io}=[
      ellipse,
      minimum width=5cm,
      minimum height=2cm,
      fill=green!20,
      draw=green!33,
      transform shape,
      font={\huge}
    ]
    \tikzstyle{component}=[
      text centered,
      thick,
      rectangle,
      minimum width=11cm,
      minimum height=2.5cm,
      fill=cyan!20,
      draw=cyan!33,
      transform shape,
      font={\huge\bfseries}
    ]

    \centering
    \begin{tikzpicture}[thin,
                        align=center,
                        scale=.45,
                        node distance=2cm,
                        every node/.style={text centered, transform shape}]
      \node[io] (document) {document};
      \node[component] (preprocessing) [right=of document] {Prétraitement Linguistique};
      \node[component] (candidate_extraction) [below=of preprocessing] {Extraction des candidats};
      \node[component] (candidate_classification_and_ranking) [below=of candidate_extraction] {
        \begin{tabular}{r|lcl}
          Classification & des candidats\\
          Ordonnancement &\\
        \end{tabular}
      };
      \node[component] (keyphrase_selection) [below=of candidate_classification_and_ranking] {Sélection des termes-clés};
      \node[io] (keyphrases) [right=of keyphrase_selection] {termes-clés};

      \path[->, thick] (document) edge (preprocessing);
      \path[->, thick] (preprocessing) edge (candidate_extraction);
      \path[->, thick] (candidate_extraction) edge (candidate_classification_and_ranking);
      \path[->, thick] (candidate_classification_and_ranking) edge (keyphrase_selection);
      \path[->, thick] (keyphrase_selection) edge (keyphrases);
    \end{tikzpicture}
    \caption{Les quatre principales étapes de l'extraction automatique de
             termes-clés. \label{fig:etapes_de_l_extraction_de_termes_cles}}
  \end{figure}

  \subsection{Extraction de termes-clés candidats}
  \label{subsec:extraction_de_termes_cles_candidats}
    % Quel est l'objectif ?
    L'objectif de l'extraction de termes-clés candidats est de réduire l'espace
    des solutions possibles aux seules unités textuelles ayant des
    particularités semblables à celles des termes-clés (tels qu'ils peuvent être
    donnés par des humains). Deux avantages directs à cela sont la réduction du
    temps de calcul nécessaire et la suppression d'unités textuelles non
    pertinentes pouvant engendrer du bruit affectant les performances de
    l'extraction de termes-clés. Pour distinguer les différents candidats
    extraits, nous définissons deux catégories~: les candidats positifs, qui
    sont de réels termes-clés, et les candidats non positifs, qui ne sont pas de
    réels termes-clés. Parmi les candidats non positifs, nous distinguons aussi
    deux catégories~: les candidats porteurs d'informations utiles à la
    promotions de candidats positifs et les candidats non pertinents, qui sont
    considérés comme des erreurs d'extraction.

    % Quels sont les différentes méthodes utilisées pour extraire les
    % termes-clés candidats ?
    Dans les travaux précédents pour l'extraction automatique de termes-clés,
    trois méthodes d'extraction de candidats sont classiquement utilisées~:
    l'extraction de n-grammes filtrés avec une liste de mots outils,
    l'extraction d'unités minimales de sens ayant pour tête un nom (chunks
    nominaux) ou l'extraction des unités textuelles respectant certains patrons
    syntaxiques.

    \subsubsection{Méthodes utilisées}
    \label{subsubsec:methodes_explorees}

      L'extraction de n-grammes consiste en l'extraction de séquences ordonnées
      de $n$ mots. Cette extraction est très exhaustive, elle fournit une grande
      quantité de termes-clés candidats, maximisant ainsi la quantité de
      candidats positifs ou porteurs d'informations, mais aussi la quantité de
      candidats non pertinents. Pour supprimer un grand nombre de ces candidats
      non pertinents, il est courant d'utiliser une liste de mots outils
      (conjonctions, prépositions, mots usuels, etc.). Une unité textuelle
      contenant un mot outils au début ou à la fin ne doit pas être considérée
      comme un terme-clé candidat. Bien que l'extraction de n-grammes filtrés
      fournisse un ensemble bruité de candidats, elle est encore largement
      utilisée dans les méthodes supervisées d'extraction automatique de
      termes-clés~\cite{witten1999kea,turney1999learningalgorithms,hulth2003keywordextraction}.
      En effet, du fait de leur phase d'apprentissage, les méthodes
      non-supervisées sont très robustes et donc peu sensibles aux bruits.

      L'extraction de chunks nominaux consiste en l'extraction d'unités
      minimales de sens ayant pour tête un nom. Contrairement aux n-grammes, les
      chunks sont toujours des unités textuelles grammaticalement correctes. De
      ce fait, elles sont moins arbitraires. D'un point de vue linguistique,
      l'extraction de chunks nominaux est plus justifiée que l'extraction de
      n-grammes. Cependant, son caractère plus restrictif ne permet pas
      d'extraire autant de candidats positifs. Il est donc important de
      s'assurer que les propriétés des chunks nominaux sont en accord avec les
      propriétés des termes-clés (tels qu'ils peuvent être donnés par des
      humains). Les expériences mennées par \newcite{hulth2003keywordextraction}
      et \newcite{eichler2010keywe} avec les chunks nominaux montrent une
      amélioration des performances vis-à-vis de l'usage de n-grammes.
      Cependant, \newcite{hulth2003keywordextraction} montre aussi qu'en tirant
      partie de l'étiquetage en parties du discours, l'extraction de termes-clés
      à partir de n-grammes donne des performances au-dessus de celles obtenues
      avec les chunks nominaux.

      L'extraction d'unités textuelles respectant certains patrons syntaxiques
      permet l'extraction de candidats qui sont grammaticalement et
      syntaxiquement contrôlés\footnote{Il est possible d'extraire les
      chunks nominaux à partir de patrons syntaxiques.}. Alors que
      \newcite{hulth2003keywordextraction} extraient des candidats avec les
      patrons de termes-clés les plus fréquents (plus de 10 occurrences) dans
      une collection de documents annotés, d'autres chercheurs tels que
      \newcite{wan2008expandrank} se concentrent uniquement sur les plus longues
      séquences de nom (nom propres inclus) et d'adjectifs.

      % Que veut-on apporter ?
      Dans le but d'améliorer la qualité des candidats extraits avec les
      méthodes susmentionnées, \newcite{kim2009termextraction} proposent de
      filtrer les candidats en fonction de leur spécificité vis-à-vis du domaine
      du document analysé. Cette spécificité est déterminée en fonction du
      rapport entre la fréquence d'un candidat dans le document et le nombre de
      documents dans lesquels il est présent \cite[TF-IDF]{jones1972tfidf}.
      Intuitivement, un candidat très fréquent dans le document analysé est
      d'autant plus spécifique à celui-ci s'il est présent dans très peu
      d'autres documents.

    \subsubsection{Méthodes pouvant être utilisées}
    \label{subsubsec:methodes_pouvant_etre_utilisees}
    
      L'idée de \newcite{kim2009termextraction} d'extraire des candidats
      spécifiques est une idée intéressant qu'il nous est possible d'étendre à
      l'usage de méthodes d'extraction terminologique. Les méthodes automatiques
      d'extraction terminologique, dont
      \newcite{castellvi2001automatictermdetection} en font la revue, extraient
      des mots ou expressions symbolisant un concept et devant, à ce titre,
      faire partie des entrées d'un index terminologique.

      Le système TermSuite\footnote{\url{http://www.ttc-project.eu}} est un
      outils état de l'art pour l'extraction terminologique monolingue et
      bilingue. Il combine à la fois le savoir linguistique et des statistiques
      calculées dans une collection de documents de spécialité (d'un domaine
      particulier), afin d'extraire les termes et leurs variantes (fautes
      d'orthographe incluses).

      En recherche d'information, \newcite{evans1996nounphraseanalysis} font le
      constat que les unités textuelles décrivant le mieux le contenu d'un
      document, et donc permettant de mieux l'indexer, sont majoritairement des
      termes. Ils proposent alors une méthode qui extrait dans un premier temps
      les groupes nominaux d'un document, puis en extrait les sous-composants.
      Ils définissent trois types distincts de sous-composants~:
      \begin{itemize}
        \item{les collocations~;}
        \item{les couples tête-modifieur~;}
        \item{les variantes des groupes prépositionnels.}
      \end{itemize}
      Dans le cas de l'extraction automatique de termes-clés, seuls les
      sous-composants étant présents dans le document analysé peuvent être
      extraits. Dans notre cas, il n'est donc pas toujours possible d'utiliser
      les couples tête-modifieurs et les variantes des groupes prépositionnels.

  \subsection{Classification/Ordonnancement des termes-clés candidats}
  \label{subsec:classification_ordonnancement_des_termes_cles_candidats}
    % Quel est l'objectif ?
    L'étape de classification/ordonnancement intervient après l'extraction des
    termes-clés candidats. Son rôle est de déterminer quels sont, parmi les
    candidats, les termes-clés du document analysé. La classification est
    majoritairement utilisée par les méthodes supervisées. Les méthodes
    non-supervisées, quant à elles,  effectuent en général un ordonnancement des
    candidats. Dans cet article, nous nous intéressons aux méthodes
    non-supervisées, nous ne présentons donc que ces dernières. De plus, les
    différences notables entre les différentes méthodes supervisées ne sont que
    dans le choix du classifieur (classifieur naïf bayésien, arbres de
    décisions, perceptron multi-couches, etc) ou des traits (TF-IDF, première
    position, parties du discours, etc.).

    % Quels sont les différentes méthodes non-supervisées existantes pour
    % l'extraction de termes-clés ?
    % Quels sont les inconvénients des méthodes actuelles ?
    % Que veut-on apporter ?
    Les méthodes non-supervisées d'extraction automatique de termes-clés
    emploient des méthodes très différentes, allant du simple usage de mesures
    statistiques~\cite{jones1972tfidf,paukkeri2010likey} au groupement des mots
    par fréquence de co-occurrences~\cite{liu2009keycluster}, en passant par
    l'utilisation de modèles de langues obtenus à partir de données
    non-annotées~\cite{tomokiyo2003languagemodel}, ou encore la construction
    d'un graphe de co-occurrences~\cite{mihalcea2004textrank}. Bien que dans la
    méthode que nous proposons nous faisons aussi du groupement, celle-ci
    appartient à la dernière catégorie de méthodes mentionnée. Nous nous
    intéressons donc, ici, aux méthodes dites \og à base de graphe~\fg.

    \newcite{mihalcea2004textrank} proposent une méthode d'ordonnancement
    d'unités textuelles à partir d'un graphe. Leur méthode, utilisée pour le
    résumé automatique et l'extraction de termes-clés, s'inspire de la méthode
    PageRank~\cite{brin1998pagerank} qui détermine l'importance d'une page Web
    grâce aux autres pages Web qui s'y réfèrent, ainsi qu'aux autres pages Web
    auxquelles elle se réfère. Le plus une pages Web est citée dans des pages
    Web différentes, le plus elle est importante, et le plus elle est
    importante, le plus elle donne d'importance aux pages Web auxquelles elle
    fait référence. Cette notion de référence entre les pages Web est
    représentée par un graphe dans lequel les n\oe{}uds sont des pages Web et
    les références les liens entre elles. Ensuite, une mesure de centralité,
    inspirée de la mesure de centralité eigenvector, est appliquée pour ordonner
    les pages Web par importance. Pour l'extraction de termes-clés avec TextRank
    les pages Web sont remplacées par les mots (nom et adjectifs) du document
    analysé et les liens entre eux symbolisent leur(s) co-occurrence(s) dans une
    fenêtre de $2$ mots. L'étape finale de la méthode consiste à générer les
    termes-clés à partir des $k$ mots les plus importants, les mots-clés. Les
    mots-clés sont marqués dans le documents et les plus longues séquences de
    mots-clés adjacents sont extraits comme termes-clés.

    \newcite{wan2008expandrank} proposent SingleRank. Cette méthode présente
    deux améliorations à TextRank. La première amélioration étend l'usage de la
    co-occurrence comme lien entre les mots et la seconde remplace l'étape de
    génération des termes-clés. Dans un premier temps, les auteurs pondèrent les
    liens de co-occurrence par leur nombre de co-occurrences calculées avec une
    fenêtres de $10$ mots. Ainsi un mot co-occurrent deux fois avec un autre est
    relié à celui-ci par un poids de $2$. Ce poids est ensuite utilisé pour
    transférer plus ou moins d'importance lors de l'ordonnancement.
    %Ce nouvel ordonnancement utilise plus d'informations présentes dans le
    %document analysé et est dotant plus efficace quand le document analysé est
    %de grande taille.
    L'importance ainsi calculée pour chaque mots est maintenant utilisée
    pour donner un score aux termes-clés candidats et ainsi les ordonner.
    Celui-ci est calculé en faisant la somme du score d'importance de chacun des
    mots qui le compose. Bien que la méthode SingleRank donne, dans la majorité
    des cas, des résultats meilleurs que ceux de TextRank, faire la somme du
    score d'importance des mots pour ordonner les candidats est une approche
    maladroite. En effet, cela a pour effet de faire monter dans le classement
    des candidats qui se recouvrent. Ainsi, dans le document
    \textit{as\_2002\_000700ar} de la collection DEFT (présentée dans la
    section~\ref{subsec:corpus_pour_l_extraction_de_termes_cles}) le candidat
    positif \og bio-politique~\fg\ est classé neuvième, alors que les autres
    candidats contenant \og bio-politique~\fg\ plus d'autres mots non
    nécessairement importants occupent les classements $2$ à $8$. Dans nos
    travaux nous ordonnons les termes-clés candidats en tenant compte de
    l'importance du sujet qu'ils représentent puis choisissons un représentant
    par sujet, nous évitons ainsi le problème rencontré avec SingleRank.

    Toujours dans l'optique d'utiliser plus d'informations pour améliorer
    l'efficacité de l'ordonnancement, \newcite{wan2008expandrank} étendent
    SingleRank en utilisant des documents similaires (documents voisins) au
    document en cours d'analyse. Leur approche consiste à observer les
    co-occurrences dans les documents voisins pour renforcer ou ajouter des
    liens dans le graphe initial. Cependant, en fonction de la similarité entre
    un document voisin et le document en cours d'analyse, des liens non
    pertinents risques d'être ajoutés. Pour y remédier, les auteurs utilisent le
    score de similarité entre les deux documents comme facteur d'atténuation de
    l'ajout ou du renforcement de liens. Cette approche donne des résultats au
    delà de ceux de SingleRank, mais il est important de noter que ses
    performances sont fortement liées à la possession de documents du même
    domaine que les documents analysés.

    A l'instar de \newcite{wan2008expandrank},
    \newcite{tsatsaronis2010semanticrank} tentent d'améliorer TextRank en
    modifiant le processus de création des liens entre les n\oe{}uds (mots) du
    graphe. Dans leur approche, un lien entre deux mots est créé et pondéré en
    fonction du lien sémantique de ces derniers selon
    WordNet~\cite{miller1995wordnet} ou
    Wikipedia~\cite{milne2008wikipediasemanticrelatedness}. Les expériences
    menées par les auteurs ne montrent pas d'amélioration vis-à-vis de TextRank.
    Ils montrent toute fois que l'ajout de connaissances lors de
    l'ordonnancement -- en biaisant l'ordonnancement en faveur des mots
    apparaissant dans le titre du document analysé ou en ajoutant le TFIDF dans
    le calcul du score d'importance -- améliore les résultats de sorte que
    ceux-ci soient au-dessus de ceux de TextRank.

    L'usage de sujets dans le processus d'ordonnancement avec TextRank est
    proposé par \newcite{liu2010topicalpagerank}. Reposant sur un modèle
    LDA~\cite[Latent Dirichlet Allocation]{blei2003lda}, leur méthode effectue
    un ordonnancement particulier pour chaque sujet, puis fusionne les rangs des
    mots pour chacun de ces ordonnancements, afin d'obtenir un ordonnancement
    global. L'ordonnancement en fonction de chaque sujet est biaisé par la
    probabilité de trouver un mot donné sachant ce sujet. Dan notre travail,
    nous émettons aussi l'hypothèse que le sujet auquel appartient une unité
    textuelle doit jouer un rôle majeur dans le processus d'ordonnancement.
    Cependant, nous tentons de nous abstraire de tout usage de documents
    supplémentaires et n'utilisons donc pas le modèle LDA. De même, il nous
    semble plus judicieux d'effectuer un seule ordonnancement, en prenant
    directement en compte l'appartenance d'une unité textuelle à un sujet
    particulier.


  \section{Collections de données}
\label{sec:presentation_des_donnees}
  Nous disposons de cinq collections de notices bibliographiques~: Archéologie,
  Sciences de l'Information, Linguistique, Psychologie et Chimie. Ces notices
  sont fournies par l'Institut de l’Information Scientifique et
  Technique\footnote{\url{http://www.inist.fr}} (INIST). Chacune d'elles
  contient le titre, le résumé et les termes-clés associés à un article. Les
  termes-clés sont obtenus semi-automatiquement %~:
  %
  à partir des textes intégraux (non disponibles pour nos travaux) et à partir
  de ressources disciplinaires telles qu'une terminologie ou des spécifications
  précises quant aux types d'informations que les termes-clés doivent
  représenter (e.g.~lieu, période et autre, en Archéologie).
  %des indexeurs
  %professionnels vérifient, corrigent et complètent les sorties d'un système
  %dont les entrées sont un document et un (ou plusieurs) référentiel(s)
  %terminologique(s). Un référentiel terminologique est une liste disciplinaire
  %de termes associés manuellement à des déclencheurs, i.e. des unités textuelles
  %qui, lorsqu'elles sont présentes dans le document, impliquent l'usage du terme
  %associé en tant que terme-clé. En addition, les indexeurs disposent de règles
  %d'indexation (non formatées) qui précisent quels types d'informations doivent
  %être présents dans l'ensemble de termes-clés (e.g. en Archéologie il est
  %important de connaître la période et la localisation de ce qui fait l'objet de
  %l'article).

  Le corpus d'\textbf{Archéologie} est composé de 718 notices INIST. Celles-ci
  représentent des articles parus entre 2001 et 2012 dans 22 revues différentes
  (\textit{Paléo}, \textit{Le bulletin de la Société préhistorique française},
  etc.).

  Le corpus de \textbf{Sciences de l'Information} contient 706 notices INIST
  d'articles publiés entre 2001 et 2012 dans six revues différentes
  (\textit{Documentaliste -- Sciences de l'information}, \textit{Document
  numérique}, etc.).

  Le corpus de \textbf{Linguistique} est constitué de 716 notices INIST
  d'articles parus entre 2000 à 2012 dans 12 revues différentes
  (\textit{Linx -- Revue des linguistes de l'Université Paris Ouest Nanterre La
  Défense}, \textit{Travaux de linguistique}, etc.).

  Le corpus de \textbf{Psychologie} contient 720 notices INIST d'articles
  publiés entre 2001 et 2012 dans sept revues différentes
  (\textit{Enfance}, \textit{Revue internationale de psychologie et de gestion
  des comportements organisationnels}, etc.).

  Le corpus de \textbf{Chimie} est composé de 782 notices INIST d'articles
  publiés entre 1983 et 2012 dans quatre revues (\textit{Comptes Rendus de
  l'Académie des Sciences}, \textit{Comptes Rendus Chimie}, etc.).

  Le tableau~\ref{tab:statistiques_des_corpus} présente les caractéristiques des
  cinq collections de données présentées ci-dessus.
  %
  Les notices INIST sont de petite taille et sont rédigées différemment selon
  les disciplines (voir la figure~\ref{fig:exemple_notice_inist}). Les notices
  d'Archéologie font l'objet d'un effort de présentation des contextes
  historiques liés aux travaux présentés, tandis que les notices de Chimie,
  principalement des comptes rendus d'expériences, décrivent sommairement les
  expériences réalisées (noms des expériences, éléments chimiques impliqués,
  etc.). Les termes-clés associés aux documents varient en nombre et en
  complexité. Il est fréquent que les termes-clés en Archéologie soient composés
  d'un unique mot, tandis que les termes-clés en Chimie sont principalement des
  composés syntagmatiques. En effet, en Archéologie, nous observons qu'un grand
  nombre de termes-clés sont des entités nommées (41,3\% des termes-clés de
  référence présents dans les documents sont étiquetés Np) de type
  \textit{période} (e.g.~\og{}Paléolithique\fg{}, \og{}Mésolithique\fg{},
  \og{}Néolithique\fg{}, etc.) ou \textit{lieu} (e.g.~\og{}Asie\fg{},
  \og{}Europe\fg{}, \og{}France\fg{}, etc.) comptant principalement un seul mot,
  tandis qu'en Chimie, nous observons un usage fréquent de notions générales
  nécessitant une spécialisation (e.g.~\og{}\underline{réaction}
  solvothermale\fg{}, \og{}\underline{réaction} électrochimique\fg{},
  \og{}\underline{réaction} thermique\fg{}, etc.). Enfin, il est important de
  noter le faible rappel maximum pouvant être obtenu pour ces corpus. Par
  exemple, dans le corpus de Sciences de l'Information, uniquement 1,3
  termes-clés peuvent être extraits des notices parmi les 5,8 associés aux
  notices, en moyenne.
  %
  %Nous observons tout d'abord
  %une différence concernant la taille des documents. Les notices archéologiques
  %ont plus de contenu que les autres notices et les notices de Chimie en ont
  %moins. Ceci est dû au fait que les notices archéologiques font l'objet d'un
  %effort de présentation du contexte historique auquel s'intéresse l'article,
  %tandis que les notices de Chimie, représentant en grande partie des comptes
  %rendus, ne donnent parfois que le nom de l'expérience réalisée et les noms
  %des éléments chimiques qui entrent en jeu. %Nous remarquons aussi qu'en
  %%fonction des collections, le nombre moyen de termes-clés assignés aux
  %%documents varie de 5,8 termes-clés en Sciences de l'Information à 17,7 en
  %%Archéologie. Cette variation est due à différents facteurs %~:
  %
  %dont les ressources disciplinaires.
  %
  %\begin{itemize}
  %  \item{taille des notices~: \TODO{plus de contenu = potentiellement plus de
  %        termes-clés}~;}
  %  \item{disponibilité du contenu de l'article~: les indexeurs ont parfois
  %        accès aux articles intégraux pour compléter l'ensemble de termes-clés
  %        extrait automatiquement~;}
  %  \item{référentiel terminologique~: plus un référentiel est précis, plus il
  %        contient de termes et de déclencheurs, alors plus le nombre de
  %        termes-clés extraits peut être important~;}
  %  \item{règles d'indexation~: plus il y a de types d'information nécessaires,
  %        alors plus il doit y avoir de termes-clés à extraire.}
  %\end{itemize}
  %
  %La longueur des termes-clés varie aussi selon les disciplines. Il est fréquent
  %que les termes-clés en Archéologie ne soient que des mots, en général des
  %entités nommées (49,6\% des termes-clés de références présents dans les
  %notices contiennent des noms propres) de type \textit{période}
  %(\og{}Paléolithique\fg{}, \og{}Mésolithique\fg{}, \og{}Néolithique\fg{}, etc.)
  %ou \textit{lieu} (\og{}Asie\fg{}, \og{}Europe\fg{}, \og{}France\fg{}, etc.),
  %tandis que les termes-clés en Chimie sont principalement des composés
  %syntagmatiques, du fait de la spécialisation systématique de certains termes
  %tels que \og{}composé\fg{}, qui peut être \og{}organique\fg{},
  %\og{}aliphatique\fg{}, \og{}éthylénique\fg{}, etc. Enfin, nous observons
  %d'importantes différences entre les proportions de termes-clés ne pouvant être
  %extraits (sous une quelconque forme fléchie). Cette différence, due à l'usage
  %de ressources externes lors de l'indexation, a une influence sur les résultats
  %des méthodes qui extraient uniquement des termes-clés à partir d'unités
  %textuelles présentes dans les documents.
  \begin{table}
    \centering
    \begin{tabular}{@{~}r|ccccc@{~}}
      \toprule
        & & \textbf{Sciences} & & &\\
        \textbf{Statistique} & \textbf{Archéologie} & \textbf{de} & \textbf{Linguistique} & \textbf{Psychologie} & \textbf{Chimie}\\
        & & \textbf{l'Information} & & &\\
      \hline
        Documents & 718 & 706 & 716 & 720 & 782\\
        Mots/doc. & 219,1 & 119,7 & 156,4 & 185,8 & 104,9\\
        Termes-clés/doc. & 17,7 & 5,8 & 8,0 & 11,0 & 12,9\\
        Mots/terme-clé & 1,3 & 1,7 & 1,7 & 1,6 & 2,2\\
        Termes-clés extractibles (Rappel max.) & 64.3\% & 21.9\% & 61.2\% & 27.6\% & 40.2\%\\
        %Termes-clés extractibles avec Np & 49,6\% & 18,7\% & 9,8\% & 12,5\% & 9,2\%\\
        Termes-clés ne contenant que des Np & 41,3\% & 13,9\% & 7,7\% & 9,3\% & 6,7\%\\
      \bottomrule
    \end{tabular}
    \caption{Caractéristiques des corpus disciplinaires. Le pourcentage de
             termes-clés ne contenant que des Np correspond au pourcentage de
             termes-clés de références qui contiennent uniquement des mots
             étiquetés Np (nom propre) par notre outils d'étiquetage
             morphosyntaxique (MElt).
             \label{tab:statistiques_des_corpus}}
  \end{table}
  \begin{figure}
    %\subfigure[Archéologie]{
      % Archeologie_09-0054907_TEI_final.xml
      \framebox[\linewidth]{
        \parbox{.99\linewidth}{
          \textbf{Variabilité du \underline{gravettien} de \underline{Kostienki}
                  (bassin moyen du Don) et des territoires associés}
          \hfill\underline{\textit{Archéologie}}\\

          Dans la région de Kostienki-\underline{Borschevo}, on observe
          l'expression, à ce jour, la plus orientale du modèle européen de
          l'évolution du \underline{Paléolithique supérieur}. Elle est
          différente à la fois du modèle Sibérien et du modèle de l'Asie
          centrale. Comme ailleurs en \underline{Europe}, le Gravettien
          apparaît à Kostienki vers 28 ka (Kostienki 8 /II/). Par la suite,
          entre 24-20 ka, les techno-complexes gravettiens sont représentés au
          moins par quatre faciès dont deux, ceux de Kostienki 21/III/ et
          Kostienki 4 /II/, ressemblent au Gravettien occidental et deux autres,
          Kostienki-\underline{Avdeevo} et Kostienki 11/II/, sont des faciès
          propres à l'Europe de l'Est, sans analogie à l'Ouest.
        }
      }
    %}
    ~\\~\\
    %\subfigure[Chimie]{
      % Chimie_88-0340321_TEI_final.xml
      \framebox[\linewidth]{
        \parbox{.99\linewidth}{
          \textbf{Etude de la metallation des carbamates d'hydroxy-2,3,4
                  quinoléines}
          \hfill\underline{\textit{Chimie}}\\

          \underline{Lithiations} régiosélectives en position ortho (C3 et C4)
          puis réactions électrophiles.
        }
      }
    %}
    \caption{Exemple de notices INIST.
             \label{fig:exemple_notice_inist}}
  \end{figure}


  \section{Extraction automatique de termes-clés}
\label{sec:extraction_automatique_de_termes_cles}
  L'extraction non-supervisée de termes-clés peut se décomposer en quatre étapes
  (cf.~figure~\ref{fig:processing_steps}). Tout d'abord, les documents sont un à
  un enrichis linguistiquement (segmentés en phrases, segmentés en mots et
  étiquetés en parties du discours), des termes-clés candidats y sont ensuite
  sélectionnés, puis ordonnés par importance et enfin, les $k$ plus importants
  sont sélectionnés en tant que termes-clés. Les étapes les plus importantes
  d'un système d'extraction automatique de termes-clés sont celles de sélection
  des candidats et d'ordonnancement de ceux-ci. Intuitivement, l'ordonnancement
  des candidats est le c\oe{}ur du système, mais la performance de celui-ci est
  limitée par la qualité de l'ensemble de termes-clés candidats qui lui est
  fourni. Un ensemble de candidats est de bonne qualité lorsqu'il fournit un
  maximum de candidats présents dans l'ensemble des termes-clés de référence et
  lorsqu'il fournit peu de candidats non-pertinents, c'est-à-dire des candidats
  qui ne sont pas dans l'ensemble des termes-clés de référence et qui peuvent
  dégrader la performance du système d'extraction de termes-clés utilisé.
  \begin{figure}
    \tikzstyle{io}=[
      ellipse,
      minimum width=5cm,
      minimum height=2cm,
      %fill=green!20,
      draw,%=green!33,
      transform shape,
      font={\huge}
    ]
    \tikzstyle{component}=[
      text centered,
      %thick,
      rectangle,
      minimum width=11cm,
      minimum height=2cm,
      %fill=cyan!20,
      draw,%=cyan!33,
      transform shape,
      font={\huge\bfseries}
    ]

    \centering
    \begin{tikzpicture}[thin,
                        align=center,
                        scale=.45,
                        node distance=2cm,
                        every node/.style={text centered, transform shape}]
      \node[io] (document) {document};
      \node[component] (preprocessing) [right=of document] {Prétraitement Linguistique};
      \node[component] (candidate_extraction) [below=of preprocessing] {Sélection des candidats};
      \node[component] (candidate_classification_and_ranking) [below=of candidate_extraction] {
        %\begin{tabular}{r|l}
        %  Ordonnancement & \multirow{2}{*}[-2pt]{des candidats}\\
        %  Classification & \\
        %\end{tabular}
        Ordonnancement des candidats
      };
      \node[component] (keyphrase_selection) [below=of candidate_classification_and_ranking] {Sélection des termes-clés};
      \node[io] (keyphrases) [right=of keyphrase_selection] {termes-clés};

      \path[->, thick] (document) edge (preprocessing);
      \path[->, thick] (preprocessing) edge (candidate_extraction);
      \path[->, thick] (candidate_extraction) edge (candidate_classification_and_ranking);
      \path[->, thick] (candidate_classification_and_ranking) edge (keyphrase_selection);
      \path[->, thick] (keyphrase_selection) edge (keyphrases);
%      \draw[decorate, decoration={brace, mirror, amplitude=5pt}, thick] ($(candidate_classification_and_ranking.north west)+(-0.9,0.5)$) -- ($(keyphrase_selection.south west)+(-0.9,-0.5)$) node[midway, xshift=-9em] {
%        \huge
%        \begin{tabular}{c}
%          \textbf{Extraction}\\
%          \textbf{des}\\
%          \textbf{termes-clés}\\
%        \end{tabular}
%      };
    \end{tikzpicture}
    \caption{Chaîne de traitements d'un système non-supervisé d'extraction
             automatique de termes-clés.
             \label{fig:processing_steps}}
  \end{figure}

  \subsection{Préparation des données}
  \label{subsec:preparation_des_donnees}
    Les documents des collections de données utilisées subissent tous les mêmes
    prétraitements. Ils sont tout d'abord segmentés en phrases, puis en mots et
    enfin étiquetés en parties du discours. Dans ce travail, la segmentation en
    phrase est effectuée par le \textit{PunktSentenceTokenizer} disponible avec
    la librairie Python NLTK~\cite[\textit{Natural Language
    ToolKit}]{bird2009nltk}, la segmentation en mots est effectuée par l'outil
    Bonsai du Bonsai PCFG-LA
    parser\footnote{\url{http://alpage.inria.fr/statgram/frdep/fr_stat_dep_parsing.html}}
    et l'étiquetage en parties du discours est réalisé par
    MElt~\cite{denis2009melt}. Tous ces outils sont utilisés avec leurs
    paramètres par défaut.

  \subsection{Sélection des termes-clés candidats}
  \label{subsec:extraction_de_termes_cles_candidats}
    Dans les travaux précédents, deux approches sont fréquemment utilisées. Soit
    les méthodes sélectionnent les n-grammes (filtrés) en tant que termes-clés
    candidats, soit elles sélectionnent les candidats par reconnaissance de
    forme~\cite{hulth2003keywordextraction}. Dans ce travail, nous expérimentons
    trois méthodes différentes~: deux méthodes conformes aux approches standards
    et une méthode sélectionnant les candidats termes obtenus par un extracteur
    terminologique. Aucun travail portant sur l'extraction automatique de
    termes-clés n'a, à notre connaissance, utilisé une telle approche. Compte
    tenu de la nature (disciplinaire) de nos données, nous faisons l'hypothèse
    que les candidats termes, tels que définis dans le domaine de l'extraction
    terminologique, sont des termes-clés candidats pertinants. Ces trois
    méthodes de sélection fournissent des ensembles de candidats de qualités
    différentes (cf. ci-dessous), ce qui nous permet par la suite d'identifier
    les facteurs qui influent sur la difficulté de l'extraction automatique de
    termes-clés.

    La \textbf{sélection des n-grammes filtrés} consiste à extraire du document
    toutes les séquences ordonnées de $n$ mots, puis à les filtrer avec un
    anti-dictionnaire regroupant les mots fonctionnels de la langue
    (conjonctions, prépositions, etc.) et les mots courants (\og{}près\fg{},
    \og{}beaucoup\fg{}, etc.). Dans ce travail, nous suivons
    \newcite{witten1999kea} et sélectionnons les n-grammes de taille $n \in
    \{1..3\}$ ($\{1..3\}$-grammes) lorsque leurs mots en tête et en queue ne
    sont pas présents dans l'anti-dictionnaire fourni par l'université de
    Neuchâtel\footnote{\url{http://members.unine.ch/jacques.savoy/clef/index.html}}
    (\textit{IR Multilingual Resources at UniNE}). La sélection des n-grammes
    est très exhaustive, elle fournit un grand nombre de termes-clés candidats,
    ce qui permet de maximiser la quantité de candidats présents dans l'ensemble
    des termes-clés de référence, mais ce qui maximise aussi la quantité de
    candidats erronés (bruités).
    
    \textit{Exemples de $\{1..3\}$-grammes sélectionnés à partir de \og{}bassin
    moyen du Don\fg{} dans la notice d'archéologie de la
    figure~\ref{fig:exemple_notice_inist}~: \og{}bassin\fg{}, \og{}moyen\fg{},
    \og{}Don\fg{}, \og{}bassin moyen\fg{} et \og{}moyen du Don\fg{}}.

    La \textbf{reconnaissance de formes} consiste à sélectionner les unités
    textuelles qui respectent certains patrons grammaticaux. Les termes-clés
    candidats sélectionnés par reconnaissance de forme ont l'avantage d'avoir
    une nature contrôlée avec précision (p. ex. des groupes nominaux), ce qui
    les rend plus fondés linguistiquement, ainsi que de meilleure qualité que
    les n-grammes. Dans ce travail, nous utilisons le patron
    \texttt{/(NOM | ADJ)+/} afin de sélectionner les plus longues séquences de
    noms (noms propres inclus) et d'adjectifs~\cite{hassan2010conundrums}.
    
    \textit{Exemples de \texttt{/(NOM | ADJ)+/} sélectionnés à partir de
    \og{}bassin moyen du Don\fg{} dans la notice d'archéologie de la
    figure~\ref{fig:exemple_notice_inist}~: \og{}bassin moyen\fg{} et
    \og{}Don\fg{}.}

    La \textbf{sélection de candidats termes} consiste à sélectionner les unités
    textuelles qui sont potentiellement des termes, tels que définis dans le
    domaine de l'extraction terminologique. En terminologie, un terme est un mot
    ou une séquence de mots représentant un concept spécifique à un domaine, ou
    une discipline. Dans ce travail, nous utilisons l'extracteur terminologique
    TermSuite~\cite{rocheteau2011termsuite}, qui est capable de détecter des
    candidats termes (simples et complexes) ainsi que leurs variantes.
    Contrairement à la méthode de sélection des plus longues séquences de noms
    et d'adjectifs, la sélection des candidats termes de TermSuite se fonde sur
    un travail de spécification linguistique et terminologique des termes. Les
    patrons grammaticaux utilisés par TermSuite sont donc plus précis (p.~ex.
    \texttt{/NOM à NOM/}, \texttt{/NOM en NOM/}, \texttt{/ADJ NOM à NOM ADJ/},
    etc.).
    
    \textit{Exemples de candidats termes sélectionnés à partir de \og{}bassin
    moyen du Don\fg{} dans la notice d'archéologie de la
    figure~\ref{fig:exemple_notice_inist}~: \og{}bassin\fg{}, \og{}Don\fg{},
    \og{}bassin moyen\fg{} et \og{}bassin moyen du Don\fg{}.}

  \subsection{Ordonnancement des termes-clés candidats}
  \label{subsec:extraction_de_termes_cles}
    Un grand nombre de méthodes sont proposées dans la catégorie des méthodes
    non-supervisées. Parmi elles, les méthodes d'ordonnancement
    TF-IDF~\cite{jones1972tfidf} et TopicRank~\cite{bougouin2013topicrank}. De
    part sa simplicité et sa robustesse, la méthode TF-IDF s'impose comme la
    méthode de référence pour l'extraction non-supervisée de
    termes-clés\footnote{Notons qu'une variante de la pondération TF-IDF est
    utilisée en Recherche
    d'Information~\cite[Okapi]{robertson1999okapi,claveau2012vectorisation}.
    Bien que cette variante est jugée plus efficace en Recherche d'Information,
    celle-ci n'a, à notre connaissance, jamais été employée pour l'extraction
    automatique de termes-clés. Notre objectif n'étant pas de trouver la
    meilleure méthode d'extraction de termes-clés, nous utilisons la méthode
    originale.}, tandis que les méthodes à base de graphe, telles que TopicRank,
    suscitent un intérêt grandissant. En effet, les graphes permettent de
    représenter simplement et efficacement les unités textuelles d'un document
    et leurs relations en son sein. De plus, ils bénéficient de nombreuses
    études théoriques donnant lieu à des outils et algorithmes efficaces pour
    résoudre divers problèmes. TF-IDF et TopicRank ont un fonctionnement très
    différent (cf. ci-dessous), ce qui nous permet par la suite d'identifier les
    facteurs qui influent sur la difficulté de l'extraction automatique de
    termes-clés.

    La méthode \textbf{TF-IDF} consiste à extraire en tant que termes-clés les
    candidats dont les mots sont importants. Un score d'importance (TF-IDF) est
    attribué à chaque mot des candidats et l'importance d'un candidat est
    calculé par la somme du score d'importance de ses mots. Selon TF-IDF, un mot
    est considéré important dans un document s'il y est fréquent (TF élevé) et
    s'il a une forte spécificité (IDF élevé). Cette dernière est déterminée à
    partir d'une collection de documents\footnote{Dans ce travail, nous
    utilisons la collection dont est extrait le document.}~: moins il y a de
    documents qui contiennent le mot, plus forte est sa spécificité.

    \textbf{TopicRank}~\cite{bougouin2013topicrank} extrait les termes-clés qui
    représentent les sujets les plus importants d'un document. Tout d'abord,
    TopicRank groupe les termes-clés candidats selon leur appartenance à un
    sujet, représente les documents sous la forme d'un graphe de sujets et
    ordonne les sujets selon leur importance dans le
    graphe~\cite{mihalcea2004textrank}. Enfin, le terme-clé candidat le plus
    représentatif d'un sujet, celui qui apparaît en premier dans le document,
    est extrait en tant que terme-clé\footnote{Ci nécessaire, les termes-clés
    extraits sont pondérés et ordonnés selon le score d'importance de leur sujet
    respectif}.
    
    TopicRank groupe les termes-clés candidats selon une mesure de similarité
    lexicale (cf. équation~\ref{equ:similarity}). Cependant, TermSuite fournit
    un groupement terminologique des termes et de leurs variantes. Lorsque les
    termes-clés candidats sont ceux extraits avec TermSuite, nous tirons profit
    de ce groupement terme/variantes à la place de celui fondé sur la similarité
    lexicale. Tenant compte du groupement (moins naïf) de TermSuite, TopicRank
    distingue alors les candidats \og{}Kostienki 11/II/\fg{} et \og{}Kostienki
    21/III/\fg{} qui représentent des faciès différents (cf.
    figure~\ref{fig:exemple_notice_inist}).
    \begin{align}
      \text{similarité}(c_1, c_2) &= \frac{\|c_1~\cap~c_2\|}{\|c_1~\cup~c_2\|}, \label{equ:similarity}
    \end{align}
    \begin{center}
      où $c_1$ et $c_2$ sont deux termes-clés candidats.
    \end{center}


  \section{Expériences}
\label{sec:experiences}
  Dans cette section, nous présentons les expériences menées dans le but
  d'observer l'échelle de difficulté pour l'extraction automatique de
  termes-clés en domaines de spécialité à partir des méthodes TF-IDF et
  TopicRank et en fonction des candidats qui sont extrait~: $\{1..3\}$-grammes
  filtrés, plus longues séquences de noms et d'adjectifs et candidats termes.

  \subsection{Mesure d'évaluation}
  \label{subsec:mesure_d_evaluation}
    Afin de mesurer l'échelle de difficulté pour l'extraction automatique de
    termes-clés en domaines de spécialité, nous utilisons la MAP (\textit{Mean
    Average Precision}), qui mesure la capacité d'une méthode à ordonner
    correctement les termes-clés candidats, c'est-à-dire à extraire en premier des
    candidats qui sont présents dans la liste des termes-clés de référence (cf.
    équation~\ref{equ:map}). Alors qu'il est plus courant d'utiliser la
    précision, le rappel et la f-mesures pour comparer les méthodes entre elles,
    notre choix se porte sur la MAP à cause du nombre variable de termes-clés de
    référence assignés aux documents selon les disciplines (de 8,0 en
    linguistique à 16,6 en archéologie). La MAP étant appliquée à tous les
    candidats ordonnés et non pas à un sous ensemble (p.~ex. les 10 premiers),
    comme pour la précision, le rappel et la f-mesure, il ne peut y avoir de
    biais lorsque nous comparons l'extraction de termes-clés entre deux
    disciplines.
    \begin{align}
      \text{MAP} &= \frac{1}{\|\text{DOCUMENTS}\|} \sum_{d \in \text{DOCUMENTS}} \frac{\mathlarger\sum_{t_i \in \text{extraction}(d)~\cap~\text{référence}(d)} \text{précision}@i}{\|\text{référence}(d)\|} \label{equ:map}
    \end{align}
    \hfill\begin{minipage}{.824\linewidth}
      \begin{tabular}{@{}l@{~~-~~}p{.9\linewidth}@{}}
        où & extraction$(d)$ représente l'ensemble ordonné des termes-clés
             candidats $t_i$ de rang $i$ pour le document $d$,\\
           & référence$(d)$ représente l'ensemble des termes-clés de référence
             du document $d$,\\
           & précision$@i$ représente la précision de l'extraction calculée au
             rang $i$,\\
           & DOCUMENTS est l'ensemble des documents de la collection. \\
      \end{tabular}
    \end{minipage}
    
    En accord avec l'évaluation menée dans les travaux précédents, nous
    considérons correcte l'extraction d'une variante flexionnelle d'un terme-clé
    de référence~\cite{kim2010semeval}. Les opérations de comparaison entre les
    termes-clés de référence et les termes-clés extraits sont donc effectuées à
    partir de la racine des mots qui les composent en utilisant la méthode de
    \newcite{porter1980suffixstripping}.

  \subsection{Résultats}
  \label{subsec:resultats}
    \TODO{Donner des exemple d'extraction TF-IDF et TopicRank}
    La figure~\ref{fig:resultats} montre la performance des méthodes
    d'extraction de termes-clés lorsque les candidats extraits sont soit des
    $\{1..3\}$-grammes filtrés, soit des groupes nominaux, soit des candidats
    termes non filtrés.
    %
    Notre hypothèse de départ selon laquelle la tâche d'extraction de
    termes-clés présente un degré de difficulté différent selon la discipline
    scientifique se vérifie.
    %
    %Nous distinguons effectivement une différence de difficulté pour
    %l'extraction de termes-clés dans les différentes disciplines.
    %
    L'Archéologie
    est la discipline pour laquelle la tâche d'extraction automatique de
    termes-clés est la moins difficile, la Chimie étant la discipline la plus
    difficile, précédée par la Psychologie, la Linguistique et les Sciences de
    l'Information.

    \TODO{Mieux argumenter}
    Nous observons aussi que le choix des candidats a une forte influence sur
    certaines méthodes. Avec les $\{1..3\}$-grammes, TopicRank obtient des
    résultats deux à trois fois inférieurs à ceux obtenus avec les groupes
    nominaux ou les candidats termes, %.
    %
    tandis que les résultats de la méthode TF-IDF subissent des dégradations
    négligeables. Cela est dû à l'exhaustivité de l'ensemble de
    $\{1..3\}$-grammes, faisant de celui-ci un ensemble comportant de nombreux
    candidats non pertinents qui dégradent les performances de TopicRank
    (dégradation du groupement en sujets, renforcement de liens non pertinents
    dans le graphe, etc.). Dans le cas de la méthode TF-IDF, cette dégradation
    est moins conséquente, car les candidats non pertinents ont une faible
    spécificité et se trouvent donc principalement en queue du classement par
    importance des candidats~\cite{kim2009termextraction}. En opposition,
    lorsque nous comparons les résultats obtenus à partir des candidats termes
    à ceux obtenus à partir des groupes nominaux, pour le corpus d'Archéologie,
    nous observons une légère dégradation des résultats de la méthode TF-IDF et
    une amélioration de ceux de TopicRank. Pour TF-IDF la dégradation des
    résultats est due à un ajout important, vis-à-vis des groupes nominaux, de
    candidats composés de déterminants et de prépositions\footnote{Les candidats
    termes extraits sont environ deux fois plus nombreux que les groupes
    nominaux et la moitié d'entre eux contiennent des prépositions et des
    déterminants.}, alors que très peu de termes-clés de référence en
    contiennent (3,5\%). Dans le cas de TopicRank, sa capacité à créer des liens
    entre des candidats terminologiquement fondés, regroupés terminologiquement
    et présents dans des documents riches en informations, tels qu'en
    Archéologie, est un atout important lors de l'extraction de termes-clés.
    %
    %Ceci est dû à l'exhaustivité de l'ensemble
    %de n-grammes, faisant de celui-ci un ensemble comportant de nombreux
    %candidats non pertinents qui dégradent les performances de TopicRank
    %(dégradation du groupement en sujets, renforcement de liens non pertinents
    %dans le graphe, etc.). Dans le cas de la méthode TF-IDF, cette dégradation
    %est moins conséquente, car les candidats non pertinents ne sont pas des
    %unités textuelles spécifiques~\cite{kim2009termextraction}, ils se trouvent
    %donc en queue du classement.
    %
    %Nous observons aussi que, dans le cas de TopicRank, la
    %performance n'est pas stable selon l'ensemble de candidats utilisé. Les
    %$\{1..3\}$-grammes dégradent significativement les résultats de TopicRank,
    %en comparaison avec les autres types de candidats, alors que ce n'est pas le
    %cas pour la méthode TF-IDF. Ceci est dû au fait que l'ensemble de
    %$\{1..3\}$-grammes contient un grand nombre de candidats non pertinents qui
    %ont pour effet de dégrader les résultats de TopicRank, mais pas ceux de la
    %méthode TF-IDF, car les candidats non pertinents ne sont pas
    %spécifiques~\cite{kim2009termextraction}. \TODO{Compléter pour les autres
    %types de candidats.}
    %
    %Deux
    %disciplines se distinguent sur l'échelle de difficulté, l'Archéologie se
    %veut être la discipline la plus facile pour la tâche d'extraction de
    %termes-clés, en opposition avec la Chimie qui, elle, est la plus difficile.
    %Quant aux Sciences de l'Information, à la Linguistique et à la Psychologie,
    %celles-ci ont sensiblement le même degré de difficulté. Les deux méthodes
    %d'extraction de termes-clés se distinguent aussi par leur comportement. La
    %méthode TF-IDF est plus stable que TopicRank lorsque l'on compare les
    %résultats obtenus à partir des tri-grammes et des termes. En effet, en
    %comparaison avec l'ensemble de termes, l'ensemble de tri-grammes contient
    %plus de candidats non pertinants qui dégradent la qualité des groupements en
    %sujets de TopicRank, tandis que la notion de spécificité incorporée dans
    %TF-IDF permet de placer ces candidats en queue du classement et donc de
    %diminuer leur effet négatif.
    \begin{figure}
      \centering
      \subfigure[$\{1..3\}$-grammes]{
        \begin{tikzpicture}%[scale=.75]
          \pgfkeys{/pgf/number format/.cd, use comma, fixed, fixed zerofill, precision=3}
          \begin{axis}[axis lines=left,
                       symbolic x coords={Archéologie, Sciences de l'Information, Linguistique, Psychologie, Chimie},
                       xtick=data,
                       enlarge x limits=0.125,
                       x=.1\linewidth,
                       xticklabel style={anchor=east, xshift=.5em, yshift=-.25em, rotate=22.5},
                       nodes near coords,
                       nodes near coords align={vertical},
                       every node near coord/.append style={font=\scriptsize},
                       ytick={0, 0.100, 0.200, 0.300, 0.400, 0.500},
                       y=\linewidth,
                       ymin=0,
                       ymax=0.22,
                       ybar=10pt,
                       ylabel=MAP,
                       ylabel style={at={(ticklabel* cs:1)},
                                     anchor=south,
                                     rotate=270}]
            \addplot[black!66,
                     pattern=north east lines,
                     pattern color=black!40] coordinates{
              (Archéologie, 0.156)
              (Sciences de l'Information, 0.103)
              (Linguistique, 0.095)
              (Psychologie, 0.068)
              (Chimie, 0.052)
            };
            \addplot[black!66,
                     pattern=north west lines,
                     pattern color=black!66] coordinates{
              (Archéologie, 0.035)
              (Sciences de l'Information, 0.043)
              (Linguistique, 0.038)
              (Psychologie, 0.029)
              (Chimie, 0.018)
            };
            \legend{TF-IDF, TopicRank}
          \end{axis}
        \end{tikzpicture}
      }
      \subfigure[Groupes nominaux]{
        \begin{tikzpicture}%[scale=.75]
          \pgfkeys{/pgf/number format/.cd, use comma, fixed, fixed zerofill, precision=3}
          \begin{axis}[axis lines=left,
                       symbolic x coords={Archéologie, Sciences de l'Information, Linguistique, Psychologie, Chimie},
                       xtick=data,
                       enlarge x limits=0.125,
                       x=.1\linewidth,
                       xticklabel style={anchor=east, xshift=.5em, yshift=-.25em, rotate=22.5},
                       nodes near coords,
                       nodes near coords align={vertical},
                       every node near coord/.append style={font=\scriptsize},
                       ytick={0, 0.100, 0.200, 0.300, 0.400, 0.500},
                       y=\linewidth,
                       ymin=0,
                       ymax=0.22,
                       ybar=10pt,
                       ylabel=MAP,
                       ylabel style={at={(ticklabel* cs:1)},
                                     anchor=south,
                                     rotate=270}]
            \addplot[black!66,
                     pattern=north east lines,
                     pattern color=black!40] coordinates{
              (Archéologie, 0.180)
              (Sciences de l'Information, 0.120)
              (Linguistique, 0.098)
              (Psychologie, 0.079)
              (Chimie, 0.063)
            };
            \addplot[black!66,
                     pattern=north west lines,
                     pattern color=black!66] coordinates{
              (Archéologie, 0.150)
              (Sciences de l'Information, 0.128)
              (Linguistique, 0.095)
              (Psychologie, 0.078)
              (Chimie, 0.055)
            };
            \legend{TF-IDF, TopicRank}
          \end{axis}
        \end{tikzpicture}
      }
      \subfigure[Candidats termes]{
        \begin{tikzpicture}%[scale=.75]
          \pgfkeys{/pgf/number format/.cd, use comma, fixed, fixed zerofill, precision=3}
          \begin{axis}[axis lines=left,
                       symbolic x coords={Archéologie, Sciences de l'Information, Linguistique, Psychologie, Chimie},
                       xtick=data,
                       enlarge x limits=0.125,
                       x=.1\linewidth,
                       xticklabel style={anchor=east, xshift=.5em, yshift=-.25em, rotate=22.5},
                       nodes near coords,
                       nodes near coords align={vertical},
                       every node near coord/.append style={font=\scriptsize},
                       ytick={0, 0.100, 0.200, 0.300, 0.400, 0.500},
                       y=\linewidth,
                       ymin=0,
                       ymax=0.22,
                       ybar=10pt,
                       ylabel=MAP,
                       ylabel style={at={(ticklabel* cs:1)},
                                     anchor=south,
                                     rotate=270}]
            \addplot[black!66,
                     pattern=north east lines,
                     pattern color=black!40] coordinates{
              (Archéologie, 0.166)
              (Sciences de l'Information, 0.110)
              (Linguistique, 0.104)
              (Psychologie, 0.071)
              (Chimie, 0.060)
            };
            \addplot[black!66,
                     pattern=north west lines,
                     pattern color=black!66] coordinates{
              (Archéologie, 0.173)
              (Sciences de l'Information, 0.096)
              (Linguistique, 0.104)
              (Psychologie, 0.076)
              (Chimie, 0.052)
            };
            \legend{TF-IDF, TopicRank}
          \end{axis}
        \end{tikzpicture}
      }
%      \subfigure[Candidats termes et entités nommées (NEMESIS)]{
%        \begin{tikzpicture}[scale=.75]
%          \pgfkeys{/pgf/number format/.cd, use comma, fixed, fixed zerofill, precision=3}
%          \begin{axis}[axis lines=left,
%                       symbolic x coords={Archéologie, Sciences de l'Information, Linguistique, Psychologie, Chimie},
%                       xtick=data,
%                       enlarge x limits=0.125,
%                       x=.1\linewidth,
%                       xticklabel style={anchor=east, xshift=.5em, yshift=-.25em, rotate=22.5},
%                       nodes near coords,
%                       nodes near coords align={vertical},
%                       every node near coord/.append style={font=\scriptsize},
%                       ytick={0, 0.100, 0.200, 0.300, 0.400, 0.500},
%                       y=\linewidth,
%                       ymin=0,
%                       ymax=0.22,
%                       ybar=10pt,
%                       ylabel=MAP,
%                       ylabel style={at={(ticklabel* cs:1)},
%                                     anchor=south,
%                                     rotate=270}]
%            \addplot[black!66,
%                     pattern=north east lines,
%                     pattern color=black!40] coordinates{
%              (Archéologie, 0.171)
%              (Sciences de l'Information, 0.111)
%              (Linguistique, 0.104)
%              (Psychologie, 0.071)
%              (Chimie, 0.060)
%            };
%            \addplot[black!66,
%                     pattern=north west lines,
%                     pattern color=black!66] coordinates{
%              (Archéologie, 0.152)
%              (Sciences de l'Information, 0.092)
%              (Linguistique, 0.097)
%              (Psychologie, 0.072)
%              (Chimie, 0.050)
%            };
%            \legend{TF-IDF, TopicRank}
%          \end{axis}
%        \end{tikzpicture}
%      }
%      \subfigure[Candidats termes et Np+]{
%        \begin{tikzpicture}[scale=.75]
%          \pgfkeys{/pgf/number format/.cd, use comma, fixed, fixed zerofill, precision=3}
%          \begin{axis}[axis lines=left,
%                       symbolic x coords={Archéologie, Sciences de l'Information, Linguistique, Psychologie, Chimie},
%                       xtick=data,
%                       enlarge x limits=0.125,
%                       x=.1\linewidth,
%                       xticklabel style={anchor=east, xshift=.5em, yshift=-.25em, rotate=22.5},
%                       nodes near coords,
%                       nodes near coords align={vertical},
%                       every node near coord/.append style={font=\scriptsize},
%                       ytick={0, 0.100, 0.200, 0.300, 0.400, 0.500},
%                       y=\linewidth,
%                       ymin=0,
%                       ymax=0.22,
%                       ybar=10pt,
%                       ylabel=MAP,
%                       ylabel style={at={(ticklabel* cs:1)},
%                                     anchor=south,
%                                     rotate=270}]
%            \addplot[black!66,
%                     pattern=north east lines,
%                     pattern color=black!40] coordinates{
%              (Archéologie, 0.170)
%              (Sciences de l'Information, 0.110)
%              (Linguistique, 0.103)
%              (Psychologie, 0.071)
%              (Chimie, 0.058)
%            };
%            \addplot[black!66,
%                     pattern=north west lines,
%                     pattern color=black!66] coordinates{
%              (Archéologie, 0.175)
%              (Sciences de l'Information, 0.097)
%              (Linguistique, 0.104)
%              (Psychologie, 0.076)
%              (Chimie, 0.051)
%            };
%            \legend{TF-IDF, TopicRank}
%          \end{axis}
%        \end{tikzpicture}
%      }
      \caption{Performance des méthodes d'extraction de termes-clés en domaines
               de spécialité à partir de différents type de candidats.
               \label{fig:resultats}}
    \end{figure}


  \section{Discussion}
\label{sec:discussion}
  Dans cette section, nous revenons sur les résultats présentés dans la
  section~\ref{sec:experiences} et pointons, pour les différentes disciplines,
  les variations qui, selon nous, influent sur la difficulté de la tâche
  d'extraction de termes-clés en domaines de spécialité. À partir des résultats
  obtenus, nous déduisons l'échelle de difficulté suivante (de la discipline la
  plus difficile à la plus facile)~:
  \begin{enumerate}
    \item{Chimie}
    \item{Psychologie}
    \item{Linguistique}
    \item{Sciences de l'Information}
    \item{Archéologie}
  \end{enumerate}
  Selon cette échelle de difficulté, ainsi que selon nos observations du contenu
  des notices, nous définissons trois catégories pour lesquelles la difficulté
  n'est pas la même~:
  \begin{enumerate}
    \item{Travaux expérimentaux (Chimie)}
    \item{Travaux analytiques (Psychologie, Linguistique et Sciences de
          l'Information)}
    \item{Travaux pratiques, i.e.~fondés sur des faits non sujets à subjectivité
          (Archéologie)}
  \end{enumerate}

  Dans le cas général, nous observons que la qualité de l'ensemble de candidats
  utilisé influe sur la performance des méthodes d'extraction automatique de
  termes-clés. Cependant, l'influence est différente selon la méthode
  d'extraction de termes-clés. Nous observons que les $\{1..3\}$-grammes
  dégradent fortement la performance de TopicRank en comparaison avec la méthode
  TF-IDF. TopicRank ne tirant pas profit d'une quelconque mesure de spécificité,
  nous en déduisons que la nature spécifique des termes-clés est un facteur
  important ayant des conséquences sur la difficulté d'extraction des
  termes-clés. Selon ce facteur, les disciplines pour lesquelles les termes-clés
  sont majoritairement des uni-grammes (e.g.~Archéologie) sont moins difficiles
  à traiter que des disciplines pour lesquelles les termes-clés ne sont pas
  majoritairement des uni-grammes. Par exemple en Chimie, le mot
  \og{}réaction\fg{} n'est pas spécifique dans le terme-clé \og{}réaction
  topotactique\fg{}.

  Après observation du contenu des notices, nous remarquons un second facteur~:
  l'organisation du discours dans les différentes disciplines. Pour chaque
  discipline, le lecteur visé n'est pas le même et le discours est donc organisé
  différemment. Dans le cas de documents se basant sur des faits concrets, tels
  que les documents d'Archéologie, le lecteur (archéologue ou non) a besoin
  d'une définition du contexte et des relations entre les faits donnés. Un
  document insistant sur déférents éléments importants et créant des liens entre
  ces éléments est plus aisé à traiter qu'un document se reposant sur un acquis
  supposé du lecteur (et donc non explicité). Une observation similaire
  peut-être faite pour les travaux analytiques, où les hypothèses sont
  clairement explicitées. En contradiction, les documents au sujet de travaux
  expérimentaux sont très techniques et se reposent sur un acquis supposé du
  lecteur. En Chimie, les notices sont très souvent énumératives et dépourvues
  de détails explicatifs, superflus pour un initié. Dans ce cas, moins de liens
  sont établis entre les termes-clés candidats, et la tâche d'extraction
  automatique de termes-clés est plus difficile.


  \section{Conclusion et perspectives}
\label{sec:conclusion_et_perspectives}
  Dans cet article, nous nous intéressons à la tâche d'extraction automatique de
  termes-clés dans les documents scientifiques et émettons l'hypothèse que sa
  difficulté est variable selon la discipline des documents traités. Pour
  vérifier cette hypothèse, nous disposons de notices bibliographiques réparties
  dans cinq disciplines (archéologie, linguistique, sciences de l'information,
  psychologie et chimie) auxquelles nous appliquons six systèmes d'extractions
  automatique de termes-clés différents. En comparant les termes-clés extraits
  par chaque système avec les termes-clés de référence assignés aux notices dans
  des conditions réels d'indexation, notre hypothèse se vérifie et nous
  observons l'échelle suivante (de la discipline la plus facile à la plus
  difficile)~:
  \begin{enumerate*}
    \item{Archéologie~;}
    \item{Linguistique~;}
    \item{Sciences de l'information~;}
    \item{Psychologie~;}
    \item{Chimie.}
  \end{enumerate*}

  À l'issue de nos expériences et de nos observations du contenu des notices,
  nous constatons deux facteurs ayant un impact sur la difficulté de la tâche
  d'extraction automatique de termes-clés. Tout d'abord, nous observons que
  l'organisation du résumé peut aider l'extraction de termes-clés. Un résumé
  riche en explications et en mises en relations des différents concepts est
  moins difficile à traiter qu'un résumé énumératif pauvre en explications.
  Ensuite, le vocabulaire utilisé dans une discipline peut influer sur la
  difficulté à extraire les termes-clés des documents de cette discipline. Si le
  vocabulaire spécifique contient des composés syntagmatiques dont certains
  éléments sont courants dans la discipline, alors il peut être plus difficile
  d'extraire les termes-clés des documents de cette discipline.

  Des deux facteurs identifiés émergent plusieurs perspectives de travaux
  futurs. Il peut être intéressant d'analyser le discours des documents afin de
  mesurer, en amont, le degré de difficulté de l'extraction de termes-clés. Avec
  une telle connaissance, nous pourrions proposer une méthode capable de
  s'adapter au degré de difficulté en ajustant automatiquement son paramètrage.
  Cependant, l'analyse que nous proposons dans cet article se fonde uniquement
  sur le contenu de notices appartenant à cinq disciplines. Il serait pertinent
  d'étendre cette analyse au contenu intégral des documents scientifiques, ainsi
  que d'élargir le panel de disciplines utilisées dans ce travail, afin
  d'établir des catégories de discplines plus ou moins difficiles à traiter
  (p.~ex. la chimie fait partie des disciplines expérimentales, qui sont
  difficiles à traiter). Nous oberservons aussi que le vocabulaire utilisé dans
  une discipline, en particulier celui utilisé pour les termes-clés, peut rendre
  la tâche d'extraction automatique de termes-clés plus difficile. Il est donc
  important de bénéficier de resources telles que des thésaurus pour permettre à
  une méthode d'extraction de termes-clés de s'adapter au domaine. Pour
  TopicRank, par exemple, avoir connaissance de la terminologie utilisée dans
  une discipline peut améliorer le choix du terme-clé le plus représentatif d'un
  sujet. Enfin, il serait intéressant de penser la tâche d'extraction de
  termes-clés comme une tâche d'extraction d'information pour le remplissage
  d'un formulaire. En archéologie, par exemple, il pourrait s'agir d'extraire
  les informations géographiques (pays, régions, etc.), chronologiques (période,
  culture, etc.), ou encore environnementales (animaux, végétaux, etc.).



  \section*{Remerciements}
    Ce travail a bénéficié d'une aide de l'Agence Nationale de la Recherche
    portant la référence (ANR-12-CORD-0029).

  \bibliographystyle{taln2002}
  \bibliography{../../biblio}
\end{document}

