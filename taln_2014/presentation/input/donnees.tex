\section{Données}
  \begin{frame}{Données}
    \framesubtitle{}

    \begin{itemize}
      \item{Cinq disciplines~:}
      \begin{itemize}
        \item{Archéologie}
        \item{Linguistique}
        \item{Sciences de l'Information}
        \item{Psychologie}
        \item{Chimie}
      \end{itemize}
      \item<2->{700 à 800 notices~:}
      \begin{itemize}
        \item{Titre}
        \item{Résumé}
        \item{Descripteurs (termes-clés de références)}
        \begin{itemize}
          \item{Contrôlés : appartenant au vocabulaire de la discipline}
          \item{Non-contrôlés : définis librements}
        \end{itemize}
      \end{itemize}
    \end{itemize}
  \end{frame}

  \begin{frame}{Données}
    \framesubtitle{}
    \setbeamercovered{transparent=0}

    \begin{table}
      \resizebox{\linewidth}{!}{
        \begin{tabular}{@{~}r|ccccc@{~}}
          \toprule
            & & & \textbf{Sciences} & &\\
            & \textbf{Archéologie} & \textbf{Linguistique} & \textbf{de} & \textbf{Psychologie} & \textbf{Chimie}\\
            & & & \textbf{l'information} & &\\
          \hline
            Documents & 718 & 715 & 706 & 720 & 782\\
            \onslide<2->{
              Mots/doc. & 219,1 & 156,7 & 119,7 & 185,7 & 105,2
            }\\
            \onslide<5->{
              Termes-clés/doc. & 16,6 & 8,0 & 8,5 & 11,6 & 12,8
            }\\
            \onslide<6->{
              Mots/terme-clé & 1,3 & 1,8 & 1,7 & 1,6 & 2,2
            }\\
            \onslide<9->{
              Diversité des termes-clés & 25,5~\% & 23,0~\% & 25,0~\% & 17,4~\% & 40,6~\%
            }\\
            \onslide<10->{
              Termes-clés contrôlés & 79,8~\% & 86,9\% & 85,8~\% & 90,9\% & 83,0~\%\\ 
              Termes-clés non contrôlés & 20,2~\% & 13,1\% & 14,2~\% & ~~9,1\% & 17,0~\%
            }\\
            \onslide<11->{
              Termes-clés extractibles (Rappel max.) & 62,9~\% & 38,8~\% & 32,4~\% & 27,1~\% & 23,7~\%
            }
  %          $\hookrightarrow$\hspace{3em}\small Termes-clés contrôlés extractibles & \small 48,8~\% & \small 34,9~\% & \small 27,9~\% & \small 24,9~\% & \small 21,7~\% \\
  %         $\hookrightarrow$\hspace{1.39em}\small Termes-clés non contrôlés extractibles & \small 14,1~\% & \small ~~3,9~\% & \small ~~4,5~\% & \small ~~2,2~\% & \small ~~2,0~\% \\
          \alt<11->{\\\bottomrule}{\\}
        \end{tabular}
      }
    \end{table}

    \begin{multicols}{2}
      \begin{itemize}
        \item<2->{Peu de contenu}
        \item<4->{Différence d'organisation du discours}
        %\item<5->{}
        \item<6->{Différence de complexité des termes-clés}
        \item<9->{Diversité plus importante en chimie}
        %\item<10->{}
        \item<11->{Difficulté a priori très importante}
      \end{itemize}
    \end{multicols}

    \visible<3,7>{
      \begin{textblock*}{\textwidth}(0\textwidth, -0.85\textheight)
        \setbeamertemplate{blocks}[rounded][shadow=true]

        \begin{exampleblock}{\footnotesize Variabilité du Gravettien de Kostienki
                             (bassin moyen du Don) et des territoires
                             associés}\scriptsize
          \justifying{
            Dans la région de Kostienki-Borschevo, on observe l'expression, à ce
            jour, la plus orientale du modèle européen de l'évolution du
            Paléolithique supérieur. Elle est différente à la fois du modèle
            Sibérien et du modèle de l'Asie centrale. Comme ailleurs en Europe, le
            Gravettien apparaît à Kostienki vers 28 ka (Kostienki 8 /II/). Par la
            suite, entre 24-20 ka, les techno-complexes gravettiens sont
            représentés au moins par quatre faciès dont deux, ceux de Kostienki
            21/III/ et Kostienki 4 /II/, ressemblent au Gravettien occidental et
            deux autres, Kostienki-Avdeevo et Kostienki 11/II/, sont des faciès
            propres à l'Europe de l'Est, sans analogie à
            l'Ouest.\\\vspace{.5em}

            \underline{Descripteurs (termes-clés)~:} Europe, Kostienko,
            Borschevo, variation, typologie, industrie osseuse, industrie
            lithique, Europe centrale, Avdeevo, Paléolithique
            supérieur,
            Gravettien.\\\hfill\underline{\tiny\textit{Archéologie}}
          }
        \end{exampleblock}

        \begin{exampleblock}{\footnotesize Etude d'un condensat acide
                             isocyanurique-urée-formaldéhyde}\scriptsize
          \justifying{
            La synthèse d'un condensat acide isocyanurique-urée-formaldéhyde
            utilisant la pyridine en tant que solvant a été effectuée par réaction
            sonochimique.\\\vspace{.5em}

            \underline{Descripteurs (termes-clés)~:} Réaction sonochimique,
            hétérocycle azote, cycle 6 chaînons,
            ether.\\\hfill\underline{\tiny\textit{Chimie}}
          }
        \end{exampleblock}
      \end{textblock*}
    }

    \setbeamercovered{transparent}
  \end{frame}

