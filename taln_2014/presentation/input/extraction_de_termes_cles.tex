\section{Extraction de termes-clés}
  \begin{frame}{Extraction de termes-clés}
    \framesubtitle{Chaîne de traitements}

    \tikzstyle{io}=[
      ellipse,
      minimum width=5cm,
      minimum height=2cm,
      fill=green!20,
      draw=green!33,
      transform shape,
      font={\Large\bfseries}
    ]
    \tikzstyle{component}=[
      text centered,
      thick,
      rectangle,
      minimum width=12.5cm,
      minimum height=2.5cm,
      fill=cyan!20,
      draw=cyan!33,
      transform shape,
      font={\Large\bfseries}
    ]

    \begin{center}
      \begin{tikzpicture}[thin,
                          align=center,
                          scale=.4,
                          node distance=2cm,
                          every node/.style={text centered, transform shape}]
        \node[io] (document) {Notice};
        \node[component] (preprocessing) [right=of document] {Prétraitement};
        \node[component] (candidate_extraction) [below=of preprocessing] {Sélection des candidats};
        \node[component] (candidate_classification_and_ranking) [below=of candidate_extraction] {
          \begin{tabular}{r|l}
            Classification & des candidats\\
            Ordonnancement & \\
          \end{tabular}
        };
        \node[component] (keyphrase_selection) [below=of candidate_classification_and_ranking] {Sélection de termes-clés};
        \node[io] (keyphrases) [right=of keyphrase_selection] {termes-clés};

        \path[->, thick] (document) edge (preprocessing);
        \path[->, thick] (preprocessing) edge (candidate_extraction);
        \path[->, thick] (candidate_extraction) edge (candidate_classification_and_ranking);
        \path[->, thick] (candidate_classification_and_ranking) edge (keyphrase_selection);
        \path[->, thick] (keyphrase_selection) edge (keyphrases);

        \visible<2->{
            \node[draw=red, dashed, yshift=-1cm, minimum width=13cm, minimum
          height=7.5cm, label={[xshift=-5.75cm]\Large\textbf{\textcolor{red}{c\oe{}ur}}}] (core) at (candidate_extraction.south) {};
          \visible<3->{
            \coordinate[xshift=3.5em, yshift=1em] (classification) at (candidate_classification_and_ranking.west);
            \coordinate[xshift=-6em, yshift=5em] (supervised_coordinates) at (candidate_classification_and_ranking.west);
            \coordinate[xshift=.5em, yshift=-1em] (ranking) at (candidate_classification_and_ranking.west);
            \coordinate[xshift=-6em, yshift=-5em] (unsupervised_coordinates) at (candidate_classification_and_ranking.west);

            \node (supervised) at (supervised_coordinates) {\Large\textbf{supervisé}};
            \node (unsupervised) at (unsupervised_coordinates) {\Large\textbf{non-supervisé}};

            \path[->] (supervised) edge (classification);
            \path[->] (unsupervised) edge (ranking);
          }
          \visible<4->{
            \node[draw, ellipse] (unsupervised) at (unsupervised_coordinates) {\Large\textbf{non-supervisé}};
          }
        }
      \end{tikzpicture}
    \end{center}
  \end{frame}

%  \begin{frame}{Extraction de termes-clés}
%    \framesubtitle{}
%
%    \tikzstyle{io}=[
%      ellipse,
%      minimum width=5cm,
%      minimum height=2cm,
%      fill=green!20,
%      draw=green!33,
%      transform shape,
%      font={\Large\bfseries}
%    ]
%    \tikzstyle{component}=[
%      text centered,
%      thick,
%      rectangle,
%      minimum width=12.5cm,
%      minimum height=2.5cm,
%      fill=cyan!20,
%      draw=cyan!33,
%      transform shape,
%      font={\Large\bfseries}
%    ]
%
%    \begin{center}
%      \begin{tikzpicture}[thin,
%                          align=center,
%                          scale=.4,
%                          node distance=2cm,
%                          every node/.style={text centered, transform shape}]
%        \node[io] (document) {notice};
%        \node[component] (preprocessing) [right=of document] {Prétraitement};
%        \node[component] (candidate_extraction) [below=of preprocessing] {Sélection des candidats};
%        \node[component] (candidate_classification_and_ranking) [below=of candidate_extraction] {
%          \begin{tabular}{r|l}
%            Classification & des candidats\\
%            Ordonnancement & \\
%          \end{tabular}
%        };
%        \node[component] (keyphrase_selection) [below=of candidate_classification_and_ranking] {Sélection de termes-clés};
%        \node[io] (keyphrases) [right=of keyphrase_selection] {termes-clés};
%
%        \path[->, thick] (document) edge (preprocessing);
%        \path[->, thick] (preprocessing) edge (candidate_extraction);
%        \path[->, thick] (candidate_extraction) edge (candidate_classification_and_ranking);
%        \path[->, thick] (candidate_classification_and_ranking) edge (keyphrase_selection);
%        \path[->, thick] (keyphrase_selection) edge (keyphrases);
%
%        \node[draw, thick, minimum width=12.5cm, minimum height=2.5cm] (current) at (preprocessing) {\Large\textbf{Prétraitement}};
%      \end{tikzpicture}
%    \end{center}
%  \end{frame}
%
%  \begin{frame}{Extraction de termes-clés}
%    \framesubtitle{Prétraitement}
%
%    \begin{enumerate}
%      \item{Segmentation en phrases}
%      \item{Segmentation des phrases en mots}
%      \item{Étiquetage grammatical des mots}
%    \end{enumerate}
%  \end{frame}

  \begin{frame}{Extraction de termes-clés}
    \framesubtitle{}

    \tikzstyle{io}=[
      ellipse,
      minimum width=5cm,
      minimum height=2cm,
      fill=green!20,
      draw=green!33,
      transform shape,
      font={\Large\bfseries}
    ]
    \tikzstyle{component}=[
      text centered,
      thick,
      rectangle,
      minimum width=12.5cm,
      minimum height=2.5cm,
      fill=cyan!20,
      draw=cyan!33,
      transform shape,
      font={\Large\bfseries}
    ]

    \begin{center}
      \begin{tikzpicture}[thin,
                          align=center,
                          scale=.4,
                          node distance=2cm,
                          every node/.style={text centered, transform shape}]
        \node[io] (document) {notice};
        \node[component] (preprocessing) [right=of document] {Prétraitement};
        \node[component] (candidate_extraction) [below=of preprocessing] {Sélection des candidats};
        \node[component] (candidate_classification_and_ranking) [below=of candidate_extraction] {
          \begin{tabular}{r|l}
            Classification & des candidats\\
            Ordonnancement & \\
          \end{tabular}
        };
        \node[component] (keyphrase_selection) [below=of candidate_classification_and_ranking] {Sélection de termes-clés};
        \node[io] (keyphrases) [right=of keyphrase_selection] {termes-clés};

        \path[->, thick] (document) edge (preprocessing);
        \path[->, thick] (preprocessing) edge (candidate_extraction);
        \path[->, thick] (candidate_extraction) edge (candidate_classification_and_ranking);
        \path[->, thick] (candidate_classification_and_ranking) edge (keyphrase_selection);
        \path[->, thick] (keyphrase_selection) edge (keyphrases);

        \node[draw, thick, minimum width=12.5cm, minimum height=2.5cm] (current) at (candidate_extraction) {\Large\textbf{Sélection des candidats}};
      \end{tikzpicture}
    \end{center}
  \end{frame}

  \begin{frame}{Extraction de termes-clés}
    \framesubtitle{Sélection des candidats}

    Deux approches classiques~:
    \begin{itemize}
      \item{Extraction des n-grammes}
        \begin{itemize}
          \item{$n \subseteq \{1..3\}$}
          \item{Filtrage avec un anti-dictionnaire}
          \item[\textcolor{red}{\scriptsize$\blacktriangleright$}]{Sursélection des candidats $\Rightarrow$ faible qualité}
        \end{itemize}
      \item{Reconnaissance de formes}
        \begin{itemize}
          \item{\texttt{(NOM | ADJ)+}}
        \end{itemize}
    \end{itemize}

    \uncover<2->{
      Une approche non explorée jusqu'alors~:
      \begin{itemize}
        \item{Extraction des candidats termes}
        \begin{itemize}
          \item{Utilisation de TermSuite}
          \item{Formes très précises~:}
          \begin{itemize}
            \item{\texttt{NOM à NOM}}
            \item{\texttt{NOM en NOM}}
            \item{\texttt{NOM à NOM ADJ}}
            \item{etc.}
          \end{itemize}
        \end{itemize}
      \end{itemize}
    }
  \end{frame}

  \begin{frame}{Extraction de termes-clés}
    \framesubtitle{Sélection des candidats --- Exemples}

    \resizebox{\linewidth}{!}{
      \begin{tabular}{l|l|l}
        \toprule
        \multicolumn{3}{c}{\textit{\og{}bassin moyen du Don\fg{}}}\\
        \midrule
        \multicolumn{1}{c|}{$\{1..3\}$-grammes} & \multicolumn{1}{c|}{\texttt{(NOM | ADJ)+}} & \multicolumn{1}{c}{Candidats termes} \\
        \hline
        \textit{\og{}bassin\fg{}} & \textit{\og{}bassin moyen\fg{}} & \textit{\og{}bassin moyen du Don\fg{}} \\
         \textit{\og{}moyen\fg{}} & \textit{\og{}Don\fg{}} & $\hookrightarrow$ \textit{\og{}bassin\fg{}} \\
          \textit{\og{}Don\fg{}} & & $\hookrightarrow$ \textit{\og{}moyen\fg{}} \\
        \textit{\og{}bassin moyen\fg{}} & & $\hookrightarrow$ \textit{\og{}Don\fg{}} \\
        \textit{\og{}moyen du Don\fg{}} & & $\hookrightarrow$ \textit{\og{}bassin moyen\fg{}} \\
        \bottomrule
      \end{tabular}
    }
  \end{frame}

  \begin{frame}{Extraction de termes-clés}
    \framesubtitle{}

    \tikzstyle{io}=[
      ellipse,
      minimum width=5cm,
      minimum height=2cm,
      fill=green!20,
      draw=green!33,
      transform shape,
      font={\Large\bfseries}
    ]
    \tikzstyle{component}=[
      text centered,
      thick,
      rectangle,
      minimum width=12.5cm,
      minimum height=2.5cm,
      fill=cyan!20,
      draw=cyan!33,
      transform shape,
      font={\Large\bfseries}
    ]

    \begin{center}
      \begin{tikzpicture}[thin,
                          align=center,
                          scale=.4,
                          node distance=2cm,
                          every node/.style={text centered, transform shape}]
        \node[io] (document) {notice};
        \node[component] (preprocessing) [right=of document] {Prétraitement};
        \node[component] (candidate_extraction) [below=of preprocessing] {Sélection des candidats};
        \node[component] (candidate_classification_and_ranking) [below=of candidate_extraction] {
          \begin{tabular}{r|l}
            Classification & des candidats\\
            Ordonnancement & \\
          \end{tabular}
        };
        \node[component] (keyphrase_selection) [below=of candidate_classification_and_ranking] {Sélection de termes-clés};
        \node[io] (keyphrases) [right=of keyphrase_selection] {termes-clés};

        \path[->, thick] (document) edge (preprocessing);
        \path[->, thick] (preprocessing) edge (candidate_extraction);
        \path[->, thick] (candidate_extraction) edge (candidate_classification_and_ranking);
        \path[->, thick] (candidate_classification_and_ranking) edge (keyphrase_selection);
        \path[->, thick] (keyphrase_selection) edge (keyphrases);

        \node[draw, thick, minimum width=12.5cm, minimum height=2.5cm] (current)
        at (candidate_classification_and_ranking) {\Large\bfseries
          \begin{tabular}{r|l}
            \sout{Classification} & des candidats\\
            Ordonnancement & \\
          \end{tabular}
        };
      \end{tikzpicture}
    \end{center}
  \end{frame}

  \begin{frame}{Extraction de termes-clés}
    \framesubtitle{TF$\times$IDF}

    \begin{block}{Hypothèse}
      Dans une notice, un mot est d'autant plus important qu'il  y est fréquent
      (TF) et spécifique (IDF).
    \end{block}

    \begin{align*}
      \text{importance}(\text{candidat}) = \sum_{\text{mot} \in \text{candidat}} \text{TF$\times$IDF}(\text{mot})
    \end{align*}
  \end{frame}

  \begin{frame}{Extraction de termes-clés}
    \framesubtitle{TopicRank}

    \begin{block}{Hypothèses}
      \begin{enumerate}
        \item{Plusieurs candidats désignent le même sujet (concept)}
        \item{Seul le candidat le plus représentatif du sujet doit être extrait}
        \item{Les sujets qui cooccurrent se recommandent mutuellement~:}
        \begin{itemize}
          \item{Plus un sujet cooccurre avec d'autres sujets, plus il est
                important}
          \item{Plus un sujet est important, plus les sujets avec lesquels il
                cooccurre sont important}
        \end{itemize}
      \end{enumerate}
    \end{block}
  \end{frame}

