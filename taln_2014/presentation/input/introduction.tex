\section*{Introduction}
  \begin{frame}{Introduction}
    \framesubtitle{Contexte}

    \begin{block}{Termes-clés (mots-clés)}
      \begin{itemize}
        \item{Mots ou expressions polylexicales}
        \item{Aperçu d'un document}
        \item{Donnés par les auteurs, des lecteurs ou des \alt<3->{\textbf{documentalistes}}{documentalistes}}
      \end{itemize}
    \end{block}

    \begin{block}<2->{Applications}
      \begin{minipage}{.5\linewidth}
        \begin{itemize}
          \item{\alt<3->{\textbf{Indexation de documents}}{Indexation de documents}}
          \item{Résumé automatique}
          \item{Classification de documents}
        \end{itemize}
      \end{minipage}\hfill
      \begin{minipage}{.48\linewidth}
        \begin{itemize}
          \item{Expansion de requêtes}
          \item{Ciblage (\textit{marketing})}
          \item{etc.}
        \end{itemize}
      \end{minipage}
    \end{block}
  \end{frame}

  \begin{frame}{Introduction}
    \framesubtitle{Contexte (suite)}

    \begin{block}{Indexation de documents scientifiques}
      Création de notices bibliographiques~:
      \begin{itemize}
        \item{Titre}
        \item{Auteurs}
        \item{Résumé}
        \item{Codes de classement}
        \item{\textbf{Descripteurs conceptuels $\Leftrightarrow$ termes-clés}}
      \end{itemize}
    \end{block}
  \end{frame}

  \begin{frame}{Introduction}
    \framesubtitle{Exemple}

    \begin{exampleblock}{\footnotesize Variabilité du \textbf{\normalsize
                         Gravettien} de Kostienki (bassin moyen du Don) et des
                         territoires associés}\footnotesize
      \justifying{
        Dans la région de Kostienki-\textbf{Borschevo}, on observe l'expression,
        à ce jour, la plus orientale du modèle européen de l'évolution du
        \textbf{Paléolithique supérieur}. Elle est différente à la fois du
        modèle Sibérien et du modèle de l'Asie centrale. Comme ailleurs en
        \textbf{\small Europe}, le \textbf{\normalsize Gravettien} apparaît à
        Kostienki vers 28 ka (Kostienki 8 /II/). Par la suite, entre 24-20 ka,
        les techno-complexes \textbf{\normalsize gravettiens} sont représentés
        au moins par quatre faciès dont deux, ceux de Kostienki 21/III/ et
        Kostienki 4 /II/, ressemblent au \textbf{\normalsize Gravettien}
        occidental et deux autres, Kostienki-\textbf{Avdeevo} et Kostienki
        11/II/, sont des faciès propres à l'\textbf{\small Europe} de l'Est,
        sans analogie à l'Ouest.\\\vspace{.5em}

        \underline{Descripteurs (termes-clés)~:} \textbf{Europe}, Kostienko,
        \textbf{Borschevo}, variation, typologie, industrie osseuse, industrie
        lithique, Europe centrale, \textbf{Avdeevo}, \textbf{Paléolithique
        supérieur},
        \textbf{Gravettien}.\hfill\underline{\scriptsize\textit{Archéologie}}
      }
    \end{exampleblock}
  \end{frame}

  \begin{frame}{Introduction}
    \framesubtitle{Problèmatique}

    \begin{alertblock}{Mais\dots}
        L'assignation de termes-clés est une tâche coûteuse.\\
        $\Rightarrow$ Il faut extraire les termes-clés automatiquement.
    \end{alertblock}

    \begin{block}<2->{Extraction automatique de termes-clés}
      \begin{itemize}
        \item{Supervisée/\textbf{non-supervisée}}
        \item{Extraction des termes-clés \textbf{contenus dans} le titre/résumé}
        \item{Terme-clé $=$ unité textuelle \textbf{importante} dans le
              titre/résumé}
          \begin{itemize}
            \item{Fréquente et spécifique (TF-IDF)}
            \item{Centrale~\cite[TextRank]{mihalcea2004textrank}}
          \end{itemize}
      \end{itemize}
    \end{block}
  \end{frame}

  \begin{frame}{Introduction}
    \framesubtitle{}

    \begin{block}{Hypothèse}
      Il est plus difficile d'extraire les termes-clés pour certaines disciplines
      que pour d'autres.\\~\\
      $\Rightarrow$ Quels sont les facteurs qui influent sur cette difficulté~?
    \end{block}
  \end{frame}

