\section{Introduction}
\label{sec:main:domain_specific_keyphrase_annotation-introduction}
  Dans la littérature, l'indexation automatique par termes-clés se divise en
  deux catégories~: l'indexation libre, qui fournit des termes-clés apparaissant
  dans le contenu du document, et l'indexation contrôlée, qui fournit des
  termes-clés appartenant à un vocabulaire contrôlé et n'apparaissant pas
  nécessairement dans le document. Utile pour de nombreuses tâches, telles que
  la recherche d'information~\cite{jones1999phrasier}, le résumé
  automatique~\cite{avanzo2005keyphrase} et la classification de
  document~\cite{han2007webdocumentclustering}, l'indexation par termes-clés
  fait l'objet de nombreux travaux~\cite{hasan2014state_of_the_art}. Toutefois,
  la majorité des travaux existant s'intéresse uniquement à l'indexation libre.
  
  Contrairement aux travaux de la littérature, l'indexation par termes-clés
  réalisée par des indexeurs professionnels inclut aussi bien des termes-clés
  libres que contrôlés. Si nous prenons l'exemple des ingénieurs documentaliste
  de l'Inist (Institut de l'information scientifique et technique), les
  pratiques d'indexation manuelle respectent cinq règles qui impliquent une
  indexation (hybride) libre et contrôlée~:
  \begin{enumerate}
    \item{Conformité~: les termes-clés doivent être conformes au contenu du
          document et au langage documentaire utilisé dans son domaine~;}
    \item{Exhaustivité~: les termes-clés doivent représenter tous les
          aspects importants du document, même lorsque ceux-ci sont
          implicites~;}
    \item{Homogénéité~: les termes-clés des documents d'un même domaine
          doivent être cohérents et identiques lorsqu'ils représentent le
          même concept~;}
    \item{Spécificité~: les termes-clés doivent décrire le contenu d'un
          document au niveau le plus spécifique et peuvent parfois être
          accompagnés de termes-clés plus génériques afin de restituer le
          contenu du document dans son domaine~;}
    \item{Impartialité~: les termes-clés ne doivent pas être représentatifs
          d'un jugement apporté par l'indexeur.}
  \end{enumerate}
  Si les critères de conformité et d'homogénéité sont en faveur d'une indexation
  contrôlée, les principes d'exhaustivité et de spécificité impliquent l'usage
  simultané d'une indexation libre et de l'indexation contrôlée.

  Dans cet article, nous présentons une méthode à base de graphe pour
  l'indexation par termes-clés (hybride) libre et contrôlée. Pour ce faire, nous
  unifions un graphe représentant le document à un graphe représentant la
  terminologie de son domaine de spécialité. Expressions extraites du document
  et termes du domaine sont ensuite ordonnés par importance par un processus
  d'ordonnancement conjoint, puis ceux les plus importants sont fournis en temps
  que termes-clés.

  Le reste de cet article est organisé comme suit. Dans un premier temps, nous
  présentons brièvement l'état de l'art des méthodes d'indexation automatique
  par termes-clés. Dans un second temps, nous présentons notre nouvelle méthode
  à base de graphe. Enfin, nous présentons les résultats de notre travail, puis
  concluons et présentons quelques perspectives.

%-----------------------------------------------------------------------------

\section{Indexation par termes-clés}
\label{sec:main-domain_specific_keyphrase_annotation-state_of_the_art}
  Dans cette section, nous présentons les méthodes de la littérature pour
  l'indexation automatique par termes-clés. Tout d'abord, nous présentons les
  méthodes d'indexation libre, puis les méthodes d'indexation contrôlée.

  \subsection{Indexation libre}
  \label{subsec:main-domain_specific_keyphrase_annotation-state_of_the_art-keyphrase_extraction}
    L'indexation libre est l'approche la plus employée pour l'indexation
    automatique par termes-clés. Les méthodes de la littérature utilisant cette
    approche appliquent diverses techniques~\cite{hasan2014state_of_the_art}, du
    simple ordonnancement statistique de termes-clés
    candidats~\cite{salton1975tfidf}, à la classification binaire de ces mêmes
    termes-clés candidats~\cite{witten1999kea}, en passant par un ordonnancement
    à base de graphe des mots du document~\cite{mihalcea2004textrank}. Comme
    notre travail repose sur la technique de l'ordonnancement à base de graphe,
    nous allons nous concentrer sur les méthodes qui l'emploient dans le reste
    de cette section.

    Depuis les travaux fondateurs de \newcite{mihalcea2004textrank} avec
    TextRank, la technique d'ordonnancement à base de graphe pour l'indexation
    automatique par termes-clés est devenue populaire. L'idée originale de cette
    technique est de représenter le document sous la forme d'un graphe de
    cooccurrences de mots, puis d'ordonner les mots par importance au sein du
    graphe. Ensuite, les $k$ mots les plus importants, appelés mots-clés, sont
    utilisés pour extraire les termes-clés. Ces derniers ne sont autres que les
    expressions du document qui contiennent uniquement des mots-clés.

    L'ordonnancement par importance au sein du graphe repose sur le principe de
    la recommandation, ou du vote. Soit le graph $G = (N, A)$, un ensemble de
    n\oe{}uds représentant chacun un mots et un ensemble d'arêtes reliant deux
    n\oe{}uds si les mots qu'ils représentent cooccurrent dans le document.
    Alors, un n\oe{}ud $n_i$ est d'autant plus important qu'il est connecté à
    beaucoup d'autres n\oe{}uds et que les n\oe{}uds avec lesquels il est
    connecté sont eux mêmes importants~:
    \begin{align}
      \text{Importance}(n_i) = (1 - \lambda) \times \lambda \sum_{n_j \in A(n_i)}\frac{\text{Importance}(n_j)}{|A(n_j)|}
    \end{align}
    où $A(n_i)$ est l'ensemble des n\oe{}uds connectés au n\oe{}ud $n_i$ et
    $\lambda$ est un facteur de lissage fixé à 0.85 par
    \newcite{brin1998pagerank}.

    Dans la continuité des travaux de \newcite{mihalcea2004textrank},
    \newcite{wan2008expandrank} ont proposés de pondérer les arêtes selon le
    nombre de cooccurrences entre les mots, de sorte que ceux qui
    cooccurrent le plus ensemble se transfèrent plus d'importance.
    \newcite{wan2008expandrank} ont aussi proposé de compléter le graphe à
    l'aide de cooccurrences capturés dans des documents similaires à celui
    analysé. \newcite{liu2010topicalpagerank}, quant à eux, ont préférés utiliser un
    modèle de sémantique latente~\cite[LDA]{blei2003lda} pour biaiser $k$
    ordonnancements, selon chacun des $k$ sujets identifiés par LDA. Les $k$
    ordonnancement par importance des mots sont ensuite fusionnés pour obtenir
    un unique ordonnancement.

    Plus récemment, \newcite{zhang2013wordtopicmultirank}, ainsi que
    \newcite{bougouin2013topicrank}, ont suivi la démarche de
    \newcite{liu2010topicalpagerank} et ont cherché à tirer profit des sujets
    abordés dans le document. \newcite{zhang2013wordtopicmultirank} ont décidé
    de réutiliser un modèle LDA pour détecter les sujets du document, mais cette
    fois-ci en les insérant dans le graphe en les connectant aux mots qui
    vraissemblablement les représentent et de les ordonner conjointement. Quant
    à \newcite{bougouin2013topicrank}, ils ont décidé de représenter plus
    simplement les siujets par des groupes de termes-clés cantidats partageant
    un nombre suffisant de mots, d'ordonner uniquement les sujets, puis
    d'extraire un termes-clés candidat par sujet en temps que terme-clé. Notre
    travail étend celui de \newcite{bougouin2013topicrank}, nous reviendront
    donc plus en détails sur cette méthode dans la
    section~\ref{sec:main-domain_specific_keyphrase_annotation-topiccorank}.

  \subsection{Indexation contrôlée}
  \label{subsec:main-domain_specific_keyphrase_annotation-state_of_the_art-keyphrase_extraction}
    Contrairement à l'indexation libre, l'indexation contrôlé nécessite un
    vocabulaire spécifique au domaine du document analysé. Elle a pour objectif
    d'indexer les documents de manière homogène (un seul terme-clé par concept,
    quelques soient les documents) et spécifique au domaine. Dans ce but,
    l'indexation contrôlée est plus difficile, car elle doit être capable de
    trouver des termes-clés qui ne sont pas obligatoirement présents dans le
    contenu du document.

    \newcite{medelyan2006kea++} ont proposé KEA++, une méthode d'indexation par
    termes-clés contrôlée qui assigne les termes-clés à partir d'un thésaurus.
    Dans un premier temps, les termes du thésaurus sont projetés dans le
    document. Ceux présents dans le document sont retenus comme termes-clés
    candidats. Dans un second temps, les termes-clés candidats sont classés en
    tant que \og{}terme-clé\fg{} ou \og{}non terme-clé\fg{} avec un classifieur
    naïf bayésien et trois traits~: le TF-IDF du terme-clé
    candidat~\cite{witten1999kea}, sa première position~\cite{witten1999kea} et
    le nombre de relations d'association qu'entretien le candidat avec les
    autres dans le thésaurus.

    Au cours de la campagne d'évaluation
    Bio\textsc{Asq}~\cite{partalas2013bioasq}, l'indexation contrôlée par
    termes-clés a été formulé en un problème de classification multi-étiquette
    multi-classe. Les systèmes proposés lors de cette campagne attribuent des
    étiquettes au document en choisissant leur valeur parmi les entrées du
    vocabulaire contrôlé, soit les classes que peut prendre le document. Le
    problème de classification multi-étiquette multi-classe est généralement
    considéré comme de multiples problèmes de classification binaire. Soit un
    classifieur est appris pour chaque classe~; Soit un classifieur est appris
    pour chaque paire de classes et les classes retenues sont celles proposées
    par le plus de classifieurs. Ce type de classification, comparée à celle de
    \textsc{Kea}++, est donc plus coûteux, mais il présente l'avantage de
    permettre un assignement sans se limiter aux entrées du vocabulaire contrôlé
    présentes dans le document.

%-----------------------------------------------------------------------------

\section{Notre approche}
\label{sec:main-domain_specific_keyphrase_annotation-topiccorank}
  La méthode que nous proposons étend la méthode
  TopicRank~\cite{bougouin2013topicrank}. TopicRank est une méthode à base de
  graphe fonctionnant en cinq grandes étapes~:
  \begin{enumerate}
    \item{\textbf{Sélection des termes-clés candidats.} TopicRank suit les
          travaux précédents~\cite{wan2008expandrank,hassan2010conundrums} en
          sélectionnant les plus longues séquences de noms et d'adjectifs en
          tant que termes-clés candidats.}
    \item{\textbf{Groupement en sujets.} Tous les termes-clés candidats sont
          réduits à des sacs de mots racinisés et ceux partageant
          $\unitfrac{1}{4}$ de leurs mots racinisés sont groupés et au sein du
          même sujet.}
    \item{\textbf{Construction du graphe.} Le document est représenté par un
          graphe complet où les n\oe{}uds sont les sujets. Chaque sujets est
          connecté aux autres par une arête pondéré selon la force du lien
          sémantique entre les sujets. Contrairement aux travaux précédents,
          TopicRank pondère les arête selon la distance (en nombre de mots),
          dans le document, entre les termes-clés candidats des sujets. Plus les
          termes-clés candidats de deux sujets sont proches dans le document,
          plus élévée est la pondération de l'arête entre les deux sujets.}
    \item{\textbf{Ordonnancement des sujets.} À la manière de
          TextRank~\cite{mihalcea2004textrank}, les sujets sont ordonnés par
          importance de sorte que plus un sujets est fortement connecté à un
          grand nombre de sujets, plus il est important. De plus, plus les
          sujets avec lesquels il est fortement connecté sont important, plus il
          gagne d'importance.}
    \item{\textbf{Extraction des termes-clés.} Un unique terme-clé est extrait
          pour chacun des $k$ plus importants sujets.
          \newcite{bougouin2013topicrank} ont choisi de sélectionner dans chaque
          sujet le terme-clé candidat qui apparaît en premier dans le document.}
  \end{enumerate}
  
  Notre méthode modifie TopicRank au niveau des étapes de construction du
  graphe, d'ordonnancement par importance et de sélection des termes-clés. La
  construction du graphe étend le graphe de sujet initial de TopicRank en
  l'unifiant à un graphe des termes-clés de référence du domaine~;
  l'ordonnancement est désormais conjoint pour les sujets du document et les
  termes-clés du domaine~; la sélection des termes-clés ajoute la possibilité de
  puiser dans le graphe du domaine afin de réaliser de l'assignement.

  \subsection{Construction du graphe}
  \label{subsec:main-domain_specific_keyphrase_annotation-supervised_automatic_keyphrase_extraction-topiccorank-graph_construction}
    Afin de réaliser simultanément indexation libre et contrôlée, nous unifions
    deux graphes représentant le document (graphe de sujets) et les termes-clés
    de référence de son domaine (graphe du domaine). Ce dernier graphe est
    construit à partir des termes-clés de référence de documents d'entraînement.
    Comme \newcite{chaimongkol2013technicaltermextraction} l'ont fait avant nous
    pour l'extraction de termes techniques, nous faisons l'hypothèse que les
    termes-clés de référence des documents d'entraînement constituent la
    terminologie du domaine et nous les utilisons comme substituts au
    vocabulaire contrôlé. Contrairement aux termes-clés candidats sélectionnés
    dans le document, les termes-clés de référence ne sont pas redondants et ne
    sont donc pas groupés en sujets.

    Soit le graphe unifié non orienté $G = (N, A =
    A_{\textnormal{\textit{interne}}} \cup
    A_{\textnormal{\textit{externe}}})$. $N$ dénote indifféremment les
    sujets et les termes-clés du domaine. $A$ regroupe les arêtes
    $A_{\textnormal{\textit{interne}}}$, qui connectent deux sujets ou deux
    termes-clés du domaine, et les arêtes
    $A_{\textnormal{\textit{externe}}}$, qui connectent un sujet à un
    terme-clé de référence (voir la figure~\ref{fig:topiccorank_graph}). Le
    graphe de sujets et le graphe du domaine sont unifiés grâce aux arêtes
    $A_{\textnormal{\textit{externe}}}$.
    %
    En considérant le domaine comme une carte conceptuelle, l'objectif des
    arêtes $A_{\textnormal{\textit{externe}}}$ est de connecter le document
    à son domaine par l'intermédiaire des concepts qu'ils partagent. Une
    arête $A_{\textnormal{\textit{externe}}}$ est donc créée entre un sujet
    et un terme-clé du domaine si ce dernier appartient au sujet,
    c'est-à-dire correspond à l'un de ses termes-clés candidats.
    %
%        Une arête
%        $A_{\textnormal{\textit{externe}}}$ est ajoutée pour connecter un sujet
%        et un terme-clé de référence si, et seulement si, le terme-clé fait
%        partie des termes-clés candidats qui composent le sujet
%        (\TODO{exemple}). En d'autres termes, TopicRankSpe considère le domaine
%        comme une carte conceptuelle et connecte le document au domaine par
%        l'intermédiaire des concepts qu'ils partagent.
    \begin{figure}
      \newcommand{\xslant}{0.25}
      \newcommand{\yslant}{0}

      \centering
      \begin{tikzpicture}[transform shape, scale=.667]
        % frame
        \node [draw,
               rectangle,
               minimum width=.7\linewidth,
               minimum height=8em,
               xslant=\xslant,
               yslant=\yslant] (domain_graph) {};
        \node [above=of domain_graph,
               xshift=.36\linewidth,
               yshift=8em,
               anchor=south east] (domain_graph_label) {termes-clés du domaine};

        \node [draw,
               circle,
               above=of domain_graph,
               xshift=.3\linewidth,
             yshift=5em] (domain_node1) {$V_1$};
        \node [draw,
               circle,
               above=of domain_graph,
               xshift=-.3\linewidth,
               yshift=5em] (domain_node2) {$V_2$};
        \node [draw,
               circle,
               above=of domain_graph,
               yshift=5em] (domain_node3) {$V_3$};
        \node [draw,
               circle,
               above=of domain_graph,
               xshift=.15\linewidth,
               yshift=.75em] (domain_node4) {$V_4$};
        \node [draw,
               circle,
               above=of domain_graph,
               xshift=-.15\linewidth,
               yshift=.75em] (domain_node5) {$V_5$};

        \draw (domain_node1) -- (domain_node3);
        \draw (domain_node2) -- (domain_node3);
        \draw (domain_node2) -- (domain_node4);
        \draw (domain_node4) -- (domain_node5);
        \draw (domain_node4) -- (domain_node3);

        % document
        \node [draw,
               rectangle,
               minimum width=.7\linewidth,
               minimum height=8em,
               xslant=\xslant,
               yslant=\yslant,
               above=of domain_graph,
               xshift=-2em] (document_graph) {};
        \node [below=of document_graph,
               xshift=-.36\linewidth,
               yshift=-8em,
               anchor=north west] (document_graph_label) {sujets du document};

        \node [draw,
               circle,
               regular polygon sides=8,
               below=of document_graph,
               xshift=.3\linewidth,
               yshift=-5em] (document_node1) {$V_6$};
        \node [draw,
               circle,
               regular polygon sides=8,
               below=of document_graph,
               xshift=-.3\linewidth,
               yshift=-5em] (document_node2) {$V_7$};
        \node [draw,
               circle,
               regular polygon sides=8,
               below=of document_graph,
             yshift=-5em] (document_node3) {$V_8$};
        \node [draw,
               circle,
               regular polygon sides=8,
               below=of document_graph,
               xshift=.15\linewidth,
               yshift=-.75em] (document_node4) {$V_9$};

        \draw (document_node2) -- (document_node3);
        \draw (document_node3) -- (document_node1);
        \draw (document_node1) -- (document_node4);
        \draw (document_node3) -- (document_node4);

        % extra link
        \draw [dashed] (document_node2) -- (domain_node2);
        \draw [dashed] (document_node3) -- (domain_node3);
        \draw [dashed] (document_node4) -- (domain_node1);
        \draw [dashed] (document_node3) -- (domain_node4);

        % legend
        \node [right=of document_graph, xshift=2em, yshift=-9.25em] (legend_title) {\underline{Légende~:}};
        \node [below=of legend_title, xshift=-1em, yshift=2em] (begin_inner) {};
        \node [right=of begin_inner] (end_inner) {: $A_\textnormal{\textit{interne}}$};
        \node [below=of begin_inner, yshift=1.5em] (begin_outer) {};
        \node [right=of begin_outer] (end_outer) {: $A_\textnormal{\textit{externe}}$};

        \draw (legend_title.north  -| end_outer.east) rectangle (end_outer.south -| legend_title.west);

        \draw (begin_inner) -- (end_inner);
        \draw [dashed] (begin_outer) -- (end_outer);
      \end{tikzpicture}
      \caption{Illustration du graphe unifié que nous proposons 
               \label{fig:topiccorank_graph}}
    \end{figure}

    Pour permettre un ordonnancement conjoint des sujets et des termes-clés du
    domaine, le schéma de connexion entre deux sujets et entre deux termes-clés
    du domaine (arêtes $A_\textnormal{\textit{interne}}$) doit être homogène. En
    effet, si les conditions de connexion et si la pondération des arêtes ne
    sont pas équivalentes et du même ordre de grandeur, alors l'impact du
    domaine sur le document, et du document sur le domaine, sera marginal. Pour
    obtenir un graphe unifié homogène, nous connectons deux sujets ou deux
    termes-clés du domaine $n_i$ et $n_j$ lorsqu'ils apparaissent dans le même
    contexte et pondère leur arête par le nombre de fois que cela se produit
    ($\textnormal{poids}(n_i, n_j)$). Lorsqu'il s'agit des sujets, le contexte
    est une phrase du document~; lorsqu'il s'agit des termes-clés du domaine, le
    contexte est l'ensemble des termes-clés de référence d'un document
    d'entraînement. Les contextes étant utilisés pour la création du graphe, le
    graphe de sujets n'est plus complet comme celui de TopicRank. Ici, nous
    utilisons la phrase comme alternative à la fenêtre de cooccurrence.

  \subsection{Ordonnancement conjoint des sujets et des termes-clés du domaine}
  \label{subsec:main-domain_specific_keyphrase_annotation-supervised_automatic_keyphrase_extraction-topiccorank-co_ranking}
    L'ordonnancement conjoint des sujets et des termes-clés du domaine
    établit leur ordre d'importance vis-à-vis du contenu du document et du
    domaine. Pour cela, un score d'importance est attribué simultanément aux
    sujets et aux termes-clés du domaine.
%
    Nous reprenons le principe de la recommandation de TopicRank et
    l'adaptons au problème d'ordonnancement conjoint. Les premières
    hypothèses de recommandation sont donc les mêmes que celle de
    TopicRank~:
    \begin{itemize}
      \item{un sujet est d'autant plus important s'il est fortement connecté
            à un grand nombre de sujets et si les sujets avec lesquels il
            est fortement connecté sont importants~;}
      \item{un terme-clé du domaine est d'autant plus important s'il est
            fortement connecté à un grand nombre de termes-clés du domaine
            et si les termes-clés du domaine avec lesquels il est connecté
            sont importants.}
    \end{itemize}
    Ces hypothèses de recommandation, que nous qualifions d'internes,
    permettent d'établir l'importance des sujets les uns par rapport aux
    autres et l'importance des termes-clés du domaine les uns par rapport
    aux autres. Cependant, elles ne permettent pas de tirer profit des
    relations entre sujets et termes-clés du domaine. Par ailleurs,
    l'importance des termes-clés du domaine ne  dépend pas du document. Nous
    ajoutons donc deux nouvelles hypothèses de recommandation, que nous
    qualifions d'externes~:
    \begin{itemize}
      \item{un sujet est d'autant plus important s'il est représenté par
            (connecté à) des termes-clés du domaine importants~;}
      \item{un terme-clé du domaine est d'autant plus important vis-à-vis
            du contenu du document s'il véhicule (est connecté à) l'un de
            ses sujets importants.}
    \end{itemize}
    Sujets et termes-clés du domaine sont ainsi évalués d'après leur usage
    dans le document et leur importance dans le domaine. L'ordonnancement
    des uns joue un rôle sur celui des autres et permet ainsi d'effectuer
    conjointement extraction et assignement.

    L'équation~\ref{math:topiccorank} montre le calcul de l'importance d'un
    sujet ou d'un terme-clé du domaine à partir de sa recommandation interne
    $R_{\textnormal{\textit{interne}}}$ et de sa recommandation externe
    $R_{\textnormal{\textit{externe}}}$~:
    \begin{align}
      S(n_i) &= (1 - \lambda)\ R_{\textnormal{\textit{externe}}}(n_i) + \lambda\ R_{\textnormal{\textit{interne}}}(n_i)\label{math:topiccorank}\\
      R_{\textnormal{\textit{interne}}}(n_i) &= \sum_{n_j \in A_{\textnormal{\textit{interne}}}(n_i)}{\frac{\textnormal{poids}(n_j, n_i) \times S(n_j)}{\mathlarger\sum_{n_k \in A_{\textnormal{\textit{interne}}}(n_j)}{{\textnormal{poids}(n_j, n_k)}}}}\label{math:rin}\\
      R_{\textnormal{\textit{externe}}}(n_i) &= \sum_{n_j \in A_{\textnormal{\textit{externe}}}(n_i)}{\frac{S(n_j)}{|A_{\textnormal{\textit{externe}}}(n_j)|}}\label{math:rout}
    \end{align}
    où $A_{\textnormal{\textit{interne}}}(n_i)$ représente l'ensemble de
    tous les n\oe{}uds connectés au n\oe{}ud $n_i$ par une arête
    $A_\textnormal{\textit{interne}}$, où
    $A_{\textnormal{\textit{externe}}}(n_i)$ représente l'ensemble de tous
    les n\oe{}uds connectés au n\oe{}ud $n_i$ par une arête
    $A_\textnormal{\textit{externe}}$ et où le facteur $\lambda$ permet
    désormais de définir la recommandation la plus influente entre
    $R_{\textnormal{\textit{interne}}}$ et
    $R_{\textnormal{\textit{externe}}}$. Par défaut, nous définissons
    $\lambda=0,5$.

  \subsection{Sélection des termes-clés}
  \label{subsec:main-domain_specific_keyphrase_annotation-supervised_automatic_keyphrase_extraction-topiccorank-keyphrase_selection}
    Nous utilisons l'ordre d'importance établit avec le score $S$ des sujets et
    termes-clés du domaine pour déterminer les termes-clés du document. Les $k$
    n\oe{}uds du graphe unifié ayant obtenu les meilleurs scores sont retenus,
    qu'ils soient des sujets ou des termes-clés du domaine.

    Lorsqu'un terme-clé du domaine doit être assigné (indexation contrôlée), une
    étape de vérification supplémentaire est effectuée pour s'assurer que son
    importance calculée relève aussi bien du domaine que du document. En effet,
    il est possible que le graphe du domaine soit constitué de composantes
    connexes, soit de sous-graphes dont les n\oe{}uds ne sont connectés qu'entre
    eux. Dans ce cas, il se peut qu'un terme-clé du domaine d'un sous-graphe ne
    soit connecté, ni directement, ni indirectement (par l'intermédiaire d'un
    autre n\oe{}ud), à un sujet du document. Son importance est donc déterminée
    uniquement à partir du domaine et il n'est donc pas pertinent de l'assigner
    au document.

    Lorsqu'un n\oe{}ud retenu représente un sujet, c'est la même stratégie
    que celle de Topic\-Rank qui est appliquée. Pour un sujet donné, le
    terme-clé extrait est son terme-clé candidat qui apparaît en premier
    dans le document.

  \subsection{Exemple}
  \label{subsec:main-domain_specific_keyphrase_annotation-supervised_automatic_keyphrase_extraction-topiccorank-exemple}
    La figure~\ref{fig:exemple_topiccorank} donne un exemple d'extraction et
    d'assignement de termes-clés avec notre méthode, à partir d'une notice
    bibliographique d'un article d'archéologie. Dans cet exemple, nous observons
    une meilleure indexation par termes-clés qu'avec TopicRank. Tout d'abord,
    nous voyons que le graphe du domaine permet l'assignement du termes-clés
    générique \og{}France\fg{} qui n'est pas présent dans le document. Ensuite,
    nous voyons que les relations de \og{}diffusion\fg{}, \og{}analyse\fg{} et
    \og{}répartition\fg{} dans le graphe du domaine permettent de mieux les
    ordonner.
    \begin{figure} % archeologie_525-02-11060
  \centering
  \framebox[\linewidth]{
    \parbox{.99\linewidth}{\footnotesize
      \textbf{Étude préliminaire de la céramique non tournée micacée du bas
      Languedoc occidental : typologie, chronologie et aire de diffusion}\\

      L'étude présente une variété de céramique non tournée dont la
      typologie et l'analyse des décors permettent de l'identifier
      facilement. La nature de l'argile enrichie de mica donne un aspect
      pailleté à la pâte sur laquelle le décor effectué selon la méthode du
      brunissoir apparaît en traits brillant sur fond mat. Cette première
      approche se fonde sur deux séries issues de fouilles anciennes menées
      sur les oppidums du Cayla à Mailhac (Aude) et de Mourrel-Ferrat à
      Olonzac (Hérault). La carte de répartition fait état d'échanges ou de
      commerce à l'échelon macrorégional rarement mis en évidence pour de la
      céramique non tournée. S'il est difficile de statuer sur l'origine des
      décors, il semble que la production s'insère dans une ambiance
      celtisante. La chronologie de cette production se situe dans le
      deuxième âge du Fer. La fourchette proposée entre la fin du
      IV$^\text{e}$ et la fin du II$^\text{e}$ s. av. J.-C. reste encore à
      préciser.\\

      \textbf{Termes-clés de référence~:} distribution~; mourrel-ferrat~;
      olonzac~; le cayla~; mailhac~; micassé~; céramique non-tournée~; celtes~;
      production~; echange~; commerce~; cartographie~; habitat~; oppidum~; site
      fortifié~; fouille ancienne~; identification~; décor~; analyse~;
      répartition~; diffusion~; chronologie~; typologie~; céramique~; etude du
      matériel~; hérault~; aude~; france~; europe~; la tène~; age du fer.
    }
  }~\\

  \vspace{1.5em}

  \newcommand{\xslant}{0.25}
  \newcommand{\yslant}{0}
  \centering
  \begin{tikzpicture}[transform shape, scale=.75]
    % domain %%%%%%%%%%%%%%%%%%%%%%%%%%%%%%%%%%%%%%%%%%%%%%%%%%%%%%%%%%%%%%%%%%%
    \node [draw,
           rectangle,
           minimum width=1.15\linewidth,
           minimum height=.26\textheight,
           xslant=\xslant,
           yslant=\yslant] (domain_graph) {};
    \node [above=of domain_graph,
           xshift=.585\linewidth,
           yshift=.26\textheight,
           anchor=south east] (domain_graph_label) {termes-clés du domaine (sous-partie)};

    % nodes
    \node [above=of domain_graph,%draw,
           xshift=-1.9em,
           yshift=.23\textheight] (france) {france};
    \node [above=of domain_graph,%draw,
           xshift=-7.3em,
           yshift=.18\textheight] (typologie) {typologie};
    \node [above=of domain_graph,%draw,
           xshift=4.3em,
           yshift=.18\textheight] (chronologie) {chronologie};
    \node [above=of domain_graph,%draw,
           xshift=-20em,
           yshift=.14\textheight] (ceramique) {céramique};
    \node [above=of domain_graph,%draw,
           xshift=16.25em,
           yshift=.14\textheight] (production) {production};
    \node [above=of domain_graph,%draw,
           xshift=2.2em,
           yshift=.059\textheight] (analyse) {analyse};
    \node [above=of domain_graph,%draw,
           xshift=-5.2em,
           yshift=.055\textheight] (decor) {décor};
    \node [above=of domain_graph,%draw,
           xshift=-14em,
           yshift=.005\textheight] (repartition) {répartition};
    \node [above=of domain_graph,%draw,
           xshift=10.5em,
           yshift=.009\textheight] (diffusion) {diffusion};

    % france
    \draw (france) -- (typologie);
    \draw (france) -- (chronologie);
    \draw (france) -- (production);
    \draw (france) -- (diffusion);
    \draw (france) -- (decor);
    \draw (france) -- (analyse);
    \draw (france) -- (repartition);
    \draw (france) -- (ceramique);
    % typologie
    \draw (typologie) -- (chronologie);
    \draw (typologie) -- (production);
    \draw (typologie) -- (diffusion);
    \draw (typologie) -- (decor);
    \draw (typologie) -- (analyse);
    \draw (typologie) -- (repartition);
    \draw (typologie) -- (ceramique);
    % chronologie
    \draw (chronologie) -- (production);
    \draw (chronologie) -- (diffusion);
    \draw (chronologie) -- (decor);
    \draw (chronologie) -- (analyse);
    \draw (chronologie) -- (repartition);
    \draw (chronologie) -- (ceramique);
    % production
    \draw (production) -- (diffusion);
    \draw (production) -- (decor);
    \draw (production) -- (analyse);
    \draw (production) -- (ceramique);
    % diffusion
    \draw (diffusion) -- (decor);
    \draw (diffusion) -- (analyse);
    \draw (diffusion) -- (ceramique);
    % decor
    \draw (decor) -- (analyse);
    \draw (decor) -- (ceramique);
    % analyse
    \draw (analyse) -- (ceramique);

    % document %%%%%%%%%%%%%%%%%%%%%%%%%%%%%%%%%%%%%%%%%%%%%%%%%%%%%%%%%%%%%%%%%
    \node [draw,
          rectangle,
          minimum width=1.15\linewidth,
          minimum height=.26\textheight,
          xslant=\xslant,
          yslant=\yslant,
          above=of domain_graph,
          xshift=-3.9em] (document_graph) {};
    \node [below=of document_graph,
           xshift=-.585\linewidth,
           yshift=-.26\textheight,
           anchor=north west] (document_graph_label) {sujets du document (sous-partie)};

    \node [minimum width=2.5em,
           below=of document_graph,%draw,
           xshift=-3.4em,
           yshift=-.025\textheight] (typologie_c) {[typologie]};
    \node [minimum width=2.5em,
           below=of document_graph,%draw,
           xshift=12.5em,
           yshift=-.025\textheight] (chronologie_c) {[chronologie]};
    \node [minimum width=2.5em,
           below=of document_graph,%draw,
           xshift=-16.1em,
           yshift=-.135\textheight] (ceramique_c) {[céramique]};
    \node [minimum width=2.5em,
           below=of document_graph,%draw,
           xshift=20.15em,
           yshift=-.08\textheight] (production_c) {[production]};
    \node [minimum width=2.5em,
           below=of document_graph,%draw,
           xshift=6.1em,
           yshift=-.135\textheight] (analyse_c) {[analyse]};
    \node [minimum width=2.5em,
           below=of document_graph,%draw,
           xshift=2em,
           yshift=-.19\textheight] (decor_c) {[décors~; décor]};
    \node [minimum width=2.5em,
           below=of document_graph,%draw,
           xshift=-10.1em,
           yshift=-.19\textheight] (repartition_c) {[répartition]};
    \node [minimum width=2.5em,
           below=of document_graph,%draw,
           xshift=19em,
           yshift=-.19\textheight] (diffusion_c) {[diffusion]};
    \node [minimum width=2.5em,
           below=of document_graph,%draw,
           xshift=10.3em,
           yshift=-.08\textheight] (etude_preliminaire_c) {[étude préliminaire]};
    \node [minimum width=2.5em,
           below=of document_graph,%draw,
           xshift=20.15em,
           yshift=-.139\textheight] (fer_c) {[fer]};

    % typologie
    \draw (typologie_c) -- (chronologie_c);
    \draw (typologie_c) -- (diffusion_c);
    \draw (typologie_c) -- (decor_c);
    \draw (typologie_c) -- (analyse_c);
    \draw (typologie_c) -- (ceramique_c);
    % chronologie
    \draw (chronologie_c) -- (production_c);
    \draw (chronologie_c) -- (diffusion_c);
    \draw (chronologie_c) -- (ceramique_c);
    % production
    \draw (production_c) -- (decor_c);
    % diffusion
    \draw (diffusion_c) -- (ceramique_c);
    % decor
    \draw (decor_c) -- (analyse_c);
    \draw (decor_c) -- (repartition_c);
    \draw (decor_c) -- (ceramique_c);
    % analyse
    \draw (analyse_c) -- (ceramique_c);
    % répartition
    \draw (repartition_c) -- (ceramique_c);
    % étude préliminaire
    \draw (etude_preliminaire_c) -- (ceramique_c);
    \draw (etude_preliminaire_c) -- (chronologie_c);
    \draw (etude_preliminaire_c) -- (typologie_c);
    \draw (etude_preliminaire_c) -- (diffusion_c);
    % fer
    \draw (fer_c) -- (chronologie_c);
    \draw (fer_c) -- (production_c);

    % extra link %%%%%%%%%%%%%%%%%%%%%%%%%%%%%%%%%%%%%%%%%%%%%%%%%%%%%%%%%%%%%%%
    \draw [dashed] (typologie) -- (typologie_c);
    \draw [dashed] (chronologie) -- (chronologie_c);
    \draw [dashed] (ceramique) -- (ceramique_c);
    \draw [dashed] (production) -- (production_c);
    \draw [dashed] (analyse) -- (analyse_c);
    \draw [dashed] (decor) -- (decor_c);
    \draw [dashed] (repartition) -- (repartition_c);
    \draw [dashed] (diffusion) -- (diffusion_c);
  \end{tikzpicture}~\\

  \vspace{1em}

  \framebox[\linewidth]{
    \parbox{.99\linewidth}{\footnotesize
      \textbf{Sortie de TopicCoRank~:} \underline{céramique}~;
      \underline{décors}~; \underline{typologie}~; \underline{chronologie}~;
      \underline{production}~; étude préliminaire~; \underline{diffusion}~;
      \underline{analyse}~; \underline{france}~; \underline{répartition}.
    }
  }~\\

  \vspace{1em}

  \framebox[\linewidth]{
    \parbox{.99\linewidth}{\footnotesize
      \textbf{Sortie de TopicRank~:} \underline{décors}~;
      \underline{céramique}~; \underline{chronologie}~; \underline{typologie}~;
      \underline{production}~; fin~; étude préliminaire~; fer~; deuxième âge~;
      aire.
    }
  }

  \caption[
    Exemple d'extraction de termes-clés avec TopicCoRank.
  ]{
    Exemple d'extraction de termes-clés avec TopicCoRank sur le résumé de la
    notice d'archéologie présentée dans la figure~\ref{fig:example_inist} de la
    section~\ref{sec:main-data_description-termith_data}
    (page~\ref{fig:example_inist}). Les termes-clés soulignés sont les
    termes-clés correctement extraits.
    \label{fig:exemple_topiccorank}
  }
\end{figure}



\section{Paramètres expérimentaux}
\label{sec:main-domain_specific_keyphrase_annotation-supervised_automatic_keyphrase_annotation-evaluation}
  \subsection{Collections de données}
  \label{subsec:main-domain_specific_keyphrase_annotation-supervised_automatic_keyphrase_annotation-evaluation-evaluation_data}
    Nous conduisons nos expériences sur quatre collections de notices
    bibliographiques en domaines de spécialité~: Linguistique, Sciences de
    l'information, Archéologie et Chimie. Le tableau~\ref{tab:termith},
    présente ces collections. Chaque collection est répartie en deux
    sous-ensembles~: un ensemble d'apprentissage, que nous utilisons pour
    représenter le domaine, et un ensemble de test, pour l'évaluation. Les
    termes-clés de référence de chaque notice ont été proposés par les indexeurs
    professionnels de l'Inist (Institut de l’information scientifique et
    technique).
    \begin{table}[!h]
      \centering
      \resizebox{\linewidth}{!}{
        \begin{tabular}{l|c@{~~}c@{~~}c@{~~}c|c@{~~}c@{~~}c@{~~}c}
          \toprule
          \multirow{2}{*}{\textbf{Collection}} & \multicolumn{4}{c|}{\textbf{Documents}} & \multicolumn{4}{c}{\textbf{Termes-clés}}\\
          \cline{2-9}
          & Langue & Genre & Quantité & Mots moy. & Annotateur & Quantité moy. & Contrôlés & Mots moy.\\
          \hline
          Linguistique & & & & & & &\\
          \hfill{}Appr. & Français & Scientifique (résumé) & 515 & 160,5 & Professionnel & $~~$8,6 & 60,6~\% & 1,7\\
           \hfill{}Test & \ditto & \ditto & 200 & 147,0 & \ditto & $~~$8,9 & 62,8~\% & 1,8\\
          \hline
          Sciences de l'info. & & & & & & &\\
          \hfill{}Appr. & Français & Scientifique (résumé) & 506 & 105,0 & Professionnel & $~~$7,8 & 67,9~\% & 1,8\\
          \hfill{}Test & \ditto & \ditto & 200 & 157,0 & \ditto & 10,2 & 66,9~\% & 1,7\\
          \hline
          Archéologie & & & & & & &\\
          \hfill{}Appr. & Français & Scientifique (résumé) & 518 & 221,1 & Professionnel & 16,9 & 37,0~\% & 1,3\\
          \hfill{}Test & \ditto & \ditto & 200 & 213,9 & \ditto & 15,6 & 37,4~\% & 1,3\\
          \hline
          Chimie & & & & & & &\\
          \hfill{}Appr. & Français & Scientifique (résumé) & 582 & 105,7 &  Professionnel & 12,2 & 75,2~\% & 2,2\\
          \hfill{}Test & \ditto & \ditto & 200 & 103,9 & \ditto & 14,6 & 78,8~\% & 2,4\\
          \bottomrule
        \end{tabular}
      }

      \caption{Collections de données
               \label{tab:termith}}
    \end{table}

    La quantité de termes-clés contrôlés indiquée dans le
    tableau~\ref{tab:termith} montre l'importance de l'indexation contrôlée. En
    effet, plus de la moitié des termes-clés ne peuvent pas être obtenus par
    indexation libre.

  \subsection{Méthodes de référence}
  \label{subsec:main-domain_specific_keyphrase_annotation-supervised_automatic_keyphrase_annotation-evaluation-baselines}
    Dans nos expériences, nous comparons notre méthode, que nous appelerons
    TopicRankSpe à \textsc{Tf-Idf}, TopicRank et \textsc{Kea++}. Pour cette
    dernière, nous utilisons les thésaurus décrivant les vocabulaires contrôlés
    de l'Inist en linguistique, sciences de l'information, archéologie et chimie.

    Afin de mesurer l'efficacité de l'ordonnancement conjoint, nous comparons
    aussi TopicRankSpe à deux variantes. La première,
    TopicRankSpe$_\textnormal{\textit{libre}}$, ne réalise que l'indexation
    libre~; la seconde, TopicRankSpe$_\textnormal{\textit{contrôlé}}$, n'effectue
    que l'indexation contrôlée.
  
  \subsection{Mesures d'évaluation}
  \label{subsec:main-domain_specific_keyphrase_annotation-supervised_automatic_keyphrase_annotation-evaluation-evaluation_measures}
    Les performances des méthodes d'extraction de termes-clés sont exprimées
    en termes de précision (P), rappel (R) et f1-mesure (F). En
    accord avec l'évaluation menée dans les travaux précédents, les
    opérations de comparaison entre les termes-clés de référence et les
    termes-clés extraits sont effectuées à partir de la racine des mots qui
    les composent. Pour cela, nous utilisons la méthode de
    \newcite{porter1980suffixstripping}.

    Nous représentons aussi les résultats sous la forme de courbes de
    rappel--précision. Celles-ci permettent d'observer si une méthode domine
    les autres pour les critères de rappel et de précision. En optimisation
    multi-critères, nous parlons de front de Pareto optimal, c'est à dire de
    la méthode pour laquelle aucune autre méthode n'obtient de meilleures
    performances. Pour générer ces courbes, nous calculons la précision et
    le rappel lorsque  le nombre de termes-clés extraits/assignés varie de
    un jusqu'au plus grand nombre commun de termes-clés pouvant être
    extraits/assignés\footnote{Si, parmi tous les documents de test, le
    nombre minimum de termes-clés extraits/assignés pour un document est de
    73, alors la précision et le rappel sont calculés pour un jusqu'à 73
    termes-clés en moyenne pour tous les documents.}.
  
\section{Résultats}
\label{sec:main-domain_specific_keyphrase_annotation-supervised_automatic_keyphrase_annotation-evaluation-topiccorank_specific_domains}
  Nous réalisons ici une série d'expériences destinées à comparer
  TopicRankSpe à l'existant, puis à observer son comportement selon
  différentes configurations.

  Le tableau~\ref{tab:topiccorank-comparison_results_termith} montre les
  performances de TopicRankSpe en domaines de spécialité (linguistique,
  sciences de l'information, archéologie, chimie) comparées à celles des
  méthodes de référence. De manière générale, les résultats montrent le
  bien fondé de TopicRankSpe~: la variante
  TopicRankSpe$_\textnormal{assign.}$ réalise les meilleures performances,
  suivie par TopicRankSpe et TopicRankSpe$_\textnormal{extr.}$. Les faibles
  performances de \textsc{Kea++} sont surprenantes, d'autant plus que la
  seule autre méthode d'assignement, TopicRankSpe$_\textnormal{assign.}$,
  est celle qui réalise les meilleures. Contrairement à
  TopicRankSpe$_\textnormal{assign.}$, \textsc{Kea++} se limite aux entrées
  du thésaurus qui occurrent dans le document, alors que la majorité des
  termes-clés des collections Termith n'apparaissent pas dans les
  documents. De plus les thésaurus de l'Inist ne sont pas aussi riches que
  ceux utilisés par \newcite{medelyan2006kea++} dans leurs expériences~:
  moins de relations y sont définies entre les concepts. TopicRankSpe et
  ses variantes sont significativement meilleurs que les méthodes de
  référence. Comparées à celles de TopicRank, les performances de
  TopicRankSpe$_\textnormal{extr.}$ montrent que le domaine apporte des
  informations permettant d'ordonner plus précisément les sujets du
  document. Le fait que TopicRankSpe$_\textnormal{assign.}$ obtienne les
  meilleures performances montre aussi que les termes-clés du domaine sont
  ordonnés efficacement d'après le contenu du document (ses sujets). La
  prédominance de termes-clés à assigner dans les données Termith est la
  principale raison pour laquelle la variante
  TopicRankSpe$_\textnormal{assign.}$ est plus performante que TopicRankSpe.
  \begin{table}
    \resizebox{\linewidth}{!}{
      \begin{tabular}{l|ccc|c@{~~~~~~~}cc|ccc|ccc}
        \toprule
        \multirow{2}{*}{\textbf{Méthode}} & \multicolumn{3}{c|}{\textbf{Linguistique}} & \multicolumn{3}{c|}{\textbf{Sciences de l'info.}} & \multicolumn{3}{c|}{\textbf{Archéologie}} & \multicolumn{3}{c}{\textbf{Chimie}}\\
        \cline{2-13}
        & P & R & F & P & R & F & P & R & F & P & R & F\\
        \hline
        \textsc{Tf-Idf} & 13,3 & 15,8 & 14,2 & 13,5 & 14,2 & 13,4$^{~~}$ & 28,2 & 19,2 & 22,3$^{~~}$ & 15,8 & 12,3 & 13,2$^{~~}$\\
        TopicRank & 11,8 & 13,8 & 12,5 & 12,2 & 12,8 & 12,2$^{~~}$ & 29,9 & 20,3 & 23,7$^{~~}$ & 14,6 & 11,5 & 12,3$^{~~}$\\
        KEA++ & 11,6 & 13,0 & 12,1 & $~~$9,5 & 10,2 & $~~$9,6$^{~~}$ & 23,5 & 16,2 & 18,8$^{~~}$ & 11,4 & $~~$8,5 & $~~$9,2$^{~~}$\\
        \hline
        TopicRankSpe$_\text{libre}$ & 14,3 & 16,5 & 15,1 & 15,4 & 15,9 & 15,2$^\ddagger$ & 36,7 & 24,6 & 28,8$^\dagger$ & 15,8 & 12,1 & 13,1$^{~~}$\\
        TopicRankSpe$_\text{contrôlé}$ & \textbf{24,5} & \textbf{28,3} & \textbf{25,8} & \textbf{19,7} & \textbf{19,8} & \textbf{19,2}$^\ddagger$ & \textbf{47,8} & \textbf{32,3} & \textbf{37,7}$^\dagger$ & \textbf{20,0} & \textbf{14,8} & \textbf{16,3}$^\dagger$\\
        \hline
        TopicRankSpe & 18,8 & 21,9 & 19,9 & 17,3 & 17,7 & 17,0$^\ddagger$ & 38,3 & 25,7 & 30,1$^\dagger$ & 17,2 & 13,4 & 14,4$^\ddagger$\\
        \bottomrule
      \end{tabular}
    }
  \caption[
    Résultat de l'extraction de dix termes-clés avec \textsc{Tf-Idf},
    TopicRank, \textsc{Kea++}, TopicRankSpe$_\textnormal{\textit{extr.}}$,
    TopicRankSpe$_\textnormal{\textit{assign.}}$ et TopicRankSpe appliqués
    aux collections Termith
  ]{
    Résultat de l'extraction de dix termes-clés avec \textsc{Tf-Idf},
    TopicRank, \textsc{Kea++}, TopicRankSpe$_\textnormal{\textit{extr.}}$,
    TopicRankSpe$_\textnormal{\textit{assign.}}$ et TopicRankSpe appliqués
    aux collections Termith. $\dagger$ et $\ddagger$ indiquent une
    amélioration significative vis-à-vis des méthodes de référence, à
    0,001 et 0,05 pour le t-test de Student, respectivement.
    \label{tab:topiccorank-comparison_results_termith}}
  \end{table}
  
  La figure~\ref{fig:topiccorank-pr_curves_termith} permet de comparer le
  comportement respectif des méthodes de référence, de TopicRankSpe et de
  ses variantes. Elle montre que TopicRankSpe et ses variantes dominent les
  méthodes de référence (front de Pareto) selon les critères de précision
  et de rappel. Parmi elles, nous observons aussi que la variante
  TopicRankSpe$_\textnormal{assign.}$ domine la variante
  TopicRankSpe$_\textnormal{extr.}$, mais que TopicRankSpe n'est, ni
  dominante, ni dominé par elles. Bien que l'amélioration significative de
  TopicRank par TopicRankSpe et ses variantes montrent l'apport de
  l'ordonnancement conjoint entre sujets du document et termes-clés du
  domaine, la réalisation simultanée de l'extraction et de l'assignement
  reste difficile.
  \begin{figure}
  \centering
  \subfigure[Linguistique \textit{(fr)}]{
    \begin{tikzpicture}[scale=.8]
      \pgfkeys{/pgf/number format/.cd, fixed}
      \begin{axis}[x=0.004275\linewidth,
                   xtick={0, 20, 40, ..., 100},
                   xmin=0,
                   xmax=80,
                   xlabel=rappel (\%),
                   x label style={yshift=.34em},
                   y=0.004275\linewidth,
                   ytick={0, 20, ..., 100},
                   ymin=0,
                   ymax=80,
                   ylabel=précision (\%),
                   y label style={yshift=-1.1em}]
        \addplot [green!66, mark=x] file {input/figure/data/linguistique_tfidf.csv};
        \addplot [red!66, mark=+] file {input/figure/data/linguistique_topicrank.csv};
        \addplot [cyan!66, mark=o] file {input/figure/data/linguistique_kea_pp.csv};
        \addplot [orange!66, mark=square] file {input/figure/data/linguistique_topiccorank_extr.csv};
        \addplot [black!66, mark=triangle] file {input/figure/data/linguistique_topiccorank_assign.csv};
        \addplot [gray!66, mark=diamond] file {input/figure/data/linguistique_topiccorank.csv};
        %%%%%%%%%%%%%%%%%%%%%%%%%%%%%%%%%%%%%%%%%%%%%%%%%%%%%%%%%%%%%%%%%%%%%%%%
        \addplot [dotted, domain=55:100] {(70 * x) / ((2 * x) - 70)};
        \addplot [dotted, domain=45:100] {(60 * x) / ((2 * x) - 60)};
        \addplot [dotted, domain=35:100] {(50 * x) / ((2 * x) - 50)};
        \addplot [dotted, domain=25:100] {(40 * x) / ((2 * x) - 40)};
        \addplot [dotted, domain=15:100] {(30 * x) / ((2 * x) - 30)};
        \addplot [dotted, domain=10:100] {(20 * x) / ((2 * x) - 20)};
        \addplot [dotted, domain=5:100] {(10 * x) / ((2 * x) - 10)};
      \end{axis}
      \node at (4.85,4.0) [anchor=east] {\footnotesize{F=70,0}};
      \node at (4.85,3.1) [anchor=east] {\footnotesize{F=60,0}};
      \node at (4.85,2.4) [anchor=east] {\footnotesize{F=50,0}};
      \node at (4.85,1.8) [anchor=east] {\footnotesize{F=40,0}};
      \node at (4.85,1.3) [anchor=east] {\footnotesize{F=30,0}};
      \node at (4.85,0.85) [anchor=east] {\footnotesize{F=20,0}};
      \node at (4.85,0.5) [anchor=east] {\footnotesize{F=10,0}};
    \end{tikzpicture}
  }
  \subfigure[Sciences de l'info. \textit{(fr)}]{
    \begin{tikzpicture}[scale=.8]
      \pgfkeys{/pgf/number format/.cd, fixed}
      \begin{axis}[x=0.004275\linewidth,
                   xtick={0, 20, 40, ..., 100},
                   xmin=0,
                   xmax=80,
                   xlabel=rappel (\%),
                   x label style={yshift=.34em},
                   y=0.004275\linewidth,
                   ytick={0, 20, ..., 100},
                   ymin=0,
                   ymax=80,
                   ylabel=précision (\%),
                   y label style={yshift=-1.1em}]
        \addplot [green!66, mark=x] file {input/figure/data/sciences_de_l_information_tfidf.csv};
        \addplot [red!66, mark=+] file {input/figure/data/sciences_de_l_information_topicrank.csv};
        \addplot [cyan!66, mark=o] file {input/figure/data/sciences_de_l_information_kea_pp.csv};
        \addplot [orange!66, mark=square] file {input/figure/data/sciences_de_l_information_topiccorank_extr.csv};
        \addplot [black!66, mark=triangle] file {input/figure/data/sciences_de_l_information_topiccorank_assign.csv};
        \addplot [gray!66, mark=diamond] file {input/figure/data/sciences_de_l_information_topiccorank.csv};
        %%%%%%%%%%%%%%%%%%%%%%%%%%%%%%%%%%%%%%%%%%%%%%%%%%%%%%%%%%%%%%%%%%%%%%%%
        %\addplot [dotted, domain=55:100] {(70 * x) / ((2 * x) - 70)};
        %\addplot [dotted, domain=45:100] {(60 * x) / ((2 * x) - 60)};
        %\addplot [dotted, domain=35:100] {(50 * x) / ((2 * x) - 50)};
        %\addplot [dotted, domain=25:100] {(40 * x) / ((2 * x) - 40)};
        \addplot [dotted, domain=15:100] {(30 * x) / ((2 * x) - 30)};
        \addplot [dotted, domain=10:100] {(20 * x) / ((2 * x) - 20)};
        \addplot [dotted, domain=5:100] {(10 * x) / ((2 * x) - 10)};
        %%%%%%%%%%%%%%%%%%%%%%%%%%%%%%%%%%%%%%%%%%%%%%%%%%%%%%%%%%%%%%%%%%%%%%%%
        \legend{\textsc{Tf-Idf}, TopicRank, \textsc{Kea++},
                TopicCoRank$_\textnormal{extr.}$,
                TopicCoRank$_\textnormal{assign.}$, TopicCoRank};
      \end{axis}
      %\node at (4.85,4.0) [anchor=east] {\footnotesize{F=70,0}};
      %\node at (4.85,3.1) [anchor=east] {\footnotesize{F=60,0}};
      %\node at (4.85,2.4) [anchor=east] {\footnotesize{F=50,0}};
      %\node at (4.85,1.8) [anchor=east] {\footnotesize{F=40,0}};
      \node at (4.85,1.3) [anchor=east] {\footnotesize{F=30,0}};
      \node at (4.85,0.85) [anchor=east] {\footnotesize{F=20,0}};
      \node at (4.85,0.5) [anchor=east] {\footnotesize{F=10,0}};
    \end{tikzpicture}
  }
  \subfigure[Archeologie \textit{(fr)}]{
    \begin{tikzpicture}[scale=.8]
      \pgfkeys{/pgf/number format/.cd, fixed}
      \begin{axis}[x=0.004275\linewidth,
                   xtick={0, 20, 40, ..., 100},
                   xmin=0,
                   xmax=80,
                   xlabel=rappel (\%),
                   x label style={yshift=.34em},
                   y=0.004275\linewidth,
                   ytick={0, 20, ..., 100},
                   ymin=0,
                   ymax=80,
                   ylabel=précision (\%),
                   y label style={yshift=-1.1em}]
        \addplot [green!66, mark=x] file {input/figure/data/archeologie_tfidf.csv};
        \addplot [red!66, mark=+] file {input/figure/data/archeologie_topicrank.csv};
        \addplot [cyan!66, mark=o] file {input/figure/data/archeologie_kea_pp.csv};
        \addplot [orange!66, mark=square] file {input/figure/data/archeologie_topiccorank_extr.csv};
        \addplot [black!66, mark=triangle] file {input/figure/data/archeologie_topiccorank_assign.csv};
        \addplot [gray!66, mark=diamond] file {input/figure/data/archeologie_topiccorank.csv};
        %%%%%%%%%%%%%%%%%%%%%%%%%%%%%%%%%%%%%%%%%%%%%%%%%%%%%%%%%%%%%%%%%%%%%%%%
        \addplot [dotted, domain=55:100] {(70 * x) / ((2 * x) - 70)};
        \addplot [dotted, domain=45:100] {(60 * x) / ((2 * x) - 60)};
        \addplot [dotted, domain=35:100] {(50 * x) / ((2 * x) - 50)};
        \addplot [dotted, domain=25:100] {(40 * x) / ((2 * x) - 40)};
        \addplot [dotted, domain=15:100] {(30 * x) / ((2 * x) - 30)};
        \addplot [dotted, domain=10:100] {(20 * x) / ((2 * x) - 20)};
        \addplot [dotted, domain=5:100] {(10 * x) / ((2 * x) - 10)};
      \end{axis}
      \node at (4.85,4.0) [anchor=east] {\footnotesize{F=70,0}};
      \node at (4.85,3.1) [anchor=east] {\footnotesize{F=60,0}};
      \node at (4.85,2.4) [anchor=east] {\footnotesize{F=50,0}};
      \node at (4.85,1.8) [anchor=east] {\footnotesize{F=40,0}};
      \node at (4.85,1.3) [anchor=east] {\footnotesize{F=30,0}};
      \node at (4.85,0.85) [anchor=east] {\footnotesize{F=20,0}};
      \node at (4.85,0.5) [anchor=east] {\footnotesize{F=10,0}};
    \end{tikzpicture}
  }
  \subfigure[Chimie \textit{(fr)}]{
    \begin{tikzpicture}[scale=.8]
      \pgfkeys{/pgf/number format/.cd, fixed}
      \begin{axis}[x=0.00855\linewidth,
                   xtick={0, 20, 40, ..., 100},
                   xmin=0,
                   xmax=40,
                   xlabel=rappel (\%),
                   x label style={yshift=.34em},
                   y=0.00855\linewidth,
                   ytick={0, 20, ..., 100},
                   ymin=0,
                   ymax=40,
                   ylabel=précision (\%),
                   y label style={yshift=-1.1em}]
        \addplot [green!66, mark=x] file {input/figure/data/chimie_tfidf.csv};
        \addplot [red!66, mark=+] file {input/figure/data/chimie_topicrank.csv};
        \addplot [cyan!66, mark=o] file {input/figure/data/chimie_kea_pp.csv};
        \addplot [orange!66, mark=square] file {input/figure/data/chimie_topiccorank_extr.csv};
        \addplot [black!66, mark=triangle] file {input/figure/data/chimie_topiccorank_assign.csv};
        \addplot [gray!66, mark=diamond] file {input/figure/data/chimie_topiccorank.csv};
        %%%%%%%%%%%%%%%%%%%%%%%%%%%%%%%%%%%%%%%%%%%%%%%%%%%%%%%%%%%%%%%%%%%%%%%%
        %\addplot [dotted, domain=30:100] {(40 * x) / ((2 * x) - 40)};
        \addplot [dotted, domain=20:100] {(30 * x) / ((2 * x) - 30)};
        \addplot [dotted, domain=10:100] {(20 * x) / ((2 * x) - 20)};
        \addplot [dotted, domain=5:100] {(10 * x) / ((2 * x) - 10)};
      \end{axis}
      %\node at (4.85,4.5) [anchor=east] {\footnotesize{F=40,0}};
      \node at (4.85,3.1) [anchor=east] {\footnotesize{F=30,0}};
      \node at (4.85,1.8) [anchor=east] {\footnotesize{F=20,0}};
      \node at (4.85,0.9) [anchor=east] {\footnotesize{F=10,0}};
    \end{tikzpicture}
  }
  \caption{Courbes de rappel-précision de \textsc{Tf-Idf}, TopicRank
           \textit{\textsc{Kea}++}, TopicCoRank$_\textnormal{extr.}$,
           TopicCoRank$_\textnormal{assign.}$ et TopicCoRank appliqués aux
           données Termith
           \label{fig:topiccorank-pr_curves_termith}}
\end{figure}



  Afin d'observer la place que prend l'assignement dans TopicRankSpe, et
  pour comprendre pourquoi sa variante TopicRankSpe$_\textnormal{assign.}$
  est plus performante, nous nous intéressons maintenant aux taux de
  termes-clés extraits et assignés par TopicRankSpe, présentés dans le
  tableau~\ref{tab:assignment_ratio_termith}. Nous observons que
  l'extraction est légèrement prédominante face à \mbox{l'assignement}. Les deux
  catégories d'indexation par termes-clés sont effectivement réalisées
  conjointement, mais l'ordonnancement donne plus d'importance aux sujets
  du document qu'aux termes-clés de référence du domaine. En domaines de
  spécialité où l'assignement est préféré, cela peut être résolu en
  travaillant sur un affinage des schémas de connexion des n\oe{}uds de
  chaque graphe et d'unification de ceux-ci.
  \begin{table}
    \centering
    \begin{tabular}{l|c|c}
        \toprule
        & Extraction (\%) & Assignement (\%)\\
        \hline
        Linguistique & 61,7 & 38,3\\
        Sciences de l'info. & 66,4 & 33,6\\
        Archéologie & 69,1 & 30,9\\
        Chimie & 68,4 & 31,6\\
        \bottomrule
    \end{tabular}
    \caption{Taux moyens d'extraction et d'assignement réalisés par
             TopicRankSpe sur les données Termith
             \label{tab:assignment_ratio_termith}}
  \end{table}

  Au delà du fait que TopicRankSpe$_\textnormal{assign.}$ obtient de
  meilleures performances que TopicRankSpe et
  TopicRankSpe$_\textnormal{extr.}$, nous faisons une expérience dans
  laquelle nous forçons le taux d'assignement afin de déterminer si
  l'ordonnancement des termes-clés du domaine est efficace.
  Un ordonnancement efficace des termes-clés du domaine doit induire une
  courbe de performance cumulative quand nous faisons croître le taux
  d'assignement\footnote{Dans cette situation, cela signifie que la
  performance obtenue avec TopicRankSpe$_\textnormal{assign.}$ est la
  performance maximale avec TopicRankSpe}. La
  figure~\ref{fig:assignment_variations_termith} montre la performance de
  TopicRankSpe lorsque le taux d'assignement varie de 0~\% à 100~\% avec un
  pas de 10~\%. À chaque augmentation du taux d'assignement, la
  performance de TopicRankSpe augmente. L'ordonnancement des termes-clés du
  domaine fait donc émerger efficacement ceux les plus
  importants vis-à-vis du document.
  \begin{figure}[h!]
  \centering
  \begin{tikzpicture}
    \pgfkeys{/pgf/number format/.cd, fixed}
    \begin{axis}[x=0.0040\linewidth,
                 xtick={0, 20, ..., 100},
                 xmin=0,
                 xmax=100,
                 xlabel=Assignement (\%),
                 x label style={yshift=.34em},
                 y=0.007\linewidth,
                 ytick={0, 20, ..., 100},
                 ymin=0,
                 ymax=60,
                 ylabel=F1-mesure (\%),
                 y label style={yshift=-1.1em}]
      \addplot[green!66, mark=x] coordinates{
        (0, 15.1)
        (10, 15.9)
        (20, 16.5)
        (30, 17.3)
        (40, 18.1)
        (50, 19.1)
        (60, 20.4)
        (70, 22.0)
        (80, 23.4)
        (90, 24.7)
        (100, 25.7)
      };
      \addplot[red!66, mark=+] coordinates{
        (0, 15.1992)
        (10, 15.8659)
        (20, 16.1269)
        (30, 16.5223)
        (40, 16.8308)
        (50, 17.1875)
        (60, 17.3450)
        (70, 17.8887)
        (80, 18.1184)
        (90, 18.7733)
        (100, 19.4089)
      };
      \addplot[cyan!66, mark=o] coordinates{
        (0, 28.7887)
        (10, 28.7239)
        (20, 28.8927)
        (30, 29.4833)
        (40, 29.4880)
        (50, 29.9271)
        (60, 31.2943)
        (70, 32.6718)
        (80, 34.4101)
        (90, 35.8757)
        (100, 37.5003)
      };
      \addplot[orange!66, mark=square] coordinates{
        (0, 13.0605)
        (10, 13.4498)
        (20, 13.8944)
        (30, 14.1412)
        (40, 14.5673)
        (50, 15.0916)
        (60, 15.4902)
        (70, 16.1045)
        (80, 16.2055)
        (90, 16.0077)
        (100, 16.0506)
      };
      \legend{Linguistique \textit{(fr)}, Sciences de l'info. \textit{(fr)}, Archéologie \textit{(fr)}, Chimie \textit{(fr)}};
    \end{axis}
  \end{tikzpicture}
  \caption{Comportement de TopicCoRank en fonction du taux d'assignement
           \label{fig:assignment_variations}}
\end{figure}



  Enfin, nous réalisons une dernière expérience dans laquelle nous faisons
  varier la valeur du paramètre $\lambda$. Plus sa valeur est élevée, plus
  l'influence de la recommandation interne est forte. La
  figure~\ref{fig:lambda_variations_termith} montre le comportement de
  TopicRankSpe lorsque nous faisons varier sa valeur de 0 à 1
  avec un pas de 0,1. En accord avec notre hypothèse que sujets et
  termes-clés du domaine doivent se recommander les uns les autres, les
  résultats montrent que les performances de TopicRankSpe se dégradent au
  delà de $\lambda = 0,5$, valeur quasi-optimale.
  \begin{figure}[h!]
  \centering
  \begin{tikzpicture}
    \pgfkeys{/pgf/number format/.cd, fixed}
    \begin{axis}[x=0.4\linewidth,
                 xtick={0, 0.2, ..., 1.0},
                 xmin=0,
                 xmax=1.0,
                 xlabel=$\lambda$,
                 x label style={yshift=.34em},
                 y=0.007\linewidth,
                 ytick={0, 20, ..., 100},
                 ymin=0,
                 ymax=60,
                 ylabel=F1-mesure (\%),
                 y label style={yshift=-1.1em}]
      \addplot[green!66, mark=x] coordinates{
        (0.1, 19.5942)
        (0.2, 19.8314)
        (0.3, 20.0452)
        (0.4, 19.6420)
        (0.5, 19.9371)
        (0.6, 18.5649)
        (0.7, 17.4499)
        (0.8, 16.1558)
        (0.9, 16.4438)
      };
      \addplot[red!66, mark=+] coordinates{
        (0.1, 17.3100)
        (0.2, 17.4176)
        (0.3, 17.4420)
        (0.4, 17.3653)
        (0.5, 17.0094)
        (0.6, 15.7458)
        (0.7, 14.5758)
        (0.8, 12.9788)
        (0.9, 11.4421)
      };
      \addplot[cyan!66, mark=o] coordinates{
        (0.1, 31.5290)
        (0.2, 31.3748)
        (0.3, 31.0511)
        (0.4, 30.7214)
        (0.5, 30.1052)
        (0.6, 31.1819)
        (0.7, 29.3852)
        (0.8, 27.3498)
        (0.9, 24.8661)
      };
      \addplot[orange!66, mark=square] coordinates{
        (0.1, 14.7061)
        (0.2, 14.8227)
        (0.3, 14.6960)
        (0.4, 14.2947)
        (0.5, 14.3646)
        (0.6, 14.1136)
        (0.7, 13.7321)
        (0.8, 12.3091)
        (0.9, 10.6809)
      };
      \legend{Linguistique \textit{(fr)}, Sciences de l'info. \textit{(fr)}, Archéologie \textit{(fr)}, Chimie \textit{(fr)}};
    \end{axis}
  \end{tikzpicture}
  \caption{Comportement de TopicCoRank en fonction du taux d'assignement
           \label{fig:assignment_variations}}
\end{figure}



%-----------------------------------------------------------------------------

\section{Conclusion}
\label{sec:main-domain_specific_keyphrase_annotation-conclusion}
  \TODO{...}

